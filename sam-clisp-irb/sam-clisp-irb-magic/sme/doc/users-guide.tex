\batchmode
\documentstyle[11pt,alltt]{article}
\pagestyle{myheadings}
\makeindex
\markboth{SME User's Manual}{SME User's Manual}
\setlength{\oddsidemargin}{-0.1in}
\setlength{\evensidemargin}{0in}
\setlength{\textwidth}{6.5in}
\setlength{\topmargin}{-0.3in}
\setlength{\textheight}{9in}
%% Declaring SME lisp hooks
\newcommand{\decvar}[1]{\noindent\makebox[\textwidth]{{\tt #1}
                                   \hfill [{\it Variable}]}\index{#1 variable}}
\newcommand{\decfunc}[2]{\noindent\makebox[\textwidth]{{\tt #1}\ \ {\it #2}
                                               \hfill [{\it Function}]}\index{#1 function}}
\newcommand{\decmacro}[2]{\noindent\makebox[\textwidth]{{\tt #1}\ \ {\it #2}
                                                \hfill [{\it Macro}]}\index{#1 macro}}
\newcommand{\decsubst}[2]{\noindent\makebox[\textwidth]{{\tt #1}\ \ {\it #2}
                                                \hfill [{\it Subst}]}\index{#1 subst}}
%% Handy abbreviations
\newcommand{\metasyntax}[1]{$\langle #1 \rangle$}
\newcommand{\spc}[1]{\vspace{#1\baselineskip}}
\newcommand{\gthan}{\( >\)}
\newcommand{\border}{\rule{\textwidth}{0.04in}}
\newcommand{\SME}{{\tt SME}}
\newcommand{\SMET}{\mbox{${\tt SME}_{{\it SMT}}$}}
\newcommand{\Phineas}{{\tt PHINEAS}}
\newcommand{\keypoint}[2]{\begin{itemize}
                             \item \parbox[t]{0.88\textwidth}{{\it #1:} #2}
                          \end{itemize}
                          \vspace{0.08in}}
%
\begin{document}

\begin{titlepage}
{\center {\Large Technical Report UIUCDCS-R-88-1421 \\ }}
\ \ \ \vspace{0.5in} \\
{\center {\bf {\huge The SME User's Manual} \\[2ex]
              {\LARGE  (SME Version 2E)} \\ }}
\vspace{0.7in}
\begin{center}
            {\Large Brian Falkenhainer}       \\[2ex]
{\large     Qualitative Reasoning Group       \\
          Department of Computer Science      \\
         University of Illinois at Urbana-Champaign \\
\vspace{0.75in}
                December, 1988 \\ }
\end{center}
\vspace{0.75in}
\begin{abstract}
This paper documents the {\it Structure-Mapping Engine} (\SME), a
general-purpose program for studying analogical processing.  It provides a
comprehensive description of the program and instructions for using it,
including techniques for integrating it into larger systems. One section
demonstrates methods for configuring \SME\ to a variety of mapping
preferences and suggests the range of theoretical variations available.
\end{abstract}

\end{titlepage}
\pagenumbering{roman}
\tableofcontents
\markboth{SME User's Manual}{SME User's Manual}
\newpage
\pagenumbering{arabic}

\section{Introduction}

The Structure-Mapping Engine (\SME) is a general tool for performing various
types of analogical mappings. \SME\ was originally developed to simulate
Gentner's {\it Structure-Mapping} theory of analogy
\cite{smt-cogsci,allerton,analogica}. It was hoped that the developed system
would also be able to model the other types of similarity comparisons
sanctioned by Gentner's theory, such as {\it literal similarity} and {\it
mere appearance}. What ended up being developed was an extremely flexible
and efficient system. Most theoretical assumptions are left out of the
program and are supplied through match rules. Thus, while \SME\ was
originally designed to simulate the comparisons of structure-mapping theory,
it may simulate many others as well. Given a set of theoretical restrictions
on what constitutes a reasonable analogical mapping, one may implement these
restrictions in the form of rules and use \SME\ to interactively test their
consequences. This report is intended to make that task easier.

This paper is designed for those interested in using \SME\ for studying
analogical processing, testing alternate theories, or as the mapping
component in a larger system. It describes the options and user support
provided in \SME, how to use it for testing theories, and how to integrate
it with other programs. For a discussion of the theory behind \SME, the
general algorithm, and descriptions of the program in operation, one should
consult \cite{aaai86,sme-aij} prior to reading this manual. Descriptions of
the use of \SME\ in various research projects may be found in
\cite{VBAL,MLW,DBR,Falken-phd,allerton,tinlap3,aaai87}, while descriptions
of Gentner's Structure-Mapping theory appear in
\cite{Gentner80,smt-cogsci,allerton,analogica,framework,sme-aij}.

\subsection{Conventions}\index{conventions}

Throughout this guide, a few conventions will be used which should be
explained at this time.

\begin{enumerate}

 \item {\it CommonLisp Packages}.\index{packages} The \SME\ system resides
in its own package, {\tt SME}, which is defined to use CommonLisp. As a
result, any reference to an \SME\ function or variable must specify the {\tt
SME} package, as in the function {\tt sme:define-predicate}. To simplify the
discussion, we will omit the package prefix when describing \SME\ functions,
macros, and variables. In addition, while the \SME\ routines reside in the
{\tt SME} package, the structures it manipulates reside in the general {\tt
USER} package.
 \item {\it The declarative interface.}\index{conventions, function names}
In general, the routines used to present data items to \SME, such as
predicate definitions and concept descriptions, appear in two,
functionally-equivalent forms. These two types have a naming convention
associated with each. The most common type is the {\it declarative} or macro
interface. The declarative routines do not evaluate their arguments and
match the syntactic form {\tt def}{\it routine-name}. For example, to define
the entity {\tt sun} declaratively, one writes {\tt (defEntity sun)}. The
second type is the {\it functional} interface, which is present to support
declarations by external programs.  These routines evaluate their arguments
and match the syntactic form {\tt define-}{\it routine-name}. For example,
to define the entity {\tt sun} functionally, one writes {\tt (define-entity
'sun)}.

\end{enumerate}


\subsection{File Organization}\index{file organization}

\SME\ is contained within the following twelve files:

\begin{description}
 \item[config.lisp]  The declarations for site specific
parameters.\index{parameters, site specific}
 \item[defs.lisp]  The basic structure definitions and macros used throughout
\SME. 
 \item[bits.lisp]  Routines for creating and manipulating bit vectors.
 \item[bms.lisp]  The belief-maintenance system (BMS) - a probabilistic TMS.
 \item[bms-tre.lisp]  The rule system and problem-solver front end for the BMS.
 \item[sme.lisp]  The \SME\ top-level routines, such as initialization,
defining facts about a concept, and fetching and storing facts and concept
descriptions.
 \item[match.lisp]  The \SME\ mapping algorithm.
 \item[match-rules-support.lisp]  A few functions useful for writing \SME\
match rules.
 \item[display.lisp]  Machine independent output routines.
 \item[windowing.lisp]  Symbolics dependent interface routines.
 \item[batch.lisp]  Routines to enable execution in {\it batch} mode with a
final report generated.
 \item[generalize.lisp]  Inductive generalization support.
\end{description}


\section{System Review}

The Structure-Mapping Engine can simulate a class of {\it structural
approaches} to analogical mapping. In these approaches, there is a distinct
stage of matching and carryover of predicates from one domain (the {\it
base}) into another (the {\it target}) within the larger analogy process.
Furthermore, although there are a number of differences, there is widespread
agreement among these techniques on one fundamental restriction
\cite{Burstein83,Carbonell83,Greiner,Indurkhya,KedarCabelli,Reed,Rumelhart,Winston82}:
\begin{enumerate}
 \item {\it Structural consistency}. If a final analogical mapping includes
a predicate in the base paired with a predicate in the target, then it must
also include corresponding pairings between each of their arguments. This
criterion simply asserts that an analogical mapping must not produce
syntactically meaningless predicate calculus forms.
\end{enumerate}
In \SME, this restriction was enforced by the requirement of simulating
Structure-Mapping theory; its impact on the algorithm is described in
\cite{sme-aij}. However, this restriction is only part of that theory and
alone does not uniquely define a matching algorithm. Additional theoretical
restrictions must be supplied through {\it match rules}. This enables \SME\
to be used in exploring the space of theories consistent with this single
criterion. An additional restriction is enforced by default:

\keypoint{One-to-one mapping}{No base item (predicate or object) may be
paired with multiple target items. Likewise, no target item may be paired
with multiple base items.}

Enforcement of the one-to-one restriction is a global parameter which may be
disabled. Support is provided to implement variations of one-to-one within
the match rules.

Match rules specify what pairwise matches are possible and provide local
measures of evidence used in computing the evaluation score. These rules are
the key to \SME's flexibility. To build a new matcher one simply loads a new
set of match rules.  This has several important advantages.  First, we can
simulate all of the types of comparisons sanctioned by Structure-Mapping
theory with one program.  Second, the rules could in theory be ``tuned'' if
needed to simulate particular kinds of human performance. Third, a variety
of other analogical mapping systems may be simulated for comparison and
theoretical investigation.  The breadth of the space of these structural
approaches is suggested by the examples in Section~\ref{ss:rule-system}.

In this section, the \SME\ matching algorithm is briefly reviewed, followed
by a short discussion of how theoretical guidelines may be added to the
general mechanism. It is a summary of the algorithm description appearing in
\cite{sme-aij}, annotated with the \SME\ functions that carry out each step.


\begin{figure}
\border
\vspace{2.1in}
\caption{Simplified water flow and heat flow descriptions.}
\label{fg:wfhf-calculus}
\border
\end{figure}


\subsection{Algorithm Review}\index{algorithm, summary}

Given descriptions of a base and a target (called Dgroups), \SME\ builds all
structurally consistent interpretations of the comparison between them.
Each interpretation of the match is called a {\em global mapping}, or {\em
Gmap}.  Gmaps consist of three parts:

\begin{enumerate}
\item{\em Correspondences:} A set of pairwise matches between the
expressions and entities of the two dgroups.
\item {\em Candidate Inferences:} A set of new expressions which the
comparison suggests holds in the target dgroup.
\item {\em Structural Evaluation Score:} (Called {\em SES} for
brevity) A numerical estimate of match quality.
\end{enumerate}

For example, given the descriptions of water flow and heat flow shown in
Figure~\ref{fg:wfhf-calculus}, \SME\ might, depending on the current
theoretical configuration, offer several alternative interpretations for
this potential analogy.  In one interpretation, the central inference is
that water flowing from the beaker to the vial corresponds to heat flowing
from the coffee to the ice cube. Alternatively, one could map water to
coffee, since they are both liquids.


\begin{figure}
\border
\begin{small}
\begin{tt}
\begin{itemize}
 \item Run MHC rules to construct match hypotheses (create-match-hypotheses).
 \item Calculate the {\it Conflicting} set for each match hypothesis (calculate-nogoods).
 \item Calculate the {\it EMaps} and {\it NoGood} sets for each match
        hypothesis by upward propagation from entity mappings
        (generate-justifications and propagate-descendants).
 \item During the propagation, delete any match hypotheses that have
        justification holes (propagate-death).
 \item Merge match hypotheses into Gmaps (generate-gmaps).
      \begin{enumerate}
         \item Interconnected and consistent (generate-structure-groups).
         \item Consistent members of same base structure (merge-base).
         \item Any further consistent combinations (full-gmap-merge).
      \end{enumerate}
 \item Calculate the candidate inferences for each GMap (gather-inferences).
 \item Score the matches (run-rules).
      \begin{enumerate}
         \item Local match scores.
         \item Global structural evaluation scores.
      \end{enumerate}
\end{itemize}
\end{tt}
\end{small}
  \caption{Summary of SME algorithm.}
  \label{fg:sme-overview}
\border
\end{figure}


The \SME\  algorithm (see Figure~\ref{fg:sme-overview}) is logically divided
into four stages:
\begin{enumerate} 
\item {\em Local match construction:} Finds all pairs of ({\it BaseItem},
{\it TargetItem}) that potentially can match.  A {\em Match Hypothesis} is
created for each such pair to represent the possibility that this local
match is part of a global match.
\item {\em Gmap construction:} Combines the local matches into maximal
consistent collections of correspondences.
\item {\em Candidate inference construction:} Derives the inferences
suggested by each Gmap.
\item {\em Match Evaluation:} Attaches evidence to each local match and
uses this evidence to compute structural evaluation scores for each Gmap. 
\end{enumerate}

\noindent
Each computation will now be reviewed, using a simple example to
illustrate their operation. In this example, the rules of structure-mapping
theory are in use. It is important to distinguish the general \SME\ system
from its behavior when using the rules of a particular theory. Hence, when
using the rules of structure-mapping theory, it will be called \SMET.

\subsubsection{Step 1: Local match construction (create-match-hypotheses)}\index{match hypothesis}\index{algorithm}

\begin{figure}
\border
\vspace{3.1in}
\caption[Local Match Construction]{Local Match Construction. The water flow
and heat flow descriptions of Figure~\ref{fg:wfhf-calculus} have been drawn
in the abstract and placed to the left and right, respectively. The objects
in the middle depict match hypotheses.}
\label{fg:3D-MH}\index{algorithm}
\border
\end{figure}


\SME\ begins by finding for each entity and predicate in the base the set of
entities or predicates in the target that could plausibly match that item
(see Figure~\ref{fg:3D-MH}). Plausibility is determined by {\it match
constructor} rules, which are of the form:
\begin{tt}
\begin{tabbing}
\ \ \= (MHCrule \= (\metasyntax{Trigger} \= \metasyntax{BaseVariable} \metasyntax{TargetVariable} \\
\> \> \> {\it [}:test \metasyntax{TestForm}{\it ]}) \\
\> \ \ \ \  \metasyntax{Body}) \\
\end{tabbing}
\end{tt}

The body of these rules is run on each pair of items (one from the base and
one from the target) that satisfy the condition and installs a {\it match
hypothesis}\index{match hypothesis} which represents the possibility of them
matching. For example, to state that an expression in the base may match an
expression in the target whose functor is identical, we write:
\begin{verbatim}
   (MHC-rule (:filter ?b ?t :test (equal (expression-functor ?b)
                                         (expression-functor ?t)))
         (install-MH ?b ?t))
\end{verbatim}


\begin{figure}
\border
\vspace{4.4in}
\caption[Water Flow / Heat Flow Analogy]{Water Flow / Heat Flow
Analogy After Local Match Construction. Here we show the graph of match
hypotheses depicted schematically in Figure~\ref{fg:3D-MH}, augmented by links
indicating expression-to-arguments relationships. Match hypotheses which are not
descended from others are called {\it roots} (e.g., the matches
between the {\tt GREATER} predicates, {\tt MH-1} and {\tt MH-6}, and the
match for the predicate {\tt FLOW}, {\tt MH-9}). Match hypotheses between
entities are called {\it Emaps} (e.g., the match between beaker and coffee,
{\tt MH-4}).  Emaps play an important role in algorithms based on structural
consistency.}
\label{fg:wfhf-local}\index{algorithm}
\border
\end{figure}


The likelyhood of each match hypothesis\index{match hypothesis} is found by
running {\it match evidence} rules and combining their results. The evidence
rules provide support for a match hypothesis by examining the structural and
syntactic properties of the items matched. For example, the rule
\begin{verbatim}
   (MHErule ((:intern (MH ?b ?t) :test (and (expression? ?b) (expression? ?t)
                                            (eq (expression-functor ?b)
                                                (expression-functor ?t)))))
       (assert! (implies same-functor (MH ?b ?t) (0.5 . 0.0))))
\end{verbatim}
states ``If the two items are expressions and their functors are the same, then
supply 0.5 evidence in favor of the match hypothesis.'' The rules may also
examine match hypotheses associated with the arguments of these items to
provide support based on systematicity. This causes evidence for a match
hypothesis to increase with the amount of higher-order structure supporting
it.

The state of the match between the water flow and heat flow descriptions of
Figure~\ref{fg:wfhf-calculus} after running these first two sets of rules is
shown in Figure~\ref{fg:wfhf-local}.  There are several important things to
notice in this figure.  First, there can be more than one match
hypothesis\index{match hypothesis} involving any particular base or target
item.  Second, our rules required predicates to match identically while they
allowed entities to match on the basis of their roles in the predicate
structure.  Thus while {\tt TEMPERATURE} can match either {\tt PRESSURE} or
{\tt DIAMETER}, {\tt IMPLIES} cannot match anything but {\tt IMPLIES}.
Third, not every possible correspondence is created. Local matches between
entities are only created when justified by some other identity.  This
significantly constrains the number of possible matches in the typical case.


\begin{figure}
\border
\vspace{3.1in}
\caption[Conflicting relationships.]{Water Flow - Heat Flow analogy after
computation of {\it Conflicting} relationships. Simple lines show the
tree-like graph that the grounding criteria imposes upon match hypotheses.
Lines with circular endpoints indicate the {\it Conflicting} relationships
between matches. Some of the original lines from MH construction have been
left in to show the source of a few {\it Conflicting} relations.}
\label{3D-nogoods}
\border
\vspace{3.1in}
\caption{GMap Construction. (a) Merge step 1: Interconnected and consistent.
(b) Merge step 2: Consistent members of the same base structure. (c) Merge
step 3: Any further consistent combinations.}
\label{fg:wfhf-merging}\index{global mapping}\index{algorithm}
\border
\end{figure}


\subsubsection{Step 2: Global Match Construction}\index{global mapping}\index{algorithm}

The second step in the \SME\ algorithm combines local match hypotheses into
collections of global matches (Gmaps).  Intuitively, each global match is
the largest possible set of match hypotheses that depend on the same one to
one object correspondences.

More formally, Gmaps consist of {\em maximal, structurally consistent}
collections of match hypotheses.  A collection of match hypotheses is
structurally consistent if it satisfies two criteria:
\begin{enumerate}
\item {\em One-to-one:} No two match hypotheses assign the same
base item to multiple target items or any target item to multiple
base items.
\item {\em Support:} If a match hypothesis {\tt MH} is in the
collection, then so are the match hypotheses which pair up all of
the arguments of {\tt MH}'s base and target items.
\end{enumerate}
The preservation criteria enforces strict one to one mappings.  The
grounding criteria preserves connected predicate structure.  A collection is
maximal if adding any additional match hypothesis would render the collection
structurally inconsistent.

The formation of global matches\index{global mapping} is composed of two primary stages:

\begin{enumerate}
\item {\em Compute consistency relationships {\tt (calculate-nogoods)}:}
Here we generate for each match hypothesis the sets of entity mappings it
entails, what match hypotheses it locally conflicts with, and which match
hypotheses it is structurally inconsistent with.  This information
simplifies the detection of contradictory sets of match hypotheses, a
critical operation in the rest of the algorithm. The result of this stage of
processing appears in Figure~\ref{3D-nogoods}.

\item {\em Merge match hypotheses {\tt (generate-gmaps)}:} Compute Gmaps by
successively combining match hypotheses as follows:
 \begin{enumerate}
 \item {\em Form initial combinations {\tt (generate-structure-groups)}:}
Combine interconnected and consistent match hypotheses into an initial set
of Gmaps (Figure~\ref{fg:wfhf-merging}a).
 \item {\em Combine dependent Gmaps {\tt (merge-base)}:} Since base and
target dgroups are rarely isomorphic, some Gmaps in the initial set will
overlap in ways that allow them to be merged.  The advantage in merging them
is that the new combination may provide structural support for candidate
inferences (Figure~\ref{fg:wfhf-merging}b).
 \item {\em Combine independent collections {\tt
(full-gmap-merge)}:}\footnote{These two merge steps (b and c) are called by
{\tt merge-gmaps}, which is in turn called by {\tt generate-gmaps}.} The
results of the previous step are next combined to form maximal consistent
collections (Figure~\ref{fg:wfhf-merging}c).
 \end{enumerate}
\end{enumerate}

A parameter option allows the support criterion to be weakened so that it
does not cross the boundaries of a {\it relational group} \cite{Falken-phd}.
A relational group is distinguished as an unordered collection of relational
structures that may be collectively referred to as a unit.  They correspond
to the abstract notion of a ``set'' and are associated to predicates taking
any number of arguments.  For example, a set of relations joined by the
predicate {\tt AND} defines a relational group.  Other examples include the
axioms of a theory, a decomposable compound object, or the relations holding
over an interval of time.  Intuitively, we would like to say that two groups
correspond without requiring that their contents are exhaustively mapped.

If base and target propositions each contain a group as an argument, the
propositions should not be prevented from matching if the groups' members
cannot be exhaustively paired. For example, the set of relations
\begin{tt}
\begin{tabbing}
~~~B:~~~\= Implies[And(P$_{1}$,P$_{2}$,P$_{3}$), P$_{4}$] \` {\rm (1)} \\
~~~T:   \> Implies[And(P$^{\prime}_{1}$,P$^{\prime}_{2}$), P$^{\prime}_{4}$]
\end{tabbing}
\end{tt}

\noindent
should match better than the set of relations
\begin{tt}
\begin{tabbing}
~~~B:~~~\= Implies[And(P$_{1}$,P$_{2}$,P$_{3}$), P$_{4}$] \` {\rm (2)} \\
~~~T$^{\prime}$: \> P$^{\prime}_{1}$, P$^{\prime}_{2}$, P$^{\prime}_{4}$
\end{tabbing}
\end{tt}

\noindent 
The original model of structural consistency would score (1) and (2)
equally, since the {\tt Implies} relations of (1) would not be allowed to
match. This is a particularly important consideration when matching
sequential, state-based descriptions (e.g., the behavior of a system through
time). The set of relations describing a pair of states often do not
exhaustively match or are of different cardinality. Yet, higher-order
relations over states, such as temporal orderings, are vital and must appear
in the mapping.



\subsubsection{Step 3: Compute Candidate Inferences (gather-inferences)}\index{candidate inference}\index{algorithm}

Associated with each Gmap\index{global mapping} is a (possibly empty) set of
candidate inferences.  Candidate inferences are base predicates that would
fill in structure which is not in the Gmap (and hence not already in the
target). If a candidate inference contains a base entity that has no
corresponding target entity (i.e., the base entity is not part of any match
hypothesis for that gmap), \SME\ introduces a new, hypothetical entity into
the target.  Such entities are represented as a skolem function of the
original base entity (i.e., {\tt (:skolem base-entity)}).

In Figure~\ref{fg:wfhf-done}, Gmap \#1 has the top level {\tt CAUSE}
predicate as its sole candidate inference.  In other words, this Gmap
suggests that the cause of the flow in the heat dgroup is the difference in
temperatures.  If the {\tt FLOW} predicate was not present in the target,
then the candidate inferences for a Gmap corresponding to the pressure
inequality would be both {\tt CAUSE} and {\tt FLOW}. Note that {\tt
GREATER-THAN[DIAMETER(coffee), DIAMETER(ice cube)]} is not a valid candidate
inference for the first Gmap because it does not intersect the existing Gmap
structure.\index{candidate inference}

\begin{figure}
\border
\begin{small}
\begin{verbatim}
  Rule File: literal-similarity.rules     Number of Match Hypotheses: 14

  Gmap #1: { (>PRESSURE  >TEMPERATURE)  (PRESSURE-BEAKER  TEMP-COFFEE)
             (PRESSURE-VIAL  TEMP-ICE-CUBE)  (WFLOW  HFLOW) }
      Emaps:  { (beaker coffee)  (vial ice-cube)  (water heat)  (pipe bar) }
      Weight:  5.99
      Candidate Inferences: (CAUSE >TEMPERATURE HFLOW)

  Gmap #2: { (>DIAMETER  >TEMPERATURE)  (DIAMETER-1  TEMP-COFFEE)
             (DIAMETER-2  TEMP-ICE-CUBE) }
      Emaps:  { (beaker coffee)  (vial ice-cube) }
      Weight:  3.94
      Candidate Inferences:  { }

  Gmap #3: { (LIQUID-3  LIQUID-5)  (FLAT-TOP-4  FLAT-TOP-6) }
      Emaps:  { (water coffee) }
      Weight:  2.44
      Candidate Inferences:  { }
\end{verbatim}
\end{small}

  \caption{Complete SME interpretation of Water Flow - Heat Flow Analogy.}
  \label{fg:wfhf-done}\index{algorithm}\index{global mapping}
\border
\end{figure}

\subsubsection{Step 4: Compute Structural Evaluation Scores (run-rules)}\index{global mapping, scoring}\index{algorithm}

Typically a particular pair of base and target will give rise to several
Gmaps, each representing a different interpretation of the match.  Often it
is desired to select only a single Gmap,\index{global mapping} for example
to represent the best interpretation of an analogy. Many of these evaluation
criteria (including validity, usefulness, and so forth) lie outside the
province of Structure-Mapping, and rely heavily on the domain and
application.  However, one important component of evaluation is {\em
structural} --- for example, one Gmap may be considered a better analogy
than another if it embodies a more systematic match.  \SME\ provides a
programmable mechanism for computing a {\em structural evaluation score}
(SES) for each Gmap.  This score can be used to rank-order the Gmaps in
selecting the ``best'' analogy, or as a factor in a more complex (but
external) evaluation procedure.  In \SMET, the structural evaluation score is
currently computed by simply adding the belief of each local match
hypothesis to the belief of the Gmaps it is a member of.

Returning to Figure~\ref{fg:wfhf-done}, note that the ``strongest''
interpretation (i.e., the one which has the highest structural evaluation
score) is the one we would intuitively expect.  In other words, {\tt beaker}
maps to {\tt coffee}, {\tt vial} maps to {\tt ice-cube}, {\tt water} maps to
{\tt heat}, {\tt pipe} maps to {\tt bar}, and {\tt PRESSURE} maps to {\tt
TEMPERATURE}.  Furthermore, we have the candidate inference that the
temperature difference is what causes the flow.\index{global mapping}

\subsection{Adding Theoretical Constraints}

Given the general program, we may then add theoretical constraints in the
form of rules. For instance, the example just presented used the {\it
literal similarity} rules of structure-mapping theory. These rules augment
\SME's one-to-one mapping and structural consistency criteria with two
additional restrictions. First, evidence is computed according to {\it
systematicity}, that is, highly interconnected systems of relations are
preferred over independent facts. Second, only identical relations are
allowed to match (i.e., {\tt CAUSE} is not allowed to match {\tt
GREATER-THAN}).  Had another set of rules been used, the results might have
been substantially different.  For example, the {\it mere appearance} rules
of structure-mapping theory would have determined that the {\tt water} to
{\tt coffee} mapping was the best, due to their superficial similarity.


\section{Declarations}

The descriptions given to \SME\ are constructed from a user-defined
vocabulary of entities and predicates. This section discusses the
conventions for defining languages for \SME's use.

\subsection{Declaring Predicates}

\begin{tt}\index{defPredicate macro}\index{predicate}\index{predicate, declaring}
\begin{tabbing}
defPredicate \= {\it name argument-declarations predicate-class} \` [{\it Macro}] \\
             \> \&key \= :expression-type {\it logical-type}    \\
             \>      \> :commutative?  \{t | nil\}              \\
             \>      \> :n-ary?  \{t | nil\}                    \\
             \>      \> :documentation {\it descriptive-string} \\
             \>      \> :eval {\it procedural-attachment}\ \ )
\end{tabbing}
\end{tt}
\decfunc{define-predicate}{name argument-declarations predicate-class ...}

\spc{1}
\noindent
{\it predicate-class} is either {\tt function}, {\tt attribute}, or
{\tt relation}, according to what kind of predicate {\it name} is.
The {\it argument-declarations} allows the arguments to be named and
typed.  For example, the declaration:

\begin{verbatim}
   (defPredicate CAUSE ((antecedent event) (consequent event)) relation)
\end{verbatim}
states that {\tt CAUSE} is a two-place relational predicate.  Its arguments
are called {\tt antecedent} and {\tt consequent}, both of type {\tt event}.
The names and types of arguments are for the convenience of the
representation builder and any external routines (including the match rules),
and are not currently used by \SME\ internally.  Likewise, the predicate
class may be very important to the theoretical constraints imposed in the
rules, but is ignored by \SME\ internally.

The optional declaration {\tt :expression-type} indicates the logical type
of an expression headed by the given predicate. For example, the predicate
{\tt throw} may represent a kind of {\tt action}, while the predicate {\tt
mass} may represent an {\tt extensive-quantity}.

The optional declarations {\tt :commutative?} and {\tt :n-ary?} provide
\SME\ with important syntactic information.  {\tt :commutative?} indicates
that the predicate is commutative, and thus the order of arguments is
unimportant when matching.  {\tt :n-ary?} indicates that the predicate can
take any number of arguments.  Examples of commutative nary predicates include
{\tt AND}, {\tt SUM}, and {\tt SET}.

The {\tt :documentation} option allows one to attach a descriptive string to
a predicate. This documentation may then be accessed through the lisp
machine supplied interface\index{lisp machine interface, predicate documentation}
({\tt C-Shift-D}) or some externally written routine. If no documentation is
supplied, the list of argument names is used.  Another option provided
strictly for potential user routines is the optional {\tt :eval} parameter.
This allows one to declare a procedural attachment for a predicate.


\subsection{Declaring Entities}

\indent
\decmacro{defEntity}{name {\tt \&key type constant?}}\index{entity, declaring}
\decfunc{define-Entity}{name {\tt \&key type constant?}}

\spc{0.5}
Entities are logical individuals, i.e., the objects and constants of a
domain. Typical entities include physical objects, their temperature, and
the substance they are made of.  Primitive entities are declared with the
{\tt defEntity} form (a non-primitive entity would be {\tt (temperature
sun)}, which is a functional form representing a particular numeric
temperature entity). Primitive entities declared in this way represent
global entity types, that is, they represent a class of entities rather than
an actual instance of an entity. When an entity type is actually used in a
domain description, a unique entity instance is created for that type (e.g.,
{\tt Mary} is translated to {\tt Mary43}).

Since the language is typed, each entity type can be declared as
a subtype of an existing type using the {\tt :type} option.
For example, we might have
\begin{verbatim}
   (defEntity star :type inanimate)
   (defEntity Sun :type star)
\end{verbatim}
to say that stars are inanimate objects, and our Sun is a particular star.
Constants are declared by using the {\tt :constant?} option, as in
\begin{verbatim}
   (defEntity zero :type number :constant? t)
\end{verbatim}


\subsection{Declaring Description Groups}\label{ss:new-dgroups}
\index{description group}

\begin{tt}
\begin{tabbing}
defDescription {\it description-name}  \` [{\it Macro}]   \\
\ \ \ \= expressions \= \kill
      \> entities  \> ({\it entity$_{1}$}, {\it entity$_{2}$},$\ldots$,{\it entity$_{i}$}) \\ 
      \> expressions \> ({\it expression-declarations})
\end{tabbing}
\end{tt}
\decfunc{define-description}{description-name entities expressions}

\spc{0.5}
For simplicity, predicate instances and compound terms are called {\it
expressions}. A {\em Description Group}, or {\em Dgroup}\index{Dgroup}, is a
collection of primitive entities and expressions concerning them.  Dgroups
are defined with the {\tt defDescription}\index{defDescription macro} form,
where {\it expression-declarations} take the form {\tt
\begin{tabbing}
\ \   \= {\it expression} {\rm or} \\
      \> ({\it expression} :name {\it expression-name})
\end{tabbing}
}
\noindent
For example, the description of water flow depicted in
Figure~\ref{fg:wfhf-calculus} was given to \SME\ as
\begin{verbatim}
  (defDescription simple-water-flow
        entities (water beaker vial pipe)
        expressions (((flow beaker vial water pipe) :name wflow)
                     ((pressure beaker) :name pressure-beaker)
                     ((pressure vial) :name pressure-vial)
                     ((greater pressure-beaker pressure-vial) :name >pressure)
                     ((greater (diameter beaker) (diameter vial)) :name >diameter)
                     ((cause >pressure wflow) :name cause-flow)
                     (flat-top water)
                     (liquid water)))
\end{verbatim}
All entities must have been previously defined and every entity referred to
in the Dgroup's expressions must appear in the {\tt entities} list of the {\tt
defDescription}.\index{description group}

\subsection{Adding new expressions}\label{ss:new-expressions}

\indent
\decfunc{expression}{form dgroup-name {\tt \&key expression-name update-structure?}}\index{expression, adding new expressions}

\spc{0.7}
Expressions are normally defined as a side effect of creating a description group
(Dgroup). However, the facility is provided for dynamically adding new expressions
to a Dgroup. The syntax is essentially the same as for expressions declared within
a {\tt defDescription}. The expression's {\it form} may refer to the names of
existing Dgroup expressions and the form may be given a name. When {\tt expression} is
used to add expressions to an existing Dgroup, the {\tt update-structure?} keyword
must be invoked with a non-nil (e.g., {\tt T}) value. This keyword indicates
that the Dgroup's structure must be reexamined, since the known structural
roots will change as a result of this new expression.

\subsection{Typed Logic}

A mechanism exists for attaching types to predicates and their arguments
(see {\tt defPredicate}).  This facility is designed to constrain the
operation of \SME, particularly candidate inference generation. However, it
has not been extensively used to date.  The ability to attach types may be
useful for consistency checking by external systems.


\section{Using the rule system}\label{ss:rule-system}\index{rule system}

The rule system is the heart of \SME's flexibility. It allows one to specify
what types of things might match and how strongly these matches should be
believed. This section describes the required syntax for a rule set and
different strategies for rule specification.

\subsection{Rule file syntax}\index{rule system, rule file syntax}\index{rule system}

A rule set, or rule file, consists of a {\it declaration}, a set of
{\it match constructor} rules, and a set of {\it match evidence} rules.
In order to describe each, we will examine the syntax and functionality of
each part of the {\it smt-analogy} rule file.

\subsubsection{File declarations}\index{rule system}

\indent
\decfunc{sme-rules-file}{identification-string}

\spc{0.7}
Each rule file must begin with the initialization command {\tt
sme-rules-file}. This function clears the rule system in preparation for a
new set of rules (rules are cached for efficiency) and stores the name of
the rule set for output identification purposes. For example, our sample
rules file begins with:\footnote{Notice the file extension *.rules. While
rule files are not required to end in ``.rules'', all user interface
facilities for simplifying the loading of rule files depend upon this
extension. Another useful point is that rule files are typically defined to
be in the \SME\ package to avoid having to use {\tt sme:} throughout the
rule set.}  

\begin{verbatim}
   (sme-rules-file "smt-analogy.rules")
\end{verbatim}

\noindent
The rule file must then end with the {\tt tre-save-rules} command:

\spc{0.7}
\decfunc{tre-save-rules}{}

\subsubsection{Match Constructor rules}\index{rule system, match constructor rules}\index{rule system}

\index{match hypothesis, installing}
\index{match constructor rules}
\index{MHC-rule}
\begin{tt}
\begin{tabbing}
MHC-rule \= ({\it trigger} \= {\it ?base-variable} {\it ?target-variable} \` [{\it Macro}] \\
         \>                \> [:test {\it test-form}]) \\
\ \ \  {\it body}
\end{tabbing}
\end{tt}
\decfunc{install-MH}{base-item target-item}\index{match hypothesis, installing}

\spc{0.7}
\SME\ begins by finding for each entity and predicate in the base the set of
entities or predicates in the target that could plausibly match that item.
Plausibility is determined by {\it match constructor} rules, which are
responsible for installing all match hypotheses processed by \SME.  There
are two types of constructor rules, each indicated by a different value for
{\it trigger}.  The first type of rule is indicated by a {\tt :filter}
trigger.  These rules are applied to each pair of base and target expressions,
executing the code in {\it body}.  If the {\tt :test} option is used,
{\it test-form} must return true for the body to be run.  For example, the
following rule states that an expression in the base may match an expression
in the target whose functor is identical, unless they are attributes
(a structure-mapping {\it analogy} criterion):

\begin{small}
\begin{verbatim}
   (MHC-rule (:filter ?b ?t :test (and (eq (expression-functor ?b)
                                           (expression-functor ?t))
                                       (not (attribute? (expression-functor ?b)))))
          (install-MH ?b ?t))
\end{verbatim}
\end{small}

The second type of MHC rule is indicated by a trigger of {\tt :intern}.
These rules are run on each match hypothesis as it is created. Typically
they create match hypotheses between any functions or entities that are the
arguments of the expressions joined by the match hypothesis that triggered
the rule. The following is one of two that appear in smt-analogy.rules:

\begin{small}
\begin{verbatim}
   (MHC-rule (:intern ?b ?t :test (and (expression? ?b) (expression? ?t)
                                       (not (commutative? (expression-functor ?b)))
                                       (not (commutative? (expression-functor ?t)))))
       (do ((bchildren (expression-arguments ?b) (cdr bchildren))
            (tchildren (expression-arguments ?t) (cdr tchildren)))
           ((or (null bchildren) (null tchildren)))
         (cond ((and (entity? (first bchildren)) (entity? (first tchildren)))
                (install-MH (first bchildren) (first tchildren)))
               ((and (function? (expression-functor (first bchildren)))
                     (function? (expression-functor (first tchildren))))
                (install-MH (first bchildren) (first tchildren)))
               ((and (attribute? (expression-functor (first bchildren)))
                     (eq (expression-functor (first bchildren))
                         (expression-functor (first tchildren))))
                (install-MH (first bchildren) (first tchildren))))))
\end{verbatim}
\end{small}

Notice that the third test allows identical attributes to match, whereas the
previous MHC rule did not allow such matches. This design does not allow
isolated attributes to match, but recognizes that attributes appearing in a
larger overall structure should be matched.


\subsubsection{Match Evidence rules}\label{ss:ME-rules}\index{match evidence rules}\index{rule system, match evidence rules}\index{rule system}

\indent
\decmacro{rule}{nested-triggers body}\index{match hypothesis, scoring}
\decmacro{rassert!}{expression\ {\tt \&optional (}belief+ {\tt 1.0) (}belief- {\tt 0.0)}}
\decfunc{assert!}{expression\ {\tt \&optional (}belief+ {\tt 1.0) (}belief- {\tt 0.0)}}
\decmacro{initial-assertion}{assertion-form}

\spc{0.8}
The structural evaluation score is computed in two phases. First, each match
hypothesis is assigned some local degree of evidence, independently of what
Gmaps it belongs to.  Second, the score for each Gmap is computed based on
the evidence for its match hypotheses.  The management of evidence rules is
performed by the {\it Belief Maintenance System} (BMS) \cite{BMS}. A BMS is
a form of Truth-Maintenance system, extended to handle numerical weights for
evidence and degree of belief (see \cite{BMS} for a description of what the
weights mean). Pattern-directed rules are provided that trigger on certain
events in the knowledge base. 

The following is a simple rule for giving evidence to match hypotheses
between expressions that have the same predicate:

\begin{small}
\begin{verbatim}
   (initial-assertion (assert! 'same-functor))

   (rule ((:intern (MH ?b ?t) :test (and (expression? ?b) (expression? ?t)
                                         (eq (expression-functor ?b)
                                             (expression-functor ?t)))))
      (if (function? (expression-functor ?b))
          (rassert! (implies same-functor (MH ?b ?t) (0.2 . 0.0)))
          (rassert! (implies same-functor (MH ?b ?t) (0.5 . 0.0)))))
\end{verbatim}
\end{small}

There are two things to notice here in addition to the evidence rule. First,
the proposition {\tt same-functor} was asserted to be true (a belief of
1.0) and then used as the antecedent for the implication of evidence. In
this way, the source of this particular piece of evidence is identified and
is available for inspection. Second, the assertion of {\tt same-functor} was
placed inside the {\tt initial-assertion} form. Since \SME\ caches the
current rule file, it must be told if there are any functions embedded in
the rule file that must be invoked each time \SME\ is initialized.

\spc{0.8}
\decfunc{children-of?}{base-child target-child base-expression target-expression}

\spc{0.3}
Nested triggers within an evidence rule may be used to locate
interdependencies between different match hypotheses. For example,
structure-mapping's systematicity principle is implemented in a local
fashion by propagating evidence from a match hypothesis to its
children:\footnote{A number of functions (e.g., {\tt children-of?}) are
provided to simplify the writing of rules. These appear in the file {\tt
match-rules-support.lisp}.}\index{match hypothesis, scoring}

\begin{small}
\begin{verbatim}
   (rule ((:intern (MH ?b1 ?t1) :test (and (expression? ?b1) (expression? ?t1)
                                           (not (commutative? (expression-functor ?b1)))))
          (:intern (MH ?b2 ?t2) :test (children-of? ?b2 ?t2 ?b1 ?t1)))
       (rassert! (implies (MH ?b1 ?t1) (MH ?b2 ?t2) (0.8 . 0.0))))
\end{verbatim}
\end{small}

\noindent
Evidence for a Gmap is given by:

\begin{small}
\begin{verbatim}
   (rule ((:intern (GMAP ?gm) :var ?the-group))
      (dolist (mh (gm-elements ?gm))
         (assert! `(implies ,(mh-form mh) ,?the-group))))
\end{verbatim}
\end{small}

\index{global mapping, scoring}\index{rule system}

The {\tt BMS} allows a set of nodes to be declared special and will treat
evidence to these nodes differently. An {\tt additive-nodes} function is
provided which takes a set of {\tt BMS} nodes and modifies them so that
their evidence is added rather than normalized using Dempster's rule. \SME\
automatically invokes {\tt additive-nodes} on the derived set of Gmaps once
they are created. Thus, when the above Gmap rule is executed and the {\tt
implies} statement is used to supply evidence from each match hypothesis to
the Gmap, that evidence is simply automatically added to the total Gmap
evidence rather than propagated using Dempster's probabilistic sum.

The following destructive rule is often used instead of the previous one to
give a significant speed up:

\begin{small}
\begin{verbatim}
   (rule ((:intern (GMAP ?gm) :var ?the-group))
       (setf (node-belief+ (gm-bms-node ?gm)) 0)
       (dolist (mh (gm-elements ?gm))
          (incf (node-belief+ (gm-bms-node ?gm))
                (node-belief+ (mh-bms-node mh)))))
\end{verbatim}
\end{small}

This rule bypasses the BMS entirely, thus increasing speed by not creating
justification links. It also renders the {\tt additive-nodes} distinction
irrelevant. However, such rules must be used with extreme caution.  For
example, the source of a Gmap's evidence cannot be inspected when using this
type of operation (see Section~\ref{ss:inspecting-evidence}).

\subsection{Making SME simulate structure-mapping theory}\index{rule system}\index{rule system, simulating structure-mapping theory}

The previous section examined the general structure of an \SME\ rule file.
In the process, the basic elements of the {\it structure-mapping-theory
analogy} rule set were presented. The {\it literal similarity} and
{\it mere-appearance} rules are essentially the same as the {\it analogy}
rules. They differ in the first match constructor rule. The {\it analogy}
rule set has the test {\tt (not (attribute? (expression-functor ?b)))} which is
absent from the corresponding {\it literal similarity} rule. Conversely, the
corresponding {\it mere appearance} rule forces the opposite condition
{\tt (attribute? (expression-functor ?b))}. One should consult Appendix A of
\cite{sme-aij} for listings of all three structure-mapping rule sets.

\subsection{Making SME perform as SPROUTER}\index{rule system}\index{rule system, simulating SPROUTER}

The {\tt SPROUTER} program \cite{hayes-roth} was
developed as an approach to the problem of inductively forming
characteristic concept descriptions. That is, given a sequence of events
(e.g., a list of pictures), produce a single, conjunctive description which
represents a generalized, characteristic description of the sequence. {\tt
SPROUTER} generalized a sequence of {\it N} descriptions by finding the
commonalities between the first two descriptions, generalizing these common
elements (i.e., variablize the literals), and then repeating the process
using this generalized description and the next, unprocessed description.
These steps would be repeated until the generalization had propagated
through the entire list of input descriptions.

\SME\ may be used to implement {\tt SPROUTER}'s {\it interference matching}
technique by giving it a set of match constructor rules which require all
matching predicates to have the same name (i.e., the {\it literal
similarity} rules without the condition that allows functions with different
names to match). The {\tt SPROUTER} generalization\index{generalize}
mechanism may then be implemented with the following algorithm:

\begin{small}
\begin{verbatim}
  Procedure SPROUTER (event-list)
     begin
       generalization := pop(event-list)
       while event-list
          pairwise-match := match(pop(event-list), generalization)
          generalization := generalize(pairwise-match)
       return generalization
     end
\end{verbatim}
\end{small}


\subsection{Relaxing the identical predicates constraint}\index{rule system}\index{rule system, relaxing identical predicates}

The current structure-mapping theory rules are sensitive to representation
by requiring that relational predicates match only if they are identical.
This is an important restriction that ensures the structures being compared
are semantically similar. However, it can also be overly restrictive.  We
are currently exploring different methods to relax the identicality
requirement while still maintaining a strong sense of semantic similarity.
One approach, called the {\it minimal ascension principle}, allows relations
to match if they share a common ancestor in a multi-root is-a hierarchy of
expression types \cite{Falken-phd} (i.e., the identicality test in the match
constructor rule is replaced by a call to {\tt
predicate-type-intersection?}). The local evidence score for their match is
inversely proportional (exponentially) to the relations' distance in the
hierarchy.  This enables \SME\ to match non-identical relations if such a
match is supported by the surrounding structure, while still maintaining a
strong preference for matching semantically close relations. This is similar
to approaches used in \cite{Burstein83,Greiner,Winston80}.

Problems with an unconstrained minimal ascension match technique are
discussed in \cite{Falken-phd}. A mapping approach which considers the
current context when determining pairwise similarity is also discussed.

\subsection{Pure isomorphisms}\label{ss:isomorphisms}
\index{rule system}\index{rule system, pure isomorphisms}

While it is important to assure that the structures being compared are
semantically similar, one can in principle remove all semantic comparisons.
This would allow match creation to be guided strictly by \SME's structural
consistency and 1-1 mapping criteria and match selection to be based
strictly on systematicity.

\begin{figure}
\border
\begin{small}
\begin{tt}
\begin{tabbing}
\ \ \ \ \ \ {\rm Addition}\ \ \ \ \ \ \ \ \ \ \ \ \ \ \ \ \ \ \ \ \ \ \ \ \ \ \ \ \ \ \ \ {\rm Union} \\
\ \ \ \= N1 + N2 $=$ N2 + N1\ \ \ \ \ \ \ \ \ \ \ \ \ \ \ \ \ \ \ \ \ \= S1 $\cup$ S2 $\equiv$ S2 $\cup$ S1 \\
\> N3+(N4+N5) $=$ (N3+N4)+N5 \> S3 $\cup$ [S4 $\cup$ S5] $\equiv$ [S3 $\cup$ S4] $\cup$ S5 \\
\> N6+0 $=$ N6            \> S6 $\cup$ $\emptyset$ $\equiv$ S6
\end{tabbing}
\end{tt}
\end{small}
  \caption{Formal descriptions for addition and union.}
  \label{fg:addition-union}
\border
\end{figure}


Consider the isomorphic mapping between the formal definitions of numeric
addition and set union shown in
Figure~\ref{fg:addition-union}.\footnote{This example is taken from an
advance copy of a paper by Holyoak and Thagard \cite{ACME}. I include it
here simply to demonstrate the range of matching preferences available in
\SME.} These formal descriptions may be given to \SME\ in the standard
manner, as in {\tt (plus N3 (plus N4 N5))} for the left side of the
associativity rule (the representation {\tt (plus N4 N5 result45)} also
works, although it results in a slightly longer run time due to the
flattening of structure).  When presented as formal definitions, the
concepts of addition and union are structurally isomorphic, independent of
the meaning of the predicates. Thus, while it could be argued that the
predicates {\tt plus} and {\tt union} share a certain degree of semantic
overlap, this example demonstrates that it is possible to make \SME\ ignore
predicates entirely and simply look for isomorphic mappings.  The rule set
for isomorphic mappings is shown in Figure~\ref{fg:isomorphism}. (This is
called the {\it ACME} rule set, as it configures \SME\ to emulate the {\tt
ACME} program on this example \cite{ACME}). The only constraint this rule
set enforces is that each predicate has the same number of arguments. While
it includes the Structure-Mapping notion of systematicity to prefer systems
of relations, it does not enforce identicality of predicates.  Using this
rule set, \SME\ produces the unique best mapping that we would expect
between the formal definitions of addition and union.

Since \SME\ enforces the ``same number of arguments'' restriction by
defeating any match hypotheses that are not structurally sound, we could in
principle effectively remove the rule file entirely. This could be done with
one match constructor rule to match everything with everything and one
evidence rule to measure systematicity. When this {\it free-for-all} rules
file was given to \SME, the same single best Gmap was produced, but at the
expense of increasing the run time from 13 seconds to 3.25 minutes.


\begin{figure}[t]
\border
\begin{small}
\begin{alltt}
   (MHC-rule (:filter ?b ?t :test (= (numargs (expression-functor ?b))
                                     (numargs (expression-functor ?t))))
       (install-MH ?b ?t))

   {\it ;;; Intern rule to match entities (non-commutative predicates)}
   (MHC-rule (:intern ?b ?t :test (and (expression? ?b) (expression? ?t)))
       (do ((bchildren (expression-arguments ?b) (cdr bchildren))
            (tchildren (expression-arguments ?t) (cdr tchildren)))
           ((or (null bchildren) (null tchildren)))
         (if (and (entity? (first bchildren)) (entity? (first tchildren)))
             (install-MH (first bchildren) (first tchildren)))))

   {\it ;;; Give a uniform initial priming to each MH}
   (initial-assertion (assert! 'initial-priming))

   (rule ((:intern (MH ?b ?t)))
      (rassert! (implies initial-priming (MH ?b ?t) (0.2 . 0.0))))

   {\it ;;;propagate interconnections - systematicity}
   (rule ((:intern (MH ?b1 ?t1) :test (and (expression? ?b1) (expression? ?t1)))
          (:intern (MH ?b2 ?t2) :test (children-of? ?b2 ?t2 ?b1 ?t1)))
      (rassert! (implies (MH ?b1 ?t1) (MH ?b2 ?t2) (0.8 . 0.0))))

   {\it ;;; Support from its MH's}
   (rule ((:intern (GMAP ?gm) :var ?the-group))
      (dolist (mh (gm-elements ?gm))
         (assert! `(implies ,(mh-form mh) ,?the-group))))
\end{alltt}
\end{small}
  \caption{Rule set for forming general isomorphic mappings.}
  \label{fg:isomorphism}\index{rule system}
\border
\end{figure}



\subsection{Imposing externally established pairings}\index{rule system}\index{rule system, analogical hints}

In certain situations, a number of entity and predicate mappings may already
be known prior to invoking \SME. These mappings may have been provided as an
analogical hint\index{analogical hint} from an instructor or derived by the
application program during earlier processing.  For example, \Phineas\
\cite{VBAL,MLW} uses \SME\ to analogically relate observed physical
phenomena to known theories of the world. \Phineas\ uses two analogical
mappings to learn about a new physical process. First, behavioral
correspondences are established (i.e., what entities and quantities are
behaving in the same manner). Second, the relevant base theories are
analogically mapped into the new domain, guided by the behavioral
correspondences. The two-stage mapping process solves the problem of using
analogy in cases where one does not have a pre-existing theory, as occurs
with truely novel learning. The assumption made in \Phineas\ is that similar
behaviors will have similar theoretical explanations. The first mapping
provides the correspondences between entities and functions required to
guide the importation of an old theory to explain a new domain in the second
mapping.

\SME\ includes facilities to simplify writing \Phineas-like programs, by
enabling the results of earlier processing to constrain subsequent mapping
tasks. These routines are divided into two categories, {\it declaration} and
{\it test}. The declaration routines tell \SME\ what predicate and entity
correspondences are known {\it a-priori}. The test routines enable the match
constructor rules to adhere to these imposed constraints. Known mappings are
declared through the following functions:

\spc{0.6}
\noindent
\begin{minipage}{\textwidth}
\begin{tt}
\begin{tabbing}
defGiven-Mappings  \` [{\it Macro}]   \\
\ \ \ \= entities \ \ \= (\= ({\it base-entity$_{1}$ target-entity$_{1}$})   \\
      \>              \>  \> ({\it base-entity$_{i}$ target-entity$_{j}$}) ...)  \\
      \> predicates   \> (({\it base-predicate$_{1}$ target-predicate$_{1}$})   \\
      \>              \>  \> ({\it base-predicate$_{i}$ target-predicate$_{j}$}) ...)
\end{tabbing}
\end{tt}
\end{minipage}
\decfunc{declare-given-mappings}{entities predicates}
\decfunc{clear-given-mappings}{}

\spc{0.8}
Both {\tt defGiven-Mappings} and {\tt declare-given-mappings} have identical
functionality. The first does not evaluate its arguments while the second
one does. Disjunctive constraints may be imposed by including all of the
possible pairings (e.g., defining both {\tt (}{\it base-entity$_{i}$
target-entity$_{j}$}{\tt )} and {\tt (}{\it base-entity$_{i}$
target-entity$_{k}$}{\tt )}).

Once a set of given mappings has been declared, the following test routines
may be used within the match constructor rules to enforce these mappings:

\spc{0.8}
\decfunc{sanctioned-pairing?}{base-item target-item}
\decfunc{paired-item?}{{\tt \&key base-item target-item}}

\spc{0.4}
\noindent
{\tt sanctioned-pairing?} tests if the given pair is one of the {\it a-prior}
pairings. {\tt paired-item?} takes either a base item or a target item and returns
true if the mapping for that item has been externally determined.

These functions help in writing rules which respect established mappings.
For example, the following two rules are used in the \Phineas\ system to
allow observed behavioral correspondences to constrain the mapping of the
relevant theory:

\begin{small}
\begin{verbatim}
(MHC-rule (:filter ?b ?t :test (and (eq (expression-functor ?b) (expression-functor ?t))
                                    (not (paired-item? :base-item (expression-functor ?b)))
                                    (not (paired-item? :target-item (expression-functor ?t)))))
    (install-MH ?b ?t))

(MHC-rule (:filter ?b ?t :test (sanctioned-pairing? (expression-functor ?b)
                                                    (expression-functor ?t)))
    (install-MH ?b ?t))
\end{verbatim}
\end{small}

When an analogy is being made between two behaviors, {\tt clear-given-mappings}
is used to make \SME\ perform in normal analogy mode. The discovered entity and
function correspondences are then given to {\tt declare-given-mappings} prior to
using \SME\ to map the relevant theory.\index{rule system}


\section{Representation Issues}\index{representation issues}

The proper representation becomes an issue in \SME\ due to its significant
impact on speed performance. Hierarchical representations provide an
important source of constraint on generating potential matches. They tend to
make the semantic interrelations explicit in the structure of the syntax.
For example, Section~\ref{ss:isomorphisms} described a comparison between
the laws of addition and union. There it was noted that part of the additive
associativity rule may be represented as {\tt (plus N3 (plus N4 N5))} or as
the pair {\tt (plus N4 N5 result45)} and {\tt (plus N3 result45
result3-45)}. The latter ``flat'' representation takes more time for \SME\
to process, sometimes a significant difference for large domain
descriptions. This is because the functional representation makes the
associativity rule structurally explicit, while the flat representation
buries it among the tokens appearing as arguments to {\tt plus}. However, it
is important to note that \SME\ is able to process domain descriptions in
any predicate-based format. It is simply speed considerations that render
standard, flat forms of representation undesirable.

Due to SME's ability to accept commutative, n-ary predicates, it is able to
match arbitrary sets (which must be of equal size at this time). This has
two consequences. First, the explicit use of sets becomes a viable form of
representation. Thus, a theory might be represented concisely as

\begin{verbatim}
   (Theory T1 (SET axiom-8 axiom-14 ...))
\end{verbatim}

\noindent
rather than as {\tt (Axiom-of T1 axiom-8)}, etc. Second, sets may be used to
add structure to descriptions. For example, the set representation for
theories results in greatly reduced run times compared to the non-set
representation.\footnote{The difference in speed is due to the operation of
merge step 2, which combines matches sharing a common base structure. The
set notation for theory {\tt T1} enables merge step 2 to know that matches
for {\tt axiom-8} and {\tt axiom-14} should be placed in the same gmap, thus
reducing the number of possibilities in merge step 3.} I am currently
investigating the use of a similar representation for temporal states, as in:

\begin{verbatim}
   (Situation S1 (SET (Increasing (Amount-of water1))
                      (Increasing (Pressure water1))
                      o o o ))
\end{verbatim}

\noindent
A \Phineas\ problem which took \SME\ 53 minutes (using {\tt (Increasing
(Amount-of (at water1 S1)))}) was reduced to 34 seconds using this more
structured representation.


\section{Using SME}

This section describes how to install \SME\ on your machine, load it, and
operate it.

\subsection{Installing SME}\index{installation}\label{ss:site-specifics}

\indent
\decvar{*sme-language-file*}\index{parameters, site specific}\index{site specific information}
\decvar{*sme-default-rules*}\index{site specific information, pathnames}
\decvar{*sme-rules-pathname*}
\decvar{*sme-dgroup-pathname*}

\spc{0.8}
To configure \SME\ to a particular site, a handful of variables storing
system directory information must be edited and set to the appropriate
values. These variables appear in {\tt config.lisp}, a separate file for
this purpose. Of primary importance are {\tt *sme-language-file*} and {\tt
*sme-default-rules*}. These are used by {\tt sme-init} to initialize the
language and rule systems. The two variables storing the rules and dgroup
pathnames are used by the user interface routines.

\spc{0.7}
\decvar{*the-lisp-package*}\index{packages,site specific}
\decvar{*the-user-package*}\index{site specific information, packages}

\spc{0.3}
In most Common Lisp implementations, one package exists for general user
definitions and another exists for the lisp implementation. It is important
to notify \SME\ what these are for the Common Lisp in use. For example, on a
Symbolics (version 6.2), the lisp package is called {\tt common-lisp} and
the user package is called {\tt cl-user}. These are the default settings.

\spc{0.7}
\decvar{*sme-system-pathname*}
\decvar{*sme-files*}

\spc{0.8}
These variables are used to automate compiling and loading. If the system is
being loaded on something other than a Symbolics or TI Explorer, the file
{\tt windowing} should not be included in the list {\tt *sme-files*}.
Otherwise, it should be left in the list of \SME\ files, which is the
default.

The \SME\ routines assume a set of naming conventions on domain description
and rule files.\index{conventions, file names} The names of files containing
domain descriptions ({\tt defDescription}) should end with a {\tt *.dgroup}
extension. The name of a file containing a rule set should end with the {\tt
*.rules} extension.


\subsection{Running SME}\index{running SME}\index{execution}

\indent
\decfunc{sme-init}{{\tt \&optional (}initialize-language? {\tt T) (}initialize-rules? {\tt T)}}

\spc{0.7}
This section gives a brief overview of the process of using \SME\ for
matching, generalization, and inspection tasks.


\begin{figure}
\border
\begin{footnotesize}
\begin{alltt}
{\bf\gthan\ (sme:sme-init)}
Initializing SME...
   Loading default language file: prof:>falken>sme>language
   Loading default rules file: prof:>falken>sme>literal-similarity.bin
Complete.
T
{\bf\gthan\ (load "prof:\gthan falken\gthan sme\gthan simple-water-flow.dgroup")}
Loading PROF:>falken>sme>simple-water-flow.dgroup into package USER
#P"PROF:>falken>sme>simple-water-flow.dgroup"
NIL
{\bf\gthan\ (load "prof:\gthan falken\gthan sme\gthan simple-heat-flow.dgroup")}
Loading PROF:>falken>sme>simple-heat-flow.dgroup into package USER
#P"PROF:>falken>sme>simple-heat-flow.dgroup"
NIL
{\bf\gthan\ (sme:match 'swater-flow 'sheat-flow T)}
              SME Version 2E
    Analogical Match from SWATER-FLOW to SHEAT-FLOW.

Rule File: literal-similarity.rules
--------------------------------------------------------------------------
                  | # Entities | # Expr. | Maximum order | Average order |
  Base Statistics |       4    |     11  |        3      |      1.36     |
Target Statistics |       4    |      6  |        2      |      1.17     |
--------------------------------------------------------------------------
# MH's | # Gmaps | Merge Step 3 | CI Generation | Show Best Only |
   14  |     3   |   ACTIVE     |    ACTIVE     |       OFF      |
------------------------------------------------------------------
Total Run Time:    0 Minutes,  0.821 Seconds
BMS Run Time:      0 Minutes,  0.530 Seconds
Best Gmaps: { 3 }

Match Hypotheses:
    (0.6320  0.0000)  (PIPE4 BAR7)
    (0.7900  0.0000)  (FLAT-WATER FLAT-COFFEE)
    {\it   o       o         ;a number of match hypotheses appeared here}
    (0.8646  0.0000)  (WATER1 COFFEE5)
    (0.7900  0.0000)  (LIQUID-WATER LIQUID-COFFEE)

Gmap #1:   (BEAKER2 COFFEE5)  (DIAM-BEAKER TEMP-COFFEE)  (VIAL3 ICE-CUBE6)
           (DIAM-VIAL TEMP-ICE-CUBE)  (>DIAMETER >TEMP)
  Emaps:  (BEAKER2 COFFEE5) (VIAL3 ICE-CUBE6)
  Weight: 3.937660
  Candidate Inferences:  { }

    {\it   o       o         ;Gmap #2 appeared here...}

Gmap #3:   (>PRESSURE >TEMP)  (PRESS-VIAL TEMP-ICE-CUBE)  (PRESS-BEAKER TEMP-COFFEE)
           (BEAKER2 COFFEE5)  (VIAL3 ICE-CUBE6)  (WATER1 HEAT8)
           (PIPE4 BAR7)  (WFLOW HFLOW)
  Emaps:  (BEAKER2 COFFEE5) (VIAL3 ICE-CUBE6) (WATER1 HEAT8) (PIPE4 BAR7)
  Weight: 5.991660
  Candidate Inferences:  (CAUSE >TEMP HFLOW)
\end{alltt}
\end{footnotesize}
  \caption{Initializing and running SME.}
  \label{fg:sme-trace}
\border
\end{figure}


\begin{enumerate}

\item {\it Loading the files.} To load or compile \SME, the file {\tt
config} should be loaded and then {\tt (load-sme)} or {\tt (compile-sme)}
called. A {\tt defSystem} definition is provided in {\tt system.lisp} for
Symbolics machines.

\item {\it System startup.} This stage is only appropriate for full, lisp
machine startup. The \SME\ window environment may be created with Select-S
on a Symbolics or System-S on a TI Explorer.

\item {\it Initialization.} The function {\tt sme-init} should be called to
initialize the database. If {\it initialize-language?} is non-nil, the
default language file (predicate definitions) will be loaded. If {\it
initialize-rules?} is non-nil, the default rules file will be loaded. Prior
to operating \SME, the language and rule systems must be established.

\item {\it Loading Dgroups.} Any description groups that are to be matched
must be declared. These declarations are typically stored in files, with the
extension {\tt *.dgroup}. If the windowing system is active, the command
{\tt Load Dgroup} will offer a menu of all {\tt *.dgroup} files in
{\tt *sme-dgroup-pathname*} to select what to load.

\item {\it Analogical mapping.} The function {\tt match} may be called to
form a mapping between two given Dgroups. This is discussed in
Section~\ref{ss:match}. If the windowing system is active, the command {\tt
Match} will offer a menu to select base and target Dgroups. It is prior to
this step that one might want to think about wether to modify any system
parameters (e.g., print the match hypotheses, print only the best Gmaps,
generate candidate inferences, etc.). 

\item {\it Describing Dgroups.} Once Dgroups are defined, the {\tt
describe-dgroup} facility will provide a description of any particular
Dgroup.

\item {\it Graphically displaying Dgroups.} If the windowing system is
active, Dgroups may also be displayed graphically, through the {\tt Display
Dgroup} utility.

\item {\it Generalizing.} Once a mapping is formed, it may be generalized
using the {\tt generalize} function or the {\tt Generalize} command in the
system menu.

\item {\it Saving the results of a session.} If the windowing system is
active, the results of commands like {\tt Match} and {\tt Generalize} are
sent to the scroll window by default. These results may be written to a file
using the {\tt dump-scroll} system utility.

\item {\it Comparing two apparently identical Gmaps.} When two Gmaps are
formed that appear to be identical, their differences can be identified
using the {\tt compare-gmaps} system utility.

\end{enumerate}

\noindent
A trace of \SME\ performing the basic mapping task is given in
Figure~\ref{fg:sme-trace}. Each of the other options are described in
greater detail in the following sections.


\subsection{Batch mode}
\index{batch mode}\index{report generation}

\indent
\decfunc{run-batch-file}{pathname {\tt \&key} {\tt (}gmap-display {\tt :all) (}gmap-statistics {\tt :none)}}
\decmacro{language-file}{pathname}
\decmacro{dgroup-directory}{pathname}
\decmacro{dgroup-file}{file-name}
\decmacro{rule-directory}{pathname}
\decmacro{rule-file}{file-name}
\decmacro{rule-sets}{{\tt \&rest} rule-file-names}
\decmacro{report-comments}{string}
\decmacro{send-report-to}{pathname {\tt \&key (}text-driver {\tt :LPR) (}style {\tt :STANDARD)}}
\decmacro{run-matcher-on}{base-name target-name}
\decmacro{defPostMatcher}{function}


\begin{figure}
\border
\begin{footnotesize}
\begin{alltt}
(sme:Dgroup-Directory "prof:>falken>sme>")

(sme:Dgroup-File "solar-system")
(sme:Dgroup-File "rutherford")
(sme:Dgroup-File "simple-water-flow")
(sme:Dgroup-File "simple-heat-flow")

(sme:Rule-Directory "prof:>falken>sme>")

(sme:Rule-Sets "literal-similarity" "true-analogy" "attribute-only")   {\it; iterate over each rule set}

(sme:Report-Comments "Sample run of SME to demonstrate batch mode.")

(sme:Send-Report-To "heath:>falken>sample.dmp" :text-driver :LATEX)

(sme:Run-Matcher-On solar-system rutherford-atom)                 {\it; map this pair once for each rule set}
(sme:Run-Matcher-On swater-flow sheat-flow)                       {\it; map this pair once for each rule set}
\end{alltt}
\end{footnotesize}
  \caption{Sample SME batch file.}
  \label{fg:batch-file}
\border
\end{figure}


\spc{0.8}
\SME\ is normally used as an interactive utility or as a module to some
larger program. However, when performing statistical analyses across a broad
space of matching preferences (i.e., rule sets) and domain descriptions, an
interactive format soon becomes inconvenient. Utilities are provided so that
a file of \SME\ instructions may be defined and then executed using {\tt
run-batch-file} (e.g., Figure~\ref{fg:batch-file}). This would instruct
\SME\ to perform a series of matches, potentially over a variety of rule
sets and domain descriptions, and generate a detailed report of the
execution and a summary of the results. When a single rule set is specified
using {\tt rule-file}, all subsequent matches (invoked by {\tt
run-matcher-on}) will use this rule file until another one is specified.
Using {\tt rule-sets}, one may instead specify a series of rule files to be
used, so that a single {\tt run-matcher-on} command will cause \SME\ to run
once for each rule file in the list. If a user-defined function name is
given to {\tt defPostMatcher}, this function will be called after each match
is performed, in case special post-match routines are desired or extra
information is to be added to the report being generated. A variety of text
drivers are supported for report generation ({\tt send-report-to}), such as
{\tt :lpr} (line printer), {\tt :latex}, and {\tt :troff}.


\subsection{Generalization mechanism}

\indent
\decfunc{generalize}{gmap}

\spc{0.8}
The {\tt generalize} function takes a global mapping structure and returns
three alternate generalizations (using the Common-Lisp {\tt values}
protocol), each one successively larger than the previous:


\begin{figure}
\border
\begin{small}
\begin{verbatim}
   Generalizations for Match from SWATER-FLOW to SHEAT-FLOW:

       Generalization #1 (Literally Common Aspects Only):
            (FLOW ENTITY6 ENTITY8 ENTITY13 ENTITY14)

       Generalization #2 (All Common Aspects Only):
            (FLOW ENTITY6 ENTITY8 ENTITY13 ENTITY14)
            (GREATER (FUNCTION0 ENTITY6) (FUNCTION0 ENTITY8))

       Generalization #3 (Maximal Generalization):
            (CAUSE (GREATER (FUNCTION0 ENTITY6) (FUNCTION0 ENTITY8))
                   (FLOW ENTITY6 ENTITY8 ENTITY13 ENTITY14))
\end{verbatim}
\end{small}
  \caption{\SME\ generalizations for the simple water flow -- heat flow analogy.}
  \label{fg:generalizations}
\border
\end{figure}


\begin{enumerate}
 \item {\it Literally common aspects only.} This generalization locates
those sub-structures which are identical in both base and target Dgroups.
This is a type of generalization typically found in inductive generalization
programs.
 \item {\it All common aspects only.} In addition to common, identical
substructures, this generalization includes cases where functions of a
different name were allowed to match. Where this occurs in the common
structure, a skolemized function predicate is created.
 \item {\it Maximal generalization.} The largest generalization (in terms of
amount) includes all candidate inferences sanctioned by the Gmap, as well as
the common substructure of generalization mode 2. This represents the entire
shared structure between the two Dgroups under the assumption that the
candidate inferences are valid.
\end{enumerate}

For example, given the best Gmap from the simple water flow -- heat flow
analogy described in Section 2.1 and shown in Figure~\ref{fg:wfhf-done},
\SME\ will produce the set of generalizations shown in
Figure~\ref{fg:generalizations}. The first generalization indicates that the
only thing in common between the two situations is the existence of flow. The
second generalization loosens the meaning of ``in common'' to include the
fact that a quantity associated with the source of flow was greater than the
same quantity measured for the destination. The final generalization assumes
that this inequality, which was the cause of flow in the water flow domain,
is actually the cause of flow for both situations.


\subsection{Inspecting MH and Gmap evidence}\label{ss:inspecting-evidence}

\indent
\decfunc{match-evidence-inspector}{}\index{match hypothesis, inspecting evidence justifications}\index{match hypothesis, inspecting evidence justifications}

\spc{0.8}
When developing a theory about what types of rules should be used and how
much evidence for a particular match they should provide, it is often useful
to explicitly see what the different sources of evidence were for a
particular match item. The system utility {\tt match-evidence-inspector} may
be used to display a trace of the entire evidence facility or just the
evidence for a particular match hypothesis or Gmap. For example, the
following information was printed out about the {\tt pressure} to {\tt
temperature} match hypothesis in the water flow -- heat flow analogy:

\begin{small}
\begin{verbatim}
   (MH F#PRESS-BEAKER F#TEMP-COFFEE) has evidence (0.7120, 0.0000) due to
      IMPLICATION((MH F#>PRESSURE F#>TEMP))  (0.5200, 0.0000)
      IMPLICATION(CHILDREN-POTENTIAL)  (0.4000, 0.0000)
\end{verbatim}
\end{small}
\noindent
While the following information appears for the best Gmap in this analogy:
\begin{small}
\begin{verbatim}
   (GMAP #GM3) has evidence (5.9917, 0.0000) due to
       IMPLICATION((MH F#WFLOW F#HFLOW))  (0.7900, 0.0000)
       IMPLICATION((MH I#PIPE20 I#BAR23))  (0.6320, 0.0000)
       IMPLICATION((MH I#WATER17 I#HEAT24))  (0.6320, 0.0000)
       IMPLICATION((MH I#VIAL19 I#ICE-CUBE22))  (0.9318, 0.0000)
       IMPLICATION((MH I#BEAKER18 I#COFFEE21))  (0.9318, 0.0000)
       IMPLICATION((MH F#PRESS-BEAKER F#TEMP-COFFEE))  (0.7120, 0.0000)
       IMPLICATION((MH F#PRESS-VIAL F#TEMP-ICE-CUBE))  (0.7120, 0.0000)
       IMPLICATION((MH F#>PRESSURE F#>TEMP))  (0.6500, 0.0000)
\end{verbatim}
\end{small}

\noindent
The inspection facility will not work for Gmaps if their scores were
produced by an external (to the {\tt BMS}), destructive operation. One such
destructive rule appeared at the end of Section~\ref{ss:ME-rules}.


\subsection{Windows}\index{lisp machine interface}\index{scroll windows}

\indent
\decfunc{dump-scroll-menu}{}\index{scroll windows, saving contents}
\decfunc{dump-scroll}{output-pathname}
\decfunc{clear-scroll}{}

\decfunc{select-windowing-configuration}{}
\decfunc{select-scroll}{}
\decfunc{select-double-scroll}{}
\decfunc{select-graphics}{}
\decfunc{select-large-graphics}{}
\decfunc{select-split}{}

\decvar{*sme-frame*}
\decvar{*graphics-pane*}
\decvar{*scroll-pane*}
\decvar{*spare-scroll-pane*}
\decvar{*lisp-pane*}

\spc{0.8} The windowing system is lisp machine dependent and appears in the
file {\tt windowing.lisp}. The loading of this file is optional. When the
windowing system is used, a number of window configurations are possible,
such as having a single scroll window, two side by side, a single graphics
window, or a scroll and graphics window side by side. These configurations
may be selected through their individual functions (e.g., {\tt
select-scroll}), or through the central configuration facility {\tt
select-scroll-graphics}.  By default, when the windowing system is active,
all \SME\ output is sent to the primary scroll pane.\index{scroll windows,
writing to} When both scroll windows are in use, the configuration facility
allows one to specify which scroll window is currently active. The two
scroll dumping routines write the contents of the primary scroll window to a
specified file. Output sent to the secondary ({\it scratchpad}) scroll pane
is for observation only and cannot be written to a file.


\subsection{System parameters}\index{parameters}

\indent
\decvar{*sme-parameters*}
\decvar{*parameter-menu-options*}
\decmacro{defSME-Parameter}{variable-name string-description type {\tt \&optional} type-choices}

\spc{0.8}
The {\tt defSME-Parameter} form adds a new variable to the list of known
\SME\ parameters. This list is used by the windowing interface routines to
query the user about possible parameter settings. It is provided primarily
for application programs wanting to use the standard \SME\ parameter setting
facility. The arguments to {\tt defSME-Parameter} correspond to the appropriate
definitions for the {\tt choose-variable-values} function of your particular
machine. For example, the following declaration appears in {\tt match.lisp}:
\begin{small}
\begin{verbatim}
   (defSME-Parameter *display-all-MH* "Display all the Match Hypotheses" :boolean)
\end{verbatim}
\end{small}


\begin{figure}
\border
\vspace{1.2in}
\caption{SME System Parameters.}
\label{fg:parameters}
\border
\end{figure}


\spc{0.6}
\decfunc{change-parms}{}

\spc{0.4}
The windowing system provides a menu facility for viewing and changing the
current values of system parameters. This menu is shown in
Figure~\ref{fg:parameters}.


\subsection{System utilities}\index{system utilities}

\indent
\decvar{*system-utilities-menu*}
\decmacro{defSME-Utility}{string-name lisp-form}
\decfunc{menu-utilities}{}
\decfunc{get-dgroup}{}
\decfunc{get-rules}{}

\spc{0.8}
The \SME\ system utilities are the options that appear when the {\tt
Utilities} command is evoked, the right mouse button is pressed, or the
function {\tt menu-utilities} is called. These utilities include changing
the system parameters, choosing to load a Dgroup or rule file from those in
the defined directories, and clearing or writting to file the contents of
the scroll window. These routines are lisp machine dependent. The following
declaration appears in {\tt match.lisp}:
\begin{small}
\begin{verbatim}
   (defSME-Utility "Inspect Evidence" (match-evidence-inspector))
\end{verbatim}
\end{small}


\section{User Hooks}

This section describes the global variables and routines that are available
to the user and application programs for the creation and inspection of
analogical mappings.

\subsection{Applications control over display}

\indent
\decvar{*sme-output-stream*}
\decvar{*windowing?*}
\decvar{*sme-graphics-output*}

\spc{0.8}
All \SME\ textual display routines send their output to {\tt
*sme-output-stream*}. By default, the value of this variable is {\tt T},
which causes output to be sent to {\tt *terminal-io*}, CommonLisp's default
pointer to the user's console. When {\tt *sme-output-stream*} is a scroll
window (determined by the presence of an {\tt :append-item} handler), the
appropriate scroll window routines for sending display items are invoked.
Otherwise, text is sent to the current output stream using {\tt format}. 
Text routines are machine independent.

In a similar manner, all \SME\ graphics output is sent to the current
{\tt *sme-graphics-output*} window. Graphics output is lisp machine
dependent and relies on the {\tt ZGRAPH} graphics system.

When the \SME\ windowing system is in operation, {\tt *sme-output-stream*}
is set to the primary \SME\ scroll pane ({\tt *scroll-pane*}), {\tt
*sme-graphics-output*} is set to the \SME\ graphics pane ({\tt
*graphics-pane*}), and {\tt windowing?} is set to {\tt T}.
\index{scroll windows, writing to} 

\spc{0.7}
\decmacro{sme-format}{format-string {\tt \&rest} format-args}
\decfunc{sme-print}{string}
\decfunc{sme-terpri}{{\tt \&optional (}N {\tt 1)}}

\spc{0.3}
These routines provide a general interface for sending textual output to the
current \SME\ output stream. {\tt sme-format} is equivalent to CommonLisp's
{\tt format} routine, except that the printed output is always followed by a
newline. The {\tt sme-print} routine is provided for simple situations where
only a string is printed or for situations requiring the standard use of
{\tt format}, as in building up a line of text through multiple invocations.
The printed output of {\tt sme-print} is followed by a newline. When the
routine is used for multiple calls of {\tt format}, it should be used in
conjunction with CommonLisp's {\tt with-output-to-string}, as in:

\begin{small}
\begin{verbatim}
   (sme-print
      (with-output-to-string (stream)
         (format stream "~%Beginning of a line...")
         (format stream "   middle of a line...")
         (format stream "   end of a line.")))
\end{verbatim}
\end{small}

When a whole set of operations are carried out within the context of a
single {\tt sme-print}, one must be careful not nest calls to {\tt
sme-print} (e.g., calling a function in the context of an {\tt sme-print}
which itself invokes {\tt sme-print}). Such nesting will cause output 
to appear backwards from what was intended and may cause the output stream
to close improperly.


\subsection{Useful miscellaneous functions}

Data exists within \SME\ in three forms: (1) local items such as entities,
predicates, and expressions, (2) description groups (Dgroups), and (3) analogical
mapping information. The routines to create and query these items are
described in the following sections.

\subsubsection{Entities, predicates, and expressions}

\indent
\decfunc{entity?}{item}
\decfunc{entity-type?}{item}
\decmacro{entity-name?}{symbol}\index{entity, inspection functions}
\decmacro{fetch-entity-definition}{symbol}
\decmacro{entity-domain}{symbol}
\decmacro{constant-entity?}{symbol}

\spc{0.8}
Entities declared through {\tt defEntity} represent global entity types,
that is, they represent a class of entities rather than an actual instance
of an entity. When an entity type is used in the definition of a description
group, a unique entity instance is created for that type (e.g., {\tt beaker}
is translated to {\tt beaker73}). Thus, a given entity token will represent
either a type or an instance. The structure predicate {\tt entity?} returns
true if the given item is an entity-instance structure, while {\tt
entity-type?} returns true if the item is an entity-type structure. The
macro {\tt entity-name?} returns true if the given symbol represents either
an entity type or instance. {\tt fetch-entity-definition} will return the
entity-type structure for an entity type token or the entity-instance
structure for an entity instance token. The routines {\tt entity-domain} and
{\tt constant-entity?} refer to the {\tt type} and {\tt constant?} keyword
values given to {\tt defEntity} when the corresponding entity type was
created.

\spc{1}
\decvar{*sme-predicates*}
\decmacro{fetch-predicate-definition}{predicate-symbol}\index{predicate, inspection functions}
\decmacro{predicate?}{symbol}
\decmacro{predicate-type}{predicate-symbol}
\decmacro{relation?}{predicate-symbol}
\decmacro{attribute?}{predicate-symbol}
\decmacro{function?}{predicate-symbol}
\decmacro{commutative?}{predicate-symbol}
\decmacro{n-ary?}{predicate-symbol}
\decmacro{arg-list}{predicate-symbol}
\decmacro{numargs}{predicate-symbol}
\decmacro{expression-type}{predicate-symbol}
\decmacro{eval-form}{predicate-symbol}

\spc{0.4}
These routines provide the facility to access the various predicate properties
defined with the {\tt defPredicate} form. {\tt fetch-predicate-definition}
returns the actual sme-predicate structure containing all the information
about a given predicate.

\spc{1}
\index{expression, inspection functions}
\decfunc{fetch-expression}{expression-name dgroup
                     {\tt \&optional (}error-if-absent? {\tt T)}}
\decfunc{expression-functor}{expression-structure}
\decfunc{fully-expand-expression}{expression-structure dgroup}
\decfunc{fully-expand-expression-form}{expression-form dgroup}

\spc{0.4}
An ``expression'' represents an actual predicate instance within a Dgroup.  Notice
that this includes terms corresponding to function applications as expressions.
Each use of a predicate gets its own expression with its own name, so that a
higher-order relation gets translated into several expressions, with some having
expressions as arguments. These routines allow one to retrieve and inspect expressions
in the database. {\tt fetch-expression} returns the expression structure with the given
name.

The routines {\tt fully-expand-expression} and {\tt fully-expand-expression-form} are
useful for examining the form of an expression. Typically, the expression {\tt
(greater-than (diameter beaker) 5)} is stored as the expression {\tt greater-than23},
which has the form {\tt (greater-than diameter24 5)}. These routines return
a fully expanded expression form, where all expression names are replaced by their
corresponding forms.


\subsubsection{Dgroups}

\indent
\decfunc{fetch-dgroup}{dgroup-name {\tt \&optional} create?}\index{description group, retrieval}
\decfunc{return-dgroup}{dgroup-or-dgroup-name}

\spc{0.8}
Description groups (Dgroups) are stored in a simple data base managed
primarily by routines in {\tt sme.lisp}. The general procedures for Dgroup
and expression creation were described in sections~\ref{ss:new-dgroups} and
\ref{ss:new-expressions}. A Dgroup may be retrieved by name using
{\tt fetch-dgroup}, or created if {\it create?} is non-nil and no Dgroup
with the given name currently exists. {\tt return-dgroup} is designed for
routines that may take either an actual Dgroup or simply a Dgroup name
(e.g., {\tt fetch-expression}). It will cause an error if the Dgroup does not
previously exist.

\spc{0.8}
\decfunc{describe-dgroup}{dgroup}\index{description group, textual description}
\decfunc{menu-display-dgroup}{}\index{description group, graphical display}
\decfunc{menu-display-pairs}{}

\spc{0.3}
A Dgroup may be textually described using {\tt describe-Dgroup}, which
writes to the \SME\ output stream. Graphical display is provided in the
windowing system by {\tt menu-display-dgroup} for a single Dgroup or {\tt
menu-display-pairs} for a display of two Dgroups side by side.


\subsubsection{Creating and inspecting global matches}\label{ss:match}

\indent
\decfunc{match}{base-name target-name {\tt \&optional} display?}\index{global mapping, creation}
\decfunc{best-gmaps}{{\tt \&optional (}gmaps {\tt *gmaps*) (}percentage-range {\tt 0.02)}}\index{global mapping, selecting the best}
\decfunc{display-match}{base target {\tt \&optional (}total-run-time {\tt 0) (}bms-run-time {\tt 0)}}\index{global mapping, textual display}

\spc{0.8}
The {\tt match} function is the central \SME\ procedure. Given the names of two
Dgroups, it forms the complete set of global mappings between them. If {\it display?}
is non-nil, a description of the results will be sent to the current \SME\
output stream. The function itself returns two values, the total run time of
the match process in seconds and the subset of that time spent running the
BMS evidence rules. The analogical mapping results are stored in the
following global variables, which are then accessible by the user or
application program.

\spc{0.5}
\decvar{*base*}
\decvar{*target*}
\decvar{*match-hypotheses*}\index{match hypothesis, *match-hypotheses*}
\decvar{*gmaps*}\index{global mapping, *gmaps*}
\spc{0.2}

\noindent
The Gmap(s) with the highest evaluation score are retrieved by
{\tt best-gmaps}, which returns all Gmaps having a score within a given
percentage (default is 2\%) of the highest score. {\tt best-gmaps} returns
two values: the list of best Gmaps and the actual real-valued highest score.

\spc{0.8}
\decfunc{compare-gmaps}{gmap1 gmap2 {\tt \&optional} display?}\index{global mapping, comparing apparently identical gmaps}

\spc{0.4}
Occasionally, \SME\ will produce two or more ``best'' Gmaps that appear to
be identical yet have been classified as distinct. When these Gmaps are
large, the ``here's the set of match hypotheses'' output format can make it
frustrating to find what the slight differences are between a pair of Gmaps.
When given two Gmap structures, {\tt compare-gmaps} will return (using {\tt
values}) the list of the match hypotheses that are uniquely part of the
first Gmap and a list of match hypotheses that are uniquely part of the
second Gmap (i.e., ({\it gmap1} - {\it gmap2}) and ({\it gmap2} - {\it
gmap1})). When the windowing system is active, this option is available
through the system utilities menu.


\section{Algorithm Internals}

This section quickly describes a few internal points of the program in case
one has specialized needs for interfacing to the code. It assumes knowledge
of CommonLisp and the realization that for many questions, the only feasible
answer must be to examine the \SME\ program.

\subsection{The Match Function}

\indent
\decfunc{create-match-hypotheses}{base-dgroup target-dgroup}
\decfunc{run-rules}{}\index{rule system, run-rules}
\decfunc{calculate-nogoods}{base-dgroup target-dgroup}
\decfunc{generate-gmaps}{}
\decfunc{gather-inferences}{base-dgroup target-dgroup}\index{candidate inference, gather-inferences}
\decfunc{intern-gmaps}{}\index{rule system, giving it access to gmaps}

\spc{0.8}
The function {\tt match} is primarily a sequence of calls to these
functions. {\tt create-match-hypotheses} runs the match constructor
rules to form the individual match hypotheses. The BMS evidence rules are
then run on these match hypotheses ({\tt run-rules}) and their
dependence and inconsistency relationships are determined ({\tt
calculate-nogoods}). The function {\tt generate-gmaps} executes the three
merge steps, resulting in the set of complete global mappings being placed
in the variable {\tt *gmaps*}. The candidate inferences each Gmap
sanctions is then calculated ({\tt gather-inferences}) and additive BMS
nodes for each Gmap are formed ({\tt intern-gmaps}), allowing the evidence
rules to run on each Gmap.

\subsection{Match Hypotheses}\index{match hypothesis, defstruct}

\makebox[\textwidth]{{\tt match-hypothesis} \hfill [{\it Defstruct}]}
\decsubst{mh-form}{match-hypothesis}
\decsubst{mh-base-item}{match-hypothesis}
\decsubst{mh-target-item}{match-hypothesis}
\decsubst{mh-bms-node}{match-hypothesis}
\decsubst{node-belief+}{bms-node}
\decsubst{mh-plist}{match-hypothesis}
\decvar{*match-hypotheses*}

\spc{0.8}
Most programs using \SME\ will need to interact with the {\it match
hypothesis structures}. Slots to this structure type use the {\tt mh-}
prefix. There are several slots that might be important. The MH form, which
is a list of {\tt (MH {\it $<$base-item$>$} {\it $<$target-item$>$})}, is found
using {\tt mh-form}. This is the form used for triggering the MH evidence
rules, and is asserted in the BMS for each match hypothesis. The base and
target items are expression or entity structures. The base item or target item may
be obtained directly using {\tt mh-base-item} and {\tt mh-target-item},
respectively.  The BMS node for each match hypothesis is found by {\tt
mh-bms-node}. In turn, the weight for that node may be obtained using {\tt
node-belief+}.  Finally, each match hypothesis structure has a property list
slot ({\tt mh-plist}) which may be useful for various purposes.

\subsection{Global Mappings}\index{global mapping, defstruct}

\makebox[\textwidth]{{\tt global-mapping} \hfill [{\it Defstruct}]}
\decsubst{gm-id}{global-mapping}
\decsubst{gm-elements}{global-mapping}
\decsubst{gm-emaps}{global-mapping}
\decsubst{gm-base}{global-mapping}
\decsubst{gm-target}{global-mapping}
\decsubst{gm-inferences}{global-mapping}
\decsubst{gm-bms-node}{global-mapping}
\decsubst{gm-plist}{global-mapping}
\decvar{*gmaps*}

\spc{0.8}
Each Gmap is stored as a {\it global-mapping structure}. Slots to this
structure type use the {\tt gm-} prefix. Each Gmap is assigned a unique
integer identifier, found through {\tt gm-id}. The Gmap form used by the BMS
is not explicitly available, but is asserted as {\tt (Gmap {\it
$<$gmap-structure$>$})}. The match hypotheses associated with a Gmap are stored
in {\tt gm-elements}, while the subset of these that are entity mappings is
stored in {\tt gm-emaps}. The candidate inferences sanctioned by the Gmap
appear in {\tt gm-inferences} and are stored as a simple list data type,
using the syntax defined in Section~\ref{ss:new-dgroups} for description
group expressions.

\subsection{Candidate Inference Generation}\index{candidate inference, alternate algorithms}

The original candidate inference generator, as described in \cite{aaai86},
will take any base structure ``intersecting the Gmap structure''. The newer
(V. 2) edition only takes base structure ``intersecting a Gmap root''.
Thus, the newer edition is more cautious and far more efficient than the
older edition. Both versions of the code are available (in {\tt
match.lisp}), with the default being the newer, more cautious version. There
are theoretical arguments for and against each approach. For example, one
might want to use only the inferences from the newer, more cautious approach
at first since they are supported by more target knowledge and thus more
likely valid. If an analogy proves fruitful, one may want to relax these
constraints, and use the older version to find out what additional
inferences might be made.

\subsection{Rule System}

\indent
\decfunc{tre-rules-file}{}\index{rule system, dynamically swapping rule sets}
\decfunc{tre-save-rules}{}
\decfunc{tre-init}{}
\decfunc{restore-rules}{}
\decfunc{run-rules}{}
\decvar{*tre-rules-saver*}
\decvar{*initial-assertions*}

\spc{0.8} 
When a new rules file is loaded, the sme-rules-file command at the top of
the file initializes the BMS rule system prior to loading the new set of
rules.  At the bottom of the rules file, the fresh, just-loaded set of rules
are saved in the global variable *tre-rules-saver* by the command
tre-save-rules.  This variable saves the status of the rule counters and the
list of initial rules. A similar variable, {\tt *initial-assertions*}, is
used to store all assertions appearing in the rules file. When the BMS is
run, new rules may be created and added to the known set of rules. As a
result, each time {\tt match} is invoked, the BMS is reinitialized to its
status just after loading the rule file, that is, it is restored to the
status indicated in {\tt *tre-rules-saver*}.

This facility may be used by application programs to save different rule
sets in memory and swap them as needed, without having to load rule files
each time. For example, suppose \SME\ is invoked, which will cause it to run
the current rule set for Gmap scoring. If a second scoring criterion is then
desired, a second set of rules may be invoked using code of the form:
\begin{small}
\begin{alltt}
   (let ((save-rules sme:*tre-rules-saver*)          {\it ;save the previous rule set}
         (save-assertions sme:*initial-assertions*))
     (setq sme:*tre-rules-saver* *my-other-rules-set-rules*)
     (setq sme:*initial-assertions* *my-other-rules-set-assertions*)
     (sme:tre-init)                                  {\it ;initialize with new rules}
     (sme:run-rules)                                 {\it ;run the new rule set}
     (setq sme:*tre-rules-saver* save-rules)         {\it ;restore the original rules}
     (setq sme:*initial-assertions* save-assertions))
\end{alltt}
\end{small}
This saves \SME's normal rule set, runs a different one, and then restores
the rule set to its previous value. In this example, {\tt tre-init} was
used, which fully initializes the BMS. If the desire is simply to supply
additional rules without destroying the current BMS state, {\tt
restore-rules} should be used in place of {\tt tre-init}.  The variables
corresponding to ``my-other-rules-set'' may be initialized by a similar
program which saves the current rule set, {\it loads} the desired ``other
rule set'' file, sets the ``my-other-rules-set'' variables from {\tt
*tre-rules-saver*} and {\tt *initial-assertions}, and then restores the
original rule set.


\section{Summary}

The \SME\ program has been described from the perspective of how to actually
use it. A number of methods have been presented about how to configure \SME\
to perform a variety of different types of matches.  It is hoped that \SME\
may serve as a general mapping tool for research on analogical mapping,
allowing researchers to focus on the more substantive issue of general
theories of analogical mapping, as opposed to worrying about implementation
details. The latter has the unfortunate effect of producing the repeated
scenario in which analogy researcher A goes to analogy researcher B and says
``My program can do X, which yours cannot'', followed by researcher B
returning a month later with this simple modification added. By testing
different theories within the same program, we may now compare the more
critical ``This is a logical consequence of my theory''. A program does not
a theory make. It can, however, function as a useful analytical tool.

Of course, not all of the problems of implementing analogical mapping have
been solved.  Most critical is redesigning the potentially combinatoric
merge step 3, perhaps using either a heuristic search or connectionist
relaxation network as suggested in \cite{sme-aij}. Of theoretical relevance
is the appropriateness of the abstract structural approach which \SME\
embodies.

\section{Acknowledgements}

The development of \SME\ has been a collaborative effort with Ken Forbus and
Dedre Gentner, with significant influence provided by Janice Skorstad. This
work has also benefited from discussions with Steve Chien, John Collins, and
Ray Mooney. 

This research is supported in part by an IBM graduate fellowship and in part
by the Office of Naval Research, Contract No. N00014-85-K-0559.

\begin{thebibliography}{99}\addcontentsline{toc}{section}{References}

 \bibitem{Burstein83}
Burstein, M., Concept formation by incremental analogical reasoning and
debugging, {\it Proceedings of the Second International Workshop on
Machine Learning}, University of Illinois, Monticello, Illinois, June,
1983.  A revised version appears in {\it Machine Learning: An Artificial
Intelligence Approach Vol. II}, R.S. Michalski, J.G. Carbonell, and T.M.
Mitchell (Eds.), Morgan Kaufman, 1986.
 \bibitem{Carbonell83}
Carbonell, J.G., Learning by Analogy: Formulating and generalizing plans
from past experience, in: {\it Machine Learning: An Artificial
Intelligence Approach}, R.S. Michalski, J.G. Carbonell, and T.M.
Mitchell (Eds.), Morgan Kaufman, 1983.
 \bibitem{BMS}
Falkenhainer, B.,
Towards a general-purpose belief maintenance system, in: J.F. Lemmer \& L.N.
Kanal (Eds.), {\it Uncertainty in Artificial Intelligence, Volume II}, 1987.
Also Technical Report, UIUCDCS-R-87-1717, Department of Computer Science,
University of Illinois, 1987.
 \bibitem{VBAL}
Falkenhainer, B.,
An examination of the third stage in the analogy process:
Verification-Based Analogical Learning, Technical Report
UIUCDCS-R-86-1302, Department of Computer Science, University of Illinois,
October, 1986. A summary appears in {\it Proceedings of the Tenth
International Joint Conference on Artificial Intelligence}, Milan, Italy,
August, 1987.
 \bibitem{MLW}
Falkenhainer, B.,
Scientific theory formation through analogical inference, {\it
Proceedings of the Fourth International Machine Learning Workshop}, Irvine,
CA, June, 1987.
 \bibitem{DBR}
Falkenhainer, B.
The utility of difference-based reasoning, {\it Proceedings of the Seventh
National Conference on Artificial Intelligence}, St. Paul, August, 1988.
 \bibitem{Falken-phd}
Falkenhainer, B., Learning from physical analogies: A study in analogy
and the explanation process, Ph.D. Thesis, University of Illinois,
December, 1988.
 \bibitem{aaai86}
Falkenhainer, B., K.D. Forbus, D. Gentner,
The Structure-Mapping Engine, {\it Proceedings of the Fifth National
Conference on Artificial Intelligence}, August, 1986.
 \bibitem{sme-aij}
Falkenhainer, B., K.D. Forbus, D. Gentner,
The Structure-Mapping Engine:  Algorithm and Examples, Technical Report
UIUCDCS-R-87-1361, Department of Computer Science, University of Illinois,
July, 1987. To appear in {\it Artificial Intelligence}.
 \bibitem{framework}
Forbus, K.D. and D. Gentner.
Learning physical domains: Towards a theoretical framework,
in: {\it Machine Learning: An Artificial Intelligence Approach Vol. II}, R.S. Michalski,
J.G. Carbonell, and T.M. Mitchell (Eds.), Morgan Kaufmann, 1986.
 \bibitem{Gentner80}
Gentner, D.
{\it The Structure of Analogical Models in Science}, BBN Tech.
Report No. 4451, Cambridge, MA., Bolt Beranek and Newman Inc., 1980.
 \bibitem{smt-cogsci}
Gentner, D.
Structure-Mapping: A Theoretical Framework for Analogy, {\it Cognitive
Science} {\bf 7}(2), 1983. 
 \bibitem{allerton}
Gentner, D.
Mechanisms of analogy.  To appear in S. Vosniadou and A. Ortony, (Eds.),
{\em Similarity and analogical reasoning}, Cambridge University Press,
Oxford.
 \bibitem{analogica}
Gentner, D.
Analogical inference and analogical access, To appear in A. Preiditis (Ed.),
{\em Analogica: Proceedings of the First Workshop on Analogical Reasoning},
London, Pitman Publishing Co.  Presented in December, 1986.
 \bibitem{tinlap3}
Gentner, D., B. Falkenhainer, and J. Skorstad
Metaphor: The good, the bad and the ugly. {\it Proceedings of the Third
Conference on Theoretical Issues in Natural Language Processing}, Las
Cruces, New Mexico, January, 1987.  
 \bibitem{Greiner}
Greiner, R., Learning by understanding analogies, PhD Thesis (STAN-CS-1071),
Department of Computer Science, Stanford University, September, 1985.
 \bibitem{hayes-roth}
Hayes-Roth, F., J. McDermott, An interference matching technique for
inducing abstractions, {\it Communications of the ACM}, {\bf 21}(5),
May, 1978.
 \bibitem{ACME}
Holyoak, K., \& Thagard, P. Analogical mapping by constraint satisfaction,
June, 1988, ({\it submitted for publication}).
 \bibitem{Indurkhya}
Indurkhya, B., Constrained semantic transference: A formal theory of
metaphors, Technical Report 85/008, Boston University, Department of
Computer Science, October, 1985.
 \bibitem{KedarCabelli}
Kedar-Cabelli, S., Purpose-directed analogy. {\it Proceedings of the Seventh
Annual Conference of the Cognitive Science Society}, Irvine, CA, 1985.
 \bibitem{Reed}
Reed, S.K., A structure-mapping model for word problems. Paper presented
at the meeting of the Psychonomic Society, Boston, 1985.
 \bibitem{Rumelhart}
Rumelhart, D.E., \& Norman, D.A., Analogical processes in learning.  In J.R.
Anderson (Ed.), {\it Cognitive skills and their acquisition}, Hillsdale,
N.J., Erlbaum, 1981.
 \bibitem{aaai87}
Skorstad, J., B. Falkenhainer and D. Gentner
Analogical Processing: A simulation and empirical corroboration, {\it
Proceedings of the Sixth National Conference on Artificial Intelligence},
Seattle, WA, August, 1987.
 \bibitem{Winston80}
Winston, P.H., Learning and Reasoning by Analogy, {\it Communications of the
ACM}, {\bf 23}(12), 1980.
 \bibitem{Winston82}
Winston, P.H., Learning new principles from precedents and exercises,
{\it Artificial Intelligence}, {\bf 19}, 321-350, 1982.

\end{thebibliography}

\newpage
\addcontentsline{toc}{section}{Index}

\begin{theindex}

\item algorithm, 4-10
	\subitem  summary, 3
\item analogical hint, 17
\item arg-list macro, 28
\item {\tt assert!} function, 14
\item {\tt attribute?} macro, 28
\indexspace
\item {\tt *base*} variable, 29
\item batch mode, 22
\item {\tt best-gmaps} function, 29
\indexspace
\item {\tt calculate-nogoods} function, 30
\item candidate inference, 9
	\subitem  alternate algorithms, 31
	\subitem  gather-inferences, 30
\item {\tt change-parms} function, 26
\item {\tt children-of?} function, 15
\item {\tt clear-given-mappings} function, 18
\item {\tt clear-scroll} function, 25
\item {\tt commutative?} macro, 28
\item {\tt compare-gmaps} function, 29
\item {\tt constant-entity?} macro, 27
\item conventions, 1
	\subitem  file names, 20
	\subitem  function names, 1
\item {\tt create-match-hypotheses} function, 30
\indexspace
\item {\tt declare-given-mappings} function, 18
\item {\tt defDescription} macro, 12
\item {\tt defEntity} macro, 11
\item {\tt define-description} function, 12
\item {\tt define-Entity} function, 11
\item {\tt define-predicate} function, 10
\item {\tt defPostMatcher} macro, 23
\item {\tt defPredicate} macro, 10
\item {\tt defSME-Parameter} macro, 25
\item {\tt defSME-Utility} macro, 26
\item {\tt describe-dgroup} function, 29
\item description group, 11-12
	\subitem  graphical display, 29
	\subitem  retrieval, 29
	\subitem  textual description, 29
\item Dgroup, 12
\item {\tt dgroup-directory} macro, 22
\item {\tt dgroup-file} macro, 22
\item {\tt display-match} function, 29
\item {\tt dump-scroll} function, 25
\item {\tt dump-scroll-menu} function, 25
\indexspace
\item entity, 11
	\subitem  declaring, 11
\item {\tt entity?} function, 27
	\subitem  inspection functions, 27
\item {\tt entity-domain} macro, 27
\item {\tt entity-name?} macro, 27
\item {\tt entity-type?} function, 27
\item {\tt eval-form} macro, 28
\item execution, 20
\item expression, 12
	\subitem  adding new expressions, 12
\item {\tt expression} function, 12
	\subitem  inspection functions, 28
\item {\tt expression-functor} function, 28
\item {\tt expression-type} macro, 28
\indexspace
\item {\tt fetch-dgroup} function, 29
\item {\tt fetch-entity-definition} macro, 27
\item {\tt fetch-expression} function, 28
\item {\tt fetch-predicate-definition} macro, 28
\item file organization, 1
\item {\tt fully-expand-expression} function, 28
\item {\tt fully-expand-expression-form} function, 28
\item {\tt function?} macro, 28
\indexspace
\item {\tt gather-inferences} function, 30
\item generalize, 16
\item {\tt generalize} function, 23
\item {\tt generate-gmaps} function, 30
\item {\tt get-dgroup} function, 26
\item {\tt get-rules} function, 26
\item global mapping, 7-10
	\subitem  comparing apparently identical gmaps, 29
	\subitem  creation, 29
	\subitem  defstruct, 30
	\subitem  {\tt *gmaps*}, 29
	\subitem  scoring, 9, 15
	\subitem  selecting the best, 29
	\subitem  textual display, 29
\item {\tt *gmaps*} variable, 29, 31
\item {\tt gm-base} subst, 31
\item {\tt gm-bms-node} subst, 31
\item {\tt gm-elements} subst, 31
\item {\tt gm-emaps} subst, 31
\item {\tt gm-id} subst, 30
\item {\tt gm-inferences} subst, 31
\item {\tt gm-plist} subst, 31
\item {\tt gm-target} subst, 31
\item {\tt *graphics-pane*} variable, 25
\indexspace
\item {\tt initial-assertion} macro, 14
\item {\tt *initial-assertions*} variable, 31
\item installation, 20
\item {\tt install-MH} function, 13
\item {\tt intern-gmaps} function, 30
\indexspace
\item {\tt language-file} macro, 22
\item lisp machine interface, 25
	\subitem  predicate documentation, 11
\item {\tt *lisp-pane*} variable, 25
\indexspace
\item match constructor rules, 13
\item match evidence rules, 14
\item {\tt match} function, 29
\item match hypothesis, 4-5
	\subitem  defstruct, 30
	\subitem  inspecting evidence justifications, 24
	\subitem  installing, 13
	\subitem  {\tt *match-hypotheses*}, 29
	\subitem  scoring, 14-15
\item {\tt match-evidence-inspector} function, 24
\item {\tt *match-hypotheses*} variable, 29-30
\item {\tt menu-display-dgroup} function, 29
\item {\tt menu-display-pairs} function, 29
\item {\tt menu-utilities} function, 26
\item {\tt mh-base-item} subst, 30
\item {\tt mh-bms-node} subst, 30
\item MHC-rule, 13
\item {\tt mh-form} subst, 30
\item {\tt mh-plist} subst, 30
\item {\tt mh-target-item} subst, 30
\indexspace
\item {\tt n-ary?} macro, 28
\item {\tt node-belief+} subst, 30
\item {\tt numargs} macro, 28
\indexspace
\item packages, 1
	\subitem site specific, 20
\item {\tt paired-item?} function, 19
\item {\tt *parameter-menu-options*} variable, 25
\item parameters, 25
	\subitem  site specific, 1, 20
\item predicate, 10
	\subitem  declaring, 10
	\subitem  inspection functions, 28
\item {\tt predicate?} macro, 28
\item {\tt predicate-type} macro, 28
\indexspace
\item {\tt rassert!} macro, 14
\item {\tt relation?} macro, 28
\item report generation, 22
\item {\tt report-comments} macro, 23
\item representation issues, 19
\item {\tt restore-rules} function, 31
\item {\tt return-dgroup} function, 29
\item {\tt rule} macro, 14
\item rule system, 12-19, 19
	\subitem  analogical hints, 17
	\subitem  dynamically swapping rule sets, 31
	\subitem  giving it access to gmaps, 30
	\subitem  match constructor rules, 13
	\subitem  match evidence rules, 14
	\subitem  pure isomorphisms, 16
	\subitem  relaxing identical predicates, 16
	\subitem  rule file syntax, 13
	\subitem  {\tt run-rules,} 30
	\subitem  simulating SPROUTER, 16
	\subitem  simulating structure-mapping theory, 15
\item {\tt rule-directory} macro, 22
\item {\tt rule-file} macro, 22
\item {\tt rule-sets} macro, 22
\item {\tt run-batch-file} function, 22
\item {\tt run-matcher-on} macro, 23
\item running SME, 20
\item {\tt run-rules} function, 30-31
\indexspace
\item {\tt sanctioned-pairing?} function, 19
\item scroll windows, 25
	\subitem  saving contents, 25
	\subitem  writing to, 25, 27
\item {\tt *scroll-pane*} variable, 25
\item {\tt select-double-scroll} function, 25
\item {\tt select-graphics} function, 25
\item {\tt select-large-graphics} function, 25
\item {\tt select-scroll} function, 25
\item {\tt select-split} function, 25
\item {\tt select-windowing-configuration} function, 25
\item {\tt send-report-to} macro, 23
\item site specific information, 20
	\subitem  packages, 20
	\subitem  pathnames, 20
\item {\tt *sme-default-rules*} variable, 20
\item {\tt *sme-dgroup-pathname*} variable, 20
\item {\tt *sme-files*} variable, 20
\item {\tt sme-format} macro, 27
\item {\tt *sme-frame*} variable, 25
\item {\tt *sme-graphics-output*} variable, 26
\item {\tt sme-init} function, 20
\item {\tt *sme-language-file*} variable, 20
\item {\tt *sme-output-stream*} variable, 26
\item {\tt *sme-parameters*} variable, 25
\item {\tt *sme-predicates*} variable, 28
\item {\tt sme-print} function, 27
\item {\tt sme-rules-file} function, 13
\item {\tt *sme-rules-pathname*} variable, 20
\item {\tt *sme-system-pathname*} variable, 20
\item {\tt sme-terpri} function, 27
\item {\tt *spare-scroll-pane*} variable, 25
\item system utilities, 26
\item {\tt *system-utilities-menu*} variable, 26
\indexspace
\item {\tt *target*} variable, 29
\item {\tt *the-lisp-package*} variable, 20
\item {\tt *the-user-package*} variable, 20
\item {\tt tre-init} function, 31
\item {\tt tre-rules-file} function, 31
\item {\tt *tre-rules-saver*} variable, 31
\item {\tt tre-save-rules} function, 13, 31
\indexspace
\item {\tt *windowing?*} variable, 26
\indexspace
\end{theindex}

%%\batchmode
\documentstyle[11pt,alltt]{article}
\pagestyle{myheadings}
\makeindex
\markboth{SME User's Manual}{SME User's Manual}
\setlength{\oddsidemargin}{-0.1in}
\setlength{\evensidemargin}{0in}
\setlength{\textwidth}{6.5in}
\setlength{\topmargin}{-0.3in}
\setlength{\textheight}{9in}
%% Declaring SME lisp hooks
\newcommand{\decvar}[1]{\noindent\makebox[\textwidth]{{\tt #1}
                                   \hfill [{\it Variable}]}\index{#1 variable}}
\newcommand{\decfunc}[2]{\noindent\makebox[\textwidth]{{\tt #1}\ \ {\it #2}
                                               \hfill [{\it Function}]}\index{#1 function}}
\newcommand{\decmacro}[2]{\noindent\makebox[\textwidth]{{\tt #1}\ \ {\it #2}
                                                \hfill [{\it Macro}]}\index{#1 macro}}
\newcommand{\decsubst}[2]{\noindent\makebox[\textwidth]{{\tt #1}\ \ {\it #2}
                                                \hfill [{\it Subst}]}\index{#1 subst}}
%% Handy abbreviations
\newcommand{\metasyntax}[1]{$\langle #1 \rangle$}
\newcommand{\spc}[1]{\vspace{#1\baselineskip}}
\newcommand{\gthan}{\( >\)}
\newcommand{\border}{\rule{\textwidth}{0.04in}}
\newcommand{\SME}{{\tt SME}}
\newcommand{\SMET}{\mbox{${\tt SME}_{{\it SMT}}$}}
\newcommand{\Phineas}{{\tt PHINEAS}}
\newcommand{\keypoint}[2]{\begin{itemize}
                             \item \parbox[t]{0.88\textwidth}{{\it #1:} #2}
                          \end{itemize}
                          \vspace{0.08in}}
%
\begin{document}

\begin{titlepage}
{\center {\Large Technical Report UIUCDCS-R-88-1421 \\ }}
\ \ \ \vspace{0.5in} \\
{\center {\bf {\huge The SME User's Manual} \\[2ex]
              {\LARGE  (SME Version 2E)} \\ }}
\vspace{0.7in}
\begin{center}
            {\Large Brian Falkenhainer}       \\[2ex]
{\large     Qualitative Reasoning Group       \\
          Department of Computer Science      \\
         University of Illinois at Urbana-Champaign \\
\vspace{0.75in}
                December, 1988 \\ }
\end{center}
\vspace{0.75in}
\begin{abstract}
This paper documents the {\it Structure-Mapping Engine} (\SME), a
general-purpose program for studying analogical processing.  It provides a
comprehensive description of the program and instructions for using it,
including techniques for integrating it into larger systems. One section
demonstrates methods for configuring \SME\ to a variety of mapping
preferences and suggests the range of theoretical variations available.
\end{abstract}

\end{titlepage}
\pagenumbering{roman}
\tableofcontents
\markboth{SME User's Manual}{SME User's Manual}
\newpage
\pagenumbering{arabic}

\section{Introduction}

The Structure-Mapping Engine (\SME) is a general tool for performing various
types of analogical mappings. \SME\ was originally developed to simulate
Gentner's {\it Structure-Mapping} theory of analogy
\cite{smt-cogsci,allerton,analogica}. It was hoped that the developed system
would also be able to model the other types of similarity comparisons
sanctioned by Gentner's theory, such as {\it literal similarity} and {\it
mere appearance}. What ended up being developed was an extremely flexible
and efficient system. Most theoretical assumptions are left out of the
program and are supplied through match rules. Thus, while \SME\ was
originally designed to simulate the comparisons of structure-mapping theory,
it may simulate many others as well. Given a set of theoretical restrictions
on what constitutes a reasonable analogical mapping, one may implement these
restrictions in the form of rules and use \SME\ to interactively test their
consequences. This report is intended to make that task easier.

This paper is designed for those interested in using \SME\ for studying
analogical processing, testing alternate theories, or as the mapping
component in a larger system. It describes the options and user support
provided in \SME, how to use it for testing theories, and how to integrate
it with other programs. For a discussion of the theory behind \SME, the
general algorithm, and descriptions of the program in operation, one should
consult \cite{aaai86,sme-aij} prior to reading this manual. Descriptions of
the use of \SME\ in various research projects may be found in
\cite{VBAL,MLW,DBR,Falken-phd,allerton,tinlap3,aaai87}, while descriptions
of Gentner's Structure-Mapping theory appear in
\cite{Gentner80,smt-cogsci,allerton,analogica,framework,sme-aij}.

\subsection{Conventions}\index{conventions}

Throughout this guide, a few conventions will be used which should be
explained at this time.

\begin{enumerate}

 \item {\it CommonLisp Packages}.\index{packages} The \SME\ system resides
in its own package, {\tt SME}, which is defined to use CommonLisp. As a
result, any reference to an \SME\ function or variable must specify the {\tt
SME} package, as in the function {\tt sme:define-predicate}. To simplify the
discussion, we will omit the package prefix when describing \SME\ functions,
macros, and variables. In addition, while the \SME\ routines reside in the
{\tt SME} package, the structures it manipulates reside in the general {\tt
USER} package.
 \item {\it The declarative interface.}\index{conventions, function names}
In general, the routines used to present data items to \SME, such as
predicate definitions and concept descriptions, appear in two,
functionally-equivalent forms. These two types have a naming convention
associated with each. The most common type is the {\it declarative} or macro
interface. The declarative routines do not evaluate their arguments and
match the syntactic form {\tt def}{\it routine-name}. For example, to define
the entity {\tt sun} declaratively, one writes {\tt (defEntity sun)}. The
second type is the {\it functional} interface, which is present to support
declarations by external programs.  These routines evaluate their arguments
and match the syntactic form {\tt define-}{\it routine-name}. For example,
to define the entity {\tt sun} functionally, one writes {\tt (define-entity
'sun)}.

\end{enumerate}


\subsection{File Organization}\index{file organization}

\SME\ is contained within the following twelve files:

\begin{description}
 \item[config.lisp]  The declarations for site specific
parameters.\index{parameters, site specific}
 \item[defs.lisp]  The basic structure definitions and macros used throughout
\SME. 
 \item[bits.lisp]  Routines for creating and manipulating bit vectors.
 \item[bms.lisp]  The belief-maintenance system (BMS) - a probabilistic TMS.
 \item[bms-tre.lisp]  The rule system and problem-solver front end for the BMS.
 \item[sme.lisp]  The \SME\ top-level routines, such as initialization,
defining facts about a concept, and fetching and storing facts and concept
descriptions.
 \item[match.lisp]  The \SME\ mapping algorithm.
 \item[match-rules-support.lisp]  A few functions useful for writing \SME\
match rules.
 \item[display.lisp]  Machine independent output routines.
 \item[windowing.lisp]  Symbolics dependent interface routines.
 \item[batch.lisp]  Routines to enable execution in {\it batch} mode with a
final report generated.
 \item[generalize.lisp]  Inductive generalization support.
\end{description}


\section{System Review}

The Structure-Mapping Engine can simulate a class of {\it structural
approaches} to analogical mapping. In these approaches, there is a distinct
stage of matching and carryover of predicates from one domain (the {\it
base}) into another (the {\it target}) within the larger analogy process.
Furthermore, although there are a number of differences, there is widespread
agreement among these techniques on one fundamental restriction
\cite{Burstein83,Carbonell83,Greiner,Indurkhya,KedarCabelli,Reed,Rumelhart,Winston82}:
\begin{enumerate}
 \item {\it Structural consistency}. If a final analogical mapping includes
a predicate in the base paired with a predicate in the target, then it must
also include corresponding pairings between each of their arguments. This
criterion simply asserts that an analogical mapping must not produce
syntactically meaningless predicate calculus forms.
\end{enumerate}
In \SME, this restriction was enforced by the requirement of simulating
Structure-Mapping theory; its impact on the algorithm is described in
\cite{sme-aij}. However, this restriction is only part of that theory and
alone does not uniquely define a matching algorithm. Additional theoretical
restrictions must be supplied through {\it match rules}. This enables \SME\
to be used in exploring the space of theories consistent with this single
criterion. An additional restriction is enforced by default:

\keypoint{One-to-one mapping}{No base item (predicate or object) may be
paired with multiple target items. Likewise, no target item may be paired
with multiple base items.}

Enforcement of the one-to-one restriction is a global parameter which may be
disabled. Support is provided to implement variations of one-to-one within
the match rules.

Match rules specify what pairwise matches are possible and provide local
measures of evidence used in computing the evaluation score. These rules are
the key to \SME's flexibility. To build a new matcher one simply loads a new
set of match rules.  This has several important advantages.  First, we can
simulate all of the types of comparisons sanctioned by Structure-Mapping
theory with one program.  Second, the rules could in theory be ``tuned'' if
needed to simulate particular kinds of human performance. Third, a variety
of other analogical mapping systems may be simulated for comparison and
theoretical investigation.  The breadth of the space of these structural
approaches is suggested by the examples in Section~\ref{ss:rule-system}.

In this section, the \SME\ matching algorithm is briefly reviewed, followed
by a short discussion of how theoretical guidelines may be added to the
general mechanism. It is a summary of the algorithm description appearing in
\cite{sme-aij}, annotated with the \SME\ functions that carry out each step.


\begin{figure}
\border
\vspace{2.1in}
\caption{Simplified water flow and heat flow descriptions.}
\label{fg:wfhf-calculus}
\border
\end{figure}


\subsection{Algorithm Review}\index{algorithm, summary}

Given descriptions of a base and a target (called Dgroups), \SME\ builds all
structurally consistent interpretations of the comparison between them.
Each interpretation of the match is called a {\em global mapping}, or {\em
Gmap}.  Gmaps consist of three parts:

\begin{enumerate}
\item{\em Correspondences:} A set of pairwise matches between the
expressions and entities of the two dgroups.
\item {\em Candidate Inferences:} A set of new expressions which the
comparison suggests holds in the target dgroup.
\item {\em Structural Evaluation Score:} (Called {\em SES} for
brevity) A numerical estimate of match quality.
\end{enumerate}

For example, given the descriptions of water flow and heat flow shown in
Figure~\ref{fg:wfhf-calculus}, \SME\ might, depending on the current
theoretical configuration, offer several alternative interpretations for
this potential analogy.  In one interpretation, the central inference is
that water flowing from the beaker to the vial corresponds to heat flowing
from the coffee to the ice cube. Alternatively, one could map water to
coffee, since they are both liquids.


\begin{figure}
\border
\begin{small}
\begin{tt}
\begin{itemize}
 \item Run MHC rules to construct match hypotheses (create-match-hypotheses).
 \item Calculate the {\it Conflicting} set for each match hypothesis (calculate-nogoods).
 \item Calculate the {\it EMaps} and {\it NoGood} sets for each match
        hypothesis by upward propagation from entity mappings
        (generate-justifications and propagate-descendants).
 \item During the propagation, delete any match hypotheses that have
        justification holes (propagate-death).
 \item Merge match hypotheses into Gmaps (generate-gmaps).
      \begin{enumerate}
         \item Interconnected and consistent (generate-structure-groups).
         \item Consistent members of same base structure (merge-base).
         \item Any further consistent combinations (full-gmap-merge).
      \end{enumerate}
 \item Calculate the candidate inferences for each GMap (gather-inferences).
 \item Score the matches (run-rules).
      \begin{enumerate}
         \item Local match scores.
         \item Global structural evaluation scores.
      \end{enumerate}
\end{itemize}
\end{tt}
\end{small}
  \caption{Summary of SME algorithm.}
  \label{fg:sme-overview}
\border
\end{figure}


The \SME\  algorithm (see Figure~\ref{fg:sme-overview}) is logically divided
into four stages:
\begin{enumerate} 
\item {\em Local match construction:} Finds all pairs of ({\it BaseItem},
{\it TargetItem}) that potentially can match.  A {\em Match Hypothesis} is
created for each such pair to represent the possibility that this local
match is part of a global match.
\item {\em Gmap construction:} Combines the local matches into maximal
consistent collections of correspondences.
\item {\em Candidate inference construction:} Derives the inferences
suggested by each Gmap.
\item {\em Match Evaluation:} Attaches evidence to each local match and
uses this evidence to compute structural evaluation scores for each Gmap. 
\end{enumerate}

\noindent
Each computation will now be reviewed, using a simple example to
illustrate their operation. In this example, the rules of structure-mapping
theory are in use. It is important to distinguish the general \SME\ system
from its behavior when using the rules of a particular theory. Hence, when
using the rules of structure-mapping theory, it will be called \SMET.

\subsubsection{Step 1: Local match construction (create-match-hypotheses)}\index{match hypothesis}\index{algorithm}

\begin{figure}
\border
\vspace{3.1in}
\caption[Local Match Construction]{Local Match Construction. The water flow
and heat flow descriptions of Figure~\ref{fg:wfhf-calculus} have been drawn
in the abstract and placed to the left and right, respectively. The objects
in the middle depict match hypotheses.}
\label{fg:3D-MH}\index{algorithm}
\border
\end{figure}


\SME\ begins by finding for each entity and predicate in the base the set of
entities or predicates in the target that could plausibly match that item
(see Figure~\ref{fg:3D-MH}). Plausibility is determined by {\it match
constructor} rules, which are of the form:
\begin{tt}
\begin{tabbing}
\ \ \= (MHCrule \= (\metasyntax{Trigger} \= \metasyntax{BaseVariable} \metasyntax{TargetVariable} \\
\> \> \> {\it [}:test \metasyntax{TestForm}{\it ]}) \\
\> \ \ \ \  \metasyntax{Body}) \\
\end{tabbing}
\end{tt}

The body of these rules is run on each pair of items (one from the base and
one from the target) that satisfy the condition and installs a {\it match
hypothesis}\index{match hypothesis} which represents the possibility of them
matching. For example, to state that an expression in the base may match an
expression in the target whose functor is identical, we write:
\begin{verbatim}
   (MHC-rule (:filter ?b ?t :test (equal (expression-functor ?b)
                                         (expression-functor ?t)))
         (install-MH ?b ?t))
\end{verbatim}


\begin{figure}
\border
\vspace{4.4in}
\caption[Water Flow / Heat Flow Analogy]{Water Flow / Heat Flow
Analogy After Local Match Construction. Here we show the graph of match
hypotheses depicted schematically in Figure~\ref{fg:3D-MH}, augmented by links
indicating expression-to-arguments relationships. Match hypotheses which are not
descended from others are called {\it roots} (e.g., the matches
between the {\tt GREATER} predicates, {\tt MH-1} and {\tt MH-6}, and the
match for the predicate {\tt FLOW}, {\tt MH-9}). Match hypotheses between
entities are called {\it Emaps} (e.g., the match between beaker and coffee,
{\tt MH-4}).  Emaps play an important role in algorithms based on structural
consistency.}
\label{fg:wfhf-local}\index{algorithm}
\border
\end{figure}


The likelyhood of each match hypothesis\index{match hypothesis} is found by
running {\it match evidence} rules and combining their results. The evidence
rules provide support for a match hypothesis by examining the structural and
syntactic properties of the items matched. For example, the rule
\begin{verbatim}
   (MHErule ((:intern (MH ?b ?t) :test (and (expression? ?b) (expression? ?t)
                                            (eq (expression-functor ?b)
                                                (expression-functor ?t)))))
       (assert! (implies same-functor (MH ?b ?t) (0.5 . 0.0))))
\end{verbatim}
states ``If the two items are expressions and their functors are the same, then
supply 0.5 evidence in favor of the match hypothesis.'' The rules may also
examine match hypotheses associated with the arguments of these items to
provide support based on systematicity. This causes evidence for a match
hypothesis to increase with the amount of higher-order structure supporting
it.

The state of the match between the water flow and heat flow descriptions of
Figure~\ref{fg:wfhf-calculus} after running these first two sets of rules is
shown in Figure~\ref{fg:wfhf-local}.  There are several important things to
notice in this figure.  First, there can be more than one match
hypothesis\index{match hypothesis} involving any particular base or target
item.  Second, our rules required predicates to match identically while they
allowed entities to match on the basis of their roles in the predicate
structure.  Thus while {\tt TEMPERATURE} can match either {\tt PRESSURE} or
{\tt DIAMETER}, {\tt IMPLIES} cannot match anything but {\tt IMPLIES}.
Third, not every possible correspondence is created. Local matches between
entities are only created when justified by some other identity.  This
significantly constrains the number of possible matches in the typical case.


\begin{figure}
\border
\vspace{3.1in}
\caption[Conflicting relationships.]{Water Flow - Heat Flow analogy after
computation of {\it Conflicting} relationships. Simple lines show the
tree-like graph that the grounding criteria imposes upon match hypotheses.
Lines with circular endpoints indicate the {\it Conflicting} relationships
between matches. Some of the original lines from MH construction have been
left in to show the source of a few {\it Conflicting} relations.}
\label{3D-nogoods}
\border
\vspace{3.1in}
\caption{GMap Construction. (a) Merge step 1: Interconnected and consistent.
(b) Merge step 2: Consistent members of the same base structure. (c) Merge
step 3: Any further consistent combinations.}
\label{fg:wfhf-merging}\index{global mapping}\index{algorithm}
\border
\end{figure}


\subsubsection{Step 2: Global Match Construction}\index{global mapping}\index{algorithm}

The second step in the \SME\ algorithm combines local match hypotheses into
collections of global matches (Gmaps).  Intuitively, each global match is
the largest possible set of match hypotheses that depend on the same one to
one object correspondences.

More formally, Gmaps consist of {\em maximal, structurally consistent}
collections of match hypotheses.  A collection of match hypotheses is
structurally consistent if it satisfies two criteria:
\begin{enumerate}
\item {\em One-to-one:} No two match hypotheses assign the same
base item to multiple target items or any target item to multiple
base items.
\item {\em Support:} If a match hypothesis {\tt MH} is in the
collection, then so are the match hypotheses which pair up all of
the arguments of {\tt MH}'s base and target items.
\end{enumerate}
The preservation criteria enforces strict one to one mappings.  The
grounding criteria preserves connected predicate structure.  A collection is
maximal if adding any additional match hypothesis would render the collection
structurally inconsistent.

The formation of global matches\index{global mapping} is composed of two primary stages:

\begin{enumerate}
\item {\em Compute consistency relationships {\tt (calculate-nogoods)}:}
Here we generate for each match hypothesis the sets of entity mappings it
entails, what match hypotheses it locally conflicts with, and which match
hypotheses it is structurally inconsistent with.  This information
simplifies the detection of contradictory sets of match hypotheses, a
critical operation in the rest of the algorithm. The result of this stage of
processing appears in Figure~\ref{3D-nogoods}.

\item {\em Merge match hypotheses {\tt (generate-gmaps)}:} Compute Gmaps by
successively combining match hypotheses as follows:
 \begin{enumerate}
 \item {\em Form initial combinations {\tt (generate-structure-groups)}:}
Combine interconnected and consistent match hypotheses into an initial set
of Gmaps (Figure~\ref{fg:wfhf-merging}a).
 \item {\em Combine dependent Gmaps {\tt (merge-base)}:} Since base and
target dgroups are rarely isomorphic, some Gmaps in the initial set will
overlap in ways that allow them to be merged.  The advantage in merging them
is that the new combination may provide structural support for candidate
inferences (Figure~\ref{fg:wfhf-merging}b).
 \item {\em Combine independent collections {\tt
(full-gmap-merge)}:}\footnote{These two merge steps (b and c) are called by
{\tt merge-gmaps}, which is in turn called by {\tt generate-gmaps}.} The
results of the previous step are next combined to form maximal consistent
collections (Figure~\ref{fg:wfhf-merging}c).
 \end{enumerate}
\end{enumerate}

A parameter option allows the support criterion to be weakened so that it
does not cross the boundaries of a {\it relational group} \cite{Falken-phd}.
A relational group is distinguished as an unordered collection of relational
structures that may be collectively referred to as a unit.  They correspond
to the abstract notion of a ``set'' and are associated to predicates taking
any number of arguments.  For example, a set of relations joined by the
predicate {\tt AND} defines a relational group.  Other examples include the
axioms of a theory, a decomposable compound object, or the relations holding
over an interval of time.  Intuitively, we would like to say that two groups
correspond without requiring that their contents are exhaustively mapped.

If base and target propositions each contain a group as an argument, the
propositions should not be prevented from matching if the groups' members
cannot be exhaustively paired. For example, the set of relations
\begin{tt}
\begin{tabbing}
~~~B:~~~\= Implies[And(P$_{1}$,P$_{2}$,P$_{3}$), P$_{4}$] \` {\rm (1)} \\
~~~T:   \> Implies[And(P$^{\prime}_{1}$,P$^{\prime}_{2}$), P$^{\prime}_{4}$]
\end{tabbing}
\end{tt}

\noindent
should match better than the set of relations
\begin{tt}
\begin{tabbing}
~~~B:~~~\= Implies[And(P$_{1}$,P$_{2}$,P$_{3}$), P$_{4}$] \` {\rm (2)} \\
~~~T$^{\prime}$: \> P$^{\prime}_{1}$, P$^{\prime}_{2}$, P$^{\prime}_{4}$
\end{tabbing}
\end{tt}

\noindent 
The original model of structural consistency would score (1) and (2)
equally, since the {\tt Implies} relations of (1) would not be allowed to
match. This is a particularly important consideration when matching
sequential, state-based descriptions (e.g., the behavior of a system through
time). The set of relations describing a pair of states often do not
exhaustively match or are of different cardinality. Yet, higher-order
relations over states, such as temporal orderings, are vital and must appear
in the mapping.



\subsubsection{Step 3: Compute Candidate Inferences (gather-inferences)}\index{candidate inference}\index{algorithm}

Associated with each Gmap\index{global mapping} is a (possibly empty) set of
candidate inferences.  Candidate inferences are base predicates that would
fill in structure which is not in the Gmap (and hence not already in the
target). If a candidate inference contains a base entity that has no
corresponding target entity (i.e., the base entity is not part of any match
hypothesis for that gmap), \SME\ introduces a new, hypothetical entity into
the target.  Such entities are represented as a skolem function of the
original base entity (i.e., {\tt (:skolem base-entity)}).

In Figure~\ref{fg:wfhf-done}, Gmap \#1 has the top level {\tt CAUSE}
predicate as its sole candidate inference.  In other words, this Gmap
suggests that the cause of the flow in the heat dgroup is the difference in
temperatures.  If the {\tt FLOW} predicate was not present in the target,
then the candidate inferences for a Gmap corresponding to the pressure
inequality would be both {\tt CAUSE} and {\tt FLOW}. Note that {\tt
GREATER-THAN[DIAMETER(coffee), DIAMETER(ice cube)]} is not a valid candidate
inference for the first Gmap because it does not intersect the existing Gmap
structure.\index{candidate inference}

\begin{figure}
\border
\begin{small}
\begin{verbatim}
  Rule File: literal-similarity.rules     Number of Match Hypotheses: 14

  Gmap #1: { (>PRESSURE  >TEMPERATURE)  (PRESSURE-BEAKER  TEMP-COFFEE)
             (PRESSURE-VIAL  TEMP-ICE-CUBE)  (WFLOW  HFLOW) }
      Emaps:  { (beaker coffee)  (vial ice-cube)  (water heat)  (pipe bar) }
      Weight:  5.99
      Candidate Inferences: (CAUSE >TEMPERATURE HFLOW)

  Gmap #2: { (>DIAMETER  >TEMPERATURE)  (DIAMETER-1  TEMP-COFFEE)
             (DIAMETER-2  TEMP-ICE-CUBE) }
      Emaps:  { (beaker coffee)  (vial ice-cube) }
      Weight:  3.94
      Candidate Inferences:  { }

  Gmap #3: { (LIQUID-3  LIQUID-5)  (FLAT-TOP-4  FLAT-TOP-6) }
      Emaps:  { (water coffee) }
      Weight:  2.44
      Candidate Inferences:  { }
\end{verbatim}
\end{small}

  \caption{Complete SME interpretation of Water Flow - Heat Flow Analogy.}
  \label{fg:wfhf-done}\index{algorithm}\index{global mapping}
\border
\end{figure}

\subsubsection{Step 4: Compute Structural Evaluation Scores (run-rules)}\index{global mapping, scoring}\index{algorithm}

Typically a particular pair of base and target will give rise to several
Gmaps, each representing a different interpretation of the match.  Often it
is desired to select only a single Gmap,\index{global mapping} for example
to represent the best interpretation of an analogy. Many of these evaluation
criteria (including validity, usefulness, and so forth) lie outside the
province of Structure-Mapping, and rely heavily on the domain and
application.  However, one important component of evaluation is {\em
structural} --- for example, one Gmap may be considered a better analogy
than another if it embodies a more systematic match.  \SME\ provides a
programmable mechanism for computing a {\em structural evaluation score}
(SES) for each Gmap.  This score can be used to rank-order the Gmaps in
selecting the ``best'' analogy, or as a factor in a more complex (but
external) evaluation procedure.  In \SMET, the structural evaluation score is
currently computed by simply adding the belief of each local match
hypothesis to the belief of the Gmaps it is a member of.

Returning to Figure~\ref{fg:wfhf-done}, note that the ``strongest''
interpretation (i.e., the one which has the highest structural evaluation
score) is the one we would intuitively expect.  In other words, {\tt beaker}
maps to {\tt coffee}, {\tt vial} maps to {\tt ice-cube}, {\tt water} maps to
{\tt heat}, {\tt pipe} maps to {\tt bar}, and {\tt PRESSURE} maps to {\tt
TEMPERATURE}.  Furthermore, we have the candidate inference that the
temperature difference is what causes the flow.\index{global mapping}

\subsection{Adding Theoretical Constraints}

Given the general program, we may then add theoretical constraints in the
form of rules. For instance, the example just presented used the {\it
literal similarity} rules of structure-mapping theory. These rules augment
\SME's one-to-one mapping and structural consistency criteria with two
additional restrictions. First, evidence is computed according to {\it
systematicity}, that is, highly interconnected systems of relations are
preferred over independent facts. Second, only identical relations are
allowed to match (i.e., {\tt CAUSE} is not allowed to match {\tt
GREATER-THAN}).  Had another set of rules been used, the results might have
been substantially different.  For example, the {\it mere appearance} rules
of structure-mapping theory would have determined that the {\tt water} to
{\tt coffee} mapping was the best, due to their superficial similarity.


\section{Declarations}

The descriptions given to \SME\ are constructed from a user-defined
vocabulary of entities and predicates. This section discusses the
conventions for defining languages for \SME's use.

\subsection{Declaring Predicates}

\begin{tt}\index{defPredicate macro}\index{predicate}\index{predicate, declaring}
\begin{tabbing}
defPredicate \= {\it name argument-declarations predicate-class} \` [{\it Macro}] \\
             \> \&key \= :expression-type {\it logical-type}    \\
             \>      \> :commutative?  \{t | nil\}              \\
             \>      \> :n-ary?  \{t | nil\}                    \\
             \>      \> :documentation {\it descriptive-string} \\
             \>      \> :eval {\it procedural-attachment}\ \ )
\end{tabbing}
\end{tt}
\decfunc{define-predicate}{name argument-declarations predicate-class ...}

\spc{1}
\noindent
{\it predicate-class} is either {\tt function}, {\tt attribute}, or
{\tt relation}, according to what kind of predicate {\it name} is.
The {\it argument-declarations} allows the arguments to be named and
typed.  For example, the declaration:

\begin{verbatim}
   (defPredicate CAUSE ((antecedent event) (consequent event)) relation)
\end{verbatim}
states that {\tt CAUSE} is a two-place relational predicate.  Its arguments
are called {\tt antecedent} and {\tt consequent}, both of type {\tt event}.
The names and types of arguments are for the convenience of the
representation builder and any external routines (including the match rules),
and are not currently used by \SME\ internally.  Likewise, the predicate
class may be very important to the theoretical constraints imposed in the
rules, but is ignored by \SME\ internally.

The optional declaration {\tt :expression-type} indicates the logical type
of an expression headed by the given predicate. For example, the predicate
{\tt throw} may represent a kind of {\tt action}, while the predicate {\tt
mass} may represent an {\tt extensive-quantity}.

The optional declarations {\tt :commutative?} and {\tt :n-ary?} provide
\SME\ with important syntactic information.  {\tt :commutative?} indicates
that the predicate is commutative, and thus the order of arguments is
unimportant when matching.  {\tt :n-ary?} indicates that the predicate can
take any number of arguments.  Examples of commutative nary predicates include
{\tt AND}, {\tt SUM}, and {\tt SET}.

The {\tt :documentation} option allows one to attach a descriptive string to
a predicate. This documentation may then be accessed through the lisp
machine supplied interface\index{lisp machine interface, predicate documentation}
({\tt C-Shift-D}) or some externally written routine. If no documentation is
supplied, the list of argument names is used.  Another option provided
strictly for potential user routines is the optional {\tt :eval} parameter.
This allows one to declare a procedural attachment for a predicate.


\subsection{Declaring Entities}

\indent
\decmacro{defEntity}{name {\tt \&key type constant?}}\index{entity, declaring}
\decfunc{define-Entity}{name {\tt \&key type constant?}}

\spc{0.5}
Entities are logical individuals, i.e., the objects and constants of a
domain. Typical entities include physical objects, their temperature, and
the substance they are made of.  Primitive entities are declared with the
{\tt defEntity} form (a non-primitive entity would be {\tt (temperature
sun)}, which is a functional form representing a particular numeric
temperature entity). Primitive entities declared in this way represent
global entity types, that is, they represent a class of entities rather than
an actual instance of an entity. When an entity type is actually used in a
domain description, a unique entity instance is created for that type (e.g.,
{\tt Mary} is translated to {\tt Mary43}).

Since the language is typed, each entity type can be declared as
a subtype of an existing type using the {\tt :type} option.
For example, we might have
\begin{verbatim}
   (defEntity star :type inanimate)
   (defEntity Sun :type star)
\end{verbatim}
to say that stars are inanimate objects, and our Sun is a particular star.
Constants are declared by using the {\tt :constant?} option, as in
\begin{verbatim}
   (defEntity zero :type number :constant? t)
\end{verbatim}


\subsection{Declaring Description Groups}\label{ss:new-dgroups}
\index{description group}

\begin{tt}
\begin{tabbing}
defDescription {\it description-name}  \` [{\it Macro}]   \\
\ \ \ \= expressions \= \kill
      \> entities  \> ({\it entity$_{1}$}, {\it entity$_{2}$},$\ldots$,{\it entity$_{i}$}) \\ 
      \> expressions \> ({\it expression-declarations})
\end{tabbing}
\end{tt}
\decfunc{define-description}{description-name entities expressions}

\spc{0.5}
For simplicity, predicate instances and compound terms are called {\it
expressions}. A {\em Description Group}, or {\em Dgroup}\index{Dgroup}, is a
collection of primitive entities and expressions concerning them.  Dgroups
are defined with the {\tt defDescription}\index{defDescription macro} form,
where {\it expression-declarations} take the form {\tt
\begin{tabbing}
\ \   \= {\it expression} {\rm or} \\
      \> ({\it expression} :name {\it expression-name})
\end{tabbing}
}
\noindent
For example, the description of water flow depicted in
Figure~\ref{fg:wfhf-calculus} was given to \SME\ as
\begin{verbatim}
  (defDescription simple-water-flow
        entities (water beaker vial pipe)
        expressions (((flow beaker vial water pipe) :name wflow)
                     ((pressure beaker) :name pressure-beaker)
                     ((pressure vial) :name pressure-vial)
                     ((greater pressure-beaker pressure-vial) :name >pressure)
                     ((greater (diameter beaker) (diameter vial)) :name >diameter)
                     ((cause >pressure wflow) :name cause-flow)
                     (flat-top water)
                     (liquid water)))
\end{verbatim}
All entities must have been previously defined and every entity referred to
in the Dgroup's expressions must appear in the {\tt entities} list of the {\tt
defDescription}.\index{description group}

\subsection{Adding new expressions}\label{ss:new-expressions}

\indent
\decfunc{expression}{form dgroup-name {\tt \&key expression-name update-structure?}}\index{expression, adding new expressions}

\spc{0.7}
Expressions are normally defined as a side effect of creating a description group
(Dgroup). However, the facility is provided for dynamically adding new expressions
to a Dgroup. The syntax is essentially the same as for expressions declared within
a {\tt defDescription}. The expression's {\it form} may refer to the names of
existing Dgroup expressions and the form may be given a name. When {\tt expression} is
used to add expressions to an existing Dgroup, the {\tt update-structure?} keyword
must be invoked with a non-nil (e.g., {\tt T}) value. This keyword indicates
that the Dgroup's structure must be reexamined, since the known structural
roots will change as a result of this new expression.

\subsection{Typed Logic}

A mechanism exists for attaching types to predicates and their arguments
(see {\tt defPredicate}).  This facility is designed to constrain the
operation of \SME, particularly candidate inference generation. However, it
has not been extensively used to date.  The ability to attach types may be
useful for consistency checking by external systems.


\section{Using the rule system}\label{ss:rule-system}\index{rule system}

The rule system is the heart of \SME's flexibility. It allows one to specify
what types of things might match and how strongly these matches should be
believed. This section describes the required syntax for a rule set and
different strategies for rule specification.

\subsection{Rule file syntax}\index{rule system, rule file syntax}\index{rule system}

A rule set, or rule file, consists of a {\it declaration}, a set of
{\it match constructor} rules, and a set of {\it match evidence} rules.
In order to describe each, we will examine the syntax and functionality of
each part of the {\it smt-analogy} rule file.

\subsubsection{File declarations}\index{rule system}

\indent
\decfunc{sme-rules-file}{identification-string}

\spc{0.7}
Each rule file must begin with the initialization command {\tt
sme-rules-file}. This function clears the rule system in preparation for a
new set of rules (rules are cached for efficiency) and stores the name of
the rule set for output identification purposes. For example, our sample
rules file begins with:\footnote{Notice the file extension *.rules. While
rule files are not required to end in ``.rules'', all user interface
facilities for simplifying the loading of rule files depend upon this
extension. Another useful point is that rule files are typically defined to
be in the \SME\ package to avoid having to use {\tt sme:} throughout the
rule set.}  

\begin{verbatim}
   (sme-rules-file "smt-analogy.rules")
\end{verbatim}

\noindent
The rule file must then end with the {\tt tre-save-rules} command:

\spc{0.7}
\decfunc{tre-save-rules}{}

\subsubsection{Match Constructor rules}\index{rule system, match constructor rules}\index{rule system}

\index{match hypothesis, installing}
\index{match constructor rules}
\index{MHC-rule}
\begin{tt}
\begin{tabbing}
MHC-rule \= ({\it trigger} \= {\it ?base-variable} {\it ?target-variable} \` [{\it Macro}] \\
         \>                \> [:test {\it test-form}]) \\
\ \ \  {\it body}
\end{tabbing}
\end{tt}
\decfunc{install-MH}{base-item target-item}\index{match hypothesis, installing}

\spc{0.7}
\SME\ begins by finding for each entity and predicate in the base the set of
entities or predicates in the target that could plausibly match that item.
Plausibility is determined by {\it match constructor} rules, which are
responsible for installing all match hypotheses processed by \SME.  There
are two types of constructor rules, each indicated by a different value for
{\it trigger}.  The first type of rule is indicated by a {\tt :filter}
trigger.  These rules are applied to each pair of base and target expressions,
executing the code in {\it body}.  If the {\tt :test} option is used,
{\it test-form} must return true for the body to be run.  For example, the
following rule states that an expression in the base may match an expression
in the target whose functor is identical, unless they are attributes
(a structure-mapping {\it analogy} criterion):

\begin{small}
\begin{verbatim}
   (MHC-rule (:filter ?b ?t :test (and (eq (expression-functor ?b)
                                           (expression-functor ?t))
                                       (not (attribute? (expression-functor ?b)))))
          (install-MH ?b ?t))
\end{verbatim}
\end{small}

The second type of MHC rule is indicated by a trigger of {\tt :intern}.
These rules are run on each match hypothesis as it is created. Typically
they create match hypotheses between any functions or entities that are the
arguments of the expressions joined by the match hypothesis that triggered
the rule. The following is one of two that appear in smt-analogy.rules:

\begin{small}
\begin{verbatim}
   (MHC-rule (:intern ?b ?t :test (and (expression? ?b) (expression? ?t)
                                       (not (commutative? (expression-functor ?b)))
                                       (not (commutative? (expression-functor ?t)))))
       (do ((bchildren (expression-arguments ?b) (cdr bchildren))
            (tchildren (expression-arguments ?t) (cdr tchildren)))
           ((or (null bchildren) (null tchildren)))
         (cond ((and (entity? (first bchildren)) (entity? (first tchildren)))
                (install-MH (first bchildren) (first tchildren)))
               ((and (function? (expression-functor (first bchildren)))
                     (function? (expression-functor (first tchildren))))
                (install-MH (first bchildren) (first tchildren)))
               ((and (attribute? (expression-functor (first bchildren)))
                     (eq (expression-functor (first bchildren))
                         (expression-functor (first tchildren))))
                (install-MH (first bchildren) (first tchildren))))))
\end{verbatim}
\end{small}

Notice that the third test allows identical attributes to match, whereas the
previous MHC rule did not allow such matches. This design does not allow
isolated attributes to match, but recognizes that attributes appearing in a
larger overall structure should be matched.


\subsubsection{Match Evidence rules}\label{ss:ME-rules}\index{match evidence rules}\index{rule system, match evidence rules}\index{rule system}

\indent
\decmacro{rule}{nested-triggers body}\index{match hypothesis, scoring}
\decmacro{rassert!}{expression\ {\tt \&optional (}belief+ {\tt 1.0) (}belief- {\tt 0.0)}}
\decfunc{assert!}{expression\ {\tt \&optional (}belief+ {\tt 1.0) (}belief- {\tt 0.0)}}
\decmacro{initial-assertion}{assertion-form}

\spc{0.8}
The structural evaluation score is computed in two phases. First, each match
hypothesis is assigned some local degree of evidence, independently of what
Gmaps it belongs to.  Second, the score for each Gmap is computed based on
the evidence for its match hypotheses.  The management of evidence rules is
performed by the {\it Belief Maintenance System} (BMS) \cite{BMS}. A BMS is
a form of Truth-Maintenance system, extended to handle numerical weights for
evidence and degree of belief (see \cite{BMS} for a description of what the
weights mean). Pattern-directed rules are provided that trigger on certain
events in the knowledge base. 

The following is a simple rule for giving evidence to match hypotheses
between expressions that have the same predicate:

\begin{small}
\begin{verbatim}
   (initial-assertion (assert! 'same-functor))

   (rule ((:intern (MH ?b ?t) :test (and (expression? ?b) (expression? ?t)
                                         (eq (expression-functor ?b)
                                             (expression-functor ?t)))))
      (if (function? (expression-functor ?b))
          (rassert! (implies same-functor (MH ?b ?t) (0.2 . 0.0)))
          (rassert! (implies same-functor (MH ?b ?t) (0.5 . 0.0)))))
\end{verbatim}
\end{small}

There are two things to notice here in addition to the evidence rule. First,
the proposition {\tt same-functor} was asserted to be true (a belief of
1.0) and then used as the antecedent for the implication of evidence. In
this way, the source of this particular piece of evidence is identified and
is available for inspection. Second, the assertion of {\tt same-functor} was
placed inside the {\tt initial-assertion} form. Since \SME\ caches the
current rule file, it must be told if there are any functions embedded in
the rule file that must be invoked each time \SME\ is initialized.

\spc{0.8}
\decfunc{children-of?}{base-child target-child base-expression target-expression}

\spc{0.3}
Nested triggers within an evidence rule may be used to locate
interdependencies between different match hypotheses. For example,
structure-mapping's systematicity principle is implemented in a local
fashion by propagating evidence from a match hypothesis to its
children:\footnote{A number of functions (e.g., {\tt children-of?}) are
provided to simplify the writing of rules. These appear in the file {\tt
match-rules-support.lisp}.}\index{match hypothesis, scoring}

\begin{small}
\begin{verbatim}
   (rule ((:intern (MH ?b1 ?t1) :test (and (expression? ?b1) (expression? ?t1)
                                           (not (commutative? (expression-functor ?b1)))))
          (:intern (MH ?b2 ?t2) :test (children-of? ?b2 ?t2 ?b1 ?t1)))
       (rassert! (implies (MH ?b1 ?t1) (MH ?b2 ?t2) (0.8 . 0.0))))
\end{verbatim}
\end{small}

\noindent
Evidence for a Gmap is given by:

\begin{small}
\begin{verbatim}
   (rule ((:intern (GMAP ?gm) :var ?the-group))
      (dolist (mh (gm-elements ?gm))
         (assert! `(implies ,(mh-form mh) ,?the-group))))
\end{verbatim}
\end{small}

\index{global mapping, scoring}\index{rule system}

The {\tt BMS} allows a set of nodes to be declared special and will treat
evidence to these nodes differently. An {\tt additive-nodes} function is
provided which takes a set of {\tt BMS} nodes and modifies them so that
their evidence is added rather than normalized using Dempster's rule. \SME\
automatically invokes {\tt additive-nodes} on the derived set of Gmaps once
they are created. Thus, when the above Gmap rule is executed and the {\tt
implies} statement is used to supply evidence from each match hypothesis to
the Gmap, that evidence is simply automatically added to the total Gmap
evidence rather than propagated using Dempster's probabilistic sum.

The following destructive rule is often used instead of the previous one to
give a significant speed up:

\begin{small}
\begin{verbatim}
   (rule ((:intern (GMAP ?gm) :var ?the-group))
       (setf (node-belief+ (gm-bms-node ?gm)) 0)
       (dolist (mh (gm-elements ?gm))
          (incf (node-belief+ (gm-bms-node ?gm))
                (node-belief+ (mh-bms-node mh)))))
\end{verbatim}
\end{small}

This rule bypasses the BMS entirely, thus increasing speed by not creating
justification links. It also renders the {\tt additive-nodes} distinction
irrelevant. However, such rules must be used with extreme caution.  For
example, the source of a Gmap's evidence cannot be inspected when using this
type of operation (see Section~\ref{ss:inspecting-evidence}).

\subsection{Making SME simulate structure-mapping theory}\index{rule system}\index{rule system, simulating structure-mapping theory}

The previous section examined the general structure of an \SME\ rule file.
In the process, the basic elements of the {\it structure-mapping-theory
analogy} rule set were presented. The {\it literal similarity} and
{\it mere-appearance} rules are essentially the same as the {\it analogy}
rules. They differ in the first match constructor rule. The {\it analogy}
rule set has the test {\tt (not (attribute? (expression-functor ?b)))} which is
absent from the corresponding {\it literal similarity} rule. Conversely, the
corresponding {\it mere appearance} rule forces the opposite condition
{\tt (attribute? (expression-functor ?b))}. One should consult Appendix A of
\cite{sme-aij} for listings of all three structure-mapping rule sets.

\subsection{Making SME perform as SPROUTER}\index{rule system}\index{rule system, simulating SPROUTER}

The {\tt SPROUTER} program \cite{hayes-roth} was
developed as an approach to the problem of inductively forming
characteristic concept descriptions. That is, given a sequence of events
(e.g., a list of pictures), produce a single, conjunctive description which
represents a generalized, characteristic description of the sequence. {\tt
SPROUTER} generalized a sequence of {\it N} descriptions by finding the
commonalities between the first two descriptions, generalizing these common
elements (i.e., variablize the literals), and then repeating the process
using this generalized description and the next, unprocessed description.
These steps would be repeated until the generalization had propagated
through the entire list of input descriptions.

\SME\ may be used to implement {\tt SPROUTER}'s {\it interference matching}
technique by giving it a set of match constructor rules which require all
matching predicates to have the same name (i.e., the {\it literal
similarity} rules without the condition that allows functions with different
names to match). The {\tt SPROUTER} generalization\index{generalize}
mechanism may then be implemented with the following algorithm:

\begin{small}
\begin{verbatim}
  Procedure SPROUTER (event-list)
     begin
       generalization := pop(event-list)
       while event-list
          pairwise-match := match(pop(event-list), generalization)
          generalization := generalize(pairwise-match)
       return generalization
     end
\end{verbatim}
\end{small}


\subsection{Relaxing the identical predicates constraint}\index{rule system}\index{rule system, relaxing identical predicates}

The current structure-mapping theory rules are sensitive to representation
by requiring that relational predicates match only if they are identical.
This is an important restriction that ensures the structures being compared
are semantically similar. However, it can also be overly restrictive.  We
are currently exploring different methods to relax the identicality
requirement while still maintaining a strong sense of semantic similarity.
One approach, called the {\it minimal ascension principle}, allows relations
to match if they share a common ancestor in a multi-root is-a hierarchy of
expression types \cite{Falken-phd} (i.e., the identicality test in the match
constructor rule is replaced by a call to {\tt
predicate-type-intersection?}). The local evidence score for their match is
inversely proportional (exponentially) to the relations' distance in the
hierarchy.  This enables \SME\ to match non-identical relations if such a
match is supported by the surrounding structure, while still maintaining a
strong preference for matching semantically close relations. This is similar
to approaches used in \cite{Burstein83,Greiner,Winston80}.

Problems with an unconstrained minimal ascension match technique are
discussed in \cite{Falken-phd}. A mapping approach which considers the
current context when determining pairwise similarity is also discussed.

\subsection{Pure isomorphisms}\label{ss:isomorphisms}
\index{rule system}\index{rule system, pure isomorphisms}

While it is important to assure that the structures being compared are
semantically similar, one can in principle remove all semantic comparisons.
This would allow match creation to be guided strictly by \SME's structural
consistency and 1-1 mapping criteria and match selection to be based
strictly on systematicity.

\begin{figure}
\border
\begin{small}
\begin{tt}
\begin{tabbing}
\ \ \ \ \ \ {\rm Addition}\ \ \ \ \ \ \ \ \ \ \ \ \ \ \ \ \ \ \ \ \ \ \ \ \ \ \ \ \ \ \ \ {\rm Union} \\
\ \ \ \= N1 + N2 $=$ N2 + N1\ \ \ \ \ \ \ \ \ \ \ \ \ \ \ \ \ \ \ \ \ \= S1 $\cup$ S2 $\equiv$ S2 $\cup$ S1 \\
\> N3+(N4+N5) $=$ (N3+N4)+N5 \> S3 $\cup$ [S4 $\cup$ S5] $\equiv$ [S3 $\cup$ S4] $\cup$ S5 \\
\> N6+0 $=$ N6            \> S6 $\cup$ $\emptyset$ $\equiv$ S6
\end{tabbing}
\end{tt}
\end{small}
  \caption{Formal descriptions for addition and union.}
  \label{fg:addition-union}
\border
\end{figure}


Consider the isomorphic mapping between the formal definitions of numeric
addition and set union shown in
Figure~\ref{fg:addition-union}.\footnote{This example is taken from an
advance copy of a paper by Holyoak and Thagard \cite{ACME}. I include it
here simply to demonstrate the range of matching preferences available in
\SME.} These formal descriptions may be given to \SME\ in the standard
manner, as in {\tt (plus N3 (plus N4 N5))} for the left side of the
associativity rule (the representation {\tt (plus N4 N5 result45)} also
works, although it results in a slightly longer run time due to the
flattening of structure).  When presented as formal definitions, the
concepts of addition and union are structurally isomorphic, independent of
the meaning of the predicates. Thus, while it could be argued that the
predicates {\tt plus} and {\tt union} share a certain degree of semantic
overlap, this example demonstrates that it is possible to make \SME\ ignore
predicates entirely and simply look for isomorphic mappings.  The rule set
for isomorphic mappings is shown in Figure~\ref{fg:isomorphism}. (This is
called the {\it ACME} rule set, as it configures \SME\ to emulate the {\tt
ACME} program on this example \cite{ACME}). The only constraint this rule
set enforces is that each predicate has the same number of arguments. While
it includes the Structure-Mapping notion of systematicity to prefer systems
of relations, it does not enforce identicality of predicates.  Using this
rule set, \SME\ produces the unique best mapping that we would expect
between the formal definitions of addition and union.

Since \SME\ enforces the ``same number of arguments'' restriction by
defeating any match hypotheses that are not structurally sound, we could in
principle effectively remove the rule file entirely. This could be done with
one match constructor rule to match everything with everything and one
evidence rule to measure systematicity. When this {\it free-for-all} rules
file was given to \SME, the same single best Gmap was produced, but at the
expense of increasing the run time from 13 seconds to 3.25 minutes.


\begin{figure}[t]
\border
\begin{small}
\begin{alltt}
   (MHC-rule (:filter ?b ?t :test (= (numargs (expression-functor ?b))
                                     (numargs (expression-functor ?t))))
       (install-MH ?b ?t))

   {\it ;;; Intern rule to match entities (non-commutative predicates)}
   (MHC-rule (:intern ?b ?t :test (and (expression? ?b) (expression? ?t)))
       (do ((bchildren (expression-arguments ?b) (cdr bchildren))
            (tchildren (expression-arguments ?t) (cdr tchildren)))
           ((or (null bchildren) (null tchildren)))
         (if (and (entity? (first bchildren)) (entity? (first tchildren)))
             (install-MH (first bchildren) (first tchildren)))))

   {\it ;;; Give a uniform initial priming to each MH}
   (initial-assertion (assert! 'initial-priming))

   (rule ((:intern (MH ?b ?t)))
      (rassert! (implies initial-priming (MH ?b ?t) (0.2 . 0.0))))

   {\it ;;;propagate interconnections - systematicity}
   (rule ((:intern (MH ?b1 ?t1) :test (and (expression? ?b1) (expression? ?t1)))
          (:intern (MH ?b2 ?t2) :test (children-of? ?b2 ?t2 ?b1 ?t1)))
      (rassert! (implies (MH ?b1 ?t1) (MH ?b2 ?t2) (0.8 . 0.0))))

   {\it ;;; Support from its MH's}
   (rule ((:intern (GMAP ?gm) :var ?the-group))
      (dolist (mh (gm-elements ?gm))
         (assert! `(implies ,(mh-form mh) ,?the-group))))
\end{alltt}
\end{small}
  \caption{Rule set for forming general isomorphic mappings.}
  \label{fg:isomorphism}\index{rule system}
\border
\end{figure}



\subsection{Imposing externally established pairings}\index{rule system}\index{rule system, analogical hints}

In certain situations, a number of entity and predicate mappings may already
be known prior to invoking \SME. These mappings may have been provided as an
analogical hint\index{analogical hint} from an instructor or derived by the
application program during earlier processing.  For example, \Phineas\
\cite{VBAL,MLW} uses \SME\ to analogically relate observed physical
phenomena to known theories of the world. \Phineas\ uses two analogical
mappings to learn about a new physical process. First, behavioral
correspondences are established (i.e., what entities and quantities are
behaving in the same manner). Second, the relevant base theories are
analogically mapped into the new domain, guided by the behavioral
correspondences. The two-stage mapping process solves the problem of using
analogy in cases where one does not have a pre-existing theory, as occurs
with truely novel learning. The assumption made in \Phineas\ is that similar
behaviors will have similar theoretical explanations. The first mapping
provides the correspondences between entities and functions required to
guide the importation of an old theory to explain a new domain in the second
mapping.

\SME\ includes facilities to simplify writing \Phineas-like programs, by
enabling the results of earlier processing to constrain subsequent mapping
tasks. These routines are divided into two categories, {\it declaration} and
{\it test}. The declaration routines tell \SME\ what predicate and entity
correspondences are known {\it a-priori}. The test routines enable the match
constructor rules to adhere to these imposed constraints. Known mappings are
declared through the following functions:

\spc{0.6}
\noindent
\begin{minipage}{\textwidth}
\begin{tt}
\begin{tabbing}
defGiven-Mappings  \` [{\it Macro}]   \\
\ \ \ \= entities \ \ \= (\= ({\it base-entity$_{1}$ target-entity$_{1}$})   \\
      \>              \>  \> ({\it base-entity$_{i}$ target-entity$_{j}$}) ...)  \\
      \> predicates   \> (({\it base-predicate$_{1}$ target-predicate$_{1}$})   \\
      \>              \>  \> ({\it base-predicate$_{i}$ target-predicate$_{j}$}) ...)
\end{tabbing}
\end{tt}
\end{minipage}
\decfunc{declare-given-mappings}{entities predicates}
\decfunc{clear-given-mappings}{}

\spc{0.8}
Both {\tt defGiven-Mappings} and {\tt declare-given-mappings} have identical
functionality. The first does not evaluate its arguments while the second
one does. Disjunctive constraints may be imposed by including all of the
possible pairings (e.g., defining both {\tt (}{\it base-entity$_{i}$
target-entity$_{j}$}{\tt )} and {\tt (}{\it base-entity$_{i}$
target-entity$_{k}$}{\tt )}).

Once a set of given mappings has been declared, the following test routines
may be used within the match constructor rules to enforce these mappings:

\spc{0.8}
\decfunc{sanctioned-pairing?}{base-item target-item}
\decfunc{paired-item?}{{\tt \&key base-item target-item}}

\spc{0.4}
\noindent
{\tt sanctioned-pairing?} tests if the given pair is one of the {\it a-prior}
pairings. {\tt paired-item?} takes either a base item or a target item and returns
true if the mapping for that item has been externally determined.

These functions help in writing rules which respect established mappings.
For example, the following two rules are used in the \Phineas\ system to
allow observed behavioral correspondences to constrain the mapping of the
relevant theory:

\begin{small}
\begin{verbatim}
(MHC-rule (:filter ?b ?t :test (and (eq (expression-functor ?b) (expression-functor ?t))
                                    (not (paired-item? :base-item (expression-functor ?b)))
                                    (not (paired-item? :target-item (expression-functor ?t)))))
    (install-MH ?b ?t))

(MHC-rule (:filter ?b ?t :test (sanctioned-pairing? (expression-functor ?b)
                                                    (expression-functor ?t)))
    (install-MH ?b ?t))
\end{verbatim}
\end{small}

When an analogy is being made between two behaviors, {\tt clear-given-mappings}
is used to make \SME\ perform in normal analogy mode. The discovered entity and
function correspondences are then given to {\tt declare-given-mappings} prior to
using \SME\ to map the relevant theory.\index{rule system}


\section{Representation Issues}\index{representation issues}

The proper representation becomes an issue in \SME\ due to its significant
impact on speed performance. Hierarchical representations provide an
important source of constraint on generating potential matches. They tend to
make the semantic interrelations explicit in the structure of the syntax.
For example, Section~\ref{ss:isomorphisms} described a comparison between
the laws of addition and union. There it was noted that part of the additive
associativity rule may be represented as {\tt (plus N3 (plus N4 N5))} or as
the pair {\tt (plus N4 N5 result45)} and {\tt (plus N3 result45
result3-45)}. The latter ``flat'' representation takes more time for \SME\
to process, sometimes a significant difference for large domain
descriptions. This is because the functional representation makes the
associativity rule structurally explicit, while the flat representation
buries it among the tokens appearing as arguments to {\tt plus}. However, it
is important to note that \SME\ is able to process domain descriptions in
any predicate-based format. It is simply speed considerations that render
standard, flat forms of representation undesirable.

Due to SME's ability to accept commutative, n-ary predicates, it is able to
match arbitrary sets (which must be of equal size at this time). This has
two consequences. First, the explicit use of sets becomes a viable form of
representation. Thus, a theory might be represented concisely as

\begin{verbatim}
   (Theory T1 (SET axiom-8 axiom-14 ...))
\end{verbatim}

\noindent
rather than as {\tt (Axiom-of T1 axiom-8)}, etc. Second, sets may be used to
add structure to descriptions. For example, the set representation for
theories results in greatly reduced run times compared to the non-set
representation.\footnote{The difference in speed is due to the operation of
merge step 2, which combines matches sharing a common base structure. The
set notation for theory {\tt T1} enables merge step 2 to know that matches
for {\tt axiom-8} and {\tt axiom-14} should be placed in the same gmap, thus
reducing the number of possibilities in merge step 3.} I am currently
investigating the use of a similar representation for temporal states, as in:

\begin{verbatim}
   (Situation S1 (SET (Increasing (Amount-of water1))
                      (Increasing (Pressure water1))
                      o o o ))
\end{verbatim}

\noindent
A \Phineas\ problem which took \SME\ 53 minutes (using {\tt (Increasing
(Amount-of (at water1 S1)))}) was reduced to 34 seconds using this more
structured representation.


\section{Using SME}

This section describes how to install \SME\ on your machine, load it, and
operate it.

\subsection{Installing SME}\index{installation}\label{ss:site-specifics}

\indent
\decvar{*sme-language-file*}\index{parameters, site specific}\index{site specific information}
\decvar{*sme-default-rules*}\index{site specific information, pathnames}
\decvar{*sme-rules-pathname*}
\decvar{*sme-dgroup-pathname*}

\spc{0.8}
To configure \SME\ to a particular site, a handful of variables storing
system directory information must be edited and set to the appropriate
values. These variables appear in {\tt config.lisp}, a separate file for
this purpose. Of primary importance are {\tt *sme-language-file*} and {\tt
*sme-default-rules*}. These are used by {\tt sme-init} to initialize the
language and rule systems. The two variables storing the rules and dgroup
pathnames are used by the user interface routines.

\spc{0.7}
\decvar{*the-lisp-package*}\index{packages,site specific}
\decvar{*the-user-package*}\index{site specific information, packages}

\spc{0.3}
In most Common Lisp implementations, one package exists for general user
definitions and another exists for the lisp implementation. It is important
to notify \SME\ what these are for the Common Lisp in use. For example, on a
Symbolics (version 6.2), the lisp package is called {\tt common-lisp} and
the user package is called {\tt cl-user}. These are the default settings.

\spc{0.7}
\decvar{*sme-system-pathname*}
\decvar{*sme-files*}

\spc{0.8}
These variables are used to automate compiling and loading. If the system is
being loaded on something other than a Symbolics or TI Explorer, the file
{\tt windowing} should not be included in the list {\tt *sme-files*}.
Otherwise, it should be left in the list of \SME\ files, which is the
default.

The \SME\ routines assume a set of naming conventions on domain description
and rule files.\index{conventions, file names} The names of files containing
domain descriptions ({\tt defDescription}) should end with a {\tt *.dgroup}
extension. The name of a file containing a rule set should end with the {\tt
*.rules} extension.


\subsection{Running SME}\index{running SME}\index{execution}

\indent
\decfunc{sme-init}{{\tt \&optional (}initialize-language? {\tt T) (}initialize-rules? {\tt T)}}

\spc{0.7}
This section gives a brief overview of the process of using \SME\ for
matching, generalization, and inspection tasks.


\begin{figure}
\border
\begin{footnotesize}
\begin{alltt}
{\bf\gthan\ (sme:sme-init)}
Initializing SME...
   Loading default language file: prof:>falken>sme>language
   Loading default rules file: prof:>falken>sme>literal-similarity.bin
Complete.
T
{\bf\gthan\ (load "prof:\gthan falken\gthan sme\gthan simple-water-flow.dgroup")}
Loading PROF:>falken>sme>simple-water-flow.dgroup into package USER
#P"PROF:>falken>sme>simple-water-flow.dgroup"
NIL
{\bf\gthan\ (load "prof:\gthan falken\gthan sme\gthan simple-heat-flow.dgroup")}
Loading PROF:>falken>sme>simple-heat-flow.dgroup into package USER
#P"PROF:>falken>sme>simple-heat-flow.dgroup"
NIL
{\bf\gthan\ (sme:match 'swater-flow 'sheat-flow T)}
              SME Version 2E
    Analogical Match from SWATER-FLOW to SHEAT-FLOW.

Rule File: literal-similarity.rules
--------------------------------------------------------------------------
                  | # Entities | # Expr. | Maximum order | Average order |
  Base Statistics |       4    |     11  |        3      |      1.36     |
Target Statistics |       4    |      6  |        2      |      1.17     |
--------------------------------------------------------------------------
# MH's | # Gmaps | Merge Step 3 | CI Generation | Show Best Only |
   14  |     3   |   ACTIVE     |    ACTIVE     |       OFF      |
------------------------------------------------------------------
Total Run Time:    0 Minutes,  0.821 Seconds
BMS Run Time:      0 Minutes,  0.530 Seconds
Best Gmaps: { 3 }

Match Hypotheses:
    (0.6320  0.0000)  (PIPE4 BAR7)
    (0.7900  0.0000)  (FLAT-WATER FLAT-COFFEE)
    {\it   o       o         ;a number of match hypotheses appeared here}
    (0.8646  0.0000)  (WATER1 COFFEE5)
    (0.7900  0.0000)  (LIQUID-WATER LIQUID-COFFEE)

Gmap #1:   (BEAKER2 COFFEE5)  (DIAM-BEAKER TEMP-COFFEE)  (VIAL3 ICE-CUBE6)
           (DIAM-VIAL TEMP-ICE-CUBE)  (>DIAMETER >TEMP)
  Emaps:  (BEAKER2 COFFEE5) (VIAL3 ICE-CUBE6)
  Weight: 3.937660
  Candidate Inferences:  { }

    {\it   o       o         ;Gmap #2 appeared here...}

Gmap #3:   (>PRESSURE >TEMP)  (PRESS-VIAL TEMP-ICE-CUBE)  (PRESS-BEAKER TEMP-COFFEE)
           (BEAKER2 COFFEE5)  (VIAL3 ICE-CUBE6)  (WATER1 HEAT8)
           (PIPE4 BAR7)  (WFLOW HFLOW)
  Emaps:  (BEAKER2 COFFEE5) (VIAL3 ICE-CUBE6) (WATER1 HEAT8) (PIPE4 BAR7)
  Weight: 5.991660
  Candidate Inferences:  (CAUSE >TEMP HFLOW)
\end{alltt}
\end{footnotesize}
  \caption{Initializing and running SME.}
  \label{fg:sme-trace}
\border
\end{figure}


\begin{enumerate}

\item {\it Loading the files.} To load or compile \SME, the file {\tt
config} should be loaded and then {\tt (load-sme)} or {\tt (compile-sme)}
called. A {\tt defSystem} definition is provided in {\tt system.lisp} for
Symbolics machines.

\item {\it System startup.} This stage is only appropriate for full, lisp
machine startup. The \SME\ window environment may be created with Select-S
on a Symbolics or System-S on a TI Explorer.

\item {\it Initialization.} The function {\tt sme-init} should be called to
initialize the database. If {\it initialize-language?} is non-nil, the
default language file (predicate definitions) will be loaded. If {\it
initialize-rules?} is non-nil, the default rules file will be loaded. Prior
to operating \SME, the language and rule systems must be established.

\item {\it Loading Dgroups.} Any description groups that are to be matched
must be declared. These declarations are typically stored in files, with the
extension {\tt *.dgroup}. If the windowing system is active, the command
{\tt Load Dgroup} will offer a menu of all {\tt *.dgroup} files in
{\tt *sme-dgroup-pathname*} to select what to load.

\item {\it Analogical mapping.} The function {\tt match} may be called to
form a mapping between two given Dgroups. This is discussed in
Section~\ref{ss:match}. If the windowing system is active, the command {\tt
Match} will offer a menu to select base and target Dgroups. It is prior to
this step that one might want to think about wether to modify any system
parameters (e.g., print the match hypotheses, print only the best Gmaps,
generate candidate inferences, etc.). 

\item {\it Describing Dgroups.} Once Dgroups are defined, the {\tt
describe-dgroup} facility will provide a description of any particular
Dgroup.

\item {\it Graphically displaying Dgroups.} If the windowing system is
active, Dgroups may also be displayed graphically, through the {\tt Display
Dgroup} utility.

\item {\it Generalizing.} Once a mapping is formed, it may be generalized
using the {\tt generalize} function or the {\tt Generalize} command in the
system menu.

\item {\it Saving the results of a session.} If the windowing system is
active, the results of commands like {\tt Match} and {\tt Generalize} are
sent to the scroll window by default. These results may be written to a file
using the {\tt dump-scroll} system utility.

\item {\it Comparing two apparently identical Gmaps.} When two Gmaps are
formed that appear to be identical, their differences can be identified
using the {\tt compare-gmaps} system utility.

\end{enumerate}

\noindent
A trace of \SME\ performing the basic mapping task is given in
Figure~\ref{fg:sme-trace}. Each of the other options are described in
greater detail in the following sections.


\subsection{Batch mode}
\index{batch mode}\index{report generation}

\indent
\decfunc{run-batch-file}{pathname {\tt \&key} {\tt (}gmap-display {\tt :all) (}gmap-statistics {\tt :none)}}
\decmacro{language-file}{pathname}
\decmacro{dgroup-directory}{pathname}
\decmacro{dgroup-file}{file-name}
\decmacro{rule-directory}{pathname}
\decmacro{rule-file}{file-name}
\decmacro{rule-sets}{{\tt \&rest} rule-file-names}
\decmacro{report-comments}{string}
\decmacro{send-report-to}{pathname {\tt \&key (}text-driver {\tt :LPR) (}style {\tt :STANDARD)}}
\decmacro{run-matcher-on}{base-name target-name}
\decmacro{defPostMatcher}{function}


\begin{figure}
\border
\begin{footnotesize}
\begin{alltt}
(sme:Dgroup-Directory "prof:>falken>sme>")

(sme:Dgroup-File "solar-system")
(sme:Dgroup-File "rutherford")
(sme:Dgroup-File "simple-water-flow")
(sme:Dgroup-File "simple-heat-flow")

(sme:Rule-Directory "prof:>falken>sme>")

(sme:Rule-Sets "literal-similarity" "true-analogy" "attribute-only")   {\it; iterate over each rule set}

(sme:Report-Comments "Sample run of SME to demonstrate batch mode.")

(sme:Send-Report-To "heath:>falken>sample.dmp" :text-driver :LATEX)

(sme:Run-Matcher-On solar-system rutherford-atom)                 {\it; map this pair once for each rule set}
(sme:Run-Matcher-On swater-flow sheat-flow)                       {\it; map this pair once for each rule set}
\end{alltt}
\end{footnotesize}
  \caption{Sample SME batch file.}
  \label{fg:batch-file}
\border
\end{figure}


\spc{0.8}
\SME\ is normally used as an interactive utility or as a module to some
larger program. However, when performing statistical analyses across a broad
space of matching preferences (i.e., rule sets) and domain descriptions, an
interactive format soon becomes inconvenient. Utilities are provided so that
a file of \SME\ instructions may be defined and then executed using {\tt
run-batch-file} (e.g., Figure~\ref{fg:batch-file}). This would instruct
\SME\ to perform a series of matches, potentially over a variety of rule
sets and domain descriptions, and generate a detailed report of the
execution and a summary of the results. When a single rule set is specified
using {\tt rule-file}, all subsequent matches (invoked by {\tt
run-matcher-on}) will use this rule file until another one is specified.
Using {\tt rule-sets}, one may instead specify a series of rule files to be
used, so that a single {\tt run-matcher-on} command will cause \SME\ to run
once for each rule file in the list. If a user-defined function name is
given to {\tt defPostMatcher}, this function will be called after each match
is performed, in case special post-match routines are desired or extra
information is to be added to the report being generated. A variety of text
drivers are supported for report generation ({\tt send-report-to}), such as
{\tt :lpr} (line printer), {\tt :latex}, and {\tt :troff}.


\subsection{Generalization mechanism}

\indent
\decfunc{generalize}{gmap}

\spc{0.8}
The {\tt generalize} function takes a global mapping structure and returns
three alternate generalizations (using the Common-Lisp {\tt values}
protocol), each one successively larger than the previous:


\begin{figure}
\border
\begin{small}
\begin{verbatim}
   Generalizations for Match from SWATER-FLOW to SHEAT-FLOW:

       Generalization #1 (Literally Common Aspects Only):
            (FLOW ENTITY6 ENTITY8 ENTITY13 ENTITY14)

       Generalization #2 (All Common Aspects Only):
            (FLOW ENTITY6 ENTITY8 ENTITY13 ENTITY14)
            (GREATER (FUNCTION0 ENTITY6) (FUNCTION0 ENTITY8))

       Generalization #3 (Maximal Generalization):
            (CAUSE (GREATER (FUNCTION0 ENTITY6) (FUNCTION0 ENTITY8))
                   (FLOW ENTITY6 ENTITY8 ENTITY13 ENTITY14))
\end{verbatim}
\end{small}
  \caption{\SME\ generalizations for the simple water flow -- heat flow analogy.}
  \label{fg:generalizations}
\border
\end{figure}


\begin{enumerate}
 \item {\it Literally common aspects only.} This generalization locates
those sub-structures which are identical in both base and target Dgroups.
This is a type of generalization typically found in inductive generalization
programs.
 \item {\it All common aspects only.} In addition to common, identical
substructures, this generalization includes cases where functions of a
different name were allowed to match. Where this occurs in the common
structure, a skolemized function predicate is created.
 \item {\it Maximal generalization.} The largest generalization (in terms of
amount) includes all candidate inferences sanctioned by the Gmap, as well as
the common substructure of generalization mode 2. This represents the entire
shared structure between the two Dgroups under the assumption that the
candidate inferences are valid.
\end{enumerate}

For example, given the best Gmap from the simple water flow -- heat flow
analogy described in Section 2.1 and shown in Figure~\ref{fg:wfhf-done},
\SME\ will produce the set of generalizations shown in
Figure~\ref{fg:generalizations}. The first generalization indicates that the
only thing in common between the two situations is the existence of flow. The
second generalization loosens the meaning of ``in common'' to include the
fact that a quantity associated with the source of flow was greater than the
same quantity measured for the destination. The final generalization assumes
that this inequality, which was the cause of flow in the water flow domain,
is actually the cause of flow for both situations.


\subsection{Inspecting MH and Gmap evidence}\label{ss:inspecting-evidence}

\indent
\decfunc{match-evidence-inspector}{}\index{match hypothesis, inspecting evidence justifications}\index{match hypothesis, inspecting evidence justifications}

\spc{0.8}
When developing a theory about what types of rules should be used and how
much evidence for a particular match they should provide, it is often useful
to explicitly see what the different sources of evidence were for a
particular match item. The system utility {\tt match-evidence-inspector} may
be used to display a trace of the entire evidence facility or just the
evidence for a particular match hypothesis or Gmap. For example, the
following information was printed out about the {\tt pressure} to {\tt
temperature} match hypothesis in the water flow -- heat flow analogy:

\begin{small}
\begin{verbatim}
   (MH F#PRESS-BEAKER F#TEMP-COFFEE) has evidence (0.7120, 0.0000) due to
      IMPLICATION((MH F#>PRESSURE F#>TEMP))  (0.5200, 0.0000)
      IMPLICATION(CHILDREN-POTENTIAL)  (0.4000, 0.0000)
\end{verbatim}
\end{small}
\noindent
While the following information appears for the best Gmap in this analogy:
\begin{small}
\begin{verbatim}
   (GMAP #GM3) has evidence (5.9917, 0.0000) due to
       IMPLICATION((MH F#WFLOW F#HFLOW))  (0.7900, 0.0000)
       IMPLICATION((MH I#PIPE20 I#BAR23))  (0.6320, 0.0000)
       IMPLICATION((MH I#WATER17 I#HEAT24))  (0.6320, 0.0000)
       IMPLICATION((MH I#VIAL19 I#ICE-CUBE22))  (0.9318, 0.0000)
       IMPLICATION((MH I#BEAKER18 I#COFFEE21))  (0.9318, 0.0000)
       IMPLICATION((MH F#PRESS-BEAKER F#TEMP-COFFEE))  (0.7120, 0.0000)
       IMPLICATION((MH F#PRESS-VIAL F#TEMP-ICE-CUBE))  (0.7120, 0.0000)
       IMPLICATION((MH F#>PRESSURE F#>TEMP))  (0.6500, 0.0000)
\end{verbatim}
\end{small}

\noindent
The inspection facility will not work for Gmaps if their scores were
produced by an external (to the {\tt BMS}), destructive operation. One such
destructive rule appeared at the end of Section~\ref{ss:ME-rules}.


\subsection{Windows}\index{lisp machine interface}\index{scroll windows}

\indent
\decfunc{dump-scroll-menu}{}\index{scroll windows, saving contents}
\decfunc{dump-scroll}{output-pathname}
\decfunc{clear-scroll}{}

\decfunc{select-windowing-configuration}{}
\decfunc{select-scroll}{}
\decfunc{select-double-scroll}{}
\decfunc{select-graphics}{}
\decfunc{select-large-graphics}{}
\decfunc{select-split}{}

\decvar{*sme-frame*}
\decvar{*graphics-pane*}
\decvar{*scroll-pane*}
\decvar{*spare-scroll-pane*}
\decvar{*lisp-pane*}

\spc{0.8} The windowing system is lisp machine dependent and appears in the
file {\tt windowing.lisp}. The loading of this file is optional. When the
windowing system is used, a number of window configurations are possible,
such as having a single scroll window, two side by side, a single graphics
window, or a scroll and graphics window side by side. These configurations
may be selected through their individual functions (e.g., {\tt
select-scroll}), or through the central configuration facility {\tt
select-scroll-graphics}.  By default, when the windowing system is active,
all \SME\ output is sent to the primary scroll pane.\index{scroll windows,
writing to} When both scroll windows are in use, the configuration facility
allows one to specify which scroll window is currently active. The two
scroll dumping routines write the contents of the primary scroll window to a
specified file. Output sent to the secondary ({\it scratchpad}) scroll pane
is for observation only and cannot be written to a file.


\subsection{System parameters}\index{parameters}

\indent
\decvar{*sme-parameters*}
\decvar{*parameter-menu-options*}
\decmacro{defSME-Parameter}{variable-name string-description type {\tt \&optional} type-choices}

\spc{0.8}
The {\tt defSME-Parameter} form adds a new variable to the list of known
\SME\ parameters. This list is used by the windowing interface routines to
query the user about possible parameter settings. It is provided primarily
for application programs wanting to use the standard \SME\ parameter setting
facility. The arguments to {\tt defSME-Parameter} correspond to the appropriate
definitions for the {\tt choose-variable-values} function of your particular
machine. For example, the following declaration appears in {\tt match.lisp}:
\begin{small}
\begin{verbatim}
   (defSME-Parameter *display-all-MH* "Display all the Match Hypotheses" :boolean)
\end{verbatim}
\end{small}


\begin{figure}
\border
\vspace{1.2in}
\caption{SME System Parameters.}
\label{fg:parameters}
\border
\end{figure}


\spc{0.6}
\decfunc{change-parms}{}

\spc{0.4}
The windowing system provides a menu facility for viewing and changing the
current values of system parameters. This menu is shown in
Figure~\ref{fg:parameters}.


\subsection{System utilities}\index{system utilities}

\indent
\decvar{*system-utilities-menu*}
\decmacro{defSME-Utility}{string-name lisp-form}
\decfunc{menu-utilities}{}
\decfunc{get-dgroup}{}
\decfunc{get-rules}{}

\spc{0.8}
The \SME\ system utilities are the options that appear when the {\tt
Utilities} command is evoked, the right mouse button is pressed, or the
function {\tt menu-utilities} is called. These utilities include changing
the system parameters, choosing to load a Dgroup or rule file from those in
the defined directories, and clearing or writting to file the contents of
the scroll window. These routines are lisp machine dependent. The following
declaration appears in {\tt match.lisp}:
\begin{small}
\begin{verbatim}
   (defSME-Utility "Inspect Evidence" (match-evidence-inspector))
\end{verbatim}
\end{small}


\section{User Hooks}

This section describes the global variables and routines that are available
to the user and application programs for the creation and inspection of
analogical mappings.

\subsection{Applications control over display}

\indent
\decvar{*sme-output-stream*}
\decvar{*windowing?*}
\decvar{*sme-graphics-output*}

\spc{0.8}
All \SME\ textual display routines send their output to {\tt
*sme-output-stream*}. By default, the value of this variable is {\tt T},
which causes output to be sent to {\tt *terminal-io*}, CommonLisp's default
pointer to the user's console. When {\tt *sme-output-stream*} is a scroll
window (determined by the presence of an {\tt :append-item} handler), the
appropriate scroll window routines for sending display items are invoked.
Otherwise, text is sent to the current output stream using {\tt format}. 
Text routines are machine independent.

In a similar manner, all \SME\ graphics output is sent to the current
{\tt *sme-graphics-output*} window. Graphics output is lisp machine
dependent and relies on the {\tt ZGRAPH} graphics system.

When the \SME\ windowing system is in operation, {\tt *sme-output-stream*}
is set to the primary \SME\ scroll pane ({\tt *scroll-pane*}), {\tt
*sme-graphics-output*} is set to the \SME\ graphics pane ({\tt
*graphics-pane*}), and {\tt windowing?} is set to {\tt T}.
\index{scroll windows, writing to} 

\spc{0.7}
\decmacro{sme-format}{format-string {\tt \&rest} format-args}
\decfunc{sme-print}{string}
\decfunc{sme-terpri}{{\tt \&optional (}N {\tt 1)}}

\spc{0.3}
These routines provide a general interface for sending textual output to the
current \SME\ output stream. {\tt sme-format} is equivalent to CommonLisp's
{\tt format} routine, except that the printed output is always followed by a
newline. The {\tt sme-print} routine is provided for simple situations where
only a string is printed or for situations requiring the standard use of
{\tt format}, as in building up a line of text through multiple invocations.
The printed output of {\tt sme-print} is followed by a newline. When the
routine is used for multiple calls of {\tt format}, it should be used in
conjunction with CommonLisp's {\tt with-output-to-string}, as in:

\begin{small}
\begin{verbatim}
   (sme-print
      (with-output-to-string (stream)
         (format stream "~%Beginning of a line...")
         (format stream "   middle of a line...")
         (format stream "   end of a line.")))
\end{verbatim}
\end{small}

When a whole set of operations are carried out within the context of a
single {\tt sme-print}, one must be careful not nest calls to {\tt
sme-print} (e.g., calling a function in the context of an {\tt sme-print}
which itself invokes {\tt sme-print}). Such nesting will cause output 
to appear backwards from what was intended and may cause the output stream
to close improperly.


\subsection{Useful miscellaneous functions}

Data exists within \SME\ in three forms: (1) local items such as entities,
predicates, and expressions, (2) description groups (Dgroups), and (3) analogical
mapping information. The routines to create and query these items are
described in the following sections.

\subsubsection{Entities, predicates, and expressions}

\indent
\decfunc{entity?}{item}
\decfunc{entity-type?}{item}
\decmacro{entity-name?}{symbol}\index{entity, inspection functions}
\decmacro{fetch-entity-definition}{symbol}
\decmacro{entity-domain}{symbol}
\decmacro{constant-entity?}{symbol}

\spc{0.8}
Entities declared through {\tt defEntity} represent global entity types,
that is, they represent a class of entities rather than an actual instance
of an entity. When an entity type is used in the definition of a description
group, a unique entity instance is created for that type (e.g., {\tt beaker}
is translated to {\tt beaker73}). Thus, a given entity token will represent
either a type or an instance. The structure predicate {\tt entity?} returns
true if the given item is an entity-instance structure, while {\tt
entity-type?} returns true if the item is an entity-type structure. The
macro {\tt entity-name?} returns true if the given symbol represents either
an entity type or instance. {\tt fetch-entity-definition} will return the
entity-type structure for an entity type token or the entity-instance
structure for an entity instance token. The routines {\tt entity-domain} and
{\tt constant-entity?} refer to the {\tt type} and {\tt constant?} keyword
values given to {\tt defEntity} when the corresponding entity type was
created.

\spc{1}
\decvar{*sme-predicates*}
\decmacro{fetch-predicate-definition}{predicate-symbol}\index{predicate, inspection functions}
\decmacro{predicate?}{symbol}
\decmacro{predicate-type}{predicate-symbol}
\decmacro{relation?}{predicate-symbol}
\decmacro{attribute?}{predicate-symbol}
\decmacro{function?}{predicate-symbol}
\decmacro{commutative?}{predicate-symbol}
\decmacro{n-ary?}{predicate-symbol}
\decmacro{arg-list}{predicate-symbol}
\decmacro{numargs}{predicate-symbol}
\decmacro{expression-type}{predicate-symbol}
\decmacro{eval-form}{predicate-symbol}

\spc{0.4}
These routines provide the facility to access the various predicate properties
defined with the {\tt defPredicate} form. {\tt fetch-predicate-definition}
returns the actual sme-predicate structure containing all the information
about a given predicate.

\spc{1}
\index{expression, inspection functions}
\decfunc{fetch-expression}{expression-name dgroup
                     {\tt \&optional (}error-if-absent? {\tt T)}}
\decfunc{expression-functor}{expression-structure}
\decfunc{fully-expand-expression}{expression-structure dgroup}
\decfunc{fully-expand-expression-form}{expression-form dgroup}

\spc{0.4}
An ``expression'' represents an actual predicate instance within a Dgroup.  Notice
that this includes terms corresponding to function applications as expressions.
Each use of a predicate gets its own expression with its own name, so that a
higher-order relation gets translated into several expressions, with some having
expressions as arguments. These routines allow one to retrieve and inspect expressions
in the database. {\tt fetch-expression} returns the expression structure with the given
name.

The routines {\tt fully-expand-expression} and {\tt fully-expand-expression-form} are
useful for examining the form of an expression. Typically, the expression {\tt
(greater-than (diameter beaker) 5)} is stored as the expression {\tt greater-than23},
which has the form {\tt (greater-than diameter24 5)}. These routines return
a fully expanded expression form, where all expression names are replaced by their
corresponding forms.


\subsubsection{Dgroups}

\indent
\decfunc{fetch-dgroup}{dgroup-name {\tt \&optional} create?}\index{description group, retrieval}
\decfunc{return-dgroup}{dgroup-or-dgroup-name}

\spc{0.8}
Description groups (Dgroups) are stored in a simple data base managed
primarily by routines in {\tt sme.lisp}. The general procedures for Dgroup
and expression creation were described in sections~\ref{ss:new-dgroups} and
\ref{ss:new-expressions}. A Dgroup may be retrieved by name using
{\tt fetch-dgroup}, or created if {\it create?} is non-nil and no Dgroup
with the given name currently exists. {\tt return-dgroup} is designed for
routines that may take either an actual Dgroup or simply a Dgroup name
(e.g., {\tt fetch-expression}). It will cause an error if the Dgroup does not
previously exist.

\spc{0.8}
\decfunc{describe-dgroup}{dgroup}\index{description group, textual description}
\decfunc{menu-display-dgroup}{}\index{description group, graphical display}
\decfunc{menu-display-pairs}{}

\spc{0.3}
A Dgroup may be textually described using {\tt describe-Dgroup}, which
writes to the \SME\ output stream. Graphical display is provided in the
windowing system by {\tt menu-display-dgroup} for a single Dgroup or {\tt
menu-display-pairs} for a display of two Dgroups side by side.


\subsubsection{Creating and inspecting global matches}\label{ss:match}

\indent
\decfunc{match}{base-name target-name {\tt \&optional} display?}\index{global mapping, creation}
\decfunc{best-gmaps}{{\tt \&optional (}gmaps {\tt *gmaps*) (}percentage-range {\tt 0.02)}}\index{global mapping, selecting the best}
\decfunc{display-match}{base target {\tt \&optional (}total-run-time {\tt 0) (}bms-run-time {\tt 0)}}\index{global mapping, textual display}

\spc{0.8}
The {\tt match} function is the central \SME\ procedure. Given the names of two
Dgroups, it forms the complete set of global mappings between them. If {\it display?}
is non-nil, a description of the results will be sent to the current \SME\
output stream. The function itself returns two values, the total run time of
the match process in seconds and the subset of that time spent running the
BMS evidence rules. The analogical mapping results are stored in the
following global variables, which are then accessible by the user or
application program.

\spc{0.5}
\decvar{*base*}
\decvar{*target*}
\decvar{*match-hypotheses*}\index{match hypothesis, *match-hypotheses*}
\decvar{*gmaps*}\index{global mapping, *gmaps*}
\spc{0.2}

\noindent
The Gmap(s) with the highest evaluation score are retrieved by
{\tt best-gmaps}, which returns all Gmaps having a score within a given
percentage (default is 2\%) of the highest score. {\tt best-gmaps} returns
two values: the list of best Gmaps and the actual real-valued highest score.

\spc{0.8}
\decfunc{compare-gmaps}{gmap1 gmap2 {\tt \&optional} display?}\index{global mapping, comparing apparently identical gmaps}

\spc{0.4}
Occasionally, \SME\ will produce two or more ``best'' Gmaps that appear to
be identical yet have been classified as distinct. When these Gmaps are
large, the ``here's the set of match hypotheses'' output format can make it
frustrating to find what the slight differences are between a pair of Gmaps.
When given two Gmap structures, {\tt compare-gmaps} will return (using {\tt
values}) the list of the match hypotheses that are uniquely part of the
first Gmap and a list of match hypotheses that are uniquely part of the
second Gmap (i.e., ({\it gmap1} - {\it gmap2}) and ({\it gmap2} - {\it
gmap1})). When the windowing system is active, this option is available
through the system utilities menu.


\section{Algorithm Internals}

This section quickly describes a few internal points of the program in case
one has specialized needs for interfacing to the code. It assumes knowledge
of CommonLisp and the realization that for many questions, the only feasible
answer must be to examine the \SME\ program.

\subsection{The Match Function}

\indent
\decfunc{create-match-hypotheses}{base-dgroup target-dgroup}
\decfunc{run-rules}{}\index{rule system, run-rules}
\decfunc{calculate-nogoods}{base-dgroup target-dgroup}
\decfunc{generate-gmaps}{}
\decfunc{gather-inferences}{base-dgroup target-dgroup}\index{candidate inference, gather-inferences}
\decfunc{intern-gmaps}{}\index{rule system, giving it access to gmaps}

\spc{0.8}
The function {\tt match} is primarily a sequence of calls to these
functions. {\tt create-match-hypotheses} runs the match constructor
rules to form the individual match hypotheses. The BMS evidence rules are
then run on these match hypotheses ({\tt run-rules}) and their
dependence and inconsistency relationships are determined ({\tt
calculate-nogoods}). The function {\tt generate-gmaps} executes the three
merge steps, resulting in the set of complete global mappings being placed
in the variable {\tt *gmaps*}. The candidate inferences each Gmap
sanctions is then calculated ({\tt gather-inferences}) and additive BMS
nodes for each Gmap are formed ({\tt intern-gmaps}), allowing the evidence
rules to run on each Gmap.

\subsection{Match Hypotheses}\index{match hypothesis, defstruct}

\makebox[\textwidth]{{\tt match-hypothesis} \hfill [{\it Defstruct}]}
\decsubst{mh-form}{match-hypothesis}
\decsubst{mh-base-item}{match-hypothesis}
\decsubst{mh-target-item}{match-hypothesis}
\decsubst{mh-bms-node}{match-hypothesis}
\decsubst{node-belief+}{bms-node}
\decsubst{mh-plist}{match-hypothesis}
\decvar{*match-hypotheses*}

\spc{0.8}
Most programs using \SME\ will need to interact with the {\it match
hypothesis structures}. Slots to this structure type use the {\tt mh-}
prefix. There are several slots that might be important. The MH form, which
is a list of {\tt (MH {\it $<$base-item$>$} {\it $<$target-item$>$})}, is found
using {\tt mh-form}. This is the form used for triggering the MH evidence
rules, and is asserted in the BMS for each match hypothesis. The base and
target items are expression or entity structures. The base item or target item may
be obtained directly using {\tt mh-base-item} and {\tt mh-target-item},
respectively.  The BMS node for each match hypothesis is found by {\tt
mh-bms-node}. In turn, the weight for that node may be obtained using {\tt
node-belief+}.  Finally, each match hypothesis structure has a property list
slot ({\tt mh-plist}) which may be useful for various purposes.

\subsection{Global Mappings}\index{global mapping, defstruct}

\makebox[\textwidth]{{\tt global-mapping} \hfill [{\it Defstruct}]}
\decsubst{gm-id}{global-mapping}
\decsubst{gm-elements}{global-mapping}
\decsubst{gm-emaps}{global-mapping}
\decsubst{gm-base}{global-mapping}
\decsubst{gm-target}{global-mapping}
\decsubst{gm-inferences}{global-mapping}
\decsubst{gm-bms-node}{global-mapping}
\decsubst{gm-plist}{global-mapping}
\decvar{*gmaps*}

\spc{0.8}
Each Gmap is stored as a {\it global-mapping structure}. Slots to this
structure type use the {\tt gm-} prefix. Each Gmap is assigned a unique
integer identifier, found through {\tt gm-id}. The Gmap form used by the BMS
is not explicitly available, but is asserted as {\tt (Gmap {\it
$<$gmap-structure$>$})}. The match hypotheses associated with a Gmap are stored
in {\tt gm-elements}, while the subset of these that are entity mappings is
stored in {\tt gm-emaps}. The candidate inferences sanctioned by the Gmap
appear in {\tt gm-inferences} and are stored as a simple list data type,
using the syntax defined in Section~\ref{ss:new-dgroups} for description
group expressions.

\subsection{Candidate Inference Generation}\index{candidate inference, alternate algorithms}

The original candidate inference generator, as described in \cite{aaai86},
will take any base structure ``intersecting the Gmap structure''. The newer
(V. 2) edition only takes base structure ``intersecting a Gmap root''.
Thus, the newer edition is more cautious and far more efficient than the
older edition. Both versions of the code are available (in {\tt
match.lisp}), with the default being the newer, more cautious version. There
are theoretical arguments for and against each approach. For example, one
might want to use only the inferences from the newer, more cautious approach
at first since they are supported by more target knowledge and thus more
likely valid. If an analogy proves fruitful, one may want to relax these
constraints, and use the older version to find out what additional
inferences might be made.

\subsection{Rule System}

\indent
\decfunc{tre-rules-file}{}\index{rule system, dynamically swapping rule sets}
\decfunc{tre-save-rules}{}
\decfunc{tre-init}{}
\decfunc{restore-rules}{}
\decfunc{run-rules}{}
\decvar{*tre-rules-saver*}
\decvar{*initial-assertions*}

\spc{0.8} 
When a new rules file is loaded, the sme-rules-file command at the top of
the file initializes the BMS rule system prior to loading the new set of
rules.  At the bottom of the rules file, the fresh, just-loaded set of rules
are saved in the global variable *tre-rules-saver* by the command
tre-save-rules.  This variable saves the status of the rule counters and the
list of initial rules. A similar variable, {\tt *initial-assertions*}, is
used to store all assertions appearing in the rules file. When the BMS is
run, new rules may be created and added to the known set of rules. As a
result, each time {\tt match} is invoked, the BMS is reinitialized to its
status just after loading the rule file, that is, it is restored to the
status indicated in {\tt *tre-rules-saver*}.

This facility may be used by application programs to save different rule
sets in memory and swap them as needed, without having to load rule files
each time. For example, suppose \SME\ is invoked, which will cause it to run
the current rule set for Gmap scoring. If a second scoring criterion is then
desired, a second set of rules may be invoked using code of the form:
\begin{small}
\begin{alltt}
   (let ((save-rules sme:*tre-rules-saver*)          {\it ;save the previous rule set}
         (save-assertions sme:*initial-assertions*))
     (setq sme:*tre-rules-saver* *my-other-rules-set-rules*)
     (setq sme:*initial-assertions* *my-other-rules-set-assertions*)
     (sme:tre-init)                                  {\it ;initialize with new rules}
     (sme:run-rules)                                 {\it ;run the new rule set}
     (setq sme:*tre-rules-saver* save-rules)         {\it ;restore the original rules}
     (setq sme:*initial-assertions* save-assertions))
\end{alltt}
\end{small}
This saves \SME's normal rule set, runs a different one, and then restores
the rule set to its previous value. In this example, {\tt tre-init} was
used, which fully initializes the BMS. If the desire is simply to supply
additional rules without destroying the current BMS state, {\tt
restore-rules} should be used in place of {\tt tre-init}.  The variables
corresponding to ``my-other-rules-set'' may be initialized by a similar
program which saves the current rule set, {\it loads} the desired ``other
rule set'' file, sets the ``my-other-rules-set'' variables from {\tt
*tre-rules-saver*} and {\tt *initial-assertions}, and then restores the
original rule set.


\section{Summary}

The \SME\ program has been described from the perspective of how to actually
use it. A number of methods have been presented about how to configure \SME\
to perform a variety of different types of matches.  It is hoped that \SME\
may serve as a general mapping tool for research on analogical mapping,
allowing researchers to focus on the more substantive issue of general
theories of analogical mapping, as opposed to worrying about implementation
details. The latter has the unfortunate effect of producing the repeated
scenario in which analogy researcher A goes to analogy researcher B and says
``My program can do X, which yours cannot'', followed by researcher B
returning a month later with this simple modification added. By testing
different theories within the same program, we may now compare the more
critical ``This is a logical consequence of my theory''. A program does not
a theory make. It can, however, function as a useful analytical tool.

Of course, not all of the problems of implementing analogical mapping have
been solved.  Most critical is redesigning the potentially combinatoric
merge step 3, perhaps using either a heuristic search or connectionist
relaxation network as suggested in \cite{sme-aij}. Of theoretical relevance
is the appropriateness of the abstract structural approach which \SME\
embodies.

\section{Acknowledgements}

The development of \SME\ has been a collaborative effort with Ken Forbus and
Dedre Gentner, with significant influence provided by Janice Skorstad. This
work has also benefited from discussions with Steve Chien, John Collins, and
Ray Mooney. 

This research is supported in part by an IBM graduate fellowship and in part
by the Office of Naval Research, Contract No. N00014-85-K-0559.

\begin{thebibliography}{99}\addcontentsline{toc}{section}{References}

 \bibitem{Burstein83}
Burstein, M., Concept formation by incremental analogical reasoning and
debugging, {\it Proceedings of the Second International Workshop on
Machine Learning}, University of Illinois, Monticello, Illinois, June,
1983.  A revised version appears in {\it Machine Learning: An Artificial
Intelligence Approach Vol. II}, R.S. Michalski, J.G. Carbonell, and T.M.
Mitchell (Eds.), Morgan Kaufman, 1986.
 \bibitem{Carbonell83}
Carbonell, J.G., Learning by Analogy: Formulating and generalizing plans
from past experience, in: {\it Machine Learning: An Artificial
Intelligence Approach}, R.S. Michalski, J.G. Carbonell, and T.M.
Mitchell (Eds.), Morgan Kaufman, 1983.
 \bibitem{BMS}
Falkenhainer, B.,
Towards a general-purpose belief maintenance system, in: J.F. Lemmer \& L.N.
Kanal (Eds.), {\it Uncertainty in Artificial Intelligence, Volume II}, 1987.
Also Technical Report, UIUCDCS-R-87-1717, Department of Computer Science,
University of Illinois, 1987.
 \bibitem{VBAL}
Falkenhainer, B.,
An examination of the third stage in the analogy process:
Verification-Based Analogical Learning, Technical Report
UIUCDCS-R-86-1302, Department of Computer Science, University of Illinois,
October, 1986. A summary appears in {\it Proceedings of the Tenth
International Joint Conference on Artificial Intelligence}, Milan, Italy,
August, 1987.
 \bibitem{MLW}
Falkenhainer, B.,
Scientific theory formation through analogical inference, {\it
Proceedings of the Fourth International Machine Learning Workshop}, Irvine,
CA, June, 1987.
 \bibitem{DBR}
Falkenhainer, B.
The utility of difference-based reasoning, {\it Proceedings of the Seventh
National Conference on Artificial Intelligence}, St. Paul, August, 1988.
 \bibitem{Falken-phd}
Falkenhainer, B., Learning from physical analogies: A study in analogy
and the explanation process, Ph.D. Thesis, University of Illinois,
December, 1988.
 \bibitem{aaai86}
Falkenhainer, B., K.D. Forbus, D. Gentner,
The Structure-Mapping Engine, {\it Proceedings of the Fifth National
Conference on Artificial Intelligence}, August, 1986.
 \bibitem{sme-aij}
Falkenhainer, B., K.D. Forbus, D. Gentner,
The Structure-Mapping Engine:  Algorithm and Examples, Technical Report
UIUCDCS-R-87-1361, Department of Computer Science, University of Illinois,
July, 1987. To appear in {\it Artificial Intelligence}.
 \bibitem{framework}
Forbus, K.D. and D. Gentner.
Learning physical domains: Towards a theoretical framework,
in: {\it Machine Learning: An Artificial Intelligence Approach Vol. II}, R.S. Michalski,
J.G. Carbonell, and T.M. Mitchell (Eds.), Morgan Kaufmann, 1986.
 \bibitem{Gentner80}
Gentner, D.
{\it The Structure of Analogical Models in Science}, BBN Tech.
Report No. 4451, Cambridge, MA., Bolt Beranek and Newman Inc., 1980.
 \bibitem{smt-cogsci}
Gentner, D.
Structure-Mapping: A Theoretical Framework for Analogy, {\it Cognitive
Science} {\bf 7}(2), 1983. 
 \bibitem{allerton}
Gentner, D.
Mechanisms of analogy.  To appear in S. Vosniadou and A. Ortony, (Eds.),
{\em Similarity and analogical reasoning}, Cambridge University Press,
Oxford.
 \bibitem{analogica}
Gentner, D.
Analogical inference and analogical access, To appear in A. Preiditis (Ed.),
{\em Analogica: Proceedings of the First Workshop on Analogical Reasoning},
London, Pitman Publishing Co.  Presented in December, 1986.
 \bibitem{tinlap3}
Gentner, D., B. Falkenhainer, and J. Skorstad
Metaphor: The good, the bad and the ugly. {\it Proceedings of the Third
Conference on Theoretical Issues in Natural Language Processing}, Las
Cruces, New Mexico, January, 1987.  
 \bibitem{Greiner}
Greiner, R., Learning by understanding analogies, PhD Thesis (STAN-CS-1071),
Department of Computer Science, Stanford University, September, 1985.
 \bibitem{hayes-roth}
Hayes-Roth, F., J. McDermott, An interference matching technique for
inducing abstractions, {\it Communications of the ACM}, {\bf 21}(5),
May, 1978.
 \bibitem{ACME}
Holyoak, K., \& Thagard, P. Analogical mapping by constraint satisfaction,
June, 1988, ({\it submitted for publication}).
 \bibitem{Indurkhya}
Indurkhya, B., Constrained semantic transference: A formal theory of
metaphors, Technical Report 85/008, Boston University, Department of
Computer Science, October, 1985.
 \bibitem{KedarCabelli}
Kedar-Cabelli, S., Purpose-directed analogy. {\it Proceedings of the Seventh
Annual Conference of the Cognitive Science Society}, Irvine, CA, 1985.
 \bibitem{Reed}
Reed, S.K., A structure-mapping model for word problems. Paper presented
at the meeting of the Psychonomic Society, Boston, 1985.
 \bibitem{Rumelhart}
Rumelhart, D.E., \& Norman, D.A., Analogical processes in learning.  In J.R.
Anderson (Ed.), {\it Cognitive skills and their acquisition}, Hillsdale,
N.J., Erlbaum, 1981.
 \bibitem{aaai87}
Skorstad, J., B. Falkenhainer and D. Gentner
Analogical Processing: A simulation and empirical corroboration, {\it
Proceedings of the Sixth National Conference on Artificial Intelligence},
Seattle, WA, August, 1987.
 \bibitem{Winston80}
Winston, P.H., Learning and Reasoning by Analogy, {\it Communications of the
ACM}, {\bf 23}(12), 1980.
 \bibitem{Winston82}
Winston, P.H., Learning new principles from precedents and exercises,
{\it Artificial Intelligence}, {\bf 19}, 321-350, 1982.

\end{thebibliography}

\newpage
\addcontentsline{toc}{section}{Index}

\begin{theindex}

\item algorithm, 4-10
	\subitem  summary, 3
\item analogical hint, 17
\item arg-list macro, 28
\item {\tt assert!} function, 14
\item {\tt attribute?} macro, 28
\indexspace
\item {\tt *base*} variable, 29
\item batch mode, 22
\item {\tt best-gmaps} function, 29
\indexspace
\item {\tt calculate-nogoods} function, 30
\item candidate inference, 9
	\subitem  alternate algorithms, 31
	\subitem  gather-inferences, 30
\item {\tt change-parms} function, 26
\item {\tt children-of?} function, 15
\item {\tt clear-given-mappings} function, 18
\item {\tt clear-scroll} function, 25
\item {\tt commutative?} macro, 28
\item {\tt compare-gmaps} function, 29
\item {\tt constant-entity?} macro, 27
\item conventions, 1
	\subitem  file names, 20
	\subitem  function names, 1
\item {\tt create-match-hypotheses} function, 30
\indexspace
\item {\tt declare-given-mappings} function, 18
\item {\tt defDescription} macro, 12
\item {\tt defEntity} macro, 11
\item {\tt define-description} function, 12
\item {\tt define-Entity} function, 11
\item {\tt define-predicate} function, 10
\item {\tt defPostMatcher} macro, 23
\item {\tt defPredicate} macro, 10
\item {\tt defSME-Parameter} macro, 25
\item {\tt defSME-Utility} macro, 26
\item {\tt describe-dgroup} function, 29
\item description group, 11-12
	\subitem  graphical display, 29
	\subitem  retrieval, 29
	\subitem  textual description, 29
\item Dgroup, 12
\item {\tt dgroup-directory} macro, 22
\item {\tt dgroup-file} macro, 22
\item {\tt display-match} function, 29
\item {\tt dump-scroll} function, 25
\item {\tt dump-scroll-menu} function, 25
\indexspace
\item entity, 11
	\subitem  declaring, 11
\item {\tt entity?} function, 27
	\subitem  inspection functions, 27
\item {\tt entity-domain} macro, 27
\item {\tt entity-name?} macro, 27
\item {\tt entity-type?} function, 27
\item {\tt eval-form} macro, 28
\item execution, 20
\item expression, 12
	\subitem  adding new expressions, 12
\item {\tt expression} function, 12
	\subitem  inspection functions, 28
\item {\tt expression-functor} function, 28
\item {\tt expression-type} macro, 28
\indexspace
\item {\tt fetch-dgroup} function, 29
\item {\tt fetch-entity-definition} macro, 27
\item {\tt fetch-expression} function, 28
\item {\tt fetch-predicate-definition} macro, 28
\item file organization, 1
\item {\tt fully-expand-expression} function, 28
\item {\tt fully-expand-expression-form} function, 28
\item {\tt function?} macro, 28
\indexspace
\item {\tt gather-inferences} function, 30
\item generalize, 16
\item {\tt generalize} function, 23
\item {\tt generate-gmaps} function, 30
\item {\tt get-dgroup} function, 26
\item {\tt get-rules} function, 26
\item global mapping, 7-10
	\subitem  comparing apparently identical gmaps, 29
	\subitem  creation, 29
	\subitem  defstruct, 30
	\subitem  {\tt *gmaps*}, 29
	\subitem  scoring, 9, 15
	\subitem  selecting the best, 29
	\subitem  textual display, 29
\item {\tt *gmaps*} variable, 29, 31
\item {\tt gm-base} subst, 31
\item {\tt gm-bms-node} subst, 31
\item {\tt gm-elements} subst, 31
\item {\tt gm-emaps} subst, 31
\item {\tt gm-id} subst, 30
\item {\tt gm-inferences} subst, 31
\item {\tt gm-plist} subst, 31
\item {\tt gm-target} subst, 31
\item {\tt *graphics-pane*} variable, 25
\indexspace
\item {\tt initial-assertion} macro, 14
\item {\tt *initial-assertions*} variable, 31
\item installation, 20
\item {\tt install-MH} function, 13
\item {\tt intern-gmaps} function, 30
\indexspace
\item {\tt language-file} macro, 22
\item lisp machine interface, 25
	\subitem  predicate documentation, 11
\item {\tt *lisp-pane*} variable, 25
\indexspace
\item match constructor rules, 13
\item match evidence rules, 14
\item {\tt match} function, 29
\item match hypothesis, 4-5
	\subitem  defstruct, 30
	\subitem  inspecting evidence justifications, 24
	\subitem  installing, 13
	\subitem  {\tt *match-hypotheses*}, 29
	\subitem  scoring, 14-15
\item {\tt match-evidence-inspector} function, 24
\item {\tt *match-hypotheses*} variable, 29-30
\item {\tt menu-display-dgroup} function, 29
\item {\tt menu-display-pairs} function, 29
\item {\tt menu-utilities} function, 26
\item {\tt mh-base-item} subst, 30
\item {\tt mh-bms-node} subst, 30
\item MHC-rule, 13
\item {\tt mh-form} subst, 30
\item {\tt mh-plist} subst, 30
\item {\tt mh-target-item} subst, 30
\indexspace
\item {\tt n-ary?} macro, 28
\item {\tt node-belief+} subst, 30
\item {\tt numargs} macro, 28
\indexspace
\item packages, 1
	\subitem site specific, 20
\item {\tt paired-item?} function, 19
\item {\tt *parameter-menu-options*} variable, 25
\item parameters, 25
	\subitem  site specific, 1, 20
\item predicate, 10
	\subitem  declaring, 10
	\subitem  inspection functions, 28
\item {\tt predicate?} macro, 28
\item {\tt predicate-type} macro, 28
\indexspace
\item {\tt rassert!} macro, 14
\item {\tt relation?} macro, 28
\item report generation, 22
\item {\tt report-comments} macro, 23
\item representation issues, 19
\item {\tt restore-rules} function, 31
\item {\tt return-dgroup} function, 29
\item {\tt rule} macro, 14
\item rule system, 12-19, 19
	\subitem  analogical hints, 17
	\subitem  dynamically swapping rule sets, 31
	\subitem  giving it access to gmaps, 30
	\subitem  match constructor rules, 13
	\subitem  match evidence rules, 14
	\subitem  pure isomorphisms, 16
	\subitem  relaxing identical predicates, 16
	\subitem  rule file syntax, 13
	\subitem  {\tt run-rules,} 30
	\subitem  simulating SPROUTER, 16
	\subitem  simulating structure-mapping theory, 15
\item {\tt rule-directory} macro, 22
\item {\tt rule-file} macro, 22
\item {\tt rule-sets} macro, 22
\item {\tt run-batch-file} function, 22
\item {\tt run-matcher-on} macro, 23
\item running SME, 20
\item {\tt run-rules} function, 30-31
\indexspace
\item {\tt sanctioned-pairing?} function, 19
\item scroll windows, 25
	\subitem  saving contents, 25
	\subitem  writing to, 25, 27
\item {\tt *scroll-pane*} variable, 25
\item {\tt select-double-scroll} function, 25
\item {\tt select-graphics} function, 25
\item {\tt select-large-graphics} function, 25
\item {\tt select-scroll} function, 25
\item {\tt select-split} function, 25
\item {\tt select-windowing-configuration} function, 25
\item {\tt send-report-to} macro, 23
\item site specific information, 20
	\subitem  packages, 20
	\subitem  pathnames, 20
\item {\tt *sme-default-rules*} variable, 20
\item {\tt *sme-dgroup-pathname*} variable, 20
\item {\tt *sme-files*} variable, 20
\item {\tt sme-format} macro, 27
\item {\tt *sme-frame*} variable, 25
\item {\tt *sme-graphics-output*} variable, 26
\item {\tt sme-init} function, 20
\item {\tt *sme-language-file*} variable, 20
\item {\tt *sme-output-stream*} variable, 26
\item {\tt *sme-parameters*} variable, 25
\item {\tt *sme-predicates*} variable, 28
\item {\tt sme-print} function, 27
\item {\tt sme-rules-file} function, 13
\item {\tt *sme-rules-pathname*} variable, 20
\item {\tt *sme-system-pathname*} variable, 20
\item {\tt sme-terpri} function, 27
\item {\tt *spare-scroll-pane*} variable, 25
\item system utilities, 26
\item {\tt *system-utilities-menu*} variable, 26
\indexspace
\item {\tt *target*} variable, 29
\item {\tt *the-lisp-package*} variable, 20
\item {\tt *the-user-package*} variable, 20
\item {\tt tre-init} function, 31
\item {\tt tre-rules-file} function, 31
\item {\tt *tre-rules-saver*} variable, 31
\item {\tt tre-save-rules} function, 13, 31
\indexspace
\item {\tt *windowing?*} variable, 26
\indexspace
\end{theindex}

%%\batchmode
\documentstyle[11pt,alltt]{article}
\pagestyle{myheadings}
\makeindex
\markboth{SME User's Manual}{SME User's Manual}
\setlength{\oddsidemargin}{-0.1in}
\setlength{\evensidemargin}{0in}
\setlength{\textwidth}{6.5in}
\setlength{\topmargin}{-0.3in}
\setlength{\textheight}{9in}
%% Declaring SME lisp hooks
\newcommand{\decvar}[1]{\noindent\makebox[\textwidth]{{\tt #1}
                                   \hfill [{\it Variable}]}\index{#1 variable}}
\newcommand{\decfunc}[2]{\noindent\makebox[\textwidth]{{\tt #1}\ \ {\it #2}
                                               \hfill [{\it Function}]}\index{#1 function}}
\newcommand{\decmacro}[2]{\noindent\makebox[\textwidth]{{\tt #1}\ \ {\it #2}
                                                \hfill [{\it Macro}]}\index{#1 macro}}
\newcommand{\decsubst}[2]{\noindent\makebox[\textwidth]{{\tt #1}\ \ {\it #2}
                                                \hfill [{\it Subst}]}\index{#1 subst}}
%% Handy abbreviations
\newcommand{\metasyntax}[1]{$\langle #1 \rangle$}
\newcommand{\spc}[1]{\vspace{#1\baselineskip}}
\newcommand{\gthan}{\( >\)}
\newcommand{\border}{\rule{\textwidth}{0.04in}}
\newcommand{\SME}{{\tt SME}}
\newcommand{\SMET}{\mbox{${\tt SME}_{{\it SMT}}$}}
\newcommand{\Phineas}{{\tt PHINEAS}}
\newcommand{\keypoint}[2]{\begin{itemize}
                             \item \parbox[t]{0.88\textwidth}{{\it #1:} #2}
                          \end{itemize}
                          \vspace{0.08in}}
%
\begin{document}

\begin{titlepage}
{\center {\Large Technical Report UIUCDCS-R-88-1421 \\ }}
\ \ \ \vspace{0.5in} \\
{\center {\bf {\huge The SME User's Manual} \\[2ex]
              {\LARGE  (SME Version 2E)} \\ }}
\vspace{0.7in}
\begin{center}
            {\Large Brian Falkenhainer}       \\[2ex]
{\large     Qualitative Reasoning Group       \\
          Department of Computer Science      \\
         University of Illinois at Urbana-Champaign \\
\vspace{0.75in}
                December, 1988 \\ }
\end{center}
\vspace{0.75in}
\begin{abstract}
This paper documents the {\it Structure-Mapping Engine} (\SME), a
general-purpose program for studying analogical processing.  It provides a
comprehensive description of the program and instructions for using it,
including techniques for integrating it into larger systems. One section
demonstrates methods for configuring \SME\ to a variety of mapping
preferences and suggests the range of theoretical variations available.
\end{abstract}

\end{titlepage}
\pagenumbering{roman}
\tableofcontents
\markboth{SME User's Manual}{SME User's Manual}
\newpage
\pagenumbering{arabic}

\section{Introduction}

The Structure-Mapping Engine (\SME) is a general tool for performing various
types of analogical mappings. \SME\ was originally developed to simulate
Gentner's {\it Structure-Mapping} theory of analogy
\cite{smt-cogsci,allerton,analogica}. It was hoped that the developed system
would also be able to model the other types of similarity comparisons
sanctioned by Gentner's theory, such as {\it literal similarity} and {\it
mere appearance}. What ended up being developed was an extremely flexible
and efficient system. Most theoretical assumptions are left out of the
program and are supplied through match rules. Thus, while \SME\ was
originally designed to simulate the comparisons of structure-mapping theory,
it may simulate many others as well. Given a set of theoretical restrictions
on what constitutes a reasonable analogical mapping, one may implement these
restrictions in the form of rules and use \SME\ to interactively test their
consequences. This report is intended to make that task easier.

This paper is designed for those interested in using \SME\ for studying
analogical processing, testing alternate theories, or as the mapping
component in a larger system. It describes the options and user support
provided in \SME, how to use it for testing theories, and how to integrate
it with other programs. For a discussion of the theory behind \SME, the
general algorithm, and descriptions of the program in operation, one should
consult \cite{aaai86,sme-aij} prior to reading this manual. Descriptions of
the use of \SME\ in various research projects may be found in
\cite{VBAL,MLW,DBR,Falken-phd,allerton,tinlap3,aaai87}, while descriptions
of Gentner's Structure-Mapping theory appear in
\cite{Gentner80,smt-cogsci,allerton,analogica,framework,sme-aij}.

\subsection{Conventions}\index{conventions}

Throughout this guide, a few conventions will be used which should be
explained at this time.

\begin{enumerate}

 \item {\it CommonLisp Packages}.\index{packages} The \SME\ system resides
in its own package, {\tt SME}, which is defined to use CommonLisp. As a
result, any reference to an \SME\ function or variable must specify the {\tt
SME} package, as in the function {\tt sme:define-predicate}. To simplify the
discussion, we will omit the package prefix when describing \SME\ functions,
macros, and variables. In addition, while the \SME\ routines reside in the
{\tt SME} package, the structures it manipulates reside in the general {\tt
USER} package.
 \item {\it The declarative interface.}\index{conventions, function names}
In general, the routines used to present data items to \SME, such as
predicate definitions and concept descriptions, appear in two,
functionally-equivalent forms. These two types have a naming convention
associated with each. The most common type is the {\it declarative} or macro
interface. The declarative routines do not evaluate their arguments and
match the syntactic form {\tt def}{\it routine-name}. For example, to define
the entity {\tt sun} declaratively, one writes {\tt (defEntity sun)}. The
second type is the {\it functional} interface, which is present to support
declarations by external programs.  These routines evaluate their arguments
and match the syntactic form {\tt define-}{\it routine-name}. For example,
to define the entity {\tt sun} functionally, one writes {\tt (define-entity
'sun)}.

\end{enumerate}


\subsection{File Organization}\index{file organization}

\SME\ is contained within the following twelve files:

\begin{description}
 \item[config.lisp]  The declarations for site specific
parameters.\index{parameters, site specific}
 \item[defs.lisp]  The basic structure definitions and macros used throughout
\SME. 
 \item[bits.lisp]  Routines for creating and manipulating bit vectors.
 \item[bms.lisp]  The belief-maintenance system (BMS) - a probabilistic TMS.
 \item[bms-tre.lisp]  The rule system and problem-solver front end for the BMS.
 \item[sme.lisp]  The \SME\ top-level routines, such as initialization,
defining facts about a concept, and fetching and storing facts and concept
descriptions.
 \item[match.lisp]  The \SME\ mapping algorithm.
 \item[match-rules-support.lisp]  A few functions useful for writing \SME\
match rules.
 \item[display.lisp]  Machine independent output routines.
 \item[windowing.lisp]  Symbolics dependent interface routines.
 \item[batch.lisp]  Routines to enable execution in {\it batch} mode with a
final report generated.
 \item[generalize.lisp]  Inductive generalization support.
\end{description}


\section{System Review}

The Structure-Mapping Engine can simulate a class of {\it structural
approaches} to analogical mapping. In these approaches, there is a distinct
stage of matching and carryover of predicates from one domain (the {\it
base}) into another (the {\it target}) within the larger analogy process.
Furthermore, although there are a number of differences, there is widespread
agreement among these techniques on one fundamental restriction
\cite{Burstein83,Carbonell83,Greiner,Indurkhya,KedarCabelli,Reed,Rumelhart,Winston82}:
\begin{enumerate}
 \item {\it Structural consistency}. If a final analogical mapping includes
a predicate in the base paired with a predicate in the target, then it must
also include corresponding pairings between each of their arguments. This
criterion simply asserts that an analogical mapping must not produce
syntactically meaningless predicate calculus forms.
\end{enumerate}
In \SME, this restriction was enforced by the requirement of simulating
Structure-Mapping theory; its impact on the algorithm is described in
\cite{sme-aij}. However, this restriction is only part of that theory and
alone does not uniquely define a matching algorithm. Additional theoretical
restrictions must be supplied through {\it match rules}. This enables \SME\
to be used in exploring the space of theories consistent with this single
criterion. An additional restriction is enforced by default:

\keypoint{One-to-one mapping}{No base item (predicate or object) may be
paired with multiple target items. Likewise, no target item may be paired
with multiple base items.}

Enforcement of the one-to-one restriction is a global parameter which may be
disabled. Support is provided to implement variations of one-to-one within
the match rules.

Match rules specify what pairwise matches are possible and provide local
measures of evidence used in computing the evaluation score. These rules are
the key to \SME's flexibility. To build a new matcher one simply loads a new
set of match rules.  This has several important advantages.  First, we can
simulate all of the types of comparisons sanctioned by Structure-Mapping
theory with one program.  Second, the rules could in theory be ``tuned'' if
needed to simulate particular kinds of human performance. Third, a variety
of other analogical mapping systems may be simulated for comparison and
theoretical investigation.  The breadth of the space of these structural
approaches is suggested by the examples in Section~\ref{ss:rule-system}.

In this section, the \SME\ matching algorithm is briefly reviewed, followed
by a short discussion of how theoretical guidelines may be added to the
general mechanism. It is a summary of the algorithm description appearing in
\cite{sme-aij}, annotated with the \SME\ functions that carry out each step.


\begin{figure}
\border
\vspace{2.1in}
\caption{Simplified water flow and heat flow descriptions.}
\label{fg:wfhf-calculus}
\border
\end{figure}


\subsection{Algorithm Review}\index{algorithm, summary}

Given descriptions of a base and a target (called Dgroups), \SME\ builds all
structurally consistent interpretations of the comparison between them.
Each interpretation of the match is called a {\em global mapping}, or {\em
Gmap}.  Gmaps consist of three parts:

\begin{enumerate}
\item{\em Correspondences:} A set of pairwise matches between the
expressions and entities of the two dgroups.
\item {\em Candidate Inferences:} A set of new expressions which the
comparison suggests holds in the target dgroup.
\item {\em Structural Evaluation Score:} (Called {\em SES} for
brevity) A numerical estimate of match quality.
\end{enumerate}

For example, given the descriptions of water flow and heat flow shown in
Figure~\ref{fg:wfhf-calculus}, \SME\ might, depending on the current
theoretical configuration, offer several alternative interpretations for
this potential analogy.  In one interpretation, the central inference is
that water flowing from the beaker to the vial corresponds to heat flowing
from the coffee to the ice cube. Alternatively, one could map water to
coffee, since they are both liquids.


\begin{figure}
\border
\begin{small}
\begin{tt}
\begin{itemize}
 \item Run MHC rules to construct match hypotheses (create-match-hypotheses).
 \item Calculate the {\it Conflicting} set for each match hypothesis (calculate-nogoods).
 \item Calculate the {\it EMaps} and {\it NoGood} sets for each match
        hypothesis by upward propagation from entity mappings
        (generate-justifications and propagate-descendants).
 \item During the propagation, delete any match hypotheses that have
        justification holes (propagate-death).
 \item Merge match hypotheses into Gmaps (generate-gmaps).
      \begin{enumerate}
         \item Interconnected and consistent (generate-structure-groups).
         \item Consistent members of same base structure (merge-base).
         \item Any further consistent combinations (full-gmap-merge).
      \end{enumerate}
 \item Calculate the candidate inferences for each GMap (gather-inferences).
 \item Score the matches (run-rules).
      \begin{enumerate}
         \item Local match scores.
         \item Global structural evaluation scores.
      \end{enumerate}
\end{itemize}
\end{tt}
\end{small}
  \caption{Summary of SME algorithm.}
  \label{fg:sme-overview}
\border
\end{figure}


The \SME\  algorithm (see Figure~\ref{fg:sme-overview}) is logically divided
into four stages:
\begin{enumerate} 
\item {\em Local match construction:} Finds all pairs of ({\it BaseItem},
{\it TargetItem}) that potentially can match.  A {\em Match Hypothesis} is
created for each such pair to represent the possibility that this local
match is part of a global match.
\item {\em Gmap construction:} Combines the local matches into maximal
consistent collections of correspondences.
\item {\em Candidate inference construction:} Derives the inferences
suggested by each Gmap.
\item {\em Match Evaluation:} Attaches evidence to each local match and
uses this evidence to compute structural evaluation scores for each Gmap. 
\end{enumerate}

\noindent
Each computation will now be reviewed, using a simple example to
illustrate their operation. In this example, the rules of structure-mapping
theory are in use. It is important to distinguish the general \SME\ system
from its behavior when using the rules of a particular theory. Hence, when
using the rules of structure-mapping theory, it will be called \SMET.

\subsubsection{Step 1: Local match construction (create-match-hypotheses)}\index{match hypothesis}\index{algorithm}

\begin{figure}
\border
\vspace{3.1in}
\caption[Local Match Construction]{Local Match Construction. The water flow
and heat flow descriptions of Figure~\ref{fg:wfhf-calculus} have been drawn
in the abstract and placed to the left and right, respectively. The objects
in the middle depict match hypotheses.}
\label{fg:3D-MH}\index{algorithm}
\border
\end{figure}


\SME\ begins by finding for each entity and predicate in the base the set of
entities or predicates in the target that could plausibly match that item
(see Figure~\ref{fg:3D-MH}). Plausibility is determined by {\it match
constructor} rules, which are of the form:
\begin{tt}
\begin{tabbing}
\ \ \= (MHCrule \= (\metasyntax{Trigger} \= \metasyntax{BaseVariable} \metasyntax{TargetVariable} \\
\> \> \> {\it [}:test \metasyntax{TestForm}{\it ]}) \\
\> \ \ \ \  \metasyntax{Body}) \\
\end{tabbing}
\end{tt}

The body of these rules is run on each pair of items (one from the base and
one from the target) that satisfy the condition and installs a {\it match
hypothesis}\index{match hypothesis} which represents the possibility of them
matching. For example, to state that an expression in the base may match an
expression in the target whose functor is identical, we write:
\begin{verbatim}
   (MHC-rule (:filter ?b ?t :test (equal (expression-functor ?b)
                                         (expression-functor ?t)))
         (install-MH ?b ?t))
\end{verbatim}


\begin{figure}
\border
\vspace{4.4in}
\caption[Water Flow / Heat Flow Analogy]{Water Flow / Heat Flow
Analogy After Local Match Construction. Here we show the graph of match
hypotheses depicted schematically in Figure~\ref{fg:3D-MH}, augmented by links
indicating expression-to-arguments relationships. Match hypotheses which are not
descended from others are called {\it roots} (e.g., the matches
between the {\tt GREATER} predicates, {\tt MH-1} and {\tt MH-6}, and the
match for the predicate {\tt FLOW}, {\tt MH-9}). Match hypotheses between
entities are called {\it Emaps} (e.g., the match between beaker and coffee,
{\tt MH-4}).  Emaps play an important role in algorithms based on structural
consistency.}
\label{fg:wfhf-local}\index{algorithm}
\border
\end{figure}


The likelyhood of each match hypothesis\index{match hypothesis} is found by
running {\it match evidence} rules and combining their results. The evidence
rules provide support for a match hypothesis by examining the structural and
syntactic properties of the items matched. For example, the rule
\begin{verbatim}
   (MHErule ((:intern (MH ?b ?t) :test (and (expression? ?b) (expression? ?t)
                                            (eq (expression-functor ?b)
                                                (expression-functor ?t)))))
       (assert! (implies same-functor (MH ?b ?t) (0.5 . 0.0))))
\end{verbatim}
states ``If the two items are expressions and their functors are the same, then
supply 0.5 evidence in favor of the match hypothesis.'' The rules may also
examine match hypotheses associated with the arguments of these items to
provide support based on systematicity. This causes evidence for a match
hypothesis to increase with the amount of higher-order structure supporting
it.

The state of the match between the water flow and heat flow descriptions of
Figure~\ref{fg:wfhf-calculus} after running these first two sets of rules is
shown in Figure~\ref{fg:wfhf-local}.  There are several important things to
notice in this figure.  First, there can be more than one match
hypothesis\index{match hypothesis} involving any particular base or target
item.  Second, our rules required predicates to match identically while they
allowed entities to match on the basis of their roles in the predicate
structure.  Thus while {\tt TEMPERATURE} can match either {\tt PRESSURE} or
{\tt DIAMETER}, {\tt IMPLIES} cannot match anything but {\tt IMPLIES}.
Third, not every possible correspondence is created. Local matches between
entities are only created when justified by some other identity.  This
significantly constrains the number of possible matches in the typical case.


\begin{figure}
\border
\vspace{3.1in}
\caption[Conflicting relationships.]{Water Flow - Heat Flow analogy after
computation of {\it Conflicting} relationships. Simple lines show the
tree-like graph that the grounding criteria imposes upon match hypotheses.
Lines with circular endpoints indicate the {\it Conflicting} relationships
between matches. Some of the original lines from MH construction have been
left in to show the source of a few {\it Conflicting} relations.}
\label{3D-nogoods}
\border
\vspace{3.1in}
\caption{GMap Construction. (a) Merge step 1: Interconnected and consistent.
(b) Merge step 2: Consistent members of the same base structure. (c) Merge
step 3: Any further consistent combinations.}
\label{fg:wfhf-merging}\index{global mapping}\index{algorithm}
\border
\end{figure}


\subsubsection{Step 2: Global Match Construction}\index{global mapping}\index{algorithm}

The second step in the \SME\ algorithm combines local match hypotheses into
collections of global matches (Gmaps).  Intuitively, each global match is
the largest possible set of match hypotheses that depend on the same one to
one object correspondences.

More formally, Gmaps consist of {\em maximal, structurally consistent}
collections of match hypotheses.  A collection of match hypotheses is
structurally consistent if it satisfies two criteria:
\begin{enumerate}
\item {\em One-to-one:} No two match hypotheses assign the same
base item to multiple target items or any target item to multiple
base items.
\item {\em Support:} If a match hypothesis {\tt MH} is in the
collection, then so are the match hypotheses which pair up all of
the arguments of {\tt MH}'s base and target items.
\end{enumerate}
The preservation criteria enforces strict one to one mappings.  The
grounding criteria preserves connected predicate structure.  A collection is
maximal if adding any additional match hypothesis would render the collection
structurally inconsistent.

The formation of global matches\index{global mapping} is composed of two primary stages:

\begin{enumerate}
\item {\em Compute consistency relationships {\tt (calculate-nogoods)}:}
Here we generate for each match hypothesis the sets of entity mappings it
entails, what match hypotheses it locally conflicts with, and which match
hypotheses it is structurally inconsistent with.  This information
simplifies the detection of contradictory sets of match hypotheses, a
critical operation in the rest of the algorithm. The result of this stage of
processing appears in Figure~\ref{3D-nogoods}.

\item {\em Merge match hypotheses {\tt (generate-gmaps)}:} Compute Gmaps by
successively combining match hypotheses as follows:
 \begin{enumerate}
 \item {\em Form initial combinations {\tt (generate-structure-groups)}:}
Combine interconnected and consistent match hypotheses into an initial set
of Gmaps (Figure~\ref{fg:wfhf-merging}a).
 \item {\em Combine dependent Gmaps {\tt (merge-base)}:} Since base and
target dgroups are rarely isomorphic, some Gmaps in the initial set will
overlap in ways that allow them to be merged.  The advantage in merging them
is that the new combination may provide structural support for candidate
inferences (Figure~\ref{fg:wfhf-merging}b).
 \item {\em Combine independent collections {\tt
(full-gmap-merge)}:}\footnote{These two merge steps (b and c) are called by
{\tt merge-gmaps}, which is in turn called by {\tt generate-gmaps}.} The
results of the previous step are next combined to form maximal consistent
collections (Figure~\ref{fg:wfhf-merging}c).
 \end{enumerate}
\end{enumerate}

A parameter option allows the support criterion to be weakened so that it
does not cross the boundaries of a {\it relational group} \cite{Falken-phd}.
A relational group is distinguished as an unordered collection of relational
structures that may be collectively referred to as a unit.  They correspond
to the abstract notion of a ``set'' and are associated to predicates taking
any number of arguments.  For example, a set of relations joined by the
predicate {\tt AND} defines a relational group.  Other examples include the
axioms of a theory, a decomposable compound object, or the relations holding
over an interval of time.  Intuitively, we would like to say that two groups
correspond without requiring that their contents are exhaustively mapped.

If base and target propositions each contain a group as an argument, the
propositions should not be prevented from matching if the groups' members
cannot be exhaustively paired. For example, the set of relations
\begin{tt}
\begin{tabbing}
~~~B:~~~\= Implies[And(P$_{1}$,P$_{2}$,P$_{3}$), P$_{4}$] \` {\rm (1)} \\
~~~T:   \> Implies[And(P$^{\prime}_{1}$,P$^{\prime}_{2}$), P$^{\prime}_{4}$]
\end{tabbing}
\end{tt}

\noindent
should match better than the set of relations
\begin{tt}
\begin{tabbing}
~~~B:~~~\= Implies[And(P$_{1}$,P$_{2}$,P$_{3}$), P$_{4}$] \` {\rm (2)} \\
~~~T$^{\prime}$: \> P$^{\prime}_{1}$, P$^{\prime}_{2}$, P$^{\prime}_{4}$
\end{tabbing}
\end{tt}

\noindent 
The original model of structural consistency would score (1) and (2)
equally, since the {\tt Implies} relations of (1) would not be allowed to
match. This is a particularly important consideration when matching
sequential, state-based descriptions (e.g., the behavior of a system through
time). The set of relations describing a pair of states often do not
exhaustively match or are of different cardinality. Yet, higher-order
relations over states, such as temporal orderings, are vital and must appear
in the mapping.



\subsubsection{Step 3: Compute Candidate Inferences (gather-inferences)}\index{candidate inference}\index{algorithm}

Associated with each Gmap\index{global mapping} is a (possibly empty) set of
candidate inferences.  Candidate inferences are base predicates that would
fill in structure which is not in the Gmap (and hence not already in the
target). If a candidate inference contains a base entity that has no
corresponding target entity (i.e., the base entity is not part of any match
hypothesis for that gmap), \SME\ introduces a new, hypothetical entity into
the target.  Such entities are represented as a skolem function of the
original base entity (i.e., {\tt (:skolem base-entity)}).

In Figure~\ref{fg:wfhf-done}, Gmap \#1 has the top level {\tt CAUSE}
predicate as its sole candidate inference.  In other words, this Gmap
suggests that the cause of the flow in the heat dgroup is the difference in
temperatures.  If the {\tt FLOW} predicate was not present in the target,
then the candidate inferences for a Gmap corresponding to the pressure
inequality would be both {\tt CAUSE} and {\tt FLOW}. Note that {\tt
GREATER-THAN[DIAMETER(coffee), DIAMETER(ice cube)]} is not a valid candidate
inference for the first Gmap because it does not intersect the existing Gmap
structure.\index{candidate inference}

\begin{figure}
\border
\begin{small}
\begin{verbatim}
  Rule File: literal-similarity.rules     Number of Match Hypotheses: 14

  Gmap #1: { (>PRESSURE  >TEMPERATURE)  (PRESSURE-BEAKER  TEMP-COFFEE)
             (PRESSURE-VIAL  TEMP-ICE-CUBE)  (WFLOW  HFLOW) }
      Emaps:  { (beaker coffee)  (vial ice-cube)  (water heat)  (pipe bar) }
      Weight:  5.99
      Candidate Inferences: (CAUSE >TEMPERATURE HFLOW)

  Gmap #2: { (>DIAMETER  >TEMPERATURE)  (DIAMETER-1  TEMP-COFFEE)
             (DIAMETER-2  TEMP-ICE-CUBE) }
      Emaps:  { (beaker coffee)  (vial ice-cube) }
      Weight:  3.94
      Candidate Inferences:  { }

  Gmap #3: { (LIQUID-3  LIQUID-5)  (FLAT-TOP-4  FLAT-TOP-6) }
      Emaps:  { (water coffee) }
      Weight:  2.44
      Candidate Inferences:  { }
\end{verbatim}
\end{small}

  \caption{Complete SME interpretation of Water Flow - Heat Flow Analogy.}
  \label{fg:wfhf-done}\index{algorithm}\index{global mapping}
\border
\end{figure}

\subsubsection{Step 4: Compute Structural Evaluation Scores (run-rules)}\index{global mapping, scoring}\index{algorithm}

Typically a particular pair of base and target will give rise to several
Gmaps, each representing a different interpretation of the match.  Often it
is desired to select only a single Gmap,\index{global mapping} for example
to represent the best interpretation of an analogy. Many of these evaluation
criteria (including validity, usefulness, and so forth) lie outside the
province of Structure-Mapping, and rely heavily on the domain and
application.  However, one important component of evaluation is {\em
structural} --- for example, one Gmap may be considered a better analogy
than another if it embodies a more systematic match.  \SME\ provides a
programmable mechanism for computing a {\em structural evaluation score}
(SES) for each Gmap.  This score can be used to rank-order the Gmaps in
selecting the ``best'' analogy, or as a factor in a more complex (but
external) evaluation procedure.  In \SMET, the structural evaluation score is
currently computed by simply adding the belief of each local match
hypothesis to the belief of the Gmaps it is a member of.

Returning to Figure~\ref{fg:wfhf-done}, note that the ``strongest''
interpretation (i.e., the one which has the highest structural evaluation
score) is the one we would intuitively expect.  In other words, {\tt beaker}
maps to {\tt coffee}, {\tt vial} maps to {\tt ice-cube}, {\tt water} maps to
{\tt heat}, {\tt pipe} maps to {\tt bar}, and {\tt PRESSURE} maps to {\tt
TEMPERATURE}.  Furthermore, we have the candidate inference that the
temperature difference is what causes the flow.\index{global mapping}

\subsection{Adding Theoretical Constraints}

Given the general program, we may then add theoretical constraints in the
form of rules. For instance, the example just presented used the {\it
literal similarity} rules of structure-mapping theory. These rules augment
\SME's one-to-one mapping and structural consistency criteria with two
additional restrictions. First, evidence is computed according to {\it
systematicity}, that is, highly interconnected systems of relations are
preferred over independent facts. Second, only identical relations are
allowed to match (i.e., {\tt CAUSE} is not allowed to match {\tt
GREATER-THAN}).  Had another set of rules been used, the results might have
been substantially different.  For example, the {\it mere appearance} rules
of structure-mapping theory would have determined that the {\tt water} to
{\tt coffee} mapping was the best, due to their superficial similarity.


\section{Declarations}

The descriptions given to \SME\ are constructed from a user-defined
vocabulary of entities and predicates. This section discusses the
conventions for defining languages for \SME's use.

\subsection{Declaring Predicates}

\begin{tt}\index{defPredicate macro}\index{predicate}\index{predicate, declaring}
\begin{tabbing}
defPredicate \= {\it name argument-declarations predicate-class} \` [{\it Macro}] \\
             \> \&key \= :expression-type {\it logical-type}    \\
             \>      \> :commutative?  \{t | nil\}              \\
             \>      \> :n-ary?  \{t | nil\}                    \\
             \>      \> :documentation {\it descriptive-string} \\
             \>      \> :eval {\it procedural-attachment}\ \ )
\end{tabbing}
\end{tt}
\decfunc{define-predicate}{name argument-declarations predicate-class ...}

\spc{1}
\noindent
{\it predicate-class} is either {\tt function}, {\tt attribute}, or
{\tt relation}, according to what kind of predicate {\it name} is.
The {\it argument-declarations} allows the arguments to be named and
typed.  For example, the declaration:

\begin{verbatim}
   (defPredicate CAUSE ((antecedent event) (consequent event)) relation)
\end{verbatim}
states that {\tt CAUSE} is a two-place relational predicate.  Its arguments
are called {\tt antecedent} and {\tt consequent}, both of type {\tt event}.
The names and types of arguments are for the convenience of the
representation builder and any external routines (including the match rules),
and are not currently used by \SME\ internally.  Likewise, the predicate
class may be very important to the theoretical constraints imposed in the
rules, but is ignored by \SME\ internally.

The optional declaration {\tt :expression-type} indicates the logical type
of an expression headed by the given predicate. For example, the predicate
{\tt throw} may represent a kind of {\tt action}, while the predicate {\tt
mass} may represent an {\tt extensive-quantity}.

The optional declarations {\tt :commutative?} and {\tt :n-ary?} provide
\SME\ with important syntactic information.  {\tt :commutative?} indicates
that the predicate is commutative, and thus the order of arguments is
unimportant when matching.  {\tt :n-ary?} indicates that the predicate can
take any number of arguments.  Examples of commutative nary predicates include
{\tt AND}, {\tt SUM}, and {\tt SET}.

The {\tt :documentation} option allows one to attach a descriptive string to
a predicate. This documentation may then be accessed through the lisp
machine supplied interface\index{lisp machine interface, predicate documentation}
({\tt C-Shift-D}) or some externally written routine. If no documentation is
supplied, the list of argument names is used.  Another option provided
strictly for potential user routines is the optional {\tt :eval} parameter.
This allows one to declare a procedural attachment for a predicate.


\subsection{Declaring Entities}

\indent
\decmacro{defEntity}{name {\tt \&key type constant?}}\index{entity, declaring}
\decfunc{define-Entity}{name {\tt \&key type constant?}}

\spc{0.5}
Entities are logical individuals, i.e., the objects and constants of a
domain. Typical entities include physical objects, their temperature, and
the substance they are made of.  Primitive entities are declared with the
{\tt defEntity} form (a non-primitive entity would be {\tt (temperature
sun)}, which is a functional form representing a particular numeric
temperature entity). Primitive entities declared in this way represent
global entity types, that is, they represent a class of entities rather than
an actual instance of an entity. When an entity type is actually used in a
domain description, a unique entity instance is created for that type (e.g.,
{\tt Mary} is translated to {\tt Mary43}).

Since the language is typed, each entity type can be declared as
a subtype of an existing type using the {\tt :type} option.
For example, we might have
\begin{verbatim}
   (defEntity star :type inanimate)
   (defEntity Sun :type star)
\end{verbatim}
to say that stars are inanimate objects, and our Sun is a particular star.
Constants are declared by using the {\tt :constant?} option, as in
\begin{verbatim}
   (defEntity zero :type number :constant? t)
\end{verbatim}


\subsection{Declaring Description Groups}\label{ss:new-dgroups}
\index{description group}

\begin{tt}
\begin{tabbing}
defDescription {\it description-name}  \` [{\it Macro}]   \\
\ \ \ \= expressions \= \kill
      \> entities  \> ({\it entity$_{1}$}, {\it entity$_{2}$},$\ldots$,{\it entity$_{i}$}) \\ 
      \> expressions \> ({\it expression-declarations})
\end{tabbing}
\end{tt}
\decfunc{define-description}{description-name entities expressions}

\spc{0.5}
For simplicity, predicate instances and compound terms are called {\it
expressions}. A {\em Description Group}, or {\em Dgroup}\index{Dgroup}, is a
collection of primitive entities and expressions concerning them.  Dgroups
are defined with the {\tt defDescription}\index{defDescription macro} form,
where {\it expression-declarations} take the form {\tt
\begin{tabbing}
\ \   \= {\it expression} {\rm or} \\
      \> ({\it expression} :name {\it expression-name})
\end{tabbing}
}
\noindent
For example, the description of water flow depicted in
Figure~\ref{fg:wfhf-calculus} was given to \SME\ as
\begin{verbatim}
  (defDescription simple-water-flow
        entities (water beaker vial pipe)
        expressions (((flow beaker vial water pipe) :name wflow)
                     ((pressure beaker) :name pressure-beaker)
                     ((pressure vial) :name pressure-vial)
                     ((greater pressure-beaker pressure-vial) :name >pressure)
                     ((greater (diameter beaker) (diameter vial)) :name >diameter)
                     ((cause >pressure wflow) :name cause-flow)
                     (flat-top water)
                     (liquid water)))
\end{verbatim}
All entities must have been previously defined and every entity referred to
in the Dgroup's expressions must appear in the {\tt entities} list of the {\tt
defDescription}.\index{description group}

\subsection{Adding new expressions}\label{ss:new-expressions}

\indent
\decfunc{expression}{form dgroup-name {\tt \&key expression-name update-structure?}}\index{expression, adding new expressions}

\spc{0.7}
Expressions are normally defined as a side effect of creating a description group
(Dgroup). However, the facility is provided for dynamically adding new expressions
to a Dgroup. The syntax is essentially the same as for expressions declared within
a {\tt defDescription}. The expression's {\it form} may refer to the names of
existing Dgroup expressions and the form may be given a name. When {\tt expression} is
used to add expressions to an existing Dgroup, the {\tt update-structure?} keyword
must be invoked with a non-nil (e.g., {\tt T}) value. This keyword indicates
that the Dgroup's structure must be reexamined, since the known structural
roots will change as a result of this new expression.

\subsection{Typed Logic}

A mechanism exists for attaching types to predicates and their arguments
(see {\tt defPredicate}).  This facility is designed to constrain the
operation of \SME, particularly candidate inference generation. However, it
has not been extensively used to date.  The ability to attach types may be
useful for consistency checking by external systems.


\section{Using the rule system}\label{ss:rule-system}\index{rule system}

The rule system is the heart of \SME's flexibility. It allows one to specify
what types of things might match and how strongly these matches should be
believed. This section describes the required syntax for a rule set and
different strategies for rule specification.

\subsection{Rule file syntax}\index{rule system, rule file syntax}\index{rule system}

A rule set, or rule file, consists of a {\it declaration}, a set of
{\it match constructor} rules, and a set of {\it match evidence} rules.
In order to describe each, we will examine the syntax and functionality of
each part of the {\it smt-analogy} rule file.

\subsubsection{File declarations}\index{rule system}

\indent
\decfunc{sme-rules-file}{identification-string}

\spc{0.7}
Each rule file must begin with the initialization command {\tt
sme-rules-file}. This function clears the rule system in preparation for a
new set of rules (rules are cached for efficiency) and stores the name of
the rule set for output identification purposes. For example, our sample
rules file begins with:\footnote{Notice the file extension *.rules. While
rule files are not required to end in ``.rules'', all user interface
facilities for simplifying the loading of rule files depend upon this
extension. Another useful point is that rule files are typically defined to
be in the \SME\ package to avoid having to use {\tt sme:} throughout the
rule set.}  

\begin{verbatim}
   (sme-rules-file "smt-analogy.rules")
\end{verbatim}

\noindent
The rule file must then end with the {\tt tre-save-rules} command:

\spc{0.7}
\decfunc{tre-save-rules}{}

\subsubsection{Match Constructor rules}\index{rule system, match constructor rules}\index{rule system}

\index{match hypothesis, installing}
\index{match constructor rules}
\index{MHC-rule}
\begin{tt}
\begin{tabbing}
MHC-rule \= ({\it trigger} \= {\it ?base-variable} {\it ?target-variable} \` [{\it Macro}] \\
         \>                \> [:test {\it test-form}]) \\
\ \ \  {\it body}
\end{tabbing}
\end{tt}
\decfunc{install-MH}{base-item target-item}\index{match hypothesis, installing}

\spc{0.7}
\SME\ begins by finding for each entity and predicate in the base the set of
entities or predicates in the target that could plausibly match that item.
Plausibility is determined by {\it match constructor} rules, which are
responsible for installing all match hypotheses processed by \SME.  There
are two types of constructor rules, each indicated by a different value for
{\it trigger}.  The first type of rule is indicated by a {\tt :filter}
trigger.  These rules are applied to each pair of base and target expressions,
executing the code in {\it body}.  If the {\tt :test} option is used,
{\it test-form} must return true for the body to be run.  For example, the
following rule states that an expression in the base may match an expression
in the target whose functor is identical, unless they are attributes
(a structure-mapping {\it analogy} criterion):

\begin{small}
\begin{verbatim}
   (MHC-rule (:filter ?b ?t :test (and (eq (expression-functor ?b)
                                           (expression-functor ?t))
                                       (not (attribute? (expression-functor ?b)))))
          (install-MH ?b ?t))
\end{verbatim}
\end{small}

The second type of MHC rule is indicated by a trigger of {\tt :intern}.
These rules are run on each match hypothesis as it is created. Typically
they create match hypotheses between any functions or entities that are the
arguments of the expressions joined by the match hypothesis that triggered
the rule. The following is one of two that appear in smt-analogy.rules:

\begin{small}
\begin{verbatim}
   (MHC-rule (:intern ?b ?t :test (and (expression? ?b) (expression? ?t)
                                       (not (commutative? (expression-functor ?b)))
                                       (not (commutative? (expression-functor ?t)))))
       (do ((bchildren (expression-arguments ?b) (cdr bchildren))
            (tchildren (expression-arguments ?t) (cdr tchildren)))
           ((or (null bchildren) (null tchildren)))
         (cond ((and (entity? (first bchildren)) (entity? (first tchildren)))
                (install-MH (first bchildren) (first tchildren)))
               ((and (function? (expression-functor (first bchildren)))
                     (function? (expression-functor (first tchildren))))
                (install-MH (first bchildren) (first tchildren)))
               ((and (attribute? (expression-functor (first bchildren)))
                     (eq (expression-functor (first bchildren))
                         (expression-functor (first tchildren))))
                (install-MH (first bchildren) (first tchildren))))))
\end{verbatim}
\end{small}

Notice that the third test allows identical attributes to match, whereas the
previous MHC rule did not allow such matches. This design does not allow
isolated attributes to match, but recognizes that attributes appearing in a
larger overall structure should be matched.


\subsubsection{Match Evidence rules}\label{ss:ME-rules}\index{match evidence rules}\index{rule system, match evidence rules}\index{rule system}

\indent
\decmacro{rule}{nested-triggers body}\index{match hypothesis, scoring}
\decmacro{rassert!}{expression\ {\tt \&optional (}belief+ {\tt 1.0) (}belief- {\tt 0.0)}}
\decfunc{assert!}{expression\ {\tt \&optional (}belief+ {\tt 1.0) (}belief- {\tt 0.0)}}
\decmacro{initial-assertion}{assertion-form}

\spc{0.8}
The structural evaluation score is computed in two phases. First, each match
hypothesis is assigned some local degree of evidence, independently of what
Gmaps it belongs to.  Second, the score for each Gmap is computed based on
the evidence for its match hypotheses.  The management of evidence rules is
performed by the {\it Belief Maintenance System} (BMS) \cite{BMS}. A BMS is
a form of Truth-Maintenance system, extended to handle numerical weights for
evidence and degree of belief (see \cite{BMS} for a description of what the
weights mean). Pattern-directed rules are provided that trigger on certain
events in the knowledge base. 

The following is a simple rule for giving evidence to match hypotheses
between expressions that have the same predicate:

\begin{small}
\begin{verbatim}
   (initial-assertion (assert! 'same-functor))

   (rule ((:intern (MH ?b ?t) :test (and (expression? ?b) (expression? ?t)
                                         (eq (expression-functor ?b)
                                             (expression-functor ?t)))))
      (if (function? (expression-functor ?b))
          (rassert! (implies same-functor (MH ?b ?t) (0.2 . 0.0)))
          (rassert! (implies same-functor (MH ?b ?t) (0.5 . 0.0)))))
\end{verbatim}
\end{small}

There are two things to notice here in addition to the evidence rule. First,
the proposition {\tt same-functor} was asserted to be true (a belief of
1.0) and then used as the antecedent for the implication of evidence. In
this way, the source of this particular piece of evidence is identified and
is available for inspection. Second, the assertion of {\tt same-functor} was
placed inside the {\tt initial-assertion} form. Since \SME\ caches the
current rule file, it must be told if there are any functions embedded in
the rule file that must be invoked each time \SME\ is initialized.

\spc{0.8}
\decfunc{children-of?}{base-child target-child base-expression target-expression}

\spc{0.3}
Nested triggers within an evidence rule may be used to locate
interdependencies between different match hypotheses. For example,
structure-mapping's systematicity principle is implemented in a local
fashion by propagating evidence from a match hypothesis to its
children:\footnote{A number of functions (e.g., {\tt children-of?}) are
provided to simplify the writing of rules. These appear in the file {\tt
match-rules-support.lisp}.}\index{match hypothesis, scoring}

\begin{small}
\begin{verbatim}
   (rule ((:intern (MH ?b1 ?t1) :test (and (expression? ?b1) (expression? ?t1)
                                           (not (commutative? (expression-functor ?b1)))))
          (:intern (MH ?b2 ?t2) :test (children-of? ?b2 ?t2 ?b1 ?t1)))
       (rassert! (implies (MH ?b1 ?t1) (MH ?b2 ?t2) (0.8 . 0.0))))
\end{verbatim}
\end{small}

\noindent
Evidence for a Gmap is given by:

\begin{small}
\begin{verbatim}
   (rule ((:intern (GMAP ?gm) :var ?the-group))
      (dolist (mh (gm-elements ?gm))
         (assert! `(implies ,(mh-form mh) ,?the-group))))
\end{verbatim}
\end{small}

\index{global mapping, scoring}\index{rule system}

The {\tt BMS} allows a set of nodes to be declared special and will treat
evidence to these nodes differently. An {\tt additive-nodes} function is
provided which takes a set of {\tt BMS} nodes and modifies them so that
their evidence is added rather than normalized using Dempster's rule. \SME\
automatically invokes {\tt additive-nodes} on the derived set of Gmaps once
they are created. Thus, when the above Gmap rule is executed and the {\tt
implies} statement is used to supply evidence from each match hypothesis to
the Gmap, that evidence is simply automatically added to the total Gmap
evidence rather than propagated using Dempster's probabilistic sum.

The following destructive rule is often used instead of the previous one to
give a significant speed up:

\begin{small}
\begin{verbatim}
   (rule ((:intern (GMAP ?gm) :var ?the-group))
       (setf (node-belief+ (gm-bms-node ?gm)) 0)
       (dolist (mh (gm-elements ?gm))
          (incf (node-belief+ (gm-bms-node ?gm))
                (node-belief+ (mh-bms-node mh)))))
\end{verbatim}
\end{small}

This rule bypasses the BMS entirely, thus increasing speed by not creating
justification links. It also renders the {\tt additive-nodes} distinction
irrelevant. However, such rules must be used with extreme caution.  For
example, the source of a Gmap's evidence cannot be inspected when using this
type of operation (see Section~\ref{ss:inspecting-evidence}).

\subsection{Making SME simulate structure-mapping theory}\index{rule system}\index{rule system, simulating structure-mapping theory}

The previous section examined the general structure of an \SME\ rule file.
In the process, the basic elements of the {\it structure-mapping-theory
analogy} rule set were presented. The {\it literal similarity} and
{\it mere-appearance} rules are essentially the same as the {\it analogy}
rules. They differ in the first match constructor rule. The {\it analogy}
rule set has the test {\tt (not (attribute? (expression-functor ?b)))} which is
absent from the corresponding {\it literal similarity} rule. Conversely, the
corresponding {\it mere appearance} rule forces the opposite condition
{\tt (attribute? (expression-functor ?b))}. One should consult Appendix A of
\cite{sme-aij} for listings of all three structure-mapping rule sets.

\subsection{Making SME perform as SPROUTER}\index{rule system}\index{rule system, simulating SPROUTER}

The {\tt SPROUTER} program \cite{hayes-roth} was
developed as an approach to the problem of inductively forming
characteristic concept descriptions. That is, given a sequence of events
(e.g., a list of pictures), produce a single, conjunctive description which
represents a generalized, characteristic description of the sequence. {\tt
SPROUTER} generalized a sequence of {\it N} descriptions by finding the
commonalities between the first two descriptions, generalizing these common
elements (i.e., variablize the literals), and then repeating the process
using this generalized description and the next, unprocessed description.
These steps would be repeated until the generalization had propagated
through the entire list of input descriptions.

\SME\ may be used to implement {\tt SPROUTER}'s {\it interference matching}
technique by giving it a set of match constructor rules which require all
matching predicates to have the same name (i.e., the {\it literal
similarity} rules without the condition that allows functions with different
names to match). The {\tt SPROUTER} generalization\index{generalize}
mechanism may then be implemented with the following algorithm:

\begin{small}
\begin{verbatim}
  Procedure SPROUTER (event-list)
     begin
       generalization := pop(event-list)
       while event-list
          pairwise-match := match(pop(event-list), generalization)
          generalization := generalize(pairwise-match)
       return generalization
     end
\end{verbatim}
\end{small}


\subsection{Relaxing the identical predicates constraint}\index{rule system}\index{rule system, relaxing identical predicates}

The current structure-mapping theory rules are sensitive to representation
by requiring that relational predicates match only if they are identical.
This is an important restriction that ensures the structures being compared
are semantically similar. However, it can also be overly restrictive.  We
are currently exploring different methods to relax the identicality
requirement while still maintaining a strong sense of semantic similarity.
One approach, called the {\it minimal ascension principle}, allows relations
to match if they share a common ancestor in a multi-root is-a hierarchy of
expression types \cite{Falken-phd} (i.e., the identicality test in the match
constructor rule is replaced by a call to {\tt
predicate-type-intersection?}). The local evidence score for their match is
inversely proportional (exponentially) to the relations' distance in the
hierarchy.  This enables \SME\ to match non-identical relations if such a
match is supported by the surrounding structure, while still maintaining a
strong preference for matching semantically close relations. This is similar
to approaches used in \cite{Burstein83,Greiner,Winston80}.

Problems with an unconstrained minimal ascension match technique are
discussed in \cite{Falken-phd}. A mapping approach which considers the
current context when determining pairwise similarity is also discussed.

\subsection{Pure isomorphisms}\label{ss:isomorphisms}
\index{rule system}\index{rule system, pure isomorphisms}

While it is important to assure that the structures being compared are
semantically similar, one can in principle remove all semantic comparisons.
This would allow match creation to be guided strictly by \SME's structural
consistency and 1-1 mapping criteria and match selection to be based
strictly on systematicity.

\begin{figure}
\border
\begin{small}
\begin{tt}
\begin{tabbing}
\ \ \ \ \ \ {\rm Addition}\ \ \ \ \ \ \ \ \ \ \ \ \ \ \ \ \ \ \ \ \ \ \ \ \ \ \ \ \ \ \ \ {\rm Union} \\
\ \ \ \= N1 + N2 $=$ N2 + N1\ \ \ \ \ \ \ \ \ \ \ \ \ \ \ \ \ \ \ \ \ \= S1 $\cup$ S2 $\equiv$ S2 $\cup$ S1 \\
\> N3+(N4+N5) $=$ (N3+N4)+N5 \> S3 $\cup$ [S4 $\cup$ S5] $\equiv$ [S3 $\cup$ S4] $\cup$ S5 \\
\> N6+0 $=$ N6            \> S6 $\cup$ $\emptyset$ $\equiv$ S6
\end{tabbing}
\end{tt}
\end{small}
  \caption{Formal descriptions for addition and union.}
  \label{fg:addition-union}
\border
\end{figure}


Consider the isomorphic mapping between the formal definitions of numeric
addition and set union shown in
Figure~\ref{fg:addition-union}.\footnote{This example is taken from an
advance copy of a paper by Holyoak and Thagard \cite{ACME}. I include it
here simply to demonstrate the range of matching preferences available in
\SME.} These formal descriptions may be given to \SME\ in the standard
manner, as in {\tt (plus N3 (plus N4 N5))} for the left side of the
associativity rule (the representation {\tt (plus N4 N5 result45)} also
works, although it results in a slightly longer run time due to the
flattening of structure).  When presented as formal definitions, the
concepts of addition and union are structurally isomorphic, independent of
the meaning of the predicates. Thus, while it could be argued that the
predicates {\tt plus} and {\tt union} share a certain degree of semantic
overlap, this example demonstrates that it is possible to make \SME\ ignore
predicates entirely and simply look for isomorphic mappings.  The rule set
for isomorphic mappings is shown in Figure~\ref{fg:isomorphism}. (This is
called the {\it ACME} rule set, as it configures \SME\ to emulate the {\tt
ACME} program on this example \cite{ACME}). The only constraint this rule
set enforces is that each predicate has the same number of arguments. While
it includes the Structure-Mapping notion of systematicity to prefer systems
of relations, it does not enforce identicality of predicates.  Using this
rule set, \SME\ produces the unique best mapping that we would expect
between the formal definitions of addition and union.

Since \SME\ enforces the ``same number of arguments'' restriction by
defeating any match hypotheses that are not structurally sound, we could in
principle effectively remove the rule file entirely. This could be done with
one match constructor rule to match everything with everything and one
evidence rule to measure systematicity. When this {\it free-for-all} rules
file was given to \SME, the same single best Gmap was produced, but at the
expense of increasing the run time from 13 seconds to 3.25 minutes.


\begin{figure}[t]
\border
\begin{small}
\begin{alltt}
   (MHC-rule (:filter ?b ?t :test (= (numargs (expression-functor ?b))
                                     (numargs (expression-functor ?t))))
       (install-MH ?b ?t))

   {\it ;;; Intern rule to match entities (non-commutative predicates)}
   (MHC-rule (:intern ?b ?t :test (and (expression? ?b) (expression? ?t)))
       (do ((bchildren (expression-arguments ?b) (cdr bchildren))
            (tchildren (expression-arguments ?t) (cdr tchildren)))
           ((or (null bchildren) (null tchildren)))
         (if (and (entity? (first bchildren)) (entity? (first tchildren)))
             (install-MH (first bchildren) (first tchildren)))))

   {\it ;;; Give a uniform initial priming to each MH}
   (initial-assertion (assert! 'initial-priming))

   (rule ((:intern (MH ?b ?t)))
      (rassert! (implies initial-priming (MH ?b ?t) (0.2 . 0.0))))

   {\it ;;;propagate interconnections - systematicity}
   (rule ((:intern (MH ?b1 ?t1) :test (and (expression? ?b1) (expression? ?t1)))
          (:intern (MH ?b2 ?t2) :test (children-of? ?b2 ?t2 ?b1 ?t1)))
      (rassert! (implies (MH ?b1 ?t1) (MH ?b2 ?t2) (0.8 . 0.0))))

   {\it ;;; Support from its MH's}
   (rule ((:intern (GMAP ?gm) :var ?the-group))
      (dolist (mh (gm-elements ?gm))
         (assert! `(implies ,(mh-form mh) ,?the-group))))
\end{alltt}
\end{small}
  \caption{Rule set for forming general isomorphic mappings.}
  \label{fg:isomorphism}\index{rule system}
\border
\end{figure}



\subsection{Imposing externally established pairings}\index{rule system}\index{rule system, analogical hints}

In certain situations, a number of entity and predicate mappings may already
be known prior to invoking \SME. These mappings may have been provided as an
analogical hint\index{analogical hint} from an instructor or derived by the
application program during earlier processing.  For example, \Phineas\
\cite{VBAL,MLW} uses \SME\ to analogically relate observed physical
phenomena to known theories of the world. \Phineas\ uses two analogical
mappings to learn about a new physical process. First, behavioral
correspondences are established (i.e., what entities and quantities are
behaving in the same manner). Second, the relevant base theories are
analogically mapped into the new domain, guided by the behavioral
correspondences. The two-stage mapping process solves the problem of using
analogy in cases where one does not have a pre-existing theory, as occurs
with truely novel learning. The assumption made in \Phineas\ is that similar
behaviors will have similar theoretical explanations. The first mapping
provides the correspondences between entities and functions required to
guide the importation of an old theory to explain a new domain in the second
mapping.

\SME\ includes facilities to simplify writing \Phineas-like programs, by
enabling the results of earlier processing to constrain subsequent mapping
tasks. These routines are divided into two categories, {\it declaration} and
{\it test}. The declaration routines tell \SME\ what predicate and entity
correspondences are known {\it a-priori}. The test routines enable the match
constructor rules to adhere to these imposed constraints. Known mappings are
declared through the following functions:

\spc{0.6}
\noindent
\begin{minipage}{\textwidth}
\begin{tt}
\begin{tabbing}
defGiven-Mappings  \` [{\it Macro}]   \\
\ \ \ \= entities \ \ \= (\= ({\it base-entity$_{1}$ target-entity$_{1}$})   \\
      \>              \>  \> ({\it base-entity$_{i}$ target-entity$_{j}$}) ...)  \\
      \> predicates   \> (({\it base-predicate$_{1}$ target-predicate$_{1}$})   \\
      \>              \>  \> ({\it base-predicate$_{i}$ target-predicate$_{j}$}) ...)
\end{tabbing}
\end{tt}
\end{minipage}
\decfunc{declare-given-mappings}{entities predicates}
\decfunc{clear-given-mappings}{}

\spc{0.8}
Both {\tt defGiven-Mappings} and {\tt declare-given-mappings} have identical
functionality. The first does not evaluate its arguments while the second
one does. Disjunctive constraints may be imposed by including all of the
possible pairings (e.g., defining both {\tt (}{\it base-entity$_{i}$
target-entity$_{j}$}{\tt )} and {\tt (}{\it base-entity$_{i}$
target-entity$_{k}$}{\tt )}).

Once a set of given mappings has been declared, the following test routines
may be used within the match constructor rules to enforce these mappings:

\spc{0.8}
\decfunc{sanctioned-pairing?}{base-item target-item}
\decfunc{paired-item?}{{\tt \&key base-item target-item}}

\spc{0.4}
\noindent
{\tt sanctioned-pairing?} tests if the given pair is one of the {\it a-prior}
pairings. {\tt paired-item?} takes either a base item or a target item and returns
true if the mapping for that item has been externally determined.

These functions help in writing rules which respect established mappings.
For example, the following two rules are used in the \Phineas\ system to
allow observed behavioral correspondences to constrain the mapping of the
relevant theory:

\begin{small}
\begin{verbatim}
(MHC-rule (:filter ?b ?t :test (and (eq (expression-functor ?b) (expression-functor ?t))
                                    (not (paired-item? :base-item (expression-functor ?b)))
                                    (not (paired-item? :target-item (expression-functor ?t)))))
    (install-MH ?b ?t))

(MHC-rule (:filter ?b ?t :test (sanctioned-pairing? (expression-functor ?b)
                                                    (expression-functor ?t)))
    (install-MH ?b ?t))
\end{verbatim}
\end{small}

When an analogy is being made between two behaviors, {\tt clear-given-mappings}
is used to make \SME\ perform in normal analogy mode. The discovered entity and
function correspondences are then given to {\tt declare-given-mappings} prior to
using \SME\ to map the relevant theory.\index{rule system}


\section{Representation Issues}\index{representation issues}

The proper representation becomes an issue in \SME\ due to its significant
impact on speed performance. Hierarchical representations provide an
important source of constraint on generating potential matches. They tend to
make the semantic interrelations explicit in the structure of the syntax.
For example, Section~\ref{ss:isomorphisms} described a comparison between
the laws of addition and union. There it was noted that part of the additive
associativity rule may be represented as {\tt (plus N3 (plus N4 N5))} or as
the pair {\tt (plus N4 N5 result45)} and {\tt (plus N3 result45
result3-45)}. The latter ``flat'' representation takes more time for \SME\
to process, sometimes a significant difference for large domain
descriptions. This is because the functional representation makes the
associativity rule structurally explicit, while the flat representation
buries it among the tokens appearing as arguments to {\tt plus}. However, it
is important to note that \SME\ is able to process domain descriptions in
any predicate-based format. It is simply speed considerations that render
standard, flat forms of representation undesirable.

Due to SME's ability to accept commutative, n-ary predicates, it is able to
match arbitrary sets (which must be of equal size at this time). This has
two consequences. First, the explicit use of sets becomes a viable form of
representation. Thus, a theory might be represented concisely as

\begin{verbatim}
   (Theory T1 (SET axiom-8 axiom-14 ...))
\end{verbatim}

\noindent
rather than as {\tt (Axiom-of T1 axiom-8)}, etc. Second, sets may be used to
add structure to descriptions. For example, the set representation for
theories results in greatly reduced run times compared to the non-set
representation.\footnote{The difference in speed is due to the operation of
merge step 2, which combines matches sharing a common base structure. The
set notation for theory {\tt T1} enables merge step 2 to know that matches
for {\tt axiom-8} and {\tt axiom-14} should be placed in the same gmap, thus
reducing the number of possibilities in merge step 3.} I am currently
investigating the use of a similar representation for temporal states, as in:

\begin{verbatim}
   (Situation S1 (SET (Increasing (Amount-of water1))
                      (Increasing (Pressure water1))
                      o o o ))
\end{verbatim}

\noindent
A \Phineas\ problem which took \SME\ 53 minutes (using {\tt (Increasing
(Amount-of (at water1 S1)))}) was reduced to 34 seconds using this more
structured representation.


\section{Using SME}

This section describes how to install \SME\ on your machine, load it, and
operate it.

\subsection{Installing SME}\index{installation}\label{ss:site-specifics}

\indent
\decvar{*sme-language-file*}\index{parameters, site specific}\index{site specific information}
\decvar{*sme-default-rules*}\index{site specific information, pathnames}
\decvar{*sme-rules-pathname*}
\decvar{*sme-dgroup-pathname*}

\spc{0.8}
To configure \SME\ to a particular site, a handful of variables storing
system directory information must be edited and set to the appropriate
values. These variables appear in {\tt config.lisp}, a separate file for
this purpose. Of primary importance are {\tt *sme-language-file*} and {\tt
*sme-default-rules*}. These are used by {\tt sme-init} to initialize the
language and rule systems. The two variables storing the rules and dgroup
pathnames are used by the user interface routines.

\spc{0.7}
\decvar{*the-lisp-package*}\index{packages,site specific}
\decvar{*the-user-package*}\index{site specific information, packages}

\spc{0.3}
In most Common Lisp implementations, one package exists for general user
definitions and another exists for the lisp implementation. It is important
to notify \SME\ what these are for the Common Lisp in use. For example, on a
Symbolics (version 6.2), the lisp package is called {\tt common-lisp} and
the user package is called {\tt cl-user}. These are the default settings.

\spc{0.7}
\decvar{*sme-system-pathname*}
\decvar{*sme-files*}

\spc{0.8}
These variables are used to automate compiling and loading. If the system is
being loaded on something other than a Symbolics or TI Explorer, the file
{\tt windowing} should not be included in the list {\tt *sme-files*}.
Otherwise, it should be left in the list of \SME\ files, which is the
default.

The \SME\ routines assume a set of naming conventions on domain description
and rule files.\index{conventions, file names} The names of files containing
domain descriptions ({\tt defDescription}) should end with a {\tt *.dgroup}
extension. The name of a file containing a rule set should end with the {\tt
*.rules} extension.


\subsection{Running SME}\index{running SME}\index{execution}

\indent
\decfunc{sme-init}{{\tt \&optional (}initialize-language? {\tt T) (}initialize-rules? {\tt T)}}

\spc{0.7}
This section gives a brief overview of the process of using \SME\ for
matching, generalization, and inspection tasks.


\begin{figure}
\border
\begin{footnotesize}
\begin{alltt}
{\bf\gthan\ (sme:sme-init)}
Initializing SME...
   Loading default language file: prof:>falken>sme>language
   Loading default rules file: prof:>falken>sme>literal-similarity.bin
Complete.
T
{\bf\gthan\ (load "prof:\gthan falken\gthan sme\gthan simple-water-flow.dgroup")}
Loading PROF:>falken>sme>simple-water-flow.dgroup into package USER
#P"PROF:>falken>sme>simple-water-flow.dgroup"
NIL
{\bf\gthan\ (load "prof:\gthan falken\gthan sme\gthan simple-heat-flow.dgroup")}
Loading PROF:>falken>sme>simple-heat-flow.dgroup into package USER
#P"PROF:>falken>sme>simple-heat-flow.dgroup"
NIL
{\bf\gthan\ (sme:match 'swater-flow 'sheat-flow T)}
              SME Version 2E
    Analogical Match from SWATER-FLOW to SHEAT-FLOW.

Rule File: literal-similarity.rules
--------------------------------------------------------------------------
                  | # Entities | # Expr. | Maximum order | Average order |
  Base Statistics |       4    |     11  |        3      |      1.36     |
Target Statistics |       4    |      6  |        2      |      1.17     |
--------------------------------------------------------------------------
# MH's | # Gmaps | Merge Step 3 | CI Generation | Show Best Only |
   14  |     3   |   ACTIVE     |    ACTIVE     |       OFF      |
------------------------------------------------------------------
Total Run Time:    0 Minutes,  0.821 Seconds
BMS Run Time:      0 Minutes,  0.530 Seconds
Best Gmaps: { 3 }

Match Hypotheses:
    (0.6320  0.0000)  (PIPE4 BAR7)
    (0.7900  0.0000)  (FLAT-WATER FLAT-COFFEE)
    {\it   o       o         ;a number of match hypotheses appeared here}
    (0.8646  0.0000)  (WATER1 COFFEE5)
    (0.7900  0.0000)  (LIQUID-WATER LIQUID-COFFEE)

Gmap #1:   (BEAKER2 COFFEE5)  (DIAM-BEAKER TEMP-COFFEE)  (VIAL3 ICE-CUBE6)
           (DIAM-VIAL TEMP-ICE-CUBE)  (>DIAMETER >TEMP)
  Emaps:  (BEAKER2 COFFEE5) (VIAL3 ICE-CUBE6)
  Weight: 3.937660
  Candidate Inferences:  { }

    {\it   o       o         ;Gmap #2 appeared here...}

Gmap #3:   (>PRESSURE >TEMP)  (PRESS-VIAL TEMP-ICE-CUBE)  (PRESS-BEAKER TEMP-COFFEE)
           (BEAKER2 COFFEE5)  (VIAL3 ICE-CUBE6)  (WATER1 HEAT8)
           (PIPE4 BAR7)  (WFLOW HFLOW)
  Emaps:  (BEAKER2 COFFEE5) (VIAL3 ICE-CUBE6) (WATER1 HEAT8) (PIPE4 BAR7)
  Weight: 5.991660
  Candidate Inferences:  (CAUSE >TEMP HFLOW)
\end{alltt}
\end{footnotesize}
  \caption{Initializing and running SME.}
  \label{fg:sme-trace}
\border
\end{figure}


\begin{enumerate}

\item {\it Loading the files.} To load or compile \SME, the file {\tt
config} should be loaded and then {\tt (load-sme)} or {\tt (compile-sme)}
called. A {\tt defSystem} definition is provided in {\tt system.lisp} for
Symbolics machines.

\item {\it System startup.} This stage is only appropriate for full, lisp
machine startup. The \SME\ window environment may be created with Select-S
on a Symbolics or System-S on a TI Explorer.

\item {\it Initialization.} The function {\tt sme-init} should be called to
initialize the database. If {\it initialize-language?} is non-nil, the
default language file (predicate definitions) will be loaded. If {\it
initialize-rules?} is non-nil, the default rules file will be loaded. Prior
to operating \SME, the language and rule systems must be established.

\item {\it Loading Dgroups.} Any description groups that are to be matched
must be declared. These declarations are typically stored in files, with the
extension {\tt *.dgroup}. If the windowing system is active, the command
{\tt Load Dgroup} will offer a menu of all {\tt *.dgroup} files in
{\tt *sme-dgroup-pathname*} to select what to load.

\item {\it Analogical mapping.} The function {\tt match} may be called to
form a mapping between two given Dgroups. This is discussed in
Section~\ref{ss:match}. If the windowing system is active, the command {\tt
Match} will offer a menu to select base and target Dgroups. It is prior to
this step that one might want to think about wether to modify any system
parameters (e.g., print the match hypotheses, print only the best Gmaps,
generate candidate inferences, etc.). 

\item {\it Describing Dgroups.} Once Dgroups are defined, the {\tt
describe-dgroup} facility will provide a description of any particular
Dgroup.

\item {\it Graphically displaying Dgroups.} If the windowing system is
active, Dgroups may also be displayed graphically, through the {\tt Display
Dgroup} utility.

\item {\it Generalizing.} Once a mapping is formed, it may be generalized
using the {\tt generalize} function or the {\tt Generalize} command in the
system menu.

\item {\it Saving the results of a session.} If the windowing system is
active, the results of commands like {\tt Match} and {\tt Generalize} are
sent to the scroll window by default. These results may be written to a file
using the {\tt dump-scroll} system utility.

\item {\it Comparing two apparently identical Gmaps.} When two Gmaps are
formed that appear to be identical, their differences can be identified
using the {\tt compare-gmaps} system utility.

\end{enumerate}

\noindent
A trace of \SME\ performing the basic mapping task is given in
Figure~\ref{fg:sme-trace}. Each of the other options are described in
greater detail in the following sections.


\subsection{Batch mode}
\index{batch mode}\index{report generation}

\indent
\decfunc{run-batch-file}{pathname {\tt \&key} {\tt (}gmap-display {\tt :all) (}gmap-statistics {\tt :none)}}
\decmacro{language-file}{pathname}
\decmacro{dgroup-directory}{pathname}
\decmacro{dgroup-file}{file-name}
\decmacro{rule-directory}{pathname}
\decmacro{rule-file}{file-name}
\decmacro{rule-sets}{{\tt \&rest} rule-file-names}
\decmacro{report-comments}{string}
\decmacro{send-report-to}{pathname {\tt \&key (}text-driver {\tt :LPR) (}style {\tt :STANDARD)}}
\decmacro{run-matcher-on}{base-name target-name}
\decmacro{defPostMatcher}{function}


\begin{figure}
\border
\begin{footnotesize}
\begin{alltt}
(sme:Dgroup-Directory "prof:>falken>sme>")

(sme:Dgroup-File "solar-system")
(sme:Dgroup-File "rutherford")
(sme:Dgroup-File "simple-water-flow")
(sme:Dgroup-File "simple-heat-flow")

(sme:Rule-Directory "prof:>falken>sme>")

(sme:Rule-Sets "literal-similarity" "true-analogy" "attribute-only")   {\it; iterate over each rule set}

(sme:Report-Comments "Sample run of SME to demonstrate batch mode.")

(sme:Send-Report-To "heath:>falken>sample.dmp" :text-driver :LATEX)

(sme:Run-Matcher-On solar-system rutherford-atom)                 {\it; map this pair once for each rule set}
(sme:Run-Matcher-On swater-flow sheat-flow)                       {\it; map this pair once for each rule set}
\end{alltt}
\end{footnotesize}
  \caption{Sample SME batch file.}
  \label{fg:batch-file}
\border
\end{figure}


\spc{0.8}
\SME\ is normally used as an interactive utility or as a module to some
larger program. However, when performing statistical analyses across a broad
space of matching preferences (i.e., rule sets) and domain descriptions, an
interactive format soon becomes inconvenient. Utilities are provided so that
a file of \SME\ instructions may be defined and then executed using {\tt
run-batch-file} (e.g., Figure~\ref{fg:batch-file}). This would instruct
\SME\ to perform a series of matches, potentially over a variety of rule
sets and domain descriptions, and generate a detailed report of the
execution and a summary of the results. When a single rule set is specified
using {\tt rule-file}, all subsequent matches (invoked by {\tt
run-matcher-on}) will use this rule file until another one is specified.
Using {\tt rule-sets}, one may instead specify a series of rule files to be
used, so that a single {\tt run-matcher-on} command will cause \SME\ to run
once for each rule file in the list. If a user-defined function name is
given to {\tt defPostMatcher}, this function will be called after each match
is performed, in case special post-match routines are desired or extra
information is to be added to the report being generated. A variety of text
drivers are supported for report generation ({\tt send-report-to}), such as
{\tt :lpr} (line printer), {\tt :latex}, and {\tt :troff}.


\subsection{Generalization mechanism}

\indent
\decfunc{generalize}{gmap}

\spc{0.8}
The {\tt generalize} function takes a global mapping structure and returns
three alternate generalizations (using the Common-Lisp {\tt values}
protocol), each one successively larger than the previous:


\begin{figure}
\border
\begin{small}
\begin{verbatim}
   Generalizations for Match from SWATER-FLOW to SHEAT-FLOW:

       Generalization #1 (Literally Common Aspects Only):
            (FLOW ENTITY6 ENTITY8 ENTITY13 ENTITY14)

       Generalization #2 (All Common Aspects Only):
            (FLOW ENTITY6 ENTITY8 ENTITY13 ENTITY14)
            (GREATER (FUNCTION0 ENTITY6) (FUNCTION0 ENTITY8))

       Generalization #3 (Maximal Generalization):
            (CAUSE (GREATER (FUNCTION0 ENTITY6) (FUNCTION0 ENTITY8))
                   (FLOW ENTITY6 ENTITY8 ENTITY13 ENTITY14))
\end{verbatim}
\end{small}
  \caption{\SME\ generalizations for the simple water flow -- heat flow analogy.}
  \label{fg:generalizations}
\border
\end{figure}


\begin{enumerate}
 \item {\it Literally common aspects only.} This generalization locates
those sub-structures which are identical in both base and target Dgroups.
This is a type of generalization typically found in inductive generalization
programs.
 \item {\it All common aspects only.} In addition to common, identical
substructures, this generalization includes cases where functions of a
different name were allowed to match. Where this occurs in the common
structure, a skolemized function predicate is created.
 \item {\it Maximal generalization.} The largest generalization (in terms of
amount) includes all candidate inferences sanctioned by the Gmap, as well as
the common substructure of generalization mode 2. This represents the entire
shared structure between the two Dgroups under the assumption that the
candidate inferences are valid.
\end{enumerate}

For example, given the best Gmap from the simple water flow -- heat flow
analogy described in Section 2.1 and shown in Figure~\ref{fg:wfhf-done},
\SME\ will produce the set of generalizations shown in
Figure~\ref{fg:generalizations}. The first generalization indicates that the
only thing in common between the two situations is the existence of flow. The
second generalization loosens the meaning of ``in common'' to include the
fact that a quantity associated with the source of flow was greater than the
same quantity measured for the destination. The final generalization assumes
that this inequality, which was the cause of flow in the water flow domain,
is actually the cause of flow for both situations.


\subsection{Inspecting MH and Gmap evidence}\label{ss:inspecting-evidence}

\indent
\decfunc{match-evidence-inspector}{}\index{match hypothesis, inspecting evidence justifications}\index{match hypothesis, inspecting evidence justifications}

\spc{0.8}
When developing a theory about what types of rules should be used and how
much evidence for a particular match they should provide, it is often useful
to explicitly see what the different sources of evidence were for a
particular match item. The system utility {\tt match-evidence-inspector} may
be used to display a trace of the entire evidence facility or just the
evidence for a particular match hypothesis or Gmap. For example, the
following information was printed out about the {\tt pressure} to {\tt
temperature} match hypothesis in the water flow -- heat flow analogy:

\begin{small}
\begin{verbatim}
   (MH F#PRESS-BEAKER F#TEMP-COFFEE) has evidence (0.7120, 0.0000) due to
      IMPLICATION((MH F#>PRESSURE F#>TEMP))  (0.5200, 0.0000)
      IMPLICATION(CHILDREN-POTENTIAL)  (0.4000, 0.0000)
\end{verbatim}
\end{small}
\noindent
While the following information appears for the best Gmap in this analogy:
\begin{small}
\begin{verbatim}
   (GMAP #GM3) has evidence (5.9917, 0.0000) due to
       IMPLICATION((MH F#WFLOW F#HFLOW))  (0.7900, 0.0000)
       IMPLICATION((MH I#PIPE20 I#BAR23))  (0.6320, 0.0000)
       IMPLICATION((MH I#WATER17 I#HEAT24))  (0.6320, 0.0000)
       IMPLICATION((MH I#VIAL19 I#ICE-CUBE22))  (0.9318, 0.0000)
       IMPLICATION((MH I#BEAKER18 I#COFFEE21))  (0.9318, 0.0000)
       IMPLICATION((MH F#PRESS-BEAKER F#TEMP-COFFEE))  (0.7120, 0.0000)
       IMPLICATION((MH F#PRESS-VIAL F#TEMP-ICE-CUBE))  (0.7120, 0.0000)
       IMPLICATION((MH F#>PRESSURE F#>TEMP))  (0.6500, 0.0000)
\end{verbatim}
\end{small}

\noindent
The inspection facility will not work for Gmaps if their scores were
produced by an external (to the {\tt BMS}), destructive operation. One such
destructive rule appeared at the end of Section~\ref{ss:ME-rules}.


\subsection{Windows}\index{lisp machine interface}\index{scroll windows}

\indent
\decfunc{dump-scroll-menu}{}\index{scroll windows, saving contents}
\decfunc{dump-scroll}{output-pathname}
\decfunc{clear-scroll}{}

\decfunc{select-windowing-configuration}{}
\decfunc{select-scroll}{}
\decfunc{select-double-scroll}{}
\decfunc{select-graphics}{}
\decfunc{select-large-graphics}{}
\decfunc{select-split}{}

\decvar{*sme-frame*}
\decvar{*graphics-pane*}
\decvar{*scroll-pane*}
\decvar{*spare-scroll-pane*}
\decvar{*lisp-pane*}

\spc{0.8} The windowing system is lisp machine dependent and appears in the
file {\tt windowing.lisp}. The loading of this file is optional. When the
windowing system is used, a number of window configurations are possible,
such as having a single scroll window, two side by side, a single graphics
window, or a scroll and graphics window side by side. These configurations
may be selected through their individual functions (e.g., {\tt
select-scroll}), or through the central configuration facility {\tt
select-scroll-graphics}.  By default, when the windowing system is active,
all \SME\ output is sent to the primary scroll pane.\index{scroll windows,
writing to} When both scroll windows are in use, the configuration facility
allows one to specify which scroll window is currently active. The two
scroll dumping routines write the contents of the primary scroll window to a
specified file. Output sent to the secondary ({\it scratchpad}) scroll pane
is for observation only and cannot be written to a file.


\subsection{System parameters}\index{parameters}

\indent
\decvar{*sme-parameters*}
\decvar{*parameter-menu-options*}
\decmacro{defSME-Parameter}{variable-name string-description type {\tt \&optional} type-choices}

\spc{0.8}
The {\tt defSME-Parameter} form adds a new variable to the list of known
\SME\ parameters. This list is used by the windowing interface routines to
query the user about possible parameter settings. It is provided primarily
for application programs wanting to use the standard \SME\ parameter setting
facility. The arguments to {\tt defSME-Parameter} correspond to the appropriate
definitions for the {\tt choose-variable-values} function of your particular
machine. For example, the following declaration appears in {\tt match.lisp}:
\begin{small}
\begin{verbatim}
   (defSME-Parameter *display-all-MH* "Display all the Match Hypotheses" :boolean)
\end{verbatim}
\end{small}


\begin{figure}
\border
\vspace{1.2in}
\caption{SME System Parameters.}
\label{fg:parameters}
\border
\end{figure}


\spc{0.6}
\decfunc{change-parms}{}

\spc{0.4}
The windowing system provides a menu facility for viewing and changing the
current values of system parameters. This menu is shown in
Figure~\ref{fg:parameters}.


\subsection{System utilities}\index{system utilities}

\indent
\decvar{*system-utilities-menu*}
\decmacro{defSME-Utility}{string-name lisp-form}
\decfunc{menu-utilities}{}
\decfunc{get-dgroup}{}
\decfunc{get-rules}{}

\spc{0.8}
The \SME\ system utilities are the options that appear when the {\tt
Utilities} command is evoked, the right mouse button is pressed, or the
function {\tt menu-utilities} is called. These utilities include changing
the system parameters, choosing to load a Dgroup or rule file from those in
the defined directories, and clearing or writting to file the contents of
the scroll window. These routines are lisp machine dependent. The following
declaration appears in {\tt match.lisp}:
\begin{small}
\begin{verbatim}
   (defSME-Utility "Inspect Evidence" (match-evidence-inspector))
\end{verbatim}
\end{small}


\section{User Hooks}

This section describes the global variables and routines that are available
to the user and application programs for the creation and inspection of
analogical mappings.

\subsection{Applications control over display}

\indent
\decvar{*sme-output-stream*}
\decvar{*windowing?*}
\decvar{*sme-graphics-output*}

\spc{0.8}
All \SME\ textual display routines send their output to {\tt
*sme-output-stream*}. By default, the value of this variable is {\tt T},
which causes output to be sent to {\tt *terminal-io*}, CommonLisp's default
pointer to the user's console. When {\tt *sme-output-stream*} is a scroll
window (determined by the presence of an {\tt :append-item} handler), the
appropriate scroll window routines for sending display items are invoked.
Otherwise, text is sent to the current output stream using {\tt format}. 
Text routines are machine independent.

In a similar manner, all \SME\ graphics output is sent to the current
{\tt *sme-graphics-output*} window. Graphics output is lisp machine
dependent and relies on the {\tt ZGRAPH} graphics system.

When the \SME\ windowing system is in operation, {\tt *sme-output-stream*}
is set to the primary \SME\ scroll pane ({\tt *scroll-pane*}), {\tt
*sme-graphics-output*} is set to the \SME\ graphics pane ({\tt
*graphics-pane*}), and {\tt windowing?} is set to {\tt T}.
\index{scroll windows, writing to} 

\spc{0.7}
\decmacro{sme-format}{format-string {\tt \&rest} format-args}
\decfunc{sme-print}{string}
\decfunc{sme-terpri}{{\tt \&optional (}N {\tt 1)}}

\spc{0.3}
These routines provide a general interface for sending textual output to the
current \SME\ output stream. {\tt sme-format} is equivalent to CommonLisp's
{\tt format} routine, except that the printed output is always followed by a
newline. The {\tt sme-print} routine is provided for simple situations where
only a string is printed or for situations requiring the standard use of
{\tt format}, as in building up a line of text through multiple invocations.
The printed output of {\tt sme-print} is followed by a newline. When the
routine is used for multiple calls of {\tt format}, it should be used in
conjunction with CommonLisp's {\tt with-output-to-string}, as in:

\begin{small}
\begin{verbatim}
   (sme-print
      (with-output-to-string (stream)
         (format stream "~%Beginning of a line...")
         (format stream "   middle of a line...")
         (format stream "   end of a line.")))
\end{verbatim}
\end{small}

When a whole set of operations are carried out within the context of a
single {\tt sme-print}, one must be careful not nest calls to {\tt
sme-print} (e.g., calling a function in the context of an {\tt sme-print}
which itself invokes {\tt sme-print}). Such nesting will cause output 
to appear backwards from what was intended and may cause the output stream
to close improperly.


\subsection{Useful miscellaneous functions}

Data exists within \SME\ in three forms: (1) local items such as entities,
predicates, and expressions, (2) description groups (Dgroups), and (3) analogical
mapping information. The routines to create and query these items are
described in the following sections.

\subsubsection{Entities, predicates, and expressions}

\indent
\decfunc{entity?}{item}
\decfunc{entity-type?}{item}
\decmacro{entity-name?}{symbol}\index{entity, inspection functions}
\decmacro{fetch-entity-definition}{symbol}
\decmacro{entity-domain}{symbol}
\decmacro{constant-entity?}{symbol}

\spc{0.8}
Entities declared through {\tt defEntity} represent global entity types,
that is, they represent a class of entities rather than an actual instance
of an entity. When an entity type is used in the definition of a description
group, a unique entity instance is created for that type (e.g., {\tt beaker}
is translated to {\tt beaker73}). Thus, a given entity token will represent
either a type or an instance. The structure predicate {\tt entity?} returns
true if the given item is an entity-instance structure, while {\tt
entity-type?} returns true if the item is an entity-type structure. The
macro {\tt entity-name?} returns true if the given symbol represents either
an entity type or instance. {\tt fetch-entity-definition} will return the
entity-type structure for an entity type token or the entity-instance
structure for an entity instance token. The routines {\tt entity-domain} and
{\tt constant-entity?} refer to the {\tt type} and {\tt constant?} keyword
values given to {\tt defEntity} when the corresponding entity type was
created.

\spc{1}
\decvar{*sme-predicates*}
\decmacro{fetch-predicate-definition}{predicate-symbol}\index{predicate, inspection functions}
\decmacro{predicate?}{symbol}
\decmacro{predicate-type}{predicate-symbol}
\decmacro{relation?}{predicate-symbol}
\decmacro{attribute?}{predicate-symbol}
\decmacro{function?}{predicate-symbol}
\decmacro{commutative?}{predicate-symbol}
\decmacro{n-ary?}{predicate-symbol}
\decmacro{arg-list}{predicate-symbol}
\decmacro{numargs}{predicate-symbol}
\decmacro{expression-type}{predicate-symbol}
\decmacro{eval-form}{predicate-symbol}

\spc{0.4}
These routines provide the facility to access the various predicate properties
defined with the {\tt defPredicate} form. {\tt fetch-predicate-definition}
returns the actual sme-predicate structure containing all the information
about a given predicate.

\spc{1}
\index{expression, inspection functions}
\decfunc{fetch-expression}{expression-name dgroup
                     {\tt \&optional (}error-if-absent? {\tt T)}}
\decfunc{expression-functor}{expression-structure}
\decfunc{fully-expand-expression}{expression-structure dgroup}
\decfunc{fully-expand-expression-form}{expression-form dgroup}

\spc{0.4}
An ``expression'' represents an actual predicate instance within a Dgroup.  Notice
that this includes terms corresponding to function applications as expressions.
Each use of a predicate gets its own expression with its own name, so that a
higher-order relation gets translated into several expressions, with some having
expressions as arguments. These routines allow one to retrieve and inspect expressions
in the database. {\tt fetch-expression} returns the expression structure with the given
name.

The routines {\tt fully-expand-expression} and {\tt fully-expand-expression-form} are
useful for examining the form of an expression. Typically, the expression {\tt
(greater-than (diameter beaker) 5)} is stored as the expression {\tt greater-than23},
which has the form {\tt (greater-than diameter24 5)}. These routines return
a fully expanded expression form, where all expression names are replaced by their
corresponding forms.


\subsubsection{Dgroups}

\indent
\decfunc{fetch-dgroup}{dgroup-name {\tt \&optional} create?}\index{description group, retrieval}
\decfunc{return-dgroup}{dgroup-or-dgroup-name}

\spc{0.8}
Description groups (Dgroups) are stored in a simple data base managed
primarily by routines in {\tt sme.lisp}. The general procedures for Dgroup
and expression creation were described in sections~\ref{ss:new-dgroups} and
\ref{ss:new-expressions}. A Dgroup may be retrieved by name using
{\tt fetch-dgroup}, or created if {\it create?} is non-nil and no Dgroup
with the given name currently exists. {\tt return-dgroup} is designed for
routines that may take either an actual Dgroup or simply a Dgroup name
(e.g., {\tt fetch-expression}). It will cause an error if the Dgroup does not
previously exist.

\spc{0.8}
\decfunc{describe-dgroup}{dgroup}\index{description group, textual description}
\decfunc{menu-display-dgroup}{}\index{description group, graphical display}
\decfunc{menu-display-pairs}{}

\spc{0.3}
A Dgroup may be textually described using {\tt describe-Dgroup}, which
writes to the \SME\ output stream. Graphical display is provided in the
windowing system by {\tt menu-display-dgroup} for a single Dgroup or {\tt
menu-display-pairs} for a display of two Dgroups side by side.


\subsubsection{Creating and inspecting global matches}\label{ss:match}

\indent
\decfunc{match}{base-name target-name {\tt \&optional} display?}\index{global mapping, creation}
\decfunc{best-gmaps}{{\tt \&optional (}gmaps {\tt *gmaps*) (}percentage-range {\tt 0.02)}}\index{global mapping, selecting the best}
\decfunc{display-match}{base target {\tt \&optional (}total-run-time {\tt 0) (}bms-run-time {\tt 0)}}\index{global mapping, textual display}

\spc{0.8}
The {\tt match} function is the central \SME\ procedure. Given the names of two
Dgroups, it forms the complete set of global mappings between them. If {\it display?}
is non-nil, a description of the results will be sent to the current \SME\
output stream. The function itself returns two values, the total run time of
the match process in seconds and the subset of that time spent running the
BMS evidence rules. The analogical mapping results are stored in the
following global variables, which are then accessible by the user or
application program.

\spc{0.5}
\decvar{*base*}
\decvar{*target*}
\decvar{*match-hypotheses*}\index{match hypothesis, *match-hypotheses*}
\decvar{*gmaps*}\index{global mapping, *gmaps*}
\spc{0.2}

\noindent
The Gmap(s) with the highest evaluation score are retrieved by
{\tt best-gmaps}, which returns all Gmaps having a score within a given
percentage (default is 2\%) of the highest score. {\tt best-gmaps} returns
two values: the list of best Gmaps and the actual real-valued highest score.

\spc{0.8}
\decfunc{compare-gmaps}{gmap1 gmap2 {\tt \&optional} display?}\index{global mapping, comparing apparently identical gmaps}

\spc{0.4}
Occasionally, \SME\ will produce two or more ``best'' Gmaps that appear to
be identical yet have been classified as distinct. When these Gmaps are
large, the ``here's the set of match hypotheses'' output format can make it
frustrating to find what the slight differences are between a pair of Gmaps.
When given two Gmap structures, {\tt compare-gmaps} will return (using {\tt
values}) the list of the match hypotheses that are uniquely part of the
first Gmap and a list of match hypotheses that are uniquely part of the
second Gmap (i.e., ({\it gmap1} - {\it gmap2}) and ({\it gmap2} - {\it
gmap1})). When the windowing system is active, this option is available
through the system utilities menu.


\section{Algorithm Internals}

This section quickly describes a few internal points of the program in case
one has specialized needs for interfacing to the code. It assumes knowledge
of CommonLisp and the realization that for many questions, the only feasible
answer must be to examine the \SME\ program.

\subsection{The Match Function}

\indent
\decfunc{create-match-hypotheses}{base-dgroup target-dgroup}
\decfunc{run-rules}{}\index{rule system, run-rules}
\decfunc{calculate-nogoods}{base-dgroup target-dgroup}
\decfunc{generate-gmaps}{}
\decfunc{gather-inferences}{base-dgroup target-dgroup}\index{candidate inference, gather-inferences}
\decfunc{intern-gmaps}{}\index{rule system, giving it access to gmaps}

\spc{0.8}
The function {\tt match} is primarily a sequence of calls to these
functions. {\tt create-match-hypotheses} runs the match constructor
rules to form the individual match hypotheses. The BMS evidence rules are
then run on these match hypotheses ({\tt run-rules}) and their
dependence and inconsistency relationships are determined ({\tt
calculate-nogoods}). The function {\tt generate-gmaps} executes the three
merge steps, resulting in the set of complete global mappings being placed
in the variable {\tt *gmaps*}. The candidate inferences each Gmap
sanctions is then calculated ({\tt gather-inferences}) and additive BMS
nodes for each Gmap are formed ({\tt intern-gmaps}), allowing the evidence
rules to run on each Gmap.

\subsection{Match Hypotheses}\index{match hypothesis, defstruct}

\makebox[\textwidth]{{\tt match-hypothesis} \hfill [{\it Defstruct}]}
\decsubst{mh-form}{match-hypothesis}
\decsubst{mh-base-item}{match-hypothesis}
\decsubst{mh-target-item}{match-hypothesis}
\decsubst{mh-bms-node}{match-hypothesis}
\decsubst{node-belief+}{bms-node}
\decsubst{mh-plist}{match-hypothesis}
\decvar{*match-hypotheses*}

\spc{0.8}
Most programs using \SME\ will need to interact with the {\it match
hypothesis structures}. Slots to this structure type use the {\tt mh-}
prefix. There are several slots that might be important. The MH form, which
is a list of {\tt (MH {\it $<$base-item$>$} {\it $<$target-item$>$})}, is found
using {\tt mh-form}. This is the form used for triggering the MH evidence
rules, and is asserted in the BMS for each match hypothesis. The base and
target items are expression or entity structures. The base item or target item may
be obtained directly using {\tt mh-base-item} and {\tt mh-target-item},
respectively.  The BMS node for each match hypothesis is found by {\tt
mh-bms-node}. In turn, the weight for that node may be obtained using {\tt
node-belief+}.  Finally, each match hypothesis structure has a property list
slot ({\tt mh-plist}) which may be useful for various purposes.

\subsection{Global Mappings}\index{global mapping, defstruct}

\makebox[\textwidth]{{\tt global-mapping} \hfill [{\it Defstruct}]}
\decsubst{gm-id}{global-mapping}
\decsubst{gm-elements}{global-mapping}
\decsubst{gm-emaps}{global-mapping}
\decsubst{gm-base}{global-mapping}
\decsubst{gm-target}{global-mapping}
\decsubst{gm-inferences}{global-mapping}
\decsubst{gm-bms-node}{global-mapping}
\decsubst{gm-plist}{global-mapping}
\decvar{*gmaps*}

\spc{0.8}
Each Gmap is stored as a {\it global-mapping structure}. Slots to this
structure type use the {\tt gm-} prefix. Each Gmap is assigned a unique
integer identifier, found through {\tt gm-id}. The Gmap form used by the BMS
is not explicitly available, but is asserted as {\tt (Gmap {\it
$<$gmap-structure$>$})}. The match hypotheses associated with a Gmap are stored
in {\tt gm-elements}, while the subset of these that are entity mappings is
stored in {\tt gm-emaps}. The candidate inferences sanctioned by the Gmap
appear in {\tt gm-inferences} and are stored as a simple list data type,
using the syntax defined in Section~\ref{ss:new-dgroups} for description
group expressions.

\subsection{Candidate Inference Generation}\index{candidate inference, alternate algorithms}

The original candidate inference generator, as described in \cite{aaai86},
will take any base structure ``intersecting the Gmap structure''. The newer
(V. 2) edition only takes base structure ``intersecting a Gmap root''.
Thus, the newer edition is more cautious and far more efficient than the
older edition. Both versions of the code are available (in {\tt
match.lisp}), with the default being the newer, more cautious version. There
are theoretical arguments for and against each approach. For example, one
might want to use only the inferences from the newer, more cautious approach
at first since they are supported by more target knowledge and thus more
likely valid. If an analogy proves fruitful, one may want to relax these
constraints, and use the older version to find out what additional
inferences might be made.

\subsection{Rule System}

\indent
\decfunc{tre-rules-file}{}\index{rule system, dynamically swapping rule sets}
\decfunc{tre-save-rules}{}
\decfunc{tre-init}{}
\decfunc{restore-rules}{}
\decfunc{run-rules}{}
\decvar{*tre-rules-saver*}
\decvar{*initial-assertions*}

\spc{0.8} 
When a new rules file is loaded, the sme-rules-file command at the top of
the file initializes the BMS rule system prior to loading the new set of
rules.  At the bottom of the rules file, the fresh, just-loaded set of rules
are saved in the global variable *tre-rules-saver* by the command
tre-save-rules.  This variable saves the status of the rule counters and the
list of initial rules. A similar variable, {\tt *initial-assertions*}, is
used to store all assertions appearing in the rules file. When the BMS is
run, new rules may be created and added to the known set of rules. As a
result, each time {\tt match} is invoked, the BMS is reinitialized to its
status just after loading the rule file, that is, it is restored to the
status indicated in {\tt *tre-rules-saver*}.

This facility may be used by application programs to save different rule
sets in memory and swap them as needed, without having to load rule files
each time. For example, suppose \SME\ is invoked, which will cause it to run
the current rule set for Gmap scoring. If a second scoring criterion is then
desired, a second set of rules may be invoked using code of the form:
\begin{small}
\begin{alltt}
   (let ((save-rules sme:*tre-rules-saver*)          {\it ;save the previous rule set}
         (save-assertions sme:*initial-assertions*))
     (setq sme:*tre-rules-saver* *my-other-rules-set-rules*)
     (setq sme:*initial-assertions* *my-other-rules-set-assertions*)
     (sme:tre-init)                                  {\it ;initialize with new rules}
     (sme:run-rules)                                 {\it ;run the new rule set}
     (setq sme:*tre-rules-saver* save-rules)         {\it ;restore the original rules}
     (setq sme:*initial-assertions* save-assertions))
\end{alltt}
\end{small}
This saves \SME's normal rule set, runs a different one, and then restores
the rule set to its previous value. In this example, {\tt tre-init} was
used, which fully initializes the BMS. If the desire is simply to supply
additional rules without destroying the current BMS state, {\tt
restore-rules} should be used in place of {\tt tre-init}.  The variables
corresponding to ``my-other-rules-set'' may be initialized by a similar
program which saves the current rule set, {\it loads} the desired ``other
rule set'' file, sets the ``my-other-rules-set'' variables from {\tt
*tre-rules-saver*} and {\tt *initial-assertions}, and then restores the
original rule set.


\section{Summary}

The \SME\ program has been described from the perspective of how to actually
use it. A number of methods have been presented about how to configure \SME\
to perform a variety of different types of matches.  It is hoped that \SME\
may serve as a general mapping tool for research on analogical mapping,
allowing researchers to focus on the more substantive issue of general
theories of analogical mapping, as opposed to worrying about implementation
details. The latter has the unfortunate effect of producing the repeated
scenario in which analogy researcher A goes to analogy researcher B and says
``My program can do X, which yours cannot'', followed by researcher B
returning a month later with this simple modification added. By testing
different theories within the same program, we may now compare the more
critical ``This is a logical consequence of my theory''. A program does not
a theory make. It can, however, function as a useful analytical tool.

Of course, not all of the problems of implementing analogical mapping have
been solved.  Most critical is redesigning the potentially combinatoric
merge step 3, perhaps using either a heuristic search or connectionist
relaxation network as suggested in \cite{sme-aij}. Of theoretical relevance
is the appropriateness of the abstract structural approach which \SME\
embodies.

\section{Acknowledgements}

The development of \SME\ has been a collaborative effort with Ken Forbus and
Dedre Gentner, with significant influence provided by Janice Skorstad. This
work has also benefited from discussions with Steve Chien, John Collins, and
Ray Mooney. 

This research is supported in part by an IBM graduate fellowship and in part
by the Office of Naval Research, Contract No. N00014-85-K-0559.

\begin{thebibliography}{99}\addcontentsline{toc}{section}{References}

 \bibitem{Burstein83}
Burstein, M., Concept formation by incremental analogical reasoning and
debugging, {\it Proceedings of the Second International Workshop on
Machine Learning}, University of Illinois, Monticello, Illinois, June,
1983.  A revised version appears in {\it Machine Learning: An Artificial
Intelligence Approach Vol. II}, R.S. Michalski, J.G. Carbonell, and T.M.
Mitchell (Eds.), Morgan Kaufman, 1986.
 \bibitem{Carbonell83}
Carbonell, J.G., Learning by Analogy: Formulating and generalizing plans
from past experience, in: {\it Machine Learning: An Artificial
Intelligence Approach}, R.S. Michalski, J.G. Carbonell, and T.M.
Mitchell (Eds.), Morgan Kaufman, 1983.
 \bibitem{BMS}
Falkenhainer, B.,
Towards a general-purpose belief maintenance system, in: J.F. Lemmer \& L.N.
Kanal (Eds.), {\it Uncertainty in Artificial Intelligence, Volume II}, 1987.
Also Technical Report, UIUCDCS-R-87-1717, Department of Computer Science,
University of Illinois, 1987.
 \bibitem{VBAL}
Falkenhainer, B.,
An examination of the third stage in the analogy process:
Verification-Based Analogical Learning, Technical Report
UIUCDCS-R-86-1302, Department of Computer Science, University of Illinois,
October, 1986. A summary appears in {\it Proceedings of the Tenth
International Joint Conference on Artificial Intelligence}, Milan, Italy,
August, 1987.
 \bibitem{MLW}
Falkenhainer, B.,
Scientific theory formation through analogical inference, {\it
Proceedings of the Fourth International Machine Learning Workshop}, Irvine,
CA, June, 1987.
 \bibitem{DBR}
Falkenhainer, B.
The utility of difference-based reasoning, {\it Proceedings of the Seventh
National Conference on Artificial Intelligence}, St. Paul, August, 1988.
 \bibitem{Falken-phd}
Falkenhainer, B., Learning from physical analogies: A study in analogy
and the explanation process, Ph.D. Thesis, University of Illinois,
December, 1988.
 \bibitem{aaai86}
Falkenhainer, B., K.D. Forbus, D. Gentner,
The Structure-Mapping Engine, {\it Proceedings of the Fifth National
Conference on Artificial Intelligence}, August, 1986.
 \bibitem{sme-aij}
Falkenhainer, B., K.D. Forbus, D. Gentner,
The Structure-Mapping Engine:  Algorithm and Examples, Technical Report
UIUCDCS-R-87-1361, Department of Computer Science, University of Illinois,
July, 1987. To appear in {\it Artificial Intelligence}.
 \bibitem{framework}
Forbus, K.D. and D. Gentner.
Learning physical domains: Towards a theoretical framework,
in: {\it Machine Learning: An Artificial Intelligence Approach Vol. II}, R.S. Michalski,
J.G. Carbonell, and T.M. Mitchell (Eds.), Morgan Kaufmann, 1986.
 \bibitem{Gentner80}
Gentner, D.
{\it The Structure of Analogical Models in Science}, BBN Tech.
Report No. 4451, Cambridge, MA., Bolt Beranek and Newman Inc., 1980.
 \bibitem{smt-cogsci}
Gentner, D.
Structure-Mapping: A Theoretical Framework for Analogy, {\it Cognitive
Science} {\bf 7}(2), 1983. 
 \bibitem{allerton}
Gentner, D.
Mechanisms of analogy.  To appear in S. Vosniadou and A. Ortony, (Eds.),
{\em Similarity and analogical reasoning}, Cambridge University Press,
Oxford.
 \bibitem{analogica}
Gentner, D.
Analogical inference and analogical access, To appear in A. Preiditis (Ed.),
{\em Analogica: Proceedings of the First Workshop on Analogical Reasoning},
London, Pitman Publishing Co.  Presented in December, 1986.
 \bibitem{tinlap3}
Gentner, D., B. Falkenhainer, and J. Skorstad
Metaphor: The good, the bad and the ugly. {\it Proceedings of the Third
Conference on Theoretical Issues in Natural Language Processing}, Las
Cruces, New Mexico, January, 1987.  
 \bibitem{Greiner}
Greiner, R., Learning by understanding analogies, PhD Thesis (STAN-CS-1071),
Department of Computer Science, Stanford University, September, 1985.
 \bibitem{hayes-roth}
Hayes-Roth, F., J. McDermott, An interference matching technique for
inducing abstractions, {\it Communications of the ACM}, {\bf 21}(5),
May, 1978.
 \bibitem{ACME}
Holyoak, K., \& Thagard, P. Analogical mapping by constraint satisfaction,
June, 1988, ({\it submitted for publication}).
 \bibitem{Indurkhya}
Indurkhya, B., Constrained semantic transference: A formal theory of
metaphors, Technical Report 85/008, Boston University, Department of
Computer Science, October, 1985.
 \bibitem{KedarCabelli}
Kedar-Cabelli, S., Purpose-directed analogy. {\it Proceedings of the Seventh
Annual Conference of the Cognitive Science Society}, Irvine, CA, 1985.
 \bibitem{Reed}
Reed, S.K., A structure-mapping model for word problems. Paper presented
at the meeting of the Psychonomic Society, Boston, 1985.
 \bibitem{Rumelhart}
Rumelhart, D.E., \& Norman, D.A., Analogical processes in learning.  In J.R.
Anderson (Ed.), {\it Cognitive skills and their acquisition}, Hillsdale,
N.J., Erlbaum, 1981.
 \bibitem{aaai87}
Skorstad, J., B. Falkenhainer and D. Gentner
Analogical Processing: A simulation and empirical corroboration, {\it
Proceedings of the Sixth National Conference on Artificial Intelligence},
Seattle, WA, August, 1987.
 \bibitem{Winston80}
Winston, P.H., Learning and Reasoning by Analogy, {\it Communications of the
ACM}, {\bf 23}(12), 1980.
 \bibitem{Winston82}
Winston, P.H., Learning new principles from precedents and exercises,
{\it Artificial Intelligence}, {\bf 19}, 321-350, 1982.

\end{thebibliography}

\newpage
\addcontentsline{toc}{section}{Index}

\begin{theindex}

\item algorithm, 4-10
	\subitem  summary, 3
\item analogical hint, 17
\item arg-list macro, 28
\item {\tt assert!} function, 14
\item {\tt attribute?} macro, 28
\indexspace
\item {\tt *base*} variable, 29
\item batch mode, 22
\item {\tt best-gmaps} function, 29
\indexspace
\item {\tt calculate-nogoods} function, 30
\item candidate inference, 9
	\subitem  alternate algorithms, 31
	\subitem  gather-inferences, 30
\item {\tt change-parms} function, 26
\item {\tt children-of?} function, 15
\item {\tt clear-given-mappings} function, 18
\item {\tt clear-scroll} function, 25
\item {\tt commutative?} macro, 28
\item {\tt compare-gmaps} function, 29
\item {\tt constant-entity?} macro, 27
\item conventions, 1
	\subitem  file names, 20
	\subitem  function names, 1
\item {\tt create-match-hypotheses} function, 30
\indexspace
\item {\tt declare-given-mappings} function, 18
\item {\tt defDescription} macro, 12
\item {\tt defEntity} macro, 11
\item {\tt define-description} function, 12
\item {\tt define-Entity} function, 11
\item {\tt define-predicate} function, 10
\item {\tt defPostMatcher} macro, 23
\item {\tt defPredicate} macro, 10
\item {\tt defSME-Parameter} macro, 25
\item {\tt defSME-Utility} macro, 26
\item {\tt describe-dgroup} function, 29
\item description group, 11-12
	\subitem  graphical display, 29
	\subitem  retrieval, 29
	\subitem  textual description, 29
\item Dgroup, 12
\item {\tt dgroup-directory} macro, 22
\item {\tt dgroup-file} macro, 22
\item {\tt display-match} function, 29
\item {\tt dump-scroll} function, 25
\item {\tt dump-scroll-menu} function, 25
\indexspace
\item entity, 11
	\subitem  declaring, 11
\item {\tt entity?} function, 27
	\subitem  inspection functions, 27
\item {\tt entity-domain} macro, 27
\item {\tt entity-name?} macro, 27
\item {\tt entity-type?} function, 27
\item {\tt eval-form} macro, 28
\item execution, 20
\item expression, 12
	\subitem  adding new expressions, 12
\item {\tt expression} function, 12
	\subitem  inspection functions, 28
\item {\tt expression-functor} function, 28
\item {\tt expression-type} macro, 28
\indexspace
\item {\tt fetch-dgroup} function, 29
\item {\tt fetch-entity-definition} macro, 27
\item {\tt fetch-expression} function, 28
\item {\tt fetch-predicate-definition} macro, 28
\item file organization, 1
\item {\tt fully-expand-expression} function, 28
\item {\tt fully-expand-expression-form} function, 28
\item {\tt function?} macro, 28
\indexspace
\item {\tt gather-inferences} function, 30
\item generalize, 16
\item {\tt generalize} function, 23
\item {\tt generate-gmaps} function, 30
\item {\tt get-dgroup} function, 26
\item {\tt get-rules} function, 26
\item global mapping, 7-10
	\subitem  comparing apparently identical gmaps, 29
	\subitem  creation, 29
	\subitem  defstruct, 30
	\subitem  {\tt *gmaps*}, 29
	\subitem  scoring, 9, 15
	\subitem  selecting the best, 29
	\subitem  textual display, 29
\item {\tt *gmaps*} variable, 29, 31
\item {\tt gm-base} subst, 31
\item {\tt gm-bms-node} subst, 31
\item {\tt gm-elements} subst, 31
\item {\tt gm-emaps} subst, 31
\item {\tt gm-id} subst, 30
\item {\tt gm-inferences} subst, 31
\item {\tt gm-plist} subst, 31
\item {\tt gm-target} subst, 31
\item {\tt *graphics-pane*} variable, 25
\indexspace
\item {\tt initial-assertion} macro, 14
\item {\tt *initial-assertions*} variable, 31
\item installation, 20
\item {\tt install-MH} function, 13
\item {\tt intern-gmaps} function, 30
\indexspace
\item {\tt language-file} macro, 22
\item lisp machine interface, 25
	\subitem  predicate documentation, 11
\item {\tt *lisp-pane*} variable, 25
\indexspace
\item match constructor rules, 13
\item match evidence rules, 14
\item {\tt match} function, 29
\item match hypothesis, 4-5
	\subitem  defstruct, 30
	\subitem  inspecting evidence justifications, 24
	\subitem  installing, 13
	\subitem  {\tt *match-hypotheses*}, 29
	\subitem  scoring, 14-15
\item {\tt match-evidence-inspector} function, 24
\item {\tt *match-hypotheses*} variable, 29-30
\item {\tt menu-display-dgroup} function, 29
\item {\tt menu-display-pairs} function, 29
\item {\tt menu-utilities} function, 26
\item {\tt mh-base-item} subst, 30
\item {\tt mh-bms-node} subst, 30
\item MHC-rule, 13
\item {\tt mh-form} subst, 30
\item {\tt mh-plist} subst, 30
\item {\tt mh-target-item} subst, 30
\indexspace
\item {\tt n-ary?} macro, 28
\item {\tt node-belief+} subst, 30
\item {\tt numargs} macro, 28
\indexspace
\item packages, 1
	\subitem site specific, 20
\item {\tt paired-item?} function, 19
\item {\tt *parameter-menu-options*} variable, 25
\item parameters, 25
	\subitem  site specific, 1, 20
\item predicate, 10
	\subitem  declaring, 10
	\subitem  inspection functions, 28
\item {\tt predicate?} macro, 28
\item {\tt predicate-type} macro, 28
\indexspace
\item {\tt rassert!} macro, 14
\item {\tt relation?} macro, 28
\item report generation, 22
\item {\tt report-comments} macro, 23
\item representation issues, 19
\item {\tt restore-rules} function, 31
\item {\tt return-dgroup} function, 29
\item {\tt rule} macro, 14
\item rule system, 12-19, 19
	\subitem  analogical hints, 17
	\subitem  dynamically swapping rule sets, 31
	\subitem  giving it access to gmaps, 30
	\subitem  match constructor rules, 13
	\subitem  match evidence rules, 14
	\subitem  pure isomorphisms, 16
	\subitem  relaxing identical predicates, 16
	\subitem  rule file syntax, 13
	\subitem  {\tt run-rules,} 30
	\subitem  simulating SPROUTER, 16
	\subitem  simulating structure-mapping theory, 15
\item {\tt rule-directory} macro, 22
\item {\tt rule-file} macro, 22
\item {\tt rule-sets} macro, 22
\item {\tt run-batch-file} function, 22
\item {\tt run-matcher-on} macro, 23
\item running SME, 20
\item {\tt run-rules} function, 30-31
\indexspace
\item {\tt sanctioned-pairing?} function, 19
\item scroll windows, 25
	\subitem  saving contents, 25
	\subitem  writing to, 25, 27
\item {\tt *scroll-pane*} variable, 25
\item {\tt select-double-scroll} function, 25
\item {\tt select-graphics} function, 25
\item {\tt select-large-graphics} function, 25
\item {\tt select-scroll} function, 25
\item {\tt select-split} function, 25
\item {\tt select-windowing-configuration} function, 25
\item {\tt send-report-to} macro, 23
\item site specific information, 20
	\subitem  packages, 20
	\subitem  pathnames, 20
\item {\tt *sme-default-rules*} variable, 20
\item {\tt *sme-dgroup-pathname*} variable, 20
\item {\tt *sme-files*} variable, 20
\item {\tt sme-format} macro, 27
\item {\tt *sme-frame*} variable, 25
\item {\tt *sme-graphics-output*} variable, 26
\item {\tt sme-init} function, 20
\item {\tt *sme-language-file*} variable, 20
\item {\tt *sme-output-stream*} variable, 26
\item {\tt *sme-parameters*} variable, 25
\item {\tt *sme-predicates*} variable, 28
\item {\tt sme-print} function, 27
\item {\tt sme-rules-file} function, 13
\item {\tt *sme-rules-pathname*} variable, 20
\item {\tt *sme-system-pathname*} variable, 20
\item {\tt sme-terpri} function, 27
\item {\tt *spare-scroll-pane*} variable, 25
\item system utilities, 26
\item {\tt *system-utilities-menu*} variable, 26
\indexspace
\item {\tt *target*} variable, 29
\item {\tt *the-lisp-package*} variable, 20
\item {\tt *the-user-package*} variable, 20
\item {\tt tre-init} function, 31
\item {\tt tre-rules-file} function, 31
\item {\tt *tre-rules-saver*} variable, 31
\item {\tt tre-save-rules} function, 13, 31
\indexspace
\item {\tt *windowing?*} variable, 26
\indexspace
\end{theindex}

%%\batchmode
\documentstyle[11pt,alltt]{article}
\pagestyle{myheadings}
\makeindex
\markboth{SME User's Manual}{SME User's Manual}
\setlength{\oddsidemargin}{-0.1in}
\setlength{\evensidemargin}{0in}
\setlength{\textwidth}{6.5in}
\setlength{\topmargin}{-0.3in}
\setlength{\textheight}{9in}
%% Declaring SME lisp hooks
\newcommand{\decvar}[1]{\noindent\makebox[\textwidth]{{\tt #1}
                                   \hfill [{\it Variable}]}\index{#1 variable}}
\newcommand{\decfunc}[2]{\noindent\makebox[\textwidth]{{\tt #1}\ \ {\it #2}
                                               \hfill [{\it Function}]}\index{#1 function}}
\newcommand{\decmacro}[2]{\noindent\makebox[\textwidth]{{\tt #1}\ \ {\it #2}
                                                \hfill [{\it Macro}]}\index{#1 macro}}
\newcommand{\decsubst}[2]{\noindent\makebox[\textwidth]{{\tt #1}\ \ {\it #2}
                                                \hfill [{\it Subst}]}\index{#1 subst}}
%% Handy abbreviations
\newcommand{\metasyntax}[1]{$\langle #1 \rangle$}
\newcommand{\spc}[1]{\vspace{#1\baselineskip}}
\newcommand{\gthan}{\( >\)}
\newcommand{\border}{\rule{\textwidth}{0.04in}}
\newcommand{\SME}{{\tt SME}}
\newcommand{\SMET}{\mbox{${\tt SME}_{{\it SMT}}$}}
\newcommand{\Phineas}{{\tt PHINEAS}}
\newcommand{\keypoint}[2]{\begin{itemize}
                             \item \parbox[t]{0.88\textwidth}{{\it #1:} #2}
                          \end{itemize}
                          \vspace{0.08in}}
%
\begin{document}

\begin{titlepage}
{\center {\Large Technical Report UIUCDCS-R-88-1421 \\ }}
\ \ \ \vspace{0.5in} \\
{\center {\bf {\huge The SME User's Manual} \\[2ex]
              {\LARGE  (SME Version 2E)} \\ }}
\vspace{0.7in}
\begin{center}
            {\Large Brian Falkenhainer}       \\[2ex]
{\large     Qualitative Reasoning Group       \\
          Department of Computer Science      \\
         University of Illinois at Urbana-Champaign \\
\vspace{0.75in}
                December, 1988 \\ }
\end{center}
\vspace{0.75in}
\begin{abstract}
This paper documents the {\it Structure-Mapping Engine} (\SME), a
general-purpose program for studying analogical processing.  It provides a
comprehensive description of the program and instructions for using it,
including techniques for integrating it into larger systems. One section
demonstrates methods for configuring \SME\ to a variety of mapping
preferences and suggests the range of theoretical variations available.
\end{abstract}

\end{titlepage}
\pagenumbering{roman}
\tableofcontents
\markboth{SME User's Manual}{SME User's Manual}
\newpage
\pagenumbering{arabic}

\section{Introduction}

The Structure-Mapping Engine (\SME) is a general tool for performing various
types of analogical mappings. \SME\ was originally developed to simulate
Gentner's {\it Structure-Mapping} theory of analogy
\cite{smt-cogsci,allerton,analogica}. It was hoped that the developed system
would also be able to model the other types of similarity comparisons
sanctioned by Gentner's theory, such as {\it literal similarity} and {\it
mere appearance}. What ended up being developed was an extremely flexible
and efficient system. Most theoretical assumptions are left out of the
program and are supplied through match rules. Thus, while \SME\ was
originally designed to simulate the comparisons of structure-mapping theory,
it may simulate many others as well. Given a set of theoretical restrictions
on what constitutes a reasonable analogical mapping, one may implement these
restrictions in the form of rules and use \SME\ to interactively test their
consequences. This report is intended to make that task easier.

This paper is designed for those interested in using \SME\ for studying
analogical processing, testing alternate theories, or as the mapping
component in a larger system. It describes the options and user support
provided in \SME, how to use it for testing theories, and how to integrate
it with other programs. For a discussion of the theory behind \SME, the
general algorithm, and descriptions of the program in operation, one should
consult \cite{aaai86,sme-aij} prior to reading this manual. Descriptions of
the use of \SME\ in various research projects may be found in
\cite{VBAL,MLW,DBR,Falken-phd,allerton,tinlap3,aaai87}, while descriptions
of Gentner's Structure-Mapping theory appear in
\cite{Gentner80,smt-cogsci,allerton,analogica,framework,sme-aij}.

\subsection{Conventions}\index{conventions}

Throughout this guide, a few conventions will be used which should be
explained at this time.

\begin{enumerate}

 \item {\it CommonLisp Packages}.\index{packages} The \SME\ system resides
in its own package, {\tt SME}, which is defined to use CommonLisp. As a
result, any reference to an \SME\ function or variable must specify the {\tt
SME} package, as in the function {\tt sme:define-predicate}. To simplify the
discussion, we will omit the package prefix when describing \SME\ functions,
macros, and variables. In addition, while the \SME\ routines reside in the
{\tt SME} package, the structures it manipulates reside in the general {\tt
USER} package.
 \item {\it The declarative interface.}\index{conventions, function names}
In general, the routines used to present data items to \SME, such as
predicate definitions and concept descriptions, appear in two,
functionally-equivalent forms. These two types have a naming convention
associated with each. The most common type is the {\it declarative} or macro
interface. The declarative routines do not evaluate their arguments and
match the syntactic form {\tt def}{\it routine-name}. For example, to define
the entity {\tt sun} declaratively, one writes {\tt (defEntity sun)}. The
second type is the {\it functional} interface, which is present to support
declarations by external programs.  These routines evaluate their arguments
and match the syntactic form {\tt define-}{\it routine-name}. For example,
to define the entity {\tt sun} functionally, one writes {\tt (define-entity
'sun)}.

\end{enumerate}


\subsection{File Organization}\index{file organization}

\SME\ is contained within the following twelve files:

\begin{description}
 \item[config.lisp]  The declarations for site specific
parameters.\index{parameters, site specific}
 \item[defs.lisp]  The basic structure definitions and macros used throughout
\SME. 
 \item[bits.lisp]  Routines for creating and manipulating bit vectors.
 \item[bms.lisp]  The belief-maintenance system (BMS) - a probabilistic TMS.
 \item[bms-tre.lisp]  The rule system and problem-solver front end for the BMS.
 \item[sme.lisp]  The \SME\ top-level routines, such as initialization,
defining facts about a concept, and fetching and storing facts and concept
descriptions.
 \item[match.lisp]  The \SME\ mapping algorithm.
 \item[match-rules-support.lisp]  A few functions useful for writing \SME\
match rules.
 \item[display.lisp]  Machine independent output routines.
 \item[windowing.lisp]  Symbolics dependent interface routines.
 \item[batch.lisp]  Routines to enable execution in {\it batch} mode with a
final report generated.
 \item[generalize.lisp]  Inductive generalization support.
\end{description}


\section{System Review}

The Structure-Mapping Engine can simulate a class of {\it structural
approaches} to analogical mapping. In these approaches, there is a distinct
stage of matching and carryover of predicates from one domain (the {\it
base}) into another (the {\it target}) within the larger analogy process.
Furthermore, although there are a number of differences, there is widespread
agreement among these techniques on one fundamental restriction
\cite{Burstein83,Carbonell83,Greiner,Indurkhya,KedarCabelli,Reed,Rumelhart,Winston82}:
\begin{enumerate}
 \item {\it Structural consistency}. If a final analogical mapping includes
a predicate in the base paired with a predicate in the target, then it must
also include corresponding pairings between each of their arguments. This
criterion simply asserts that an analogical mapping must not produce
syntactically meaningless predicate calculus forms.
\end{enumerate}
In \SME, this restriction was enforced by the requirement of simulating
Structure-Mapping theory; its impact on the algorithm is described in
\cite{sme-aij}. However, this restriction is only part of that theory and
alone does not uniquely define a matching algorithm. Additional theoretical
restrictions must be supplied through {\it match rules}. This enables \SME\
to be used in exploring the space of theories consistent with this single
criterion. An additional restriction is enforced by default:

\keypoint{One-to-one mapping}{No base item (predicate or object) may be
paired with multiple target items. Likewise, no target item may be paired
with multiple base items.}

Enforcement of the one-to-one restriction is a global parameter which may be
disabled. Support is provided to implement variations of one-to-one within
the match rules.

Match rules specify what pairwise matches are possible and provide local
measures of evidence used in computing the evaluation score. These rules are
the key to \SME's flexibility. To build a new matcher one simply loads a new
set of match rules.  This has several important advantages.  First, we can
simulate all of the types of comparisons sanctioned by Structure-Mapping
theory with one program.  Second, the rules could in theory be ``tuned'' if
needed to simulate particular kinds of human performance. Third, a variety
of other analogical mapping systems may be simulated for comparison and
theoretical investigation.  The breadth of the space of these structural
approaches is suggested by the examples in Section~\ref{ss:rule-system}.

In this section, the \SME\ matching algorithm is briefly reviewed, followed
by a short discussion of how theoretical guidelines may be added to the
general mechanism. It is a summary of the algorithm description appearing in
\cite{sme-aij}, annotated with the \SME\ functions that carry out each step.


\begin{figure}
\border
\vspace{2.1in}
\caption{Simplified water flow and heat flow descriptions.}
\label{fg:wfhf-calculus}
\border
\end{figure}


\subsection{Algorithm Review}\index{algorithm, summary}

Given descriptions of a base and a target (called Dgroups), \SME\ builds all
structurally consistent interpretations of the comparison between them.
Each interpretation of the match is called a {\em global mapping}, or {\em
Gmap}.  Gmaps consist of three parts:

\begin{enumerate}
\item{\em Correspondences:} A set of pairwise matches between the
expressions and entities of the two dgroups.
\item {\em Candidate Inferences:} A set of new expressions which the
comparison suggests holds in the target dgroup.
\item {\em Structural Evaluation Score:} (Called {\em SES} for
brevity) A numerical estimate of match quality.
\end{enumerate}

For example, given the descriptions of water flow and heat flow shown in
Figure~\ref{fg:wfhf-calculus}, \SME\ might, depending on the current
theoretical configuration, offer several alternative interpretations for
this potential analogy.  In one interpretation, the central inference is
that water flowing from the beaker to the vial corresponds to heat flowing
from the coffee to the ice cube. Alternatively, one could map water to
coffee, since they are both liquids.


\begin{figure}
\border
\begin{small}
\begin{tt}
\begin{itemize}
 \item Run MHC rules to construct match hypotheses (create-match-hypotheses).
 \item Calculate the {\it Conflicting} set for each match hypothesis (calculate-nogoods).
 \item Calculate the {\it EMaps} and {\it NoGood} sets for each match
        hypothesis by upward propagation from entity mappings
        (generate-justifications and propagate-descendants).
 \item During the propagation, delete any match hypotheses that have
        justification holes (propagate-death).
 \item Merge match hypotheses into Gmaps (generate-gmaps).
      \begin{enumerate}
         \item Interconnected and consistent (generate-structure-groups).
         \item Consistent members of same base structure (merge-base).
         \item Any further consistent combinations (full-gmap-merge).
      \end{enumerate}
 \item Calculate the candidate inferences for each GMap (gather-inferences).
 \item Score the matches (run-rules).
      \begin{enumerate}
         \item Local match scores.
         \item Global structural evaluation scores.
      \end{enumerate}
\end{itemize}
\end{tt}
\end{small}
  \caption{Summary of SME algorithm.}
  \label{fg:sme-overview}
\border
\end{figure}


The \SME\  algorithm (see Figure~\ref{fg:sme-overview}) is logically divided
into four stages:
\begin{enumerate} 
\item {\em Local match construction:} Finds all pairs of ({\it BaseItem},
{\it TargetItem}) that potentially can match.  A {\em Match Hypothesis} is
created for each such pair to represent the possibility that this local
match is part of a global match.
\item {\em Gmap construction:} Combines the local matches into maximal
consistent collections of correspondences.
\item {\em Candidate inference construction:} Derives the inferences
suggested by each Gmap.
\item {\em Match Evaluation:} Attaches evidence to each local match and
uses this evidence to compute structural evaluation scores for each Gmap. 
\end{enumerate}

\noindent
Each computation will now be reviewed, using a simple example to
illustrate their operation. In this example, the rules of structure-mapping
theory are in use. It is important to distinguish the general \SME\ system
from its behavior when using the rules of a particular theory. Hence, when
using the rules of structure-mapping theory, it will be called \SMET.

\subsubsection{Step 1: Local match construction (create-match-hypotheses)}\index{match hypothesis}\index{algorithm}

\begin{figure}
\border
\vspace{3.1in}
\caption[Local Match Construction]{Local Match Construction. The water flow
and heat flow descriptions of Figure~\ref{fg:wfhf-calculus} have been drawn
in the abstract and placed to the left and right, respectively. The objects
in the middle depict match hypotheses.}
\label{fg:3D-MH}\index{algorithm}
\border
\end{figure}


\SME\ begins by finding for each entity and predicate in the base the set of
entities or predicates in the target that could plausibly match that item
(see Figure~\ref{fg:3D-MH}). Plausibility is determined by {\it match
constructor} rules, which are of the form:
\begin{tt}
\begin{tabbing}
\ \ \= (MHCrule \= (\metasyntax{Trigger} \= \metasyntax{BaseVariable} \metasyntax{TargetVariable} \\
\> \> \> {\it [}:test \metasyntax{TestForm}{\it ]}) \\
\> \ \ \ \  \metasyntax{Body}) \\
\end{tabbing}
\end{tt}

The body of these rules is run on each pair of items (one from the base and
one from the target) that satisfy the condition and installs a {\it match
hypothesis}\index{match hypothesis} which represents the possibility of them
matching. For example, to state that an expression in the base may match an
expression in the target whose functor is identical, we write:
\begin{verbatim}
   (MHC-rule (:filter ?b ?t :test (equal (expression-functor ?b)
                                         (expression-functor ?t)))
         (install-MH ?b ?t))
\end{verbatim}


\begin{figure}
\border
\vspace{4.4in}
\caption[Water Flow / Heat Flow Analogy]{Water Flow / Heat Flow
Analogy After Local Match Construction. Here we show the graph of match
hypotheses depicted schematically in Figure~\ref{fg:3D-MH}, augmented by links
indicating expression-to-arguments relationships. Match hypotheses which are not
descended from others are called {\it roots} (e.g., the matches
between the {\tt GREATER} predicates, {\tt MH-1} and {\tt MH-6}, and the
match for the predicate {\tt FLOW}, {\tt MH-9}). Match hypotheses between
entities are called {\it Emaps} (e.g., the match between beaker and coffee,
{\tt MH-4}).  Emaps play an important role in algorithms based on structural
consistency.}
\label{fg:wfhf-local}\index{algorithm}
\border
\end{figure}


The likelyhood of each match hypothesis\index{match hypothesis} is found by
running {\it match evidence} rules and combining their results. The evidence
rules provide support for a match hypothesis by examining the structural and
syntactic properties of the items matched. For example, the rule
\begin{verbatim}
   (MHErule ((:intern (MH ?b ?t) :test (and (expression? ?b) (expression? ?t)
                                            (eq (expression-functor ?b)
                                                (expression-functor ?t)))))
       (assert! (implies same-functor (MH ?b ?t) (0.5 . 0.0))))
\end{verbatim}
states ``If the two items are expressions and their functors are the same, then
supply 0.5 evidence in favor of the match hypothesis.'' The rules may also
examine match hypotheses associated with the arguments of these items to
provide support based on systematicity. This causes evidence for a match
hypothesis to increase with the amount of higher-order structure supporting
it.

The state of the match between the water flow and heat flow descriptions of
Figure~\ref{fg:wfhf-calculus} after running these first two sets of rules is
shown in Figure~\ref{fg:wfhf-local}.  There are several important things to
notice in this figure.  First, there can be more than one match
hypothesis\index{match hypothesis} involving any particular base or target
item.  Second, our rules required predicates to match identically while they
allowed entities to match on the basis of their roles in the predicate
structure.  Thus while {\tt TEMPERATURE} can match either {\tt PRESSURE} or
{\tt DIAMETER}, {\tt IMPLIES} cannot match anything but {\tt IMPLIES}.
Third, not every possible correspondence is created. Local matches between
entities are only created when justified by some other identity.  This
significantly constrains the number of possible matches in the typical case.


\begin{figure}
\border
\vspace{3.1in}
\caption[Conflicting relationships.]{Water Flow - Heat Flow analogy after
computation of {\it Conflicting} relationships. Simple lines show the
tree-like graph that the grounding criteria imposes upon match hypotheses.
Lines with circular endpoints indicate the {\it Conflicting} relationships
between matches. Some of the original lines from MH construction have been
left in to show the source of a few {\it Conflicting} relations.}
\label{3D-nogoods}
\border
\vspace{3.1in}
\caption{GMap Construction. (a) Merge step 1: Interconnected and consistent.
(b) Merge step 2: Consistent members of the same base structure. (c) Merge
step 3: Any further consistent combinations.}
\label{fg:wfhf-merging}\index{global mapping}\index{algorithm}
\border
\end{figure}


\subsubsection{Step 2: Global Match Construction}\index{global mapping}\index{algorithm}

The second step in the \SME\ algorithm combines local match hypotheses into
collections of global matches (Gmaps).  Intuitively, each global match is
the largest possible set of match hypotheses that depend on the same one to
one object correspondences.

More formally, Gmaps consist of {\em maximal, structurally consistent}
collections of match hypotheses.  A collection of match hypotheses is
structurally consistent if it satisfies two criteria:
\begin{enumerate}
\item {\em One-to-one:} No two match hypotheses assign the same
base item to multiple target items or any target item to multiple
base items.
\item {\em Support:} If a match hypothesis {\tt MH} is in the
collection, then so are the match hypotheses which pair up all of
the arguments of {\tt MH}'s base and target items.
\end{enumerate}
The preservation criteria enforces strict one to one mappings.  The
grounding criteria preserves connected predicate structure.  A collection is
maximal if adding any additional match hypothesis would render the collection
structurally inconsistent.

The formation of global matches\index{global mapping} is composed of two primary stages:

\begin{enumerate}
\item {\em Compute consistency relationships {\tt (calculate-nogoods)}:}
Here we generate for each match hypothesis the sets of entity mappings it
entails, what match hypotheses it locally conflicts with, and which match
hypotheses it is structurally inconsistent with.  This information
simplifies the detection of contradictory sets of match hypotheses, a
critical operation in the rest of the algorithm. The result of this stage of
processing appears in Figure~\ref{3D-nogoods}.

\item {\em Merge match hypotheses {\tt (generate-gmaps)}:} Compute Gmaps by
successively combining match hypotheses as follows:
 \begin{enumerate}
 \item {\em Form initial combinations {\tt (generate-structure-groups)}:}
Combine interconnected and consistent match hypotheses into an initial set
of Gmaps (Figure~\ref{fg:wfhf-merging}a).
 \item {\em Combine dependent Gmaps {\tt (merge-base)}:} Since base and
target dgroups are rarely isomorphic, some Gmaps in the initial set will
overlap in ways that allow them to be merged.  The advantage in merging them
is that the new combination may provide structural support for candidate
inferences (Figure~\ref{fg:wfhf-merging}b).
 \item {\em Combine independent collections {\tt
(full-gmap-merge)}:}\footnote{These two merge steps (b and c) are called by
{\tt merge-gmaps}, which is in turn called by {\tt generate-gmaps}.} The
results of the previous step are next combined to form maximal consistent
collections (Figure~\ref{fg:wfhf-merging}c).
 \end{enumerate}
\end{enumerate}

A parameter option allows the support criterion to be weakened so that it
does not cross the boundaries of a {\it relational group} \cite{Falken-phd}.
A relational group is distinguished as an unordered collection of relational
structures that may be collectively referred to as a unit.  They correspond
to the abstract notion of a ``set'' and are associated to predicates taking
any number of arguments.  For example, a set of relations joined by the
predicate {\tt AND} defines a relational group.  Other examples include the
axioms of a theory, a decomposable compound object, or the relations holding
over an interval of time.  Intuitively, we would like to say that two groups
correspond without requiring that their contents are exhaustively mapped.

If base and target propositions each contain a group as an argument, the
propositions should not be prevented from matching if the groups' members
cannot be exhaustively paired. For example, the set of relations
\begin{tt}
\begin{tabbing}
~~~B:~~~\= Implies[And(P$_{1}$,P$_{2}$,P$_{3}$), P$_{4}$] \` {\rm (1)} \\
~~~T:   \> Implies[And(P$^{\prime}_{1}$,P$^{\prime}_{2}$), P$^{\prime}_{4}$]
\end{tabbing}
\end{tt}

\noindent
should match better than the set of relations
\begin{tt}
\begin{tabbing}
~~~B:~~~\= Implies[And(P$_{1}$,P$_{2}$,P$_{3}$), P$_{4}$] \` {\rm (2)} \\
~~~T$^{\prime}$: \> P$^{\prime}_{1}$, P$^{\prime}_{2}$, P$^{\prime}_{4}$
\end{tabbing}
\end{tt}

\noindent 
The original model of structural consistency would score (1) and (2)
equally, since the {\tt Implies} relations of (1) would not be allowed to
match. This is a particularly important consideration when matching
sequential, state-based descriptions (e.g., the behavior of a system through
time). The set of relations describing a pair of states often do not
exhaustively match or are of different cardinality. Yet, higher-order
relations over states, such as temporal orderings, are vital and must appear
in the mapping.



\subsubsection{Step 3: Compute Candidate Inferences (gather-inferences)}\index{candidate inference}\index{algorithm}

Associated with each Gmap\index{global mapping} is a (possibly empty) set of
candidate inferences.  Candidate inferences are base predicates that would
fill in structure which is not in the Gmap (and hence not already in the
target). If a candidate inference contains a base entity that has no
corresponding target entity (i.e., the base entity is not part of any match
hypothesis for that gmap), \SME\ introduces a new, hypothetical entity into
the target.  Such entities are represented as a skolem function of the
original base entity (i.e., {\tt (:skolem base-entity)}).

In Figure~\ref{fg:wfhf-done}, Gmap \#1 has the top level {\tt CAUSE}
predicate as its sole candidate inference.  In other words, this Gmap
suggests that the cause of the flow in the heat dgroup is the difference in
temperatures.  If the {\tt FLOW} predicate was not present in the target,
then the candidate inferences for a Gmap corresponding to the pressure
inequality would be both {\tt CAUSE} and {\tt FLOW}. Note that {\tt
GREATER-THAN[DIAMETER(coffee), DIAMETER(ice cube)]} is not a valid candidate
inference for the first Gmap because it does not intersect the existing Gmap
structure.\index{candidate inference}

\begin{figure}
\border
\begin{small}
\begin{verbatim}
  Rule File: literal-similarity.rules     Number of Match Hypotheses: 14

  Gmap #1: { (>PRESSURE  >TEMPERATURE)  (PRESSURE-BEAKER  TEMP-COFFEE)
             (PRESSURE-VIAL  TEMP-ICE-CUBE)  (WFLOW  HFLOW) }
      Emaps:  { (beaker coffee)  (vial ice-cube)  (water heat)  (pipe bar) }
      Weight:  5.99
      Candidate Inferences: (CAUSE >TEMPERATURE HFLOW)

  Gmap #2: { (>DIAMETER  >TEMPERATURE)  (DIAMETER-1  TEMP-COFFEE)
             (DIAMETER-2  TEMP-ICE-CUBE) }
      Emaps:  { (beaker coffee)  (vial ice-cube) }
      Weight:  3.94
      Candidate Inferences:  { }

  Gmap #3: { (LIQUID-3  LIQUID-5)  (FLAT-TOP-4  FLAT-TOP-6) }
      Emaps:  { (water coffee) }
      Weight:  2.44
      Candidate Inferences:  { }
\end{verbatim}
\end{small}

  \caption{Complete SME interpretation of Water Flow - Heat Flow Analogy.}
  \label{fg:wfhf-done}\index{algorithm}\index{global mapping}
\border
\end{figure}

\subsubsection{Step 4: Compute Structural Evaluation Scores (run-rules)}\index{global mapping, scoring}\index{algorithm}

Typically a particular pair of base and target will give rise to several
Gmaps, each representing a different interpretation of the match.  Often it
is desired to select only a single Gmap,\index{global mapping} for example
to represent the best interpretation of an analogy. Many of these evaluation
criteria (including validity, usefulness, and so forth) lie outside the
province of Structure-Mapping, and rely heavily on the domain and
application.  However, one important component of evaluation is {\em
structural} --- for example, one Gmap may be considered a better analogy
than another if it embodies a more systematic match.  \SME\ provides a
programmable mechanism for computing a {\em structural evaluation score}
(SES) for each Gmap.  This score can be used to rank-order the Gmaps in
selecting the ``best'' analogy, or as a factor in a more complex (but
external) evaluation procedure.  In \SMET, the structural evaluation score is
currently computed by simply adding the belief of each local match
hypothesis to the belief of the Gmaps it is a member of.

Returning to Figure~\ref{fg:wfhf-done}, note that the ``strongest''
interpretation (i.e., the one which has the highest structural evaluation
score) is the one we would intuitively expect.  In other words, {\tt beaker}
maps to {\tt coffee}, {\tt vial} maps to {\tt ice-cube}, {\tt water} maps to
{\tt heat}, {\tt pipe} maps to {\tt bar}, and {\tt PRESSURE} maps to {\tt
TEMPERATURE}.  Furthermore, we have the candidate inference that the
temperature difference is what causes the flow.\index{global mapping}

\subsection{Adding Theoretical Constraints}

Given the general program, we may then add theoretical constraints in the
form of rules. For instance, the example just presented used the {\it
literal similarity} rules of structure-mapping theory. These rules augment
\SME's one-to-one mapping and structural consistency criteria with two
additional restrictions. First, evidence is computed according to {\it
systematicity}, that is, highly interconnected systems of relations are
preferred over independent facts. Second, only identical relations are
allowed to match (i.e., {\tt CAUSE} is not allowed to match {\tt
GREATER-THAN}).  Had another set of rules been used, the results might have
been substantially different.  For example, the {\it mere appearance} rules
of structure-mapping theory would have determined that the {\tt water} to
{\tt coffee} mapping was the best, due to their superficial similarity.


\section{Declarations}

The descriptions given to \SME\ are constructed from a user-defined
vocabulary of entities and predicates. This section discusses the
conventions for defining languages for \SME's use.

\subsection{Declaring Predicates}

\begin{tt}\index{defPredicate macro}\index{predicate}\index{predicate, declaring}
\begin{tabbing}
defPredicate \= {\it name argument-declarations predicate-class} \` [{\it Macro}] \\
             \> \&key \= :expression-type {\it logical-type}    \\
             \>      \> :commutative?  \{t | nil\}              \\
             \>      \> :n-ary?  \{t | nil\}                    \\
             \>      \> :documentation {\it descriptive-string} \\
             \>      \> :eval {\it procedural-attachment}\ \ )
\end{tabbing}
\end{tt}
\decfunc{define-predicate}{name argument-declarations predicate-class ...}

\spc{1}
\noindent
{\it predicate-class} is either {\tt function}, {\tt attribute}, or
{\tt relation}, according to what kind of predicate {\it name} is.
The {\it argument-declarations} allows the arguments to be named and
typed.  For example, the declaration:

\begin{verbatim}
   (defPredicate CAUSE ((antecedent event) (consequent event)) relation)
\end{verbatim}
states that {\tt CAUSE} is a two-place relational predicate.  Its arguments
are called {\tt antecedent} and {\tt consequent}, both of type {\tt event}.
The names and types of arguments are for the convenience of the
representation builder and any external routines (including the match rules),
and are not currently used by \SME\ internally.  Likewise, the predicate
class may be very important to the theoretical constraints imposed in the
rules, but is ignored by \SME\ internally.

The optional declaration {\tt :expression-type} indicates the logical type
of an expression headed by the given predicate. For example, the predicate
{\tt throw} may represent a kind of {\tt action}, while the predicate {\tt
mass} may represent an {\tt extensive-quantity}.

The optional declarations {\tt :commutative?} and {\tt :n-ary?} provide
\SME\ with important syntactic information.  {\tt :commutative?} indicates
that the predicate is commutative, and thus the order of arguments is
unimportant when matching.  {\tt :n-ary?} indicates that the predicate can
take any number of arguments.  Examples of commutative nary predicates include
{\tt AND}, {\tt SUM}, and {\tt SET}.

The {\tt :documentation} option allows one to attach a descriptive string to
a predicate. This documentation may then be accessed through the lisp
machine supplied interface\index{lisp machine interface, predicate documentation}
({\tt C-Shift-D}) or some externally written routine. If no documentation is
supplied, the list of argument names is used.  Another option provided
strictly for potential user routines is the optional {\tt :eval} parameter.
This allows one to declare a procedural attachment for a predicate.


\subsection{Declaring Entities}

\indent
\decmacro{defEntity}{name {\tt \&key type constant?}}\index{entity, declaring}
\decfunc{define-Entity}{name {\tt \&key type constant?}}

\spc{0.5}
Entities are logical individuals, i.e., the objects and constants of a
domain. Typical entities include physical objects, their temperature, and
the substance they are made of.  Primitive entities are declared with the
{\tt defEntity} form (a non-primitive entity would be {\tt (temperature
sun)}, which is a functional form representing a particular numeric
temperature entity). Primitive entities declared in this way represent
global entity types, that is, they represent a class of entities rather than
an actual instance of an entity. When an entity type is actually used in a
domain description, a unique entity instance is created for that type (e.g.,
{\tt Mary} is translated to {\tt Mary43}).

Since the language is typed, each entity type can be declared as
a subtype of an existing type using the {\tt :type} option.
For example, we might have
\begin{verbatim}
   (defEntity star :type inanimate)
   (defEntity Sun :type star)
\end{verbatim}
to say that stars are inanimate objects, and our Sun is a particular star.
Constants are declared by using the {\tt :constant?} option, as in
\begin{verbatim}
   (defEntity zero :type number :constant? t)
\end{verbatim}


\subsection{Declaring Description Groups}\label{ss:new-dgroups}
\index{description group}

\begin{tt}
\begin{tabbing}
defDescription {\it description-name}  \` [{\it Macro}]   \\
\ \ \ \= expressions \= \kill
      \> entities  \> ({\it entity$_{1}$}, {\it entity$_{2}$},$\ldots$,{\it entity$_{i}$}) \\ 
      \> expressions \> ({\it expression-declarations})
\end{tabbing}
\end{tt}
\decfunc{define-description}{description-name entities expressions}

\spc{0.5}
For simplicity, predicate instances and compound terms are called {\it
expressions}. A {\em Description Group}, or {\em Dgroup}\index{Dgroup}, is a
collection of primitive entities and expressions concerning them.  Dgroups
are defined with the {\tt defDescription}\index{defDescription macro} form,
where {\it expression-declarations} take the form {\tt
\begin{tabbing}
\ \   \= {\it expression} {\rm or} \\
      \> ({\it expression} :name {\it expression-name})
\end{tabbing}
}
\noindent
For example, the description of water flow depicted in
Figure~\ref{fg:wfhf-calculus} was given to \SME\ as
\begin{verbatim}
  (defDescription simple-water-flow
        entities (water beaker vial pipe)
        expressions (((flow beaker vial water pipe) :name wflow)
                     ((pressure beaker) :name pressure-beaker)
                     ((pressure vial) :name pressure-vial)
                     ((greater pressure-beaker pressure-vial) :name >pressure)
                     ((greater (diameter beaker) (diameter vial)) :name >diameter)
                     ((cause >pressure wflow) :name cause-flow)
                     (flat-top water)
                     (liquid water)))
\end{verbatim}
All entities must have been previously defined and every entity referred to
in the Dgroup's expressions must appear in the {\tt entities} list of the {\tt
defDescription}.\index{description group}

\subsection{Adding new expressions}\label{ss:new-expressions}

\indent
\decfunc{expression}{form dgroup-name {\tt \&key expression-name update-structure?}}\index{expression, adding new expressions}

\spc{0.7}
Expressions are normally defined as a side effect of creating a description group
(Dgroup). However, the facility is provided for dynamically adding new expressions
to a Dgroup. The syntax is essentially the same as for expressions declared within
a {\tt defDescription}. The expression's {\it form} may refer to the names of
existing Dgroup expressions and the form may be given a name. When {\tt expression} is
used to add expressions to an existing Dgroup, the {\tt update-structure?} keyword
must be invoked with a non-nil (e.g., {\tt T}) value. This keyword indicates
that the Dgroup's structure must be reexamined, since the known structural
roots will change as a result of this new expression.

\subsection{Typed Logic}

A mechanism exists for attaching types to predicates and their arguments
(see {\tt defPredicate}).  This facility is designed to constrain the
operation of \SME, particularly candidate inference generation. However, it
has not been extensively used to date.  The ability to attach types may be
useful for consistency checking by external systems.


\section{Using the rule system}\label{ss:rule-system}\index{rule system}

The rule system is the heart of \SME's flexibility. It allows one to specify
what types of things might match and how strongly these matches should be
believed. This section describes the required syntax for a rule set and
different strategies for rule specification.

\subsection{Rule file syntax}\index{rule system, rule file syntax}\index{rule system}

A rule set, or rule file, consists of a {\it declaration}, a set of
{\it match constructor} rules, and a set of {\it match evidence} rules.
In order to describe each, we will examine the syntax and functionality of
each part of the {\it smt-analogy} rule file.

\subsubsection{File declarations}\index{rule system}

\indent
\decfunc{sme-rules-file}{identification-string}

\spc{0.7}
Each rule file must begin with the initialization command {\tt
sme-rules-file}. This function clears the rule system in preparation for a
new set of rules (rules are cached for efficiency) and stores the name of
the rule set for output identification purposes. For example, our sample
rules file begins with:\footnote{Notice the file extension *.rules. While
rule files are not required to end in ``.rules'', all user interface
facilities for simplifying the loading of rule files depend upon this
extension. Another useful point is that rule files are typically defined to
be in the \SME\ package to avoid having to use {\tt sme:} throughout the
rule set.}  

\begin{verbatim}
   (sme-rules-file "smt-analogy.rules")
\end{verbatim}

\noindent
The rule file must then end with the {\tt tre-save-rules} command:

\spc{0.7}
\decfunc{tre-save-rules}{}

\subsubsection{Match Constructor rules}\index{rule system, match constructor rules}\index{rule system}

\index{match hypothesis, installing}
\index{match constructor rules}
\index{MHC-rule}
\begin{tt}
\begin{tabbing}
MHC-rule \= ({\it trigger} \= {\it ?base-variable} {\it ?target-variable} \` [{\it Macro}] \\
         \>                \> [:test {\it test-form}]) \\
\ \ \  {\it body}
\end{tabbing}
\end{tt}
\decfunc{install-MH}{base-item target-item}\index{match hypothesis, installing}

\spc{0.7}
\SME\ begins by finding for each entity and predicate in the base the set of
entities or predicates in the target that could plausibly match that item.
Plausibility is determined by {\it match constructor} rules, which are
responsible for installing all match hypotheses processed by \SME.  There
are two types of constructor rules, each indicated by a different value for
{\it trigger}.  The first type of rule is indicated by a {\tt :filter}
trigger.  These rules are applied to each pair of base and target expressions,
executing the code in {\it body}.  If the {\tt :test} option is used,
{\it test-form} must return true for the body to be run.  For example, the
following rule states that an expression in the base may match an expression
in the target whose functor is identical, unless they are attributes
(a structure-mapping {\it analogy} criterion):

\begin{small}
\begin{verbatim}
   (MHC-rule (:filter ?b ?t :test (and (eq (expression-functor ?b)
                                           (expression-functor ?t))
                                       (not (attribute? (expression-functor ?b)))))
          (install-MH ?b ?t))
\end{verbatim}
\end{small}

The second type of MHC rule is indicated by a trigger of {\tt :intern}.
These rules are run on each match hypothesis as it is created. Typically
they create match hypotheses between any functions or entities that are the
arguments of the expressions joined by the match hypothesis that triggered
the rule. The following is one of two that appear in smt-analogy.rules:

\begin{small}
\begin{verbatim}
   (MHC-rule (:intern ?b ?t :test (and (expression? ?b) (expression? ?t)
                                       (not (commutative? (expression-functor ?b)))
                                       (not (commutative? (expression-functor ?t)))))
       (do ((bchildren (expression-arguments ?b) (cdr bchildren))
            (tchildren (expression-arguments ?t) (cdr tchildren)))
           ((or (null bchildren) (null tchildren)))
         (cond ((and (entity? (first bchildren)) (entity? (first tchildren)))
                (install-MH (first bchildren) (first tchildren)))
               ((and (function? (expression-functor (first bchildren)))
                     (function? (expression-functor (first tchildren))))
                (install-MH (first bchildren) (first tchildren)))
               ((and (attribute? (expression-functor (first bchildren)))
                     (eq (expression-functor (first bchildren))
                         (expression-functor (first tchildren))))
                (install-MH (first bchildren) (first tchildren))))))
\end{verbatim}
\end{small}

Notice that the third test allows identical attributes to match, whereas the
previous MHC rule did not allow such matches. This design does not allow
isolated attributes to match, but recognizes that attributes appearing in a
larger overall structure should be matched.


\subsubsection{Match Evidence rules}\label{ss:ME-rules}\index{match evidence rules}\index{rule system, match evidence rules}\index{rule system}

\indent
\decmacro{rule}{nested-triggers body}\index{match hypothesis, scoring}
\decmacro{rassert!}{expression\ {\tt \&optional (}belief+ {\tt 1.0) (}belief- {\tt 0.0)}}
\decfunc{assert!}{expression\ {\tt \&optional (}belief+ {\tt 1.0) (}belief- {\tt 0.0)}}
\decmacro{initial-assertion}{assertion-form}

\spc{0.8}
The structural evaluation score is computed in two phases. First, each match
hypothesis is assigned some local degree of evidence, independently of what
Gmaps it belongs to.  Second, the score for each Gmap is computed based on
the evidence for its match hypotheses.  The management of evidence rules is
performed by the {\it Belief Maintenance System} (BMS) \cite{BMS}. A BMS is
a form of Truth-Maintenance system, extended to handle numerical weights for
evidence and degree of belief (see \cite{BMS} for a description of what the
weights mean). Pattern-directed rules are provided that trigger on certain
events in the knowledge base. 

The following is a simple rule for giving evidence to match hypotheses
between expressions that have the same predicate:

\begin{small}
\begin{verbatim}
   (initial-assertion (assert! 'same-functor))

   (rule ((:intern (MH ?b ?t) :test (and (expression? ?b) (expression? ?t)
                                         (eq (expression-functor ?b)
                                             (expression-functor ?t)))))
      (if (function? (expression-functor ?b))
          (rassert! (implies same-functor (MH ?b ?t) (0.2 . 0.0)))
          (rassert! (implies same-functor (MH ?b ?t) (0.5 . 0.0)))))
\end{verbatim}
\end{small}

There are two things to notice here in addition to the evidence rule. First,
the proposition {\tt same-functor} was asserted to be true (a belief of
1.0) and then used as the antecedent for the implication of evidence. In
this way, the source of this particular piece of evidence is identified and
is available for inspection. Second, the assertion of {\tt same-functor} was
placed inside the {\tt initial-assertion} form. Since \SME\ caches the
current rule file, it must be told if there are any functions embedded in
the rule file that must be invoked each time \SME\ is initialized.

\spc{0.8}
\decfunc{children-of?}{base-child target-child base-expression target-expression}

\spc{0.3}
Nested triggers within an evidence rule may be used to locate
interdependencies between different match hypotheses. For example,
structure-mapping's systematicity principle is implemented in a local
fashion by propagating evidence from a match hypothesis to its
children:\footnote{A number of functions (e.g., {\tt children-of?}) are
provided to simplify the writing of rules. These appear in the file {\tt
match-rules-support.lisp}.}\index{match hypothesis, scoring}

\begin{small}
\begin{verbatim}
   (rule ((:intern (MH ?b1 ?t1) :test (and (expression? ?b1) (expression? ?t1)
                                           (not (commutative? (expression-functor ?b1)))))
          (:intern (MH ?b2 ?t2) :test (children-of? ?b2 ?t2 ?b1 ?t1)))
       (rassert! (implies (MH ?b1 ?t1) (MH ?b2 ?t2) (0.8 . 0.0))))
\end{verbatim}
\end{small}

\noindent
Evidence for a Gmap is given by:

\begin{small}
\begin{verbatim}
   (rule ((:intern (GMAP ?gm) :var ?the-group))
      (dolist (mh (gm-elements ?gm))
         (assert! `(implies ,(mh-form mh) ,?the-group))))
\end{verbatim}
\end{small}

\index{global mapping, scoring}\index{rule system}

The {\tt BMS} allows a set of nodes to be declared special and will treat
evidence to these nodes differently. An {\tt additive-nodes} function is
provided which takes a set of {\tt BMS} nodes and modifies them so that
their evidence is added rather than normalized using Dempster's rule. \SME\
automatically invokes {\tt additive-nodes} on the derived set of Gmaps once
they are created. Thus, when the above Gmap rule is executed and the {\tt
implies} statement is used to supply evidence from each match hypothesis to
the Gmap, that evidence is simply automatically added to the total Gmap
evidence rather than propagated using Dempster's probabilistic sum.

The following destructive rule is often used instead of the previous one to
give a significant speed up:

\begin{small}
\begin{verbatim}
   (rule ((:intern (GMAP ?gm) :var ?the-group))
       (setf (node-belief+ (gm-bms-node ?gm)) 0)
       (dolist (mh (gm-elements ?gm))
          (incf (node-belief+ (gm-bms-node ?gm))
                (node-belief+ (mh-bms-node mh)))))
\end{verbatim}
\end{small}

This rule bypasses the BMS entirely, thus increasing speed by not creating
justification links. It also renders the {\tt additive-nodes} distinction
irrelevant. However, such rules must be used with extreme caution.  For
example, the source of a Gmap's evidence cannot be inspected when using this
type of operation (see Section~\ref{ss:inspecting-evidence}).

\subsection{Making SME simulate structure-mapping theory}\index{rule system}\index{rule system, simulating structure-mapping theory}

The previous section examined the general structure of an \SME\ rule file.
In the process, the basic elements of the {\it structure-mapping-theory
analogy} rule set were presented. The {\it literal similarity} and
{\it mere-appearance} rules are essentially the same as the {\it analogy}
rules. They differ in the first match constructor rule. The {\it analogy}
rule set has the test {\tt (not (attribute? (expression-functor ?b)))} which is
absent from the corresponding {\it literal similarity} rule. Conversely, the
corresponding {\it mere appearance} rule forces the opposite condition
{\tt (attribute? (expression-functor ?b))}. One should consult Appendix A of
\cite{sme-aij} for listings of all three structure-mapping rule sets.

\subsection{Making SME perform as SPROUTER}\index{rule system}\index{rule system, simulating SPROUTER}

The {\tt SPROUTER} program \cite{hayes-roth} was
developed as an approach to the problem of inductively forming
characteristic concept descriptions. That is, given a sequence of events
(e.g., a list of pictures), produce a single, conjunctive description which
represents a generalized, characteristic description of the sequence. {\tt
SPROUTER} generalized a sequence of {\it N} descriptions by finding the
commonalities between the first two descriptions, generalizing these common
elements (i.e., variablize the literals), and then repeating the process
using this generalized description and the next, unprocessed description.
These steps would be repeated until the generalization had propagated
through the entire list of input descriptions.

\SME\ may be used to implement {\tt SPROUTER}'s {\it interference matching}
technique by giving it a set of match constructor rules which require all
matching predicates to have the same name (i.e., the {\it literal
similarity} rules without the condition that allows functions with different
names to match). The {\tt SPROUTER} generalization\index{generalize}
mechanism may then be implemented with the following algorithm:

\begin{small}
\begin{verbatim}
  Procedure SPROUTER (event-list)
     begin
       generalization := pop(event-list)
       while event-list
          pairwise-match := match(pop(event-list), generalization)
          generalization := generalize(pairwise-match)
       return generalization
     end
\end{verbatim}
\end{small}


\subsection{Relaxing the identical predicates constraint}\index{rule system}\index{rule system, relaxing identical predicates}

The current structure-mapping theory rules are sensitive to representation
by requiring that relational predicates match only if they are identical.
This is an important restriction that ensures the structures being compared
are semantically similar. However, it can also be overly restrictive.  We
are currently exploring different methods to relax the identicality
requirement while still maintaining a strong sense of semantic similarity.
One approach, called the {\it minimal ascension principle}, allows relations
to match if they share a common ancestor in a multi-root is-a hierarchy of
expression types \cite{Falken-phd} (i.e., the identicality test in the match
constructor rule is replaced by a call to {\tt
predicate-type-intersection?}). The local evidence score for their match is
inversely proportional (exponentially) to the relations' distance in the
hierarchy.  This enables \SME\ to match non-identical relations if such a
match is supported by the surrounding structure, while still maintaining a
strong preference for matching semantically close relations. This is similar
to approaches used in \cite{Burstein83,Greiner,Winston80}.

Problems with an unconstrained minimal ascension match technique are
discussed in \cite{Falken-phd}. A mapping approach which considers the
current context when determining pairwise similarity is also discussed.

\subsection{Pure isomorphisms}\label{ss:isomorphisms}
\index{rule system}\index{rule system, pure isomorphisms}

While it is important to assure that the structures being compared are
semantically similar, one can in principle remove all semantic comparisons.
This would allow match creation to be guided strictly by \SME's structural
consistency and 1-1 mapping criteria and match selection to be based
strictly on systematicity.

\begin{figure}
\border
\begin{small}
\begin{tt}
\begin{tabbing}
\ \ \ \ \ \ {\rm Addition}\ \ \ \ \ \ \ \ \ \ \ \ \ \ \ \ \ \ \ \ \ \ \ \ \ \ \ \ \ \ \ \ {\rm Union} \\
\ \ \ \= N1 + N2 $=$ N2 + N1\ \ \ \ \ \ \ \ \ \ \ \ \ \ \ \ \ \ \ \ \ \= S1 $\cup$ S2 $\equiv$ S2 $\cup$ S1 \\
\> N3+(N4+N5) $=$ (N3+N4)+N5 \> S3 $\cup$ [S4 $\cup$ S5] $\equiv$ [S3 $\cup$ S4] $\cup$ S5 \\
\> N6+0 $=$ N6            \> S6 $\cup$ $\emptyset$ $\equiv$ S6
\end{tabbing}
\end{tt}
\end{small}
  \caption{Formal descriptions for addition and union.}
  \label{fg:addition-union}
\border
\end{figure}


Consider the isomorphic mapping between the formal definitions of numeric
addition and set union shown in
Figure~\ref{fg:addition-union}.\footnote{This example is taken from an
advance copy of a paper by Holyoak and Thagard \cite{ACME}. I include it
here simply to demonstrate the range of matching preferences available in
\SME.} These formal descriptions may be given to \SME\ in the standard
manner, as in {\tt (plus N3 (plus N4 N5))} for the left side of the
associativity rule (the representation {\tt (plus N4 N5 result45)} also
works, although it results in a slightly longer run time due to the
flattening of structure).  When presented as formal definitions, the
concepts of addition and union are structurally isomorphic, independent of
the meaning of the predicates. Thus, while it could be argued that the
predicates {\tt plus} and {\tt union} share a certain degree of semantic
overlap, this example demonstrates that it is possible to make \SME\ ignore
predicates entirely and simply look for isomorphic mappings.  The rule set
for isomorphic mappings is shown in Figure~\ref{fg:isomorphism}. (This is
called the {\it ACME} rule set, as it configures \SME\ to emulate the {\tt
ACME} program on this example \cite{ACME}). The only constraint this rule
set enforces is that each predicate has the same number of arguments. While
it includes the Structure-Mapping notion of systematicity to prefer systems
of relations, it does not enforce identicality of predicates.  Using this
rule set, \SME\ produces the unique best mapping that we would expect
between the formal definitions of addition and union.

Since \SME\ enforces the ``same number of arguments'' restriction by
defeating any match hypotheses that are not structurally sound, we could in
principle effectively remove the rule file entirely. This could be done with
one match constructor rule to match everything with everything and one
evidence rule to measure systematicity. When this {\it free-for-all} rules
file was given to \SME, the same single best Gmap was produced, but at the
expense of increasing the run time from 13 seconds to 3.25 minutes.


\begin{figure}[t]
\border
\begin{small}
\begin{alltt}
   (MHC-rule (:filter ?b ?t :test (= (numargs (expression-functor ?b))
                                     (numargs (expression-functor ?t))))
       (install-MH ?b ?t))

   {\it ;;; Intern rule to match entities (non-commutative predicates)}
   (MHC-rule (:intern ?b ?t :test (and (expression? ?b) (expression? ?t)))
       (do ((bchildren (expression-arguments ?b) (cdr bchildren))
            (tchildren (expression-arguments ?t) (cdr tchildren)))
           ((or (null bchildren) (null tchildren)))
         (if (and (entity? (first bchildren)) (entity? (first tchildren)))
             (install-MH (first bchildren) (first tchildren)))))

   {\it ;;; Give a uniform initial priming to each MH}
   (initial-assertion (assert! 'initial-priming))

   (rule ((:intern (MH ?b ?t)))
      (rassert! (implies initial-priming (MH ?b ?t) (0.2 . 0.0))))

   {\it ;;;propagate interconnections - systematicity}
   (rule ((:intern (MH ?b1 ?t1) :test (and (expression? ?b1) (expression? ?t1)))
          (:intern (MH ?b2 ?t2) :test (children-of? ?b2 ?t2 ?b1 ?t1)))
      (rassert! (implies (MH ?b1 ?t1) (MH ?b2 ?t2) (0.8 . 0.0))))

   {\it ;;; Support from its MH's}
   (rule ((:intern (GMAP ?gm) :var ?the-group))
      (dolist (mh (gm-elements ?gm))
         (assert! `(implies ,(mh-form mh) ,?the-group))))
\end{alltt}
\end{small}
  \caption{Rule set for forming general isomorphic mappings.}
  \label{fg:isomorphism}\index{rule system}
\border
\end{figure}



\subsection{Imposing externally established pairings}\index{rule system}\index{rule system, analogical hints}

In certain situations, a number of entity and predicate mappings may already
be known prior to invoking \SME. These mappings may have been provided as an
analogical hint\index{analogical hint} from an instructor or derived by the
application program during earlier processing.  For example, \Phineas\
\cite{VBAL,MLW} uses \SME\ to analogically relate observed physical
phenomena to known theories of the world. \Phineas\ uses two analogical
mappings to learn about a new physical process. First, behavioral
correspondences are established (i.e., what entities and quantities are
behaving in the same manner). Second, the relevant base theories are
analogically mapped into the new domain, guided by the behavioral
correspondences. The two-stage mapping process solves the problem of using
analogy in cases where one does not have a pre-existing theory, as occurs
with truely novel learning. The assumption made in \Phineas\ is that similar
behaviors will have similar theoretical explanations. The first mapping
provides the correspondences between entities and functions required to
guide the importation of an old theory to explain a new domain in the second
mapping.

\SME\ includes facilities to simplify writing \Phineas-like programs, by
enabling the results of earlier processing to constrain subsequent mapping
tasks. These routines are divided into two categories, {\it declaration} and
{\it test}. The declaration routines tell \SME\ what predicate and entity
correspondences are known {\it a-priori}. The test routines enable the match
constructor rules to adhere to these imposed constraints. Known mappings are
declared through the following functions:

\spc{0.6}
\noindent
\begin{minipage}{\textwidth}
\begin{tt}
\begin{tabbing}
defGiven-Mappings  \` [{\it Macro}]   \\
\ \ \ \= entities \ \ \= (\= ({\it base-entity$_{1}$ target-entity$_{1}$})   \\
      \>              \>  \> ({\it base-entity$_{i}$ target-entity$_{j}$}) ...)  \\
      \> predicates   \> (({\it base-predicate$_{1}$ target-predicate$_{1}$})   \\
      \>              \>  \> ({\it base-predicate$_{i}$ target-predicate$_{j}$}) ...)
\end{tabbing}
\end{tt}
\end{minipage}
\decfunc{declare-given-mappings}{entities predicates}
\decfunc{clear-given-mappings}{}

\spc{0.8}
Both {\tt defGiven-Mappings} and {\tt declare-given-mappings} have identical
functionality. The first does not evaluate its arguments while the second
one does. Disjunctive constraints may be imposed by including all of the
possible pairings (e.g., defining both {\tt (}{\it base-entity$_{i}$
target-entity$_{j}$}{\tt )} and {\tt (}{\it base-entity$_{i}$
target-entity$_{k}$}{\tt )}).

Once a set of given mappings has been declared, the following test routines
may be used within the match constructor rules to enforce these mappings:

\spc{0.8}
\decfunc{sanctioned-pairing?}{base-item target-item}
\decfunc{paired-item?}{{\tt \&key base-item target-item}}

\spc{0.4}
\noindent
{\tt sanctioned-pairing?} tests if the given pair is one of the {\it a-prior}
pairings. {\tt paired-item?} takes either a base item or a target item and returns
true if the mapping for that item has been externally determined.

These functions help in writing rules which respect established mappings.
For example, the following two rules are used in the \Phineas\ system to
allow observed behavioral correspondences to constrain the mapping of the
relevant theory:

\begin{small}
\begin{verbatim}
(MHC-rule (:filter ?b ?t :test (and (eq (expression-functor ?b) (expression-functor ?t))
                                    (not (paired-item? :base-item (expression-functor ?b)))
                                    (not (paired-item? :target-item (expression-functor ?t)))))
    (install-MH ?b ?t))

(MHC-rule (:filter ?b ?t :test (sanctioned-pairing? (expression-functor ?b)
                                                    (expression-functor ?t)))
    (install-MH ?b ?t))
\end{verbatim}
\end{small}

When an analogy is being made between two behaviors, {\tt clear-given-mappings}
is used to make \SME\ perform in normal analogy mode. The discovered entity and
function correspondences are then given to {\tt declare-given-mappings} prior to
using \SME\ to map the relevant theory.\index{rule system}


\section{Representation Issues}\index{representation issues}

The proper representation becomes an issue in \SME\ due to its significant
impact on speed performance. Hierarchical representations provide an
important source of constraint on generating potential matches. They tend to
make the semantic interrelations explicit in the structure of the syntax.
For example, Section~\ref{ss:isomorphisms} described a comparison between
the laws of addition and union. There it was noted that part of the additive
associativity rule may be represented as {\tt (plus N3 (plus N4 N5))} or as
the pair {\tt (plus N4 N5 result45)} and {\tt (plus N3 result45
result3-45)}. The latter ``flat'' representation takes more time for \SME\
to process, sometimes a significant difference for large domain
descriptions. This is because the functional representation makes the
associativity rule structurally explicit, while the flat representation
buries it among the tokens appearing as arguments to {\tt plus}. However, it
is important to note that \SME\ is able to process domain descriptions in
any predicate-based format. It is simply speed considerations that render
standard, flat forms of representation undesirable.

Due to SME's ability to accept commutative, n-ary predicates, it is able to
match arbitrary sets (which must be of equal size at this time). This has
two consequences. First, the explicit use of sets becomes a viable form of
representation. Thus, a theory might be represented concisely as

\begin{verbatim}
   (Theory T1 (SET axiom-8 axiom-14 ...))
\end{verbatim}

\noindent
rather than as {\tt (Axiom-of T1 axiom-8)}, etc. Second, sets may be used to
add structure to descriptions. For example, the set representation for
theories results in greatly reduced run times compared to the non-set
representation.\footnote{The difference in speed is due to the operation of
merge step 2, which combines matches sharing a common base structure. The
set notation for theory {\tt T1} enables merge step 2 to know that matches
for {\tt axiom-8} and {\tt axiom-14} should be placed in the same gmap, thus
reducing the number of possibilities in merge step 3.} I am currently
investigating the use of a similar representation for temporal states, as in:

\begin{verbatim}
   (Situation S1 (SET (Increasing (Amount-of water1))
                      (Increasing (Pressure water1))
                      o o o ))
\end{verbatim}

\noindent
A \Phineas\ problem which took \SME\ 53 minutes (using {\tt (Increasing
(Amount-of (at water1 S1)))}) was reduced to 34 seconds using this more
structured representation.


\section{Using SME}

This section describes how to install \SME\ on your machine, load it, and
operate it.

\subsection{Installing SME}\index{installation}\label{ss:site-specifics}

\indent
\decvar{*sme-language-file*}\index{parameters, site specific}\index{site specific information}
\decvar{*sme-default-rules*}\index{site specific information, pathnames}
\decvar{*sme-rules-pathname*}
\decvar{*sme-dgroup-pathname*}

\spc{0.8}
To configure \SME\ to a particular site, a handful of variables storing
system directory information must be edited and set to the appropriate
values. These variables appear in {\tt config.lisp}, a separate file for
this purpose. Of primary importance are {\tt *sme-language-file*} and {\tt
*sme-default-rules*}. These are used by {\tt sme-init} to initialize the
language and rule systems. The two variables storing the rules and dgroup
pathnames are used by the user interface routines.

\spc{0.7}
\decvar{*the-lisp-package*}\index{packages,site specific}
\decvar{*the-user-package*}\index{site specific information, packages}

\spc{0.3}
In most Common Lisp implementations, one package exists for general user
definitions and another exists for the lisp implementation. It is important
to notify \SME\ what these are for the Common Lisp in use. For example, on a
Symbolics (version 6.2), the lisp package is called {\tt common-lisp} and
the user package is called {\tt cl-user}. These are the default settings.

\spc{0.7}
\decvar{*sme-system-pathname*}
\decvar{*sme-files*}

\spc{0.8}
These variables are used to automate compiling and loading. If the system is
being loaded on something other than a Symbolics or TI Explorer, the file
{\tt windowing} should not be included in the list {\tt *sme-files*}.
Otherwise, it should be left in the list of \SME\ files, which is the
default.

The \SME\ routines assume a set of naming conventions on domain description
and rule files.\index{conventions, file names} The names of files containing
domain descriptions ({\tt defDescription}) should end with a {\tt *.dgroup}
extension. The name of a file containing a rule set should end with the {\tt
*.rules} extension.


\subsection{Running SME}\index{running SME}\index{execution}

\indent
\decfunc{sme-init}{{\tt \&optional (}initialize-language? {\tt T) (}initialize-rules? {\tt T)}}

\spc{0.7}
This section gives a brief overview of the process of using \SME\ for
matching, generalization, and inspection tasks.


\begin{figure}
\border
\begin{footnotesize}
\begin{alltt}
{\bf\gthan\ (sme:sme-init)}
Initializing SME...
   Loading default language file: prof:>falken>sme>language
   Loading default rules file: prof:>falken>sme>literal-similarity.bin
Complete.
T
{\bf\gthan\ (load "prof:\gthan falken\gthan sme\gthan simple-water-flow.dgroup")}
Loading PROF:>falken>sme>simple-water-flow.dgroup into package USER
#P"PROF:>falken>sme>simple-water-flow.dgroup"
NIL
{\bf\gthan\ (load "prof:\gthan falken\gthan sme\gthan simple-heat-flow.dgroup")}
Loading PROF:>falken>sme>simple-heat-flow.dgroup into package USER
#P"PROF:>falken>sme>simple-heat-flow.dgroup"
NIL
{\bf\gthan\ (sme:match 'swater-flow 'sheat-flow T)}
              SME Version 2E
    Analogical Match from SWATER-FLOW to SHEAT-FLOW.

Rule File: literal-similarity.rules
--------------------------------------------------------------------------
                  | # Entities | # Expr. | Maximum order | Average order |
  Base Statistics |       4    |     11  |        3      |      1.36     |
Target Statistics |       4    |      6  |        2      |      1.17     |
--------------------------------------------------------------------------
# MH's | # Gmaps | Merge Step 3 | CI Generation | Show Best Only |
   14  |     3   |   ACTIVE     |    ACTIVE     |       OFF      |
------------------------------------------------------------------
Total Run Time:    0 Minutes,  0.821 Seconds
BMS Run Time:      0 Minutes,  0.530 Seconds
Best Gmaps: { 3 }

Match Hypotheses:
    (0.6320  0.0000)  (PIPE4 BAR7)
    (0.7900  0.0000)  (FLAT-WATER FLAT-COFFEE)
    {\it   o       o         ;a number of match hypotheses appeared here}
    (0.8646  0.0000)  (WATER1 COFFEE5)
    (0.7900  0.0000)  (LIQUID-WATER LIQUID-COFFEE)

Gmap #1:   (BEAKER2 COFFEE5)  (DIAM-BEAKER TEMP-COFFEE)  (VIAL3 ICE-CUBE6)
           (DIAM-VIAL TEMP-ICE-CUBE)  (>DIAMETER >TEMP)
  Emaps:  (BEAKER2 COFFEE5) (VIAL3 ICE-CUBE6)
  Weight: 3.937660
  Candidate Inferences:  { }

    {\it   o       o         ;Gmap #2 appeared here...}

Gmap #3:   (>PRESSURE >TEMP)  (PRESS-VIAL TEMP-ICE-CUBE)  (PRESS-BEAKER TEMP-COFFEE)
           (BEAKER2 COFFEE5)  (VIAL3 ICE-CUBE6)  (WATER1 HEAT8)
           (PIPE4 BAR7)  (WFLOW HFLOW)
  Emaps:  (BEAKER2 COFFEE5) (VIAL3 ICE-CUBE6) (WATER1 HEAT8) (PIPE4 BAR7)
  Weight: 5.991660
  Candidate Inferences:  (CAUSE >TEMP HFLOW)
\end{alltt}
\end{footnotesize}
  \caption{Initializing and running SME.}
  \label{fg:sme-trace}
\border
\end{figure}


\begin{enumerate}

\item {\it Loading the files.} To load or compile \SME, the file {\tt
config} should be loaded and then {\tt (load-sme)} or {\tt (compile-sme)}
called. A {\tt defSystem} definition is provided in {\tt system.lisp} for
Symbolics machines.

\item {\it System startup.} This stage is only appropriate for full, lisp
machine startup. The \SME\ window environment may be created with Select-S
on a Symbolics or System-S on a TI Explorer.

\item {\it Initialization.} The function {\tt sme-init} should be called to
initialize the database. If {\it initialize-language?} is non-nil, the
default language file (predicate definitions) will be loaded. If {\it
initialize-rules?} is non-nil, the default rules file will be loaded. Prior
to operating \SME, the language and rule systems must be established.

\item {\it Loading Dgroups.} Any description groups that are to be matched
must be declared. These declarations are typically stored in files, with the
extension {\tt *.dgroup}. If the windowing system is active, the command
{\tt Load Dgroup} will offer a menu of all {\tt *.dgroup} files in
{\tt *sme-dgroup-pathname*} to select what to load.

\item {\it Analogical mapping.} The function {\tt match} may be called to
form a mapping between two given Dgroups. This is discussed in
Section~\ref{ss:match}. If the windowing system is active, the command {\tt
Match} will offer a menu to select base and target Dgroups. It is prior to
this step that one might want to think about wether to modify any system
parameters (e.g., print the match hypotheses, print only the best Gmaps,
generate candidate inferences, etc.). 

\item {\it Describing Dgroups.} Once Dgroups are defined, the {\tt
describe-dgroup} facility will provide a description of any particular
Dgroup.

\item {\it Graphically displaying Dgroups.} If the windowing system is
active, Dgroups may also be displayed graphically, through the {\tt Display
Dgroup} utility.

\item {\it Generalizing.} Once a mapping is formed, it may be generalized
using the {\tt generalize} function or the {\tt Generalize} command in the
system menu.

\item {\it Saving the results of a session.} If the windowing system is
active, the results of commands like {\tt Match} and {\tt Generalize} are
sent to the scroll window by default. These results may be written to a file
using the {\tt dump-scroll} system utility.

\item {\it Comparing two apparently identical Gmaps.} When two Gmaps are
formed that appear to be identical, their differences can be identified
using the {\tt compare-gmaps} system utility.

\end{enumerate}

\noindent
A trace of \SME\ performing the basic mapping task is given in
Figure~\ref{fg:sme-trace}. Each of the other options are described in
greater detail in the following sections.


\subsection{Batch mode}
\index{batch mode}\index{report generation}

\indent
\decfunc{run-batch-file}{pathname {\tt \&key} {\tt (}gmap-display {\tt :all) (}gmap-statistics {\tt :none)}}
\decmacro{language-file}{pathname}
\decmacro{dgroup-directory}{pathname}
\decmacro{dgroup-file}{file-name}
\decmacro{rule-directory}{pathname}
\decmacro{rule-file}{file-name}
\decmacro{rule-sets}{{\tt \&rest} rule-file-names}
\decmacro{report-comments}{string}
\decmacro{send-report-to}{pathname {\tt \&key (}text-driver {\tt :LPR) (}style {\tt :STANDARD)}}
\decmacro{run-matcher-on}{base-name target-name}
\decmacro{defPostMatcher}{function}


\begin{figure}
\border
\begin{footnotesize}
\begin{alltt}
(sme:Dgroup-Directory "prof:>falken>sme>")

(sme:Dgroup-File "solar-system")
(sme:Dgroup-File "rutherford")
(sme:Dgroup-File "simple-water-flow")
(sme:Dgroup-File "simple-heat-flow")

(sme:Rule-Directory "prof:>falken>sme>")

(sme:Rule-Sets "literal-similarity" "true-analogy" "attribute-only")   {\it; iterate over each rule set}

(sme:Report-Comments "Sample run of SME to demonstrate batch mode.")

(sme:Send-Report-To "heath:>falken>sample.dmp" :text-driver :LATEX)

(sme:Run-Matcher-On solar-system rutherford-atom)                 {\it; map this pair once for each rule set}
(sme:Run-Matcher-On swater-flow sheat-flow)                       {\it; map this pair once for each rule set}
\end{alltt}
\end{footnotesize}
  \caption{Sample SME batch file.}
  \label{fg:batch-file}
\border
\end{figure}


\spc{0.8}
\SME\ is normally used as an interactive utility or as a module to some
larger program. However, when performing statistical analyses across a broad
space of matching preferences (i.e., rule sets) and domain descriptions, an
interactive format soon becomes inconvenient. Utilities are provided so that
a file of \SME\ instructions may be defined and then executed using {\tt
run-batch-file} (e.g., Figure~\ref{fg:batch-file}). This would instruct
\SME\ to perform a series of matches, potentially over a variety of rule
sets and domain descriptions, and generate a detailed report of the
execution and a summary of the results. When a single rule set is specified
using {\tt rule-file}, all subsequent matches (invoked by {\tt
run-matcher-on}) will use this rule file until another one is specified.
Using {\tt rule-sets}, one may instead specify a series of rule files to be
used, so that a single {\tt run-matcher-on} command will cause \SME\ to run
once for each rule file in the list. If a user-defined function name is
given to {\tt defPostMatcher}, this function will be called after each match
is performed, in case special post-match routines are desired or extra
information is to be added to the report being generated. A variety of text
drivers are supported for report generation ({\tt send-report-to}), such as
{\tt :lpr} (line printer), {\tt :latex}, and {\tt :troff}.


\subsection{Generalization mechanism}

\indent
\decfunc{generalize}{gmap}

\spc{0.8}
The {\tt generalize} function takes a global mapping structure and returns
three alternate generalizations (using the Common-Lisp {\tt values}
protocol), each one successively larger than the previous:


\begin{figure}
\border
\begin{small}
\begin{verbatim}
   Generalizations for Match from SWATER-FLOW to SHEAT-FLOW:

       Generalization #1 (Literally Common Aspects Only):
            (FLOW ENTITY6 ENTITY8 ENTITY13 ENTITY14)

       Generalization #2 (All Common Aspects Only):
            (FLOW ENTITY6 ENTITY8 ENTITY13 ENTITY14)
            (GREATER (FUNCTION0 ENTITY6) (FUNCTION0 ENTITY8))

       Generalization #3 (Maximal Generalization):
            (CAUSE (GREATER (FUNCTION0 ENTITY6) (FUNCTION0 ENTITY8))
                   (FLOW ENTITY6 ENTITY8 ENTITY13 ENTITY14))
\end{verbatim}
\end{small}
  \caption{\SME\ generalizations for the simple water flow -- heat flow analogy.}
  \label{fg:generalizations}
\border
\end{figure}


\begin{enumerate}
 \item {\it Literally common aspects only.} This generalization locates
those sub-structures which are identical in both base and target Dgroups.
This is a type of generalization typically found in inductive generalization
programs.
 \item {\it All common aspects only.} In addition to common, identical
substructures, this generalization includes cases where functions of a
different name were allowed to match. Where this occurs in the common
structure, a skolemized function predicate is created.
 \item {\it Maximal generalization.} The largest generalization (in terms of
amount) includes all candidate inferences sanctioned by the Gmap, as well as
the common substructure of generalization mode 2. This represents the entire
shared structure between the two Dgroups under the assumption that the
candidate inferences are valid.
\end{enumerate}

For example, given the best Gmap from the simple water flow -- heat flow
analogy described in Section 2.1 and shown in Figure~\ref{fg:wfhf-done},
\SME\ will produce the set of generalizations shown in
Figure~\ref{fg:generalizations}. The first generalization indicates that the
only thing in common between the two situations is the existence of flow. The
second generalization loosens the meaning of ``in common'' to include the
fact that a quantity associated with the source of flow was greater than the
same quantity measured for the destination. The final generalization assumes
that this inequality, which was the cause of flow in the water flow domain,
is actually the cause of flow for both situations.


\subsection{Inspecting MH and Gmap evidence}\label{ss:inspecting-evidence}

\indent
\decfunc{match-evidence-inspector}{}\index{match hypothesis, inspecting evidence justifications}\index{match hypothesis, inspecting evidence justifications}

\spc{0.8}
When developing a theory about what types of rules should be used and how
much evidence for a particular match they should provide, it is often useful
to explicitly see what the different sources of evidence were for a
particular match item. The system utility {\tt match-evidence-inspector} may
be used to display a trace of the entire evidence facility or just the
evidence for a particular match hypothesis or Gmap. For example, the
following information was printed out about the {\tt pressure} to {\tt
temperature} match hypothesis in the water flow -- heat flow analogy:

\begin{small}
\begin{verbatim}
   (MH F#PRESS-BEAKER F#TEMP-COFFEE) has evidence (0.7120, 0.0000) due to
      IMPLICATION((MH F#>PRESSURE F#>TEMP))  (0.5200, 0.0000)
      IMPLICATION(CHILDREN-POTENTIAL)  (0.4000, 0.0000)
\end{verbatim}
\end{small}
\noindent
While the following information appears for the best Gmap in this analogy:
\begin{small}
\begin{verbatim}
   (GMAP #GM3) has evidence (5.9917, 0.0000) due to
       IMPLICATION((MH F#WFLOW F#HFLOW))  (0.7900, 0.0000)
       IMPLICATION((MH I#PIPE20 I#BAR23))  (0.6320, 0.0000)
       IMPLICATION((MH I#WATER17 I#HEAT24))  (0.6320, 0.0000)
       IMPLICATION((MH I#VIAL19 I#ICE-CUBE22))  (0.9318, 0.0000)
       IMPLICATION((MH I#BEAKER18 I#COFFEE21))  (0.9318, 0.0000)
       IMPLICATION((MH F#PRESS-BEAKER F#TEMP-COFFEE))  (0.7120, 0.0000)
       IMPLICATION((MH F#PRESS-VIAL F#TEMP-ICE-CUBE))  (0.7120, 0.0000)
       IMPLICATION((MH F#>PRESSURE F#>TEMP))  (0.6500, 0.0000)
\end{verbatim}
\end{small}

\noindent
The inspection facility will not work for Gmaps if their scores were
produced by an external (to the {\tt BMS}), destructive operation. One such
destructive rule appeared at the end of Section~\ref{ss:ME-rules}.


\subsection{Windows}\index{lisp machine interface}\index{scroll windows}

\indent
\decfunc{dump-scroll-menu}{}\index{scroll windows, saving contents}
\decfunc{dump-scroll}{output-pathname}
\decfunc{clear-scroll}{}

\decfunc{select-windowing-configuration}{}
\decfunc{select-scroll}{}
\decfunc{select-double-scroll}{}
\decfunc{select-graphics}{}
\decfunc{select-large-graphics}{}
\decfunc{select-split}{}

\decvar{*sme-frame*}
\decvar{*graphics-pane*}
\decvar{*scroll-pane*}
\decvar{*spare-scroll-pane*}
\decvar{*lisp-pane*}

\spc{0.8} The windowing system is lisp machine dependent and appears in the
file {\tt windowing.lisp}. The loading of this file is optional. When the
windowing system is used, a number of window configurations are possible,
such as having a single scroll window, two side by side, a single graphics
window, or a scroll and graphics window side by side. These configurations
may be selected through their individual functions (e.g., {\tt
select-scroll}), or through the central configuration facility {\tt
select-scroll-graphics}.  By default, when the windowing system is active,
all \SME\ output is sent to the primary scroll pane.\index{scroll windows,
writing to} When both scroll windows are in use, the configuration facility
allows one to specify which scroll window is currently active. The two
scroll dumping routines write the contents of the primary scroll window to a
specified file. Output sent to the secondary ({\it scratchpad}) scroll pane
is for observation only and cannot be written to a file.


\subsection{System parameters}\index{parameters}

\indent
\decvar{*sme-parameters*}
\decvar{*parameter-menu-options*}
\decmacro{defSME-Parameter}{variable-name string-description type {\tt \&optional} type-choices}

\spc{0.8}
The {\tt defSME-Parameter} form adds a new variable to the list of known
\SME\ parameters. This list is used by the windowing interface routines to
query the user about possible parameter settings. It is provided primarily
for application programs wanting to use the standard \SME\ parameter setting
facility. The arguments to {\tt defSME-Parameter} correspond to the appropriate
definitions for the {\tt choose-variable-values} function of your particular
machine. For example, the following declaration appears in {\tt match.lisp}:
\begin{small}
\begin{verbatim}
   (defSME-Parameter *display-all-MH* "Display all the Match Hypotheses" :boolean)
\end{verbatim}
\end{small}


\begin{figure}
\border
\vspace{1.2in}
\caption{SME System Parameters.}
\label{fg:parameters}
\border
\end{figure}


\spc{0.6}
\decfunc{change-parms}{}

\spc{0.4}
The windowing system provides a menu facility for viewing and changing the
current values of system parameters. This menu is shown in
Figure~\ref{fg:parameters}.


\subsection{System utilities}\index{system utilities}

\indent
\decvar{*system-utilities-menu*}
\decmacro{defSME-Utility}{string-name lisp-form}
\decfunc{menu-utilities}{}
\decfunc{get-dgroup}{}
\decfunc{get-rules}{}

\spc{0.8}
The \SME\ system utilities are the options that appear when the {\tt
Utilities} command is evoked, the right mouse button is pressed, or the
function {\tt menu-utilities} is called. These utilities include changing
the system parameters, choosing to load a Dgroup or rule file from those in
the defined directories, and clearing or writting to file the contents of
the scroll window. These routines are lisp machine dependent. The following
declaration appears in {\tt match.lisp}:
\begin{small}
\begin{verbatim}
   (defSME-Utility "Inspect Evidence" (match-evidence-inspector))
\end{verbatim}
\end{small}


\section{User Hooks}

This section describes the global variables and routines that are available
to the user and application programs for the creation and inspection of
analogical mappings.

\subsection{Applications control over display}

\indent
\decvar{*sme-output-stream*}
\decvar{*windowing?*}
\decvar{*sme-graphics-output*}

\spc{0.8}
All \SME\ textual display routines send their output to {\tt
*sme-output-stream*}. By default, the value of this variable is {\tt T},
which causes output to be sent to {\tt *terminal-io*}, CommonLisp's default
pointer to the user's console. When {\tt *sme-output-stream*} is a scroll
window (determined by the presence of an {\tt :append-item} handler), the
appropriate scroll window routines for sending display items are invoked.
Otherwise, text is sent to the current output stream using {\tt format}. 
Text routines are machine independent.

In a similar manner, all \SME\ graphics output is sent to the current
{\tt *sme-graphics-output*} window. Graphics output is lisp machine
dependent and relies on the {\tt ZGRAPH} graphics system.

When the \SME\ windowing system is in operation, {\tt *sme-output-stream*}
is set to the primary \SME\ scroll pane ({\tt *scroll-pane*}), {\tt
*sme-graphics-output*} is set to the \SME\ graphics pane ({\tt
*graphics-pane*}), and {\tt windowing?} is set to {\tt T}.
\index{scroll windows, writing to} 

\spc{0.7}
\decmacro{sme-format}{format-string {\tt \&rest} format-args}
\decfunc{sme-print}{string}
\decfunc{sme-terpri}{{\tt \&optional (}N {\tt 1)}}

\spc{0.3}
These routines provide a general interface for sending textual output to the
current \SME\ output stream. {\tt sme-format} is equivalent to CommonLisp's
{\tt format} routine, except that the printed output is always followed by a
newline. The {\tt sme-print} routine is provided for simple situations where
only a string is printed or for situations requiring the standard use of
{\tt format}, as in building up a line of text through multiple invocations.
The printed output of {\tt sme-print} is followed by a newline. When the
routine is used for multiple calls of {\tt format}, it should be used in
conjunction with CommonLisp's {\tt with-output-to-string}, as in:

\begin{small}
\begin{verbatim}
   (sme-print
      (with-output-to-string (stream)
         (format stream "~%Beginning of a line...")
         (format stream "   middle of a line...")
         (format stream "   end of a line.")))
\end{verbatim}
\end{small}

When a whole set of operations are carried out within the context of a
single {\tt sme-print}, one must be careful not nest calls to {\tt
sme-print} (e.g., calling a function in the context of an {\tt sme-print}
which itself invokes {\tt sme-print}). Such nesting will cause output 
to appear backwards from what was intended and may cause the output stream
to close improperly.


\subsection{Useful miscellaneous functions}

Data exists within \SME\ in three forms: (1) local items such as entities,
predicates, and expressions, (2) description groups (Dgroups), and (3) analogical
mapping information. The routines to create and query these items are
described in the following sections.

\subsubsection{Entities, predicates, and expressions}

\indent
\decfunc{entity?}{item}
\decfunc{entity-type?}{item}
\decmacro{entity-name?}{symbol}\index{entity, inspection functions}
\decmacro{fetch-entity-definition}{symbol}
\decmacro{entity-domain}{symbol}
\decmacro{constant-entity?}{symbol}

\spc{0.8}
Entities declared through {\tt defEntity} represent global entity types,
that is, they represent a class of entities rather than an actual instance
of an entity. When an entity type is used in the definition of a description
group, a unique entity instance is created for that type (e.g., {\tt beaker}
is translated to {\tt beaker73}). Thus, a given entity token will represent
either a type or an instance. The structure predicate {\tt entity?} returns
true if the given item is an entity-instance structure, while {\tt
entity-type?} returns true if the item is an entity-type structure. The
macro {\tt entity-name?} returns true if the given symbol represents either
an entity type or instance. {\tt fetch-entity-definition} will return the
entity-type structure for an entity type token or the entity-instance
structure for an entity instance token. The routines {\tt entity-domain} and
{\tt constant-entity?} refer to the {\tt type} and {\tt constant?} keyword
values given to {\tt defEntity} when the corresponding entity type was
created.

\spc{1}
\decvar{*sme-predicates*}
\decmacro{fetch-predicate-definition}{predicate-symbol}\index{predicate, inspection functions}
\decmacro{predicate?}{symbol}
\decmacro{predicate-type}{predicate-symbol}
\decmacro{relation?}{predicate-symbol}
\decmacro{attribute?}{predicate-symbol}
\decmacro{function?}{predicate-symbol}
\decmacro{commutative?}{predicate-symbol}
\decmacro{n-ary?}{predicate-symbol}
\decmacro{arg-list}{predicate-symbol}
\decmacro{numargs}{predicate-symbol}
\decmacro{expression-type}{predicate-symbol}
\decmacro{eval-form}{predicate-symbol}

\spc{0.4}
These routines provide the facility to access the various predicate properties
defined with the {\tt defPredicate} form. {\tt fetch-predicate-definition}
returns the actual sme-predicate structure containing all the information
about a given predicate.

\spc{1}
\index{expression, inspection functions}
\decfunc{fetch-expression}{expression-name dgroup
                     {\tt \&optional (}error-if-absent? {\tt T)}}
\decfunc{expression-functor}{expression-structure}
\decfunc{fully-expand-expression}{expression-structure dgroup}
\decfunc{fully-expand-expression-form}{expression-form dgroup}

\spc{0.4}
An ``expression'' represents an actual predicate instance within a Dgroup.  Notice
that this includes terms corresponding to function applications as expressions.
Each use of a predicate gets its own expression with its own name, so that a
higher-order relation gets translated into several expressions, with some having
expressions as arguments. These routines allow one to retrieve and inspect expressions
in the database. {\tt fetch-expression} returns the expression structure with the given
name.

The routines {\tt fully-expand-expression} and {\tt fully-expand-expression-form} are
useful for examining the form of an expression. Typically, the expression {\tt
(greater-than (diameter beaker) 5)} is stored as the expression {\tt greater-than23},
which has the form {\tt (greater-than diameter24 5)}. These routines return
a fully expanded expression form, where all expression names are replaced by their
corresponding forms.


\subsubsection{Dgroups}

\indent
\decfunc{fetch-dgroup}{dgroup-name {\tt \&optional} create?}\index{description group, retrieval}
\decfunc{return-dgroup}{dgroup-or-dgroup-name}

\spc{0.8}
Description groups (Dgroups) are stored in a simple data base managed
primarily by routines in {\tt sme.lisp}. The general procedures for Dgroup
and expression creation were described in sections~\ref{ss:new-dgroups} and
\ref{ss:new-expressions}. A Dgroup may be retrieved by name using
{\tt fetch-dgroup}, or created if {\it create?} is non-nil and no Dgroup
with the given name currently exists. {\tt return-dgroup} is designed for
routines that may take either an actual Dgroup or simply a Dgroup name
(e.g., {\tt fetch-expression}). It will cause an error if the Dgroup does not
previously exist.

\spc{0.8}
\decfunc{describe-dgroup}{dgroup}\index{description group, textual description}
\decfunc{menu-display-dgroup}{}\index{description group, graphical display}
\decfunc{menu-display-pairs}{}

\spc{0.3}
A Dgroup may be textually described using {\tt describe-Dgroup}, which
writes to the \SME\ output stream. Graphical display is provided in the
windowing system by {\tt menu-display-dgroup} for a single Dgroup or {\tt
menu-display-pairs} for a display of two Dgroups side by side.


\subsubsection{Creating and inspecting global matches}\label{ss:match}

\indent
\decfunc{match}{base-name target-name {\tt \&optional} display?}\index{global mapping, creation}
\decfunc{best-gmaps}{{\tt \&optional (}gmaps {\tt *gmaps*) (}percentage-range {\tt 0.02)}}\index{global mapping, selecting the best}
\decfunc{display-match}{base target {\tt \&optional (}total-run-time {\tt 0) (}bms-run-time {\tt 0)}}\index{global mapping, textual display}

\spc{0.8}
The {\tt match} function is the central \SME\ procedure. Given the names of two
Dgroups, it forms the complete set of global mappings between them. If {\it display?}
is non-nil, a description of the results will be sent to the current \SME\
output stream. The function itself returns two values, the total run time of
the match process in seconds and the subset of that time spent running the
BMS evidence rules. The analogical mapping results are stored in the
following global variables, which are then accessible by the user or
application program.

\spc{0.5}
\decvar{*base*}
\decvar{*target*}
\decvar{*match-hypotheses*}\index{match hypothesis, *match-hypotheses*}
\decvar{*gmaps*}\index{global mapping, *gmaps*}
\spc{0.2}

\noindent
The Gmap(s) with the highest evaluation score are retrieved by
{\tt best-gmaps}, which returns all Gmaps having a score within a given
percentage (default is 2\%) of the highest score. {\tt best-gmaps} returns
two values: the list of best Gmaps and the actual real-valued highest score.

\spc{0.8}
\decfunc{compare-gmaps}{gmap1 gmap2 {\tt \&optional} display?}\index{global mapping, comparing apparently identical gmaps}

\spc{0.4}
Occasionally, \SME\ will produce two or more ``best'' Gmaps that appear to
be identical yet have been classified as distinct. When these Gmaps are
large, the ``here's the set of match hypotheses'' output format can make it
frustrating to find what the slight differences are between a pair of Gmaps.
When given two Gmap structures, {\tt compare-gmaps} will return (using {\tt
values}) the list of the match hypotheses that are uniquely part of the
first Gmap and a list of match hypotheses that are uniquely part of the
second Gmap (i.e., ({\it gmap1} - {\it gmap2}) and ({\it gmap2} - {\it
gmap1})). When the windowing system is active, this option is available
through the system utilities menu.


\section{Algorithm Internals}

This section quickly describes a few internal points of the program in case
one has specialized needs for interfacing to the code. It assumes knowledge
of CommonLisp and the realization that for many questions, the only feasible
answer must be to examine the \SME\ program.

\subsection{The Match Function}

\indent
\decfunc{create-match-hypotheses}{base-dgroup target-dgroup}
\decfunc{run-rules}{}\index{rule system, run-rules}
\decfunc{calculate-nogoods}{base-dgroup target-dgroup}
\decfunc{generate-gmaps}{}
\decfunc{gather-inferences}{base-dgroup target-dgroup}\index{candidate inference, gather-inferences}
\decfunc{intern-gmaps}{}\index{rule system, giving it access to gmaps}

\spc{0.8}
The function {\tt match} is primarily a sequence of calls to these
functions. {\tt create-match-hypotheses} runs the match constructor
rules to form the individual match hypotheses. The BMS evidence rules are
then run on these match hypotheses ({\tt run-rules}) and their
dependence and inconsistency relationships are determined ({\tt
calculate-nogoods}). The function {\tt generate-gmaps} executes the three
merge steps, resulting in the set of complete global mappings being placed
in the variable {\tt *gmaps*}. The candidate inferences each Gmap
sanctions is then calculated ({\tt gather-inferences}) and additive BMS
nodes for each Gmap are formed ({\tt intern-gmaps}), allowing the evidence
rules to run on each Gmap.

\subsection{Match Hypotheses}\index{match hypothesis, defstruct}

\makebox[\textwidth]{{\tt match-hypothesis} \hfill [{\it Defstruct}]}
\decsubst{mh-form}{match-hypothesis}
\decsubst{mh-base-item}{match-hypothesis}
\decsubst{mh-target-item}{match-hypothesis}
\decsubst{mh-bms-node}{match-hypothesis}
\decsubst{node-belief+}{bms-node}
\decsubst{mh-plist}{match-hypothesis}
\decvar{*match-hypotheses*}

\spc{0.8}
Most programs using \SME\ will need to interact with the {\it match
hypothesis structures}. Slots to this structure type use the {\tt mh-}
prefix. There are several slots that might be important. The MH form, which
is a list of {\tt (MH {\it $<$base-item$>$} {\it $<$target-item$>$})}, is found
using {\tt mh-form}. This is the form used for triggering the MH evidence
rules, and is asserted in the BMS for each match hypothesis. The base and
target items are expression or entity structures. The base item or target item may
be obtained directly using {\tt mh-base-item} and {\tt mh-target-item},
respectively.  The BMS node for each match hypothesis is found by {\tt
mh-bms-node}. In turn, the weight for that node may be obtained using {\tt
node-belief+}.  Finally, each match hypothesis structure has a property list
slot ({\tt mh-plist}) which may be useful for various purposes.

\subsection{Global Mappings}\index{global mapping, defstruct}

\makebox[\textwidth]{{\tt global-mapping} \hfill [{\it Defstruct}]}
\decsubst{gm-id}{global-mapping}
\decsubst{gm-elements}{global-mapping}
\decsubst{gm-emaps}{global-mapping}
\decsubst{gm-base}{global-mapping}
\decsubst{gm-target}{global-mapping}
\decsubst{gm-inferences}{global-mapping}
\decsubst{gm-bms-node}{global-mapping}
\decsubst{gm-plist}{global-mapping}
\decvar{*gmaps*}

\spc{0.8}
Each Gmap is stored as a {\it global-mapping structure}. Slots to this
structure type use the {\tt gm-} prefix. Each Gmap is assigned a unique
integer identifier, found through {\tt gm-id}. The Gmap form used by the BMS
is not explicitly available, but is asserted as {\tt (Gmap {\it
$<$gmap-structure$>$})}. The match hypotheses associated with a Gmap are stored
in {\tt gm-elements}, while the subset of these that are entity mappings is
stored in {\tt gm-emaps}. The candidate inferences sanctioned by the Gmap
appear in {\tt gm-inferences} and are stored as a simple list data type,
using the syntax defined in Section~\ref{ss:new-dgroups} for description
group expressions.

\subsection{Candidate Inference Generation}\index{candidate inference, alternate algorithms}

The original candidate inference generator, as described in \cite{aaai86},
will take any base structure ``intersecting the Gmap structure''. The newer
(V. 2) edition only takes base structure ``intersecting a Gmap root''.
Thus, the newer edition is more cautious and far more efficient than the
older edition. Both versions of the code are available (in {\tt
match.lisp}), with the default being the newer, more cautious version. There
are theoretical arguments for and against each approach. For example, one
might want to use only the inferences from the newer, more cautious approach
at first since they are supported by more target knowledge and thus more
likely valid. If an analogy proves fruitful, one may want to relax these
constraints, and use the older version to find out what additional
inferences might be made.

\subsection{Rule System}

\indent
\decfunc{tre-rules-file}{}\index{rule system, dynamically swapping rule sets}
\decfunc{tre-save-rules}{}
\decfunc{tre-init}{}
\decfunc{restore-rules}{}
\decfunc{run-rules}{}
\decvar{*tre-rules-saver*}
\decvar{*initial-assertions*}

\spc{0.8} 
When a new rules file is loaded, the sme-rules-file command at the top of
the file initializes the BMS rule system prior to loading the new set of
rules.  At the bottom of the rules file, the fresh, just-loaded set of rules
are saved in the global variable *tre-rules-saver* by the command
tre-save-rules.  This variable saves the status of the rule counters and the
list of initial rules. A similar variable, {\tt *initial-assertions*}, is
used to store all assertions appearing in the rules file. When the BMS is
run, new rules may be created and added to the known set of rules. As a
result, each time {\tt match} is invoked, the BMS is reinitialized to its
status just after loading the rule file, that is, it is restored to the
status indicated in {\tt *tre-rules-saver*}.

This facility may be used by application programs to save different rule
sets in memory and swap them as needed, without having to load rule files
each time. For example, suppose \SME\ is invoked, which will cause it to run
the current rule set for Gmap scoring. If a second scoring criterion is then
desired, a second set of rules may be invoked using code of the form:
\begin{small}
\begin{alltt}
   (let ((save-rules sme:*tre-rules-saver*)          {\it ;save the previous rule set}
         (save-assertions sme:*initial-assertions*))
     (setq sme:*tre-rules-saver* *my-other-rules-set-rules*)
     (setq sme:*initial-assertions* *my-other-rules-set-assertions*)
     (sme:tre-init)                                  {\it ;initialize with new rules}
     (sme:run-rules)                                 {\it ;run the new rule set}
     (setq sme:*tre-rules-saver* save-rules)         {\it ;restore the original rules}
     (setq sme:*initial-assertions* save-assertions))
\end{alltt}
\end{small}
This saves \SME's normal rule set, runs a different one, and then restores
the rule set to its previous value. In this example, {\tt tre-init} was
used, which fully initializes the BMS. If the desire is simply to supply
additional rules without destroying the current BMS state, {\tt
restore-rules} should be used in place of {\tt tre-init}.  The variables
corresponding to ``my-other-rules-set'' may be initialized by a similar
program which saves the current rule set, {\it loads} the desired ``other
rule set'' file, sets the ``my-other-rules-set'' variables from {\tt
*tre-rules-saver*} and {\tt *initial-assertions}, and then restores the
original rule set.


\section{Summary}

The \SME\ program has been described from the perspective of how to actually
use it. A number of methods have been presented about how to configure \SME\
to perform a variety of different types of matches.  It is hoped that \SME\
may serve as a general mapping tool for research on analogical mapping,
allowing researchers to focus on the more substantive issue of general
theories of analogical mapping, as opposed to worrying about implementation
details. The latter has the unfortunate effect of producing the repeated
scenario in which analogy researcher A goes to analogy researcher B and says
``My program can do X, which yours cannot'', followed by researcher B
returning a month later with this simple modification added. By testing
different theories within the same program, we may now compare the more
critical ``This is a logical consequence of my theory''. A program does not
a theory make. It can, however, function as a useful analytical tool.

Of course, not all of the problems of implementing analogical mapping have
been solved.  Most critical is redesigning the potentially combinatoric
merge step 3, perhaps using either a heuristic search or connectionist
relaxation network as suggested in \cite{sme-aij}. Of theoretical relevance
is the appropriateness of the abstract structural approach which \SME\
embodies.

\section{Acknowledgements}

The development of \SME\ has been a collaborative effort with Ken Forbus and
Dedre Gentner, with significant influence provided by Janice Skorstad. This
work has also benefited from discussions with Steve Chien, John Collins, and
Ray Mooney. 

This research is supported in part by an IBM graduate fellowship and in part
by the Office of Naval Research, Contract No. N00014-85-K-0559.

\begin{thebibliography}{99}\addcontentsline{toc}{section}{References}

 \bibitem{Burstein83}
Burstein, M., Concept formation by incremental analogical reasoning and
debugging, {\it Proceedings of the Second International Workshop on
Machine Learning}, University of Illinois, Monticello, Illinois, June,
1983.  A revised version appears in {\it Machine Learning: An Artificial
Intelligence Approach Vol. II}, R.S. Michalski, J.G. Carbonell, and T.M.
Mitchell (Eds.), Morgan Kaufman, 1986.
 \bibitem{Carbonell83}
Carbonell, J.G., Learning by Analogy: Formulating and generalizing plans
from past experience, in: {\it Machine Learning: An Artificial
Intelligence Approach}, R.S. Michalski, J.G. Carbonell, and T.M.
Mitchell (Eds.), Morgan Kaufman, 1983.
 \bibitem{BMS}
Falkenhainer, B.,
Towards a general-purpose belief maintenance system, in: J.F. Lemmer \& L.N.
Kanal (Eds.), {\it Uncertainty in Artificial Intelligence, Volume II}, 1987.
Also Technical Report, UIUCDCS-R-87-1717, Department of Computer Science,
University of Illinois, 1987.
 \bibitem{VBAL}
Falkenhainer, B.,
An examination of the third stage in the analogy process:
Verification-Based Analogical Learning, Technical Report
UIUCDCS-R-86-1302, Department of Computer Science, University of Illinois,
October, 1986. A summary appears in {\it Proceedings of the Tenth
International Joint Conference on Artificial Intelligence}, Milan, Italy,
August, 1987.
 \bibitem{MLW}
Falkenhainer, B.,
Scientific theory formation through analogical inference, {\it
Proceedings of the Fourth International Machine Learning Workshop}, Irvine,
CA, June, 1987.
 \bibitem{DBR}
Falkenhainer, B.
The utility of difference-based reasoning, {\it Proceedings of the Seventh
National Conference on Artificial Intelligence}, St. Paul, August, 1988.
 \bibitem{Falken-phd}
Falkenhainer, B., Learning from physical analogies: A study in analogy
and the explanation process, Ph.D. Thesis, University of Illinois,
December, 1988.
 \bibitem{aaai86}
Falkenhainer, B., K.D. Forbus, D. Gentner,
The Structure-Mapping Engine, {\it Proceedings of the Fifth National
Conference on Artificial Intelligence}, August, 1986.
 \bibitem{sme-aij}
Falkenhainer, B., K.D. Forbus, D. Gentner,
The Structure-Mapping Engine:  Algorithm and Examples, Technical Report
UIUCDCS-R-87-1361, Department of Computer Science, University of Illinois,
July, 1987. To appear in {\it Artificial Intelligence}.
 \bibitem{framework}
Forbus, K.D. and D. Gentner.
Learning physical domains: Towards a theoretical framework,
in: {\it Machine Learning: An Artificial Intelligence Approach Vol. II}, R.S. Michalski,
J.G. Carbonell, and T.M. Mitchell (Eds.), Morgan Kaufmann, 1986.
 \bibitem{Gentner80}
Gentner, D.
{\it The Structure of Analogical Models in Science}, BBN Tech.
Report No. 4451, Cambridge, MA., Bolt Beranek and Newman Inc., 1980.
 \bibitem{smt-cogsci}
Gentner, D.
Structure-Mapping: A Theoretical Framework for Analogy, {\it Cognitive
Science} {\bf 7}(2), 1983. 
 \bibitem{allerton}
Gentner, D.
Mechanisms of analogy.  To appear in S. Vosniadou and A. Ortony, (Eds.),
{\em Similarity and analogical reasoning}, Cambridge University Press,
Oxford.
 \bibitem{analogica}
Gentner, D.
Analogical inference and analogical access, To appear in A. Preiditis (Ed.),
{\em Analogica: Proceedings of the First Workshop on Analogical Reasoning},
London, Pitman Publishing Co.  Presented in December, 1986.
 \bibitem{tinlap3}
Gentner, D., B. Falkenhainer, and J. Skorstad
Metaphor: The good, the bad and the ugly. {\it Proceedings of the Third
Conference on Theoretical Issues in Natural Language Processing}, Las
Cruces, New Mexico, January, 1987.  
 \bibitem{Greiner}
Greiner, R., Learning by understanding analogies, PhD Thesis (STAN-CS-1071),
Department of Computer Science, Stanford University, September, 1985.
 \bibitem{hayes-roth}
Hayes-Roth, F., J. McDermott, An interference matching technique for
inducing abstractions, {\it Communications of the ACM}, {\bf 21}(5),
May, 1978.
 \bibitem{ACME}
Holyoak, K., \& Thagard, P. Analogical mapping by constraint satisfaction,
June, 1988, ({\it submitted for publication}).
 \bibitem{Indurkhya}
Indurkhya, B., Constrained semantic transference: A formal theory of
metaphors, Technical Report 85/008, Boston University, Department of
Computer Science, October, 1985.
 \bibitem{KedarCabelli}
Kedar-Cabelli, S., Purpose-directed analogy. {\it Proceedings of the Seventh
Annual Conference of the Cognitive Science Society}, Irvine, CA, 1985.
 \bibitem{Reed}
Reed, S.K., A structure-mapping model for word problems. Paper presented
at the meeting of the Psychonomic Society, Boston, 1985.
 \bibitem{Rumelhart}
Rumelhart, D.E., \& Norman, D.A., Analogical processes in learning.  In J.R.
Anderson (Ed.), {\it Cognitive skills and their acquisition}, Hillsdale,
N.J., Erlbaum, 1981.
 \bibitem{aaai87}
Skorstad, J., B. Falkenhainer and D. Gentner
Analogical Processing: A simulation and empirical corroboration, {\it
Proceedings of the Sixth National Conference on Artificial Intelligence},
Seattle, WA, August, 1987.
 \bibitem{Winston80}
Winston, P.H., Learning and Reasoning by Analogy, {\it Communications of the
ACM}, {\bf 23}(12), 1980.
 \bibitem{Winston82}
Winston, P.H., Learning new principles from precedents and exercises,
{\it Artificial Intelligence}, {\bf 19}, 321-350, 1982.

\end{thebibliography}

\newpage
\addcontentsline{toc}{section}{Index}

\begin{theindex}

\item algorithm, 4-10
	\subitem  summary, 3
\item analogical hint, 17
\item arg-list macro, 28
\item {\tt assert!} function, 14
\item {\tt attribute?} macro, 28
\indexspace
\item {\tt *base*} variable, 29
\item batch mode, 22
\item {\tt best-gmaps} function, 29
\indexspace
\item {\tt calculate-nogoods} function, 30
\item candidate inference, 9
	\subitem  alternate algorithms, 31
	\subitem  gather-inferences, 30
\item {\tt change-parms} function, 26
\item {\tt children-of?} function, 15
\item {\tt clear-given-mappings} function, 18
\item {\tt clear-scroll} function, 25
\item {\tt commutative?} macro, 28
\item {\tt compare-gmaps} function, 29
\item {\tt constant-entity?} macro, 27
\item conventions, 1
	\subitem  file names, 20
	\subitem  function names, 1
\item {\tt create-match-hypotheses} function, 30
\indexspace
\item {\tt declare-given-mappings} function, 18
\item {\tt defDescription} macro, 12
\item {\tt defEntity} macro, 11
\item {\tt define-description} function, 12
\item {\tt define-Entity} function, 11
\item {\tt define-predicate} function, 10
\item {\tt defPostMatcher} macro, 23
\item {\tt defPredicate} macro, 10
\item {\tt defSME-Parameter} macro, 25
\item {\tt defSME-Utility} macro, 26
\item {\tt describe-dgroup} function, 29
\item description group, 11-12
	\subitem  graphical display, 29
	\subitem  retrieval, 29
	\subitem  textual description, 29
\item Dgroup, 12
\item {\tt dgroup-directory} macro, 22
\item {\tt dgroup-file} macro, 22
\item {\tt display-match} function, 29
\item {\tt dump-scroll} function, 25
\item {\tt dump-scroll-menu} function, 25
\indexspace
\item entity, 11
	\subitem  declaring, 11
\item {\tt entity?} function, 27
	\subitem  inspection functions, 27
\item {\tt entity-domain} macro, 27
\item {\tt entity-name?} macro, 27
\item {\tt entity-type?} function, 27
\item {\tt eval-form} macro, 28
\item execution, 20
\item expression, 12
	\subitem  adding new expressions, 12
\item {\tt expression} function, 12
	\subitem  inspection functions, 28
\item {\tt expression-functor} function, 28
\item {\tt expression-type} macro, 28
\indexspace
\item {\tt fetch-dgroup} function, 29
\item {\tt fetch-entity-definition} macro, 27
\item {\tt fetch-expression} function, 28
\item {\tt fetch-predicate-definition} macro, 28
\item file organization, 1
\item {\tt fully-expand-expression} function, 28
\item {\tt fully-expand-expression-form} function, 28
\item {\tt function?} macro, 28
\indexspace
\item {\tt gather-inferences} function, 30
\item generalize, 16
\item {\tt generalize} function, 23
\item {\tt generate-gmaps} function, 30
\item {\tt get-dgroup} function, 26
\item {\tt get-rules} function, 26
\item global mapping, 7-10
	\subitem  comparing apparently identical gmaps, 29
	\subitem  creation, 29
	\subitem  defstruct, 30
	\subitem  {\tt *gmaps*}, 29
	\subitem  scoring, 9, 15
	\subitem  selecting the best, 29
	\subitem  textual display, 29
\item {\tt *gmaps*} variable, 29, 31
\item {\tt gm-base} subst, 31
\item {\tt gm-bms-node} subst, 31
\item {\tt gm-elements} subst, 31
\item {\tt gm-emaps} subst, 31
\item {\tt gm-id} subst, 30
\item {\tt gm-inferences} subst, 31
\item {\tt gm-plist} subst, 31
\item {\tt gm-target} subst, 31
\item {\tt *graphics-pane*} variable, 25
\indexspace
\item {\tt initial-assertion} macro, 14
\item {\tt *initial-assertions*} variable, 31
\item installation, 20
\item {\tt install-MH} function, 13
\item {\tt intern-gmaps} function, 30
\indexspace
\item {\tt language-file} macro, 22
\item lisp machine interface, 25
	\subitem  predicate documentation, 11
\item {\tt *lisp-pane*} variable, 25
\indexspace
\item match constructor rules, 13
\item match evidence rules, 14
\item {\tt match} function, 29
\item match hypothesis, 4-5
	\subitem  defstruct, 30
	\subitem  inspecting evidence justifications, 24
	\subitem  installing, 13
	\subitem  {\tt *match-hypotheses*}, 29
	\subitem  scoring, 14-15
\item {\tt match-evidence-inspector} function, 24
\item {\tt *match-hypotheses*} variable, 29-30
\item {\tt menu-display-dgroup} function, 29
\item {\tt menu-display-pairs} function, 29
\item {\tt menu-utilities} function, 26
\item {\tt mh-base-item} subst, 30
\item {\tt mh-bms-node} subst, 30
\item MHC-rule, 13
\item {\tt mh-form} subst, 30
\item {\tt mh-plist} subst, 30
\item {\tt mh-target-item} subst, 30
\indexspace
\item {\tt n-ary?} macro, 28
\item {\tt node-belief+} subst, 30
\item {\tt numargs} macro, 28
\indexspace
\item packages, 1
	\subitem site specific, 20
\item {\tt paired-item?} function, 19
\item {\tt *parameter-menu-options*} variable, 25
\item parameters, 25
	\subitem  site specific, 1, 20
\item predicate, 10
	\subitem  declaring, 10
	\subitem  inspection functions, 28
\item {\tt predicate?} macro, 28
\item {\tt predicate-type} macro, 28
\indexspace
\item {\tt rassert!} macro, 14
\item {\tt relation?} macro, 28
\item report generation, 22
\item {\tt report-comments} macro, 23
\item representation issues, 19
\item {\tt restore-rules} function, 31
\item {\tt return-dgroup} function, 29
\item {\tt rule} macro, 14
\item rule system, 12-19, 19
	\subitem  analogical hints, 17
	\subitem  dynamically swapping rule sets, 31
	\subitem  giving it access to gmaps, 30
	\subitem  match constructor rules, 13
	\subitem  match evidence rules, 14
	\subitem  pure isomorphisms, 16
	\subitem  relaxing identical predicates, 16
	\subitem  rule file syntax, 13
	\subitem  {\tt run-rules,} 30
	\subitem  simulating SPROUTER, 16
	\subitem  simulating structure-mapping theory, 15
\item {\tt rule-directory} macro, 22
\item {\tt rule-file} macro, 22
\item {\tt rule-sets} macro, 22
\item {\tt run-batch-file} function, 22
\item {\tt run-matcher-on} macro, 23
\item running SME, 20
\item {\tt run-rules} function, 30-31
\indexspace
\item {\tt sanctioned-pairing?} function, 19
\item scroll windows, 25
	\subitem  saving contents, 25
	\subitem  writing to, 25, 27
\item {\tt *scroll-pane*} variable, 25
\item {\tt select-double-scroll} function, 25
\item {\tt select-graphics} function, 25
\item {\tt select-large-graphics} function, 25
\item {\tt select-scroll} function, 25
\item {\tt select-split} function, 25
\item {\tt select-windowing-configuration} function, 25
\item {\tt send-report-to} macro, 23
\item site specific information, 20
	\subitem  packages, 20
	\subitem  pathnames, 20
\item {\tt *sme-default-rules*} variable, 20
\item {\tt *sme-dgroup-pathname*} variable, 20
\item {\tt *sme-files*} variable, 20
\item {\tt sme-format} macro, 27
\item {\tt *sme-frame*} variable, 25
\item {\tt *sme-graphics-output*} variable, 26
\item {\tt sme-init} function, 20
\item {\tt *sme-language-file*} variable, 20
\item {\tt *sme-output-stream*} variable, 26
\item {\tt *sme-parameters*} variable, 25
\item {\tt *sme-predicates*} variable, 28
\item {\tt sme-print} function, 27
\item {\tt sme-rules-file} function, 13
\item {\tt *sme-rules-pathname*} variable, 20
\item {\tt *sme-system-pathname*} variable, 20
\item {\tt sme-terpri} function, 27
\item {\tt *spare-scroll-pane*} variable, 25
\item system utilities, 26
\item {\tt *system-utilities-menu*} variable, 26
\indexspace
\item {\tt *target*} variable, 29
\item {\tt *the-lisp-package*} variable, 20
\item {\tt *the-user-package*} variable, 20
\item {\tt tre-init} function, 31
\item {\tt tre-rules-file} function, 31
\item {\tt *tre-rules-saver*} variable, 31
\item {\tt tre-save-rules} function, 13, 31
\indexspace
\item {\tt *windowing?*} variable, 26
\indexspace
\end{theindex}

%%\input{users-guide.index}

\end{document}


\end{document}


\end{document}


\end{document}
