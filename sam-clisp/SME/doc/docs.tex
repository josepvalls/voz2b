\documentstyle [12pt]{article}
\raggedright
\renewcommand{\baselinestretch}{1.0}
\renewcommand{\textfraction}{0.2}
\setlength{\parindent}{0.5in}
\setlength{\textwidth}{6in}
\setlength{\textheight}{8.5in}
\setlength{\headheight}{0.2in}
\setlength{\topmargin}{-.5in}
\newcommand{\cpic}[1]{\begin{figure}[htb]
        \centerline{\psfig{figure=#1}}
        \end{figure}}

\begin{document}
\vspace{2in}
\begin{center} 
{\Large Sme Window System Documentation}

\vspace {0.2in}
by

Greg Siegle


\vspace{2in}
\pagebreak

\begin{center}
{\em {\bf Sme Window System Documentation}}     
\end{center}

\vspace{0.2in}

\section{Introduction}

The following manual describes the CLIM window system interface for
SME version 2e as implemented for the IBM RS6000. The entire interface
resides in the file called {\tt windowing.lisp} and is automatically
loaded on any clim system.

The manual assumes you have a working knowledge of SME, and attempts
to preserve formalisms developed in Brian Falkenheimer's original SME
User's Manual. Special attention is given to features not included in
any previous version of SME.

\section{Getting Started} 

To begin using the interface type {\tt sme::start-windows} from the
Lisp prompt. After about 45 seconds a large window should appear in
the middle of your screen containing the entire interface. All
interaction with SME may be done via the menu options in this window.

The screen is composed of a top line menu, a scroll pane in which data
is displayed (in blue on color systems), and a lisp listener (pink on
color systems).

The top line options include:

\begin{itemize}   
\item File Options -- All interaction with outside files, including
Initialization (which loads the default lanugage file), and loading of
all rule and dgroup files. Importantly, SME must be initialized via
this menu before most other top line menu options will be useful.

\item Match Operations -- All operations governing creation and
examination of matches.

\item Utilities -- all options not listed anywhere else. Most notably
this menu governs how windows are configured, various system
parameters, and lets you access the Zgraph (generic graphing utility)
menu system. 

\item Describe Dgroup -- shows description of a dgroup in the scroll pane.

\item Display DGroup and Display DGroup Pairs -- graphically display
tree representing a dgroup on a graphics pane using Zgraph, a generic
graphing utility.

\item Inspect Dgroups -- shows dgroup concordances and pretty prints dgroups

\item Reports -- generates background reports describing dgroups in
ascii or latex formats

\item Use Lisp -- Toggles activation of the Lisp listener at the
bottom of the window. While it might, at first, seem that the lisp
listener should always be active, there are various problems with this
approach. To begin, the lisp listener traps all lisp errors, so
if problems with SME occur you may not know if the listener is
active.  In addition, due to an artifact of CLIM, the listener
automatically shuts itself off when you switch to the Zgraph
application. Use the ``Use Lisp'' option to turn it back on. Finally,
due to another artifact of CLIM, menu documentation disappears when
the Lisp Listener is active.

\end {itemize}

Each of these options is described in more depth in the next section.

\section{The Top level menu}
\subsection {File Operations}
File operations include 
\begin{itemize}
\item Initialize  -- must be executed before other options on this
menu, and most other menus are useful. This option clears out all
current dgroups and matches, but NOT rules files. Note that if the
window system is temporarily exitted and returned to, you DO NOT need
to reinitialize. All former variables are preserved.
\item Change File Parameters -- changes default files. We note that
extra language files are not yet supported
\item Load a Dgroup
\item Load a Rules File
\edn {itemize}
\subsection {Match Operations}
Most of the match options are the same as in previous versions, but
there are some new additions. They include:
\begin{itemize}
\item Set Match Parameters -- sets the base, target, and rules file
for a match
\item Run Match -- does the match
\item Examine Match -- shows base, target, rules, statistics, gmaps, etc.
\item Compare Gmaps -- shows differences in gmaps
\item Generalize -- Not supported, and generally, not working
\item Display match -- A new display which graphically illustrates a
gmap in the scroll pane. This option depicts the tree describing the
base on the x-axis and the tree describing the target on the y-axis of
a grid. All matches are displayed as x's at coordinate described by
corrosponding elements of the base and target. Because of space
limitations all elements are shown as numbers. A key to what elements
actually represent may be found on the spare scroll pane (accessed
through the ``Alter Windows'' option on the {\tt Utilities} menu. It
may also be helpful to graphically display the base and target dgroups
via the {\tt Display Dgroup Pairs} option and view these concurrently
with the Displayed match.
\end{itemize}

\subsection {Utilities}
\begin{Itemize}
\item Change System Parameters -- allows you to edit all standard
system parameters including default files
\item Alter Windowing -- Allows you to choose from 6 different window
configurations. They are relatively self explanatory. To tell your
current window configuration it is useful to note that on color
systems, the scroll and spare scroll panes are always blue. The
graphics window is a light lavender and the lisp listener is always
pink. 
\item Clear Scroll Window -- Clears all output from the scroll window.
This option is useful primarily because redisplaying the screen
eventually becomes slow if the window has a lot of text in it.
\item Dump Scroll Window -- dumps scroll window to a file. Usefull
because you cannot cut and paste from it.
\item Inspect Evidence
\item Zgraph Inspector -- Transfers you to the Zgraph menu system.
This system allows you to manipulate graphs with the greatest of ease.
It also allows you to scale graphs so they will fit on your window and
do post script output. There should be a zgraph manual floating around
somewhere... You can use the Zgraph menu options to operate on graphs
drawn in the SME graphics pane.
\item Switch Drawing Location -- Generally graphs are drawn on the
graphics pane of the SME interface. Yet, there are times when it is
helpful to have graphs directly displayed in Zgraph. This option
routes all graphs drawn with {\tt Display Dgroup} and {\tt Display
Dgroup Pairs} to the zgraph pane. Therefore, when you execute these
options, they will not appear on the SME pane. Don't panic. Blithely
switch to the zgraph pane using the ``Zgraph Inspector'' option to
view them.
\end {itemize}

\subsection {Describe Dgroup}
\subsection{Display Dgroup and Display Dgroup Pairs}
Allows you to graphically display dgroups on the SME graphics pane or
Zgraph pane.  Notably, graphs may not appear in the visible window
when you execute this option. If such an unfortunate event should
occur, do not dispair. Use the scroll bars to search around for your
graph. It should be in there somewhere.

Once a graph has been displayed the mouse serves special functions in
the graphics pane. Clicking the left button on empty space allows you
to ``drag'' the entire picture around. Click once at the place you
wish to begin dragging from. Click again where you would like to
deposit the picture.

Clicking the left button on a vertex (red dot) allows you to move the
vertex. It is ``picked up'' when you push the left button down, and
``put down'' when you release the left button.

Clicking the middle button on a vertex allows you to inspect it. All
inspecting is done in the original lisp listener window  (not the SME
lisp listener... this should change sometime). TO GET OUT OF THE
INSPECTOR, type {\tt :q} at the {\tt >>} prompt which will appear in
the original lisp listener window.

\end{document}
